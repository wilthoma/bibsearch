\documentclass{amsart}
% Package
\usepackage[all]{xy}
\usepackage{tikz}
 \tikzstyle{int}=[circle, draw,fill=black,outer sep=0,minimum size=3pt, inner sep=0]
  \tikzstyle{ext}=[circle, draw=black,outer sep=0,inner sep=1pt]
\usepackage{latexsym}
\usepackage{amssymb}
\usepackage{amsmath}
\usepackage{amsthm}
\usepackage{amscd}
\usepackage{fancyhdr}
%\usepackage[dvips]{graphicx}
\usepackage{fullpage}
\usepackage{mathrsfs}
\input cyracc.def
%\usepackage{xypic}
%\usepackage{showkeys}
\xyoption{arc}
%\usepackage{cyrillic}




% Parametres

\setlength{\parindent}{0pt} \addtolength{\headsep}{0.5cm}




  %******************************















  %******************************
  \newcommand{\ac}{\scriptstyle \textrm{!`}}
\def\id{{\mbox{1 \hskip -8pt 1}}}
\newcommand{\sgn}{{\mathit s  \mathit g\mathit  n}}
  %
 \newcommand{\lon}{\longrightarrow}
 \newcommand{\bu}{\bullet}
 \newcommand{\ad}{{\mathrm a\mathrm d}}
 \newcommand{\rar}{\rightarrow}
 \newcommand{\hook}{\hookrightarrow}
 \newcommand{\Proof}{{\bf Proof}.\, }
 \newcommand{\OM}{\Omega^1 {\cal M}}
 \newcommand{\CP}{{\mathbb C} {\mathbb P}}
 \newcommand{\End}{{\mathsf E\mathsf n \mathsf d}}
\newcommand{\p}{{\partial}}
\newcommand{\Id}{{\mathrm I\mathrm d}}
\newcommand{\no}{{\noindent}}
\newcommand{\JB}{{J\hspace{-1mm}B}}
\newcommand{\Q}{{\mathbb Q}}
 \newcommand{\Z}{{\mathbb Z}}
 \newcommand{\bS}{{\mathbb S}}
\newcommand{\bV}{{\mathcal V}}
 \renewcommand{\P}{{\mathbb P}}
 \newcommand{\C}{{\mathbb C}}
 \newcommand{\R}{{\mathbb R}}
 \newcommand{\N}{{\mathbb N}}
 \newcommand{\K}{{\mathbb K}}
 \newcommand{\bbH}{{\mathbb H}}
\newcommand{\Conf}{{\mathit C\mathit o \mathit n\mathit f}}
\newcommand{\tDef}{\mathsf{Def}}
\newcommand{\GC}{\mathsf{GC}}
\newcommand{\fGC}{\mathsf{fGC}}

%%%%%%%%%%%%%
\newcommand{\LB}{{\mathcal L}{\mathit i}{\mathit e} {\mathcal B}}
\newcommand{\LoB}{{\mathcal L}{\mathit i}{\mathit e}\hspace{-0.3mm}^\diamond\hspace{-0.4mm}
{\mathcal B}}
\newcommand{\BV}{{\mathcal B}{\mathcal V}}
\newcommand{\Ber}{{\mathit B}{\mathit e} {\mathit r}}
\newcommand{\Koz}{{\mbox {\scriptsize !`}}}
%%%%%%%%%%%%%%%%%%%%%%%%%%%%%%%%%%%%%%%%%%%%%%%%%%%%

\newcommand{\alg}[1]{\mathfrak{#1}}
  %
 \newcommand{\ot}{\otimes}
 \newcommand{\tl}{\tilde}
%
\newcommand{\rP}{\mbox{\cyr P}}
%
\newcommand{\bfA}{{\mathbf A}}
\newcommand{\bfB}{{\mathbf  B}}
\newcommand{\bfC}{{\mathbf C}}
\newcommand{\bfD}{{\mathbf D}}
\newcommand{\bfE}{{\mathbf E}}
\newcommand{\bfF}{{\mathbf F}}
\newcommand{\bfG}{{\mathbf G}}
\newcommand{\bfP}{{\mathbf P}}
%
\newcommand{\sA}{{\mathsf A}}
\newcommand{\sB}{{\mathsf  A\mathsf s\mathsf s\mathsf B}}
\newcommand{\sC}{{\mathsf C}}
\newcommand{\sD}{{\mathsf D}}
\newcommand{\sE}{{\mathsf E}}
\newcommand{\sF}{{\mathsf F}}
\newcommand{\sG}{{\mathsf G}}
\newcommand{\sP}{{\mathsf P}}
\newcommand{\sd}{{\mathsf d}}
\newcommand{\sa}{{\mathsf a}}
\newcommand{\ssf}{{\mathsf f}}
\newcommand{\sr}{{\mathsf r}}
%
  \newcommand{\Poly}{{\mathcal P}{\mathit o}{\mathit l}{\mathit y}}
  \newcommand{\CoLie}{{\mathit C \mathit o \mathit L \mathit i\mathit e}}
\newcommand{\Lieb}{{\mathsf L\mathsf i \mathsf e \mathsf B}}
\newcommand{\Lie}{\mathsf{ Lie}}
\newcommand{\Graphs}{\mathsf{ Graphs}}
\newcommand{\Com}{\mathsf{ Com}}
\newcommand{\DefQ}{{\mathsf D\mathsf e\mathsf f \mathsf Q}}
\newcommand{\Def}{{\mathsf D\mathsf e\mathsf f }}
\newcommand{\coeq}{\mathop{\rm coequalizer}}
\newcommand{\eff}{{\mathit e\mathit f\mathit f}}
%%%%%%%%%%%%%%%%%%%%%%%%%%%%%%%%%

 %**************************************
 \newcommand{\Beq}{\begin{equation}}
 \newcommand{\Eeq}{\end{equation}}
 \newcommand{\Beqr}{\begin{eqnarray}}
 \newcommand{\Eeqr}{\end{eqnarray}}
 \newcommand{\Beqrn}{\begin{eqnarray*}}
 \newcommand{\Eeqrn}{\end{eqnarray*}}
 \newcommand{\Ba}{\begin{array}}
 \newcommand{\Ea}{\end{array}}
 \newcommand{\Bi}{\begin{itemize}}
 \newcommand{\Ei}{\end{itemize}}
 \newcommand{\Bc}{\begin{center}}
 \newcommand{\Ec}{\end{center}}
 %**************************************
 \newcommand{\ff}{{\mathfrak f}}
 \newcommand{\fg}{{\mathfrak g}}
 \newcommand{\fh}{{\mathfrak h}}
 \newcommand{\fm}{{\mathfrak m}}
 \newcommand{\fl}{{\mathfrak l}}
\newcommand{\fk}{{\mathfrak k}}
\newcommand{\fs}{{\mathfrak s}}
\newcommand{\ft}{{\mathfrak t}}
\newcommand{\fr}{{\mathfrak r}}
\newcommand{\fp}{{\mathfrak p}}
\newcommand{\ii}{{\mathfrak i}}
%
\newcommand{\fA}{{\mathfrak A}}
 \newcommand{\fB}{{\mathfrak B}}
\newcommand{\fC}{{\mathfrak C}}
 \newcommand{\fG}{{\mathfrak G}}
 \newcommand{\fE}{{\mathfrak E}}
\newcommand{\fD}{{\mathfrak E}}
\newcommand{\fT}{{\mathfrak T}}
\newcommand{\fS}{{\mathfrak S}}
 %*************************************
 \newcommand{\f}{{\mathcal O}}
 \newcommand{\cA}{{\mathcal A}}
 \newcommand{\cB}{{\mathcal B}}
 \newcommand{\cC}{{\mathcal C}}
 \newcommand{\caD}{{\mathcal D}}
 \newcommand{\cE}{{\mathcal E}}
 \newcommand{\cF}{{\mathcal F}}
 \newcommand{\cG}{{\mathcal G}}
 \newcommand{\caH}{{\mathcal H}}
 \newcommand{\cI}{{\mathcal I}}
 \newcommand{\cK}{{\mathcal K}}
 \newcommand{\caL}{{\mathcal L}}
 \newcommand{\cM}{{\mathcal M}}
 \newcommand{\cN}{{\mathcal N}}
 \newcommand{\cP}{{\mathcal P}}
 \newcommand{\cQ}{{\mathcal Q}}
 \newcommand{\cR}{{\mathcal R}}
 \newcommand{\cS}{{\mathcal S}}
 \newcommand{\cT}{{\mathcal T}}
 \newcommand{\cX}{{\mathcal X}}
 \newcommand{\cY}{{\mathcal Y}}
 \newcommand{\cZ}{{\mathcal Z}}
 \newcommand{\cU}{{\mathcal U}}
 \newcommand{\cV}{{\mathcal V}}
 \newcommand{\cW}{{\mathcal W}}
 \newcommand{\cTM}{{\mathcal TM}}
 \newcommand{\cHs}{{\mathcal H \mathcal S }}

 %*************************************

 \newcommand{\al}{\alpha}
 \newcommand{\be}{\beta}
 \newcommand{\ga}{\gamma}
 \newcommand{\ka}{\kappa}
 \newcommand{\Ga}{\Gamma}
 \newcommand{\bGa}{{\mathbf \Gamma}}
 \newcommand{\Gai}{{\mit{\Upsilon}}}
 \newcommand{\var}{\varepsilon}
 \newcommand{\la}{\lambda}
 \newcommand{\om}{\omega}
 \newcommand{\Om}{\Omega}
 \newcommand{\dal}{\dot{\alpha}}
 \newcommand{\dbe}{\dot{\beta}}
 \newcommand{\dga}{\dot{\gamma}}
 \newcommand{\dmu}{\dot{\mu}}
 \newcommand{\dnu}{\dot{\nu}}
 \newcommand{\dde}{\dot{\delta}}
\newcommand{\hdelta}{{\delta^+}}
\newcommand{\hga}{{\gamma^+}}
 %****************************

 \newcommand{\tc}{{\mathbf c}}
 \newcommand{\ta}{{\mathbf a}}
 \newcommand{\tb}{{\mathbf b}}
 \newcommand{\te}{{\mathbf e}}
 \newcommand{\tm}{{\mathbf m}}
 \newcommand{\tn}{{\mathbf m}}
 \newcommand{\tr}{{\mathbf r}}

 \newcommand{\zal}{{\bar{\al}}}
 \newcommand{\zbe}{{\bar{\be}}}
 \newcommand{\zga}{{\bar{\ga}}}
 \newcommand{\zmu}{{\bar{\mu}}}
 \newcommand{\znu}{{\bar{\rnu}}}

 \newcommand{\bp}{\bar{\partial}}
 \newcommand{\ra}{{\bar{a}}}
 \newcommand{\rb}{{\bar{b}}}
 \newcommand{\rc}{{\bar{c}}}
 \newcommand{\rj}{{\bar{j}}}
 \newcommand{\ri}{\bar{i}}
 \newcommand{\bbs}{\bar{s}}
 \newcommand{\bbt}{\bar{t}}
 \newcommand{\bbz}{\bar{z}}
%
 \newcommand{\Ker}{{\mathsf K \mathsf e \mathsf r}\, }
 \newcommand{\Img}{{\mathsf I\mathsf m}\, }
 \newcommand{\Hom}{{\mathrm H\mathrm o\mathrm m}}
 \newcommand{\bH}{{\mathbf H}}
 %
 \def\hgw{{\mbox {\large $\circlearrowright$}}}
 %*****************************
 \newcommand{\sip}{\smallskip}
 \newcommand{\bip}{\bigskip}
 \newcommand{\mip}{\vspace{2.5mm}}


 % Thomas definitions ----------------------------------------
\newcommand{\hoe}{\mathrm{hoe}}
\newcommand{\Aut}{\mathrm{Aut}}
 \newcommand{\GCor}{\GC^{or}}
 \newcommand{\hGCor}{\widehat{\GC}^{or}}
 \newcommand{\conn}{\mathit{conn}}
 \newcommand{\fGCc}{\mathsf{fGCc}}
 \newcommand{\fGCor}{\mathsf{fGC}^{or}}

 \DeclareMathOperator{\Exp}{\mathrm{Exp}}

 \newcommand{\LieBi}{{\caL ie\cB}}
  \newcommand{\hLieBi}{\widehat{\LieBi}}
  \newcommand{\hLoB}{\widehat{\LoB}}
 \newcommand{\coLieBi}{{\caL ie\cB^*}}
 \newcommand{\invcoLieBi}{(\LoB)^*}
 \newcommand{\LieBiP}{\LieBi P}
 \newcommand{\LieP}{\Lie P}
  \newcommand{\coLieP}{co\Lie P}

  \newcommand{\Frob}{{\cF rob}}
  \newcommand{\hFrob}{{\widehat{\Frob}}}
 \newcommand{\invFrob}{\Frob^\diamond}
 \newcommand{\coFrob}{{\cF rob^*}}
 \newcommand{\invcoFrob}{(\cF rob^\diamond)^*}
  \newcommand{\invcoFrobtwo}{(\cF rob^\diamond_2)^*}

 \newcommand{\invLieBi}{\LoB}

% \newcommand{\sgn}{\mathrm{sgn}}


 \newcommand{\grt}{\mathfrak{grt}}
 \newcommand{\Der}{\mathrm{Der}}

 \newcommand{\gr}{\mathrm{gr}}
 \newcommand{\vecspan}{\mathrm{span}}
%--------------------------------------------------

\theoremstyle{plain}
\swapnumbers
\newtheorem{theorem}{Theorem}[subsection]
\newtheorem{corollary}[theorem]{Corollary}
\newtheorem{observation}[theorem]{Observation}
\newtheorem{lemma}[theorem]{Lemma}
\newtheorem{proposition}[theorem]{Proposition}
\newtheorem{problem}[theorem]{Problem}
\newtheorem{conjecture}[theorem]{Conjecture}
\newtheorem{prop-def}[theorem]{Proposition-definition}

\newtheorem{main-theorem}{Main~Theorem}[section]
\newtheorem{section-theorem}{Theorem}[section]
\newtheorem{section-corollary}{Corollary}[section]

\theoremstyle{definition}
\newtheorem{example}[theorem]{Example}
\newtheorem{remark}[theorem]{Remark}
\newtheorem{definition}[theorem]{Definition}

\newtheorem{fact}{Fact}[subsection]

%%%%%%%%%%%%%%%%%%%%%%%%%%%
\renewcommand{\thesection}{{\bf\arabic{section}}}
\renewcommand{\thesubsection}{{\bf\arabic{section}.\arabic{subsection}}}
\renewcommand{\thesubsubsection}{\bf\arabic{section}.\arabic{subsection}.\arabic{subsubsection}}




%%%%%%%%%%%%%%%%%%%%%%%%%%%





 %*********************************
  \begin{document}
%\pagestyle{myheadings}
%\bibliographystyle{plain}
%\baselineskip18pt plus 1pt minus 1pt
%\parskip3pt plus 1pt minus .5pt

\sloppy


 \newenvironment{proo}{\begin{trivlist} \item{\sc {Proof.}}}
  {\hfill $\square$ \end{trivlist}}

\long\def\symbolfootnote[#1]#2{\begingroup%
\def\thefootnote{\fnsymbol{footnote}}\footnote[#1]{#2}\endgroup}


 %\title{Involutive Lie bialgebras, graph complexes and the Grothendieck-Teichm\"uller group}
  %\title{GRT and BV}
  \title{The Frobenius properad is Koszul}

\author{Ricardo~Campos}
\address{Ricardo~Campos: Institute of Mathematics, University of Zurich, Zurich, Switzerland}
\email{ricardo.campos@math.uzh.ch}

\author{Sergei~Merkulov}
\address{Sergei~Merkulov:  Department of Mathematics, Stockholm University, Sweden and Mathematics Research Unit, Luxembourg University,  Grand Duchy of Luxembourg (present address) }
\email{sergei.merkulov@uni.lu}

\author{Thomas~Willwacher}
\address{Thomas~Willwacher: Institute of Mathematics, University of Zurich, Zurich, Switzerland}
\email{thomas.willwacher@math.uzh.ch}

% subject class 18D50 (operads) 55P50 (string topology) 17B62 (Lie bialgebras)

 \begin{abstract}
  We show the Koszulness of the properad governing involutive Lie bialgebras and also of the properads governing non-unital and unital-counital Frobenius algebras, solving a long-standing problem. This gives us
  minimal models for  their deformation complexes, and for deformation complexes of their algebras which are discussed in detail.

  Using an operad of graph complexes we prove, with the help of an earlier result of one of the authors \cite{Wi2}, that there is a highly non-trivial action
  of the  Grothendieck-Teichm\"uller group $GRT_1$ on (completed versions of) the minimal models of the properads governing Lie bialgebras and involutive Lie bialgebras by automorphisms.  As a corollary one obtains a large class of universal deformations of (involutive) Lie bialgebras and Frobenius algebras, parameterized by elements of the Grothendieck-Teichm\"uller Lie algebra.

We also prove that, for any given homotopy involutive Lie bialgebra structure on a vector space,
 there is an associated homotopy Batalin-Vilkovisky algebra structure on the associated Chevalley-Eilenberg complex.

%  \mip

%  {\em old version }:
%  Koszulness of the prop of involutive Lie bialgebras (and hence of the props of non-unital Frobenius algebras and of unital-counital  %Frobenius algebras) is proven. This gives us an explicit construction of the dg Lie algebra,
% $\mathsf{InvLieB}(\fg)$, controlling deformations of (strongly homotopy) involutive Lie bialgebra
% structures on an arbitrary graded vector space $\fg$. There is an operad of oriented graphs
% which acts
% on all such deformation complexes $\mathsf{InvLieB}(\fg)$ in a universal way, and the associated dg Lie algebra of coinvariants, ``an
% oriented graph complex",
% maps canonically into the Chevalley-Eilenberg complex, $CE^\bu\left(\mathsf{InvLieB}(\fg)\right)$,  of dg Lie algebra %$\mathsf{InvLieB}(\fg)$ for any $\fg$
% and any involutive Lie bialgebra structure on $\fg$. Using earlier results of Thomas
% Willwacher \cite{Wi2}, we show that the zero-th cohomology of this  oriented graph complex equals the
% Grothendieck-Teichm\"uller Lie algebra. This result implies a universal (homotopy) action
% of the  Grothendieck-Teichm\"uller group $GRT_1$ on the set of (strongly homotopy)
% involutive Lie bialgebra on an arbitrary complex $\fg$.


%  We  show that, for any (strongly homotopy) involutive Lie bialgebra structure on a complex $\fg$,
% there is associated a  $\cB\cV_\infty$ algebra structure in  the associated Chevalley-Eilenberg complex
%  $\odot^\bu (\fg[-1])$, where $\cB\cV$ stands for the operad of Batalin-Vilkovisky algebras.

%\sip
%\noindent {\sc Mathematics Subject Classifications} (2000). %53D55, 16E40, 18G55, 58A50.

%\noindent {\sc Key words}. Kontsevich graph complex, BV formalism.
\end{abstract}
 \maketitle
%\markboth{S.\ Merkulov and T.\ Willwacher}{GRT and BV}

{\large
\section{\bf Introduction}
}
\label{sec:introduction}

The notion of Lie bialgebra was introduced by Drinfeld in \cite{D1}  in the context of the
theory of Yang-Baxter equations. Later this notion played a fundamental role  in his theory
of Hopf algebra deformations of universal enveloping algebras, see the book \cite{ES} and
references cited therein.

 \sip

Many interesting examples of Lie bialgebras automatically satisfy an additional algebraic
condition, the so called {\em involutivity}, or  ``diamond'' $\diamondsuit$ constraint.
 A remarkable example of such a Lie bialgebra structure was discovered by Turaev \cite{Tu}
 on the vector space generated by all non-trivial free homotopy classes of curves on an
orientable surface. Chas proved  \cite{Ch} that such a structure is in fact always
involutive. This example was generalized to arbitrary manifolds within  the framework of
{\em string topology}: the equivariant homology
of the free loop space of a compact manifold was shown by Chas and Sullivan \cite{ChSu} to carry the
structure of a graded involutive Lie bialgebra. An involutive Lie bialgebra structure was
also found by Cieliebak and Latschev \cite{CL} in the contact homology of an arbitrary exact symplectic manifold, while Schedler \cite{Sch} introduced a natural involutive Lie bialgebra structure
on the necklace Lie algebra associated to a quiver. It is worth pointing out that
the construction of quantum $A_\infty$-algebras given in \cite{Ba1} (see also \cite{Ha})
stems from the fact that the vector space of cyclic words in elements of a graded vector space $W$ equipped with a
(skew)symmetric pairing admits a canonical involutive Lie bialgebra structure.
Therefore, involutive Lie bialgebras appear in many different areas of modern research.

% , $W_{cyc}$, spanned by cyclic words,
% $(w_1\ot \ldots \ot w_n)^{\Z_n}$, in elements of a graded vector space $W$ equipped with a
% (skew)symmetric pairing admits a canonical involutive Lie bialgebra structure.
% Therefore, involutive Lie bialgebras appear in many different areas of modern research.



\mip


In the study of the deformation theory of dg involutive Lie bialgebras
one needs to know a minimal resolution
of the associated properad. Such a minimal resolution is particularly nice and explicit if
the properad happens to be {\em Koszul} \cite{Va}. Koszulness of the prop(erad) of Lie bialgebras $\caL ie\cB$ was established by Markl and Voronov \cite{MaVo} following an idea of Kontsevich \cite{Ko}. The proof made use of a new category of  {\em small props}, which are often called
$\frac{1}{2}$-{\em props} nowadays, and a new technical tool, the {\em path
filtration}\, of a dg free properad. Attempts to settle the question of Koszulness or non-Koszulness of the properad of involutive Lie bialgebras, $\caL
ie\cB^\diamondsuit$, have been made since 2004.
The Koszulness proof of $\caL ie\cB$ does not carry over to $\LoB$ since the additional involutivity relation is not $\frac{1}{2}$-properadic in nature.
Motivated by some computer calculations the authors of
\cite{DCTT} conjectured in 2009 that the properad of involutive Lie bialgebras, $\caL
ie\cB^\diamondsuit$, {\em is}\, Koszul. In Section 2 of this paper we settle this long-standing problem. Our result in particular justifies some ad hoc definitions of ``homotopy Lie bialgebras'' which have appeared in the literature, for example in \cite{CFL}.

There are at least two not very straightforward steps in our solution. First, we extend Kontsevich's exact functor from small props to props by twisting it
with the relative simplicial cohomologies of graphs involved. This step allows us to incorporate operations in arities $(1,1)$, $(1,0)$ and $(0,1)$ into the story, which were strictly prohibited in the Kontsevich construction as they destroy the exactness of his functor. Second, we reduce
the cohomology computation of some important auxiliary dg properad to a computation checking Koszulness of some ordinary quadratic algebra, which might be of independent interest.

\bip

By Koszul duality theory of properads \cite{Va}, our result implies immediately that the properad of non-unital Frobenius algebras is Koszul. By the curved Koszul duality theory \cite{HM}, the latter result implies, after some extra work, the Koszulness  of the prop of unital-counital Frobenius algebras. These Frobenius properads also admit many applications in various areas of mathematics and mathematical physics, e.~g.\ in representation theory, algebraic geometry, combinatorics, and recently, in 2-dimensional topological quantum field theory.


\bip

Another main result of this paper is a construction of a highly non-trivial action of the
Grothendieck-Teichm\"uller group $GRT_1$ {\cite{D2}} on minimal models of the properads of involutive Lie bialgebras/Frobenius algebras, and hence on the sets of homotopy involutive Lie bialgebra/Frobenius structures
on an arbitrary dg vector space $\fg$. The Grothendieck-Teichm\"uller group $GRT_1$ has recently been shown to include a pro-unipotent subgroup freely generated by an infinite number of generators \cite{Br}, hence our construction provides a rich class of universal symmetries of the aforementioned objects.


 \mip

In \S 5 of this paper we furthermore show that the Chevalley-Eilenberg complex of an involutive Lie bialgebra carries a Batalin-Vilkovisky algebra structure, i.~e., an action of the homology operad of the framed little disks operad. This statement remains true (up to homotopy) for homotopy involutive Lie bialgebras.



\mip

\subsection*{Acknowledgements}
We are grateful to B. Vallette for helpful discussions.
R.C. and T.W. acknowledge partial support by the Swiss National Science Foundaton, grant 200021\_150012.
Research of T.W. was supported in part by the NCCR SwissMAP of the Swiss National Science Foundation.
S.M.\ is grateful to the Max Planck Institute for Mathematics in Bonn for hospitality and excellent working conditions.

We are very grateful to the anonymous referee for many valuable suggestions improving the present paper.

\mip

{\bf Some notation}. In this paper $\mathbb K$ denotes a field of characteristic $0$. The set $\{1,2, \ldots, n\}$ is abbreviated to $[n]$. Its group of automorphisms is
denoted by $\bS_n$. The sign representation of $\bS_n$ is denoted by $\sgn_n$. The
cardinality of a finite set
$A$ is denoted by $\# A$. If $V=\oplus_{i\in \Z} V^i$ is a graded vector space, then
$V[k]$ stands for the graded vector space with $V[k]^i:=V^{i+k}$. For $v\in V^i$ we set $|v|:=i$.
The phrase \emph{differential graded} is abbreviated by dg.
In some situations we will work with complete topological vector spaces. For our purposes, the following ``poor man's'' definition suffices: A complete topological graded vector space for us is a graded vector space $V$ together with a family of graded subspaces $V_p$, $p=0,1,\dots$, such that $V=\prod_{p=0}^\infty V_p$.
If a graded vector space $U$ comes with a direct sum decomposition $U=\bigoplus_{p=0}^\infty U_p$, then we call $\prod_{p=0}^\infty U_p$ the \emph{completion} of $U$ (along the given decomposition).
We define the completed tensor product of two complete graded vector spaces $V=\prod_{p=0}^\infty V_p$, $W=\prod_{q=0}^\infty W_q$ as
\[
V\hat \otimes W = \prod_{r=0}^\infty \bigoplus_{p=0}^r V_p \otimes W_{r-p}.
\]
The $n$-fold symmetric product of a (dg) vector space $V$ is denoted  by $\odot^n V$, the full symmetric product space by  $\odot^\bullet V$ or just $\odot V$ and the completed (along $\bullet$) symmetric product by $\hat \odot^\bullet V$.
For a finite group G acting on a vector space $V$, we
denote via $V^G$ the space of invariants with respect to the action of G, and by $V_G$
the space of coinvariants $V_G = V/\{gv- v| v\in V, g\in G\}$. We always work over a field $\K$ of characteristic zero so that, for finite $G$, we have a canonical isomorphism $V_G\cong V^G$.



We  use freely the language of operads and properads and their Koszul duality theory. For a background on operads we refer to the textbook \cite{LV}, while the Koszul duality theory of properads has been developed in \cite{Va}; note, however, that we always work with differentials of degree $+1$ rather than $-1$ as in the aformentioned texts.
For a properad $\cP$ we denote by $\cP\{k\}$ the unique properad which has the following property:
for any graded vector space $V$ there is a one-to-one correspondence between representations of
$\cP\{k\}$ in $V$ and representations of
$\cP$ in $V[-k]$; in particular, $\cE nd_V\{k\}=\cE nd_{V[-k]}$.
For $\cC$ a coaugmented co(pr)operad, we will denote by $\Omega(\cC)$ its cobar construction.
Concretely, $\Omega(\cC)=\cF ree\langle\overline \cC[-1]\rangle$ as a graded (pr)operad where $\overline \cC$ the cokernel of the coaugmetation and $\cF ree\langle\dots\rangle$ denotes the free (pr)operad generated by an $\bS$-(bi)module.
We will often use complexes of derivations of (pr)operads and deformation complexes of (pr)operad maps.
For a map of properads $f: \Omega(\cC){\to} \cP$, we will denote by
\begin{equation}\label{equ:Defdefi}
\Def( \Omega(\cC)\stackrel{f}{\to} \cP )\cong \prod_{m,n} \Hom_{\bS_m\times \bS_n}(\cC(m,n), \cP(m,n))
\end{equation}
the associated convolution complex.
We will also consider derivations of a the properad $\cP$. However, we will use a minor variation of the standard definition: First let us define a properad $\cP^+$ generated by $\cP$ and one other operation, say $D$, of arity $(1,1)$ and cohomological degree $+1$. On $\cP^+$ we define a differential such that whenever $\cP^+$ acts on a dg vector space $(V,d)$, then the action restricts to an action of $\cP$ on the vector space with modified differential $(V, d+D)$.
Clearly any $\cP$-algebra is a $\cP^+$-algebra by letting $D$ act trivially, so that we have a properad map $\cP^+\to \cP$.
Now, slightly abusively, we define $\Der(\cP)$ as the complex of derivations of $\cP^+$ preserving the map $\cP^+\to \cP$. Concretely, in all relevant cases $\cP=\Omega(\cC)$ is the cobar construction of a coaugmented coproperad $\cC$. The definition is then made such that $\Der(\cP)[-1]$ is identified with \eqref{equ:Defdefi} as a complex. On the other hand, if we were using ordinary derivations we had to modify \eqref{equ:Defdefi} by replacing $\cC$ by the cokernel of the coaugmentation $\overline{\cC}$ on the right-hand side, thus complicating statements of several results.
We assure the reader that this modification is minor and made for technical reasons in the cases we consider, and results about our $\Der(\cP)$ can be easily transcribed into results about the ordinary derivations if necessary.

Note however that $\Der(\cP)$ carries a natural Lie bracket through the commutator, which is not directly visible on the level of the deformation complex.

 %TODO: not correct, or is it?









\bip



{\large
\section{\bf Koszulness of the prop of involutive Lie bialgebras}
}


\subsection{Reminder on props, $\frac{1}{2}$-props, properads and operads} There are several ways to define these notions (see \cite{Ma} for a short and clear review of different approaches), but for practical computations and arguments used in our work the approach via decorated graphs is most relevant.

\subsubsection{\bf Directed graphs}\label{2: subsubsect on directed graphs} Let $m$ and $n$ be arbitrary non-negative integers.
 A {\em directed  $(m,n)$-graph} is a triple $(\Ga,f_{in},f_{out})$, where $\Ga$ is a finite
 $1$-dimensional
$CW$ complex whose 1-dimensional cells (``edges'') are oriented (``directed''), and
$$
f_{in}: [n] \rar \left\{
\Ba{c}
\mbox{\small the set of all $0$-cells, $v$, of $\Ga$}\\
\mbox{\small  which have precisely one}\\
\mbox{\small  adjacent edge directed from $v$}
\Ea
\right\},\ \
f_{out}: [m] \rar \left\{
\Ba{c}
\mbox{\small the set of all $0$-cells, $v$, of $\Ga$}\\
\mbox{\small  which have precisely one}\\
\mbox{\small  adjacent edge directed towards $v$}
\Ea
\right\}
$$
are injective maps of finite sets (called {\em labelling maps}\, or simply {\em labellings}) such that $\Img f_{in}\cap \Img f_{out}=\emptyset$.
The set, $\fG^\circlearrowright(m,n)$, of all
possible directed $(m,n)$-graphs carries an action,
$
(\Ga, f_{in}, f_{out}) \rar (\Ga, f_{in}\circ \sigma^{-1} , f_{out}\circ \tau)$,
of the group  $\bS_m\times \bS_n$ (more precisely, the {\em right}\, action of $\bS_m^{op}\times \bS_n$
but we declare this detail implicit from now on). We often abbreviate a triple  $(\Ga,f_{in},f_{out})$
to $\Ga$.
For any $\Ga\in \fG^\circlearrowright(m,n)$
the set
$$
V(\Ga):=\{\mbox{all 0-cells of}\ G\}\setminus\{\Img f_{in}\cup \Img f_{out}\},
$$
of all unlabelled
$0$-cells is called the set
of {\em vertices}\, of $\Ga$. The edges attached to labelled $0$-cells, i.e.\ the ones
 lying in $\Img f_{in}$ or in $\Img f_{out}$ are called {\em incoming}\,
  or, respectively, {\em outgoing legs}\,
of the graph $\Ga$. The set
$$
E(\Ga):=\{\mbox{all 1-cells of}\ \Ga\}\setminus\{\mbox{legs}\},
$$
is called the set of {\em (internal) edges}\, of $\Ga$. Legs and edges of $\Ga$ incident to a vertex
$v\in V(\Ga)$ are often called {\em half-edges}\, of $v$; the set of half-edges of $v$ splits naturally
 into two disjoint
sets, $In_v$ and ${\mathit O\mathit u\mathit t}_v$, consisting of incoming and, respectively,
 outgoing half-edges.
 In all our pictures the vertices of a
graph will be denoted by bullets, the edges by intervals (or, sometimes, curves) connecting the vertices,
and legs by intervals attached from one side to vertices. A choice of orientation on an edge or a leg will
be visualized by the choice of a particular direction (arrow) on the associated interval/curve;
unless otherwise explicitly shown the direction of each edge in all our pictures is assumed to go {\em
from bottom to the top}. For example, the graph
$
\Ba{c}\resizebox{10mm}{!}{
\begin{xy}
 <0mm,0mm>*{\bullet};
<0.39mm,0.39mm>*{};<3.4mm,3.4mm>*{}**@{-},
<0mm,0.39mm>*{};<0mm,3.4mm>*{}**@{-},
 <0.39mm,-0.39mm>*{};<3.4mm,-3.7mm>*{}**@{-},
 <-0.35mm,-0.35mm>*{};<-2.9mm,-2.9mm>*{}**@{-},
 <-3.4mm,-3.4mm>*{\bullet};
 <-3.4mm,-3.4mm>*{};<0mm,-6.8mm>*{}**@{-},
 %
 <3.4mm,-3.7mm>*{};<0mm,-6.8mm>*{}**@{-},
  <0mm,-6.8mm>*{\bullet};
  <0mm,-6.8mm>*{};<0mm,-10mm>*{\bullet}**@{-},
  <0mm,-6.8mm>*{};<3mm,-10mm>*{}**@{-},
<0mm,-6.8mm>*{};<-3mm,-10mm>*{}**@{-},
  <0.4mm,-8.5mm>*{};<3.8mm,-12mm>*{^2}**@{},
 <0.4mm,-6.5mm>*{};<-3.8mm,-12mm>*{^1}**@{},
 <0mm,0mm>*{};<0mm,4.5mm>*{^1}**@{},
 <0mm,0mm>*{};<3.6mm,4.5mm>*{^2}**@{},
%
(-0.39,0.39)*{}
   \ar@{->}@(ul,dl) (-3.6,-3.6)*{}
 \end{xy}}
\Ea\in \fG^\circlearrowright(2,2)
$
has four vertices, four legs and five edges; the orientation of all legs and of four
internal edges is {\em not}\, shown explicitly and hence, by default, flows {\em upwards}.
Sometimes we skip showing explicitly
 labellings of legs (as in Table 1 below, for example).
%We set $\fG^\circlearrowright:=\sqcup_{m,n\geq 0} \fG^\circlearrowright(m,n)$.
Note that elements of $\fG^\circlearrowright$ are not necessarily connected,
e.g.\
$\Ba{c}\resizebox{10mm}{!}{
\begin{xy}
 <0mm,0mm>*{\bullet};
<0.39mm,0.39mm>*{};<3.4mm,3.4mm>*{}**@{-},
<0mm,0.39mm>*{};<0mm,3.4mm>*{}**@{-},
 <0.39mm,-0.39mm>*{};<3.4mm,-3.7mm>*{}**@{-},
 <-0.35mm,-0.35mm>*{};<-2.9mm,-2.9mm>*{}**@{-},
 <-3.4mm,-3.4mm>*{\bullet};
 <-3.4mm,-3.4mm>*{};<0mm,-6.8mm>*{}**@{-},
 %
 <3.4mm,-3.7mm>*{};<0mm,-6.8mm>*{}**@{-},
  <0mm,-6.8mm>*{\bullet};
  <0mm,-6.8mm>*{};<0mm,-10mm>*{\bullet}**@{-},
  <0mm,-6.8mm>*{};<3mm,-10mm>*{}**@{-},
<0mm,-6.8mm>*{};<-3mm,-10mm>*{}**@{-},
  <0.4mm,-8.5mm>*{};<3.8mm,-12mm>*{^2}**@{},
 <0.4mm,-6.5mm>*{};<-3.8mm,-12mm>*{^3}**@{},
 <0mm,0mm>*{};<0mm,4.5mm>*{^1}**@{},
 <0mm,0mm>*{};<3.6mm,4.5mm>*{^4}**@{},
%
(-0.39,0.39)*{}
   \ar@{->}@(ul,dl) (-3.6,-3.6)*{}
 \end{xy}}\Ea
\Ba{c}\resizebox{8mm}{!}{
\begin{xy}
 <0mm,-1.3mm>*{};<0mm,-3.5mm>*{}**@{-},
 <0.38mm,-0.2mm>*{};<2.2mm,2.2mm>*{}**@{-},
 <-0.38mm,-0.2mm>*{};<-2.2mm,2.2mm>*{}**@{-},
<0mm,-0.8mm>*{\bullet};
 <2.4mm,2.4mm>*{\bullet};
 <2.5mm,2.3mm>*{};<4.4mm,-0.8mm>*{}**@{-},
 <2.4mm,2.8mm>*{};<2.4mm,5.2mm>*{}**@{-},
     <0mm,-1.3mm>*{};<0mm,-5.6mm>*{^1}**@{},
     <2.5mm,2.3mm>*{};<5.1mm,-2.7mm>*{^4}**@{},
    <2.4mm,2.5mm>*{};<2.4mm,5.7mm>*{^2}**@{},
    <-0.38mm,-0.2mm>*{};<-2.8mm,2.5mm>*{^3}**@{},
    \end{xy}}\Ea\in  \fG^\circlearrowright(4,4)
$.
A directed graph $\Ga$ is called {\em oriented}\, if it has no {\em wheels}, that is, sequences
of directed edges which for a closed path; for example, the graph
$\Ba{c}\resizebox{8mm}{!}{
\begin{xy}
 <0mm,-1.3mm>*{};<0mm,-3.5mm>*{}**@{-},
 <0.38mm,-0.2mm>*{};<2.2mm,2.2mm>*{}**@{-},
 <-0.38mm,-0.2mm>*{};<-2.2mm,2.2mm>*{}**@{-},
<0mm,-0.8mm>*{\bullet};
 <2.4mm,2.4mm>*{\bullet};
 <2.5mm,2.3mm>*{};<4.4mm,-0.8mm>*{}**@{-},
 <2.4mm,2.8mm>*{};<2.4mm,5.2mm>*{}**@{-},
     <0mm,-1.3mm>*{};<0mm,-5.6mm>*{^1}**@{},
     <2.5mm,2.3mm>*{};<5.1mm,-2.7mm>*{^2}**@{},
    <2.4mm,2.5mm>*{};<2.4mm,5.7mm>*{^2}**@{},
    <-0.38mm,-0.2mm>*{};<-2.8mm,2.5mm>*{^1}**@{},
    \end{xy}}\Ea$
 is oriented while the graph
$\Ba{c}\resizebox{10mm}{!}{
\begin{xy}
 <0mm,0mm>*{\bullet};
<0.39mm,0.39mm>*{};<3.4mm,3.4mm>*{}**@{-},
<0mm,0.39mm>*{};<0mm,3.4mm>*{}**@{-},
 <0.39mm,-0.39mm>*{};<3.4mm,-3.7mm>*{}**@{-},
 <-0.35mm,-0.35mm>*{};<-2.9mm,-2.9mm>*{}**@{-},
 <-3.4mm,-3.4mm>*{\bullet};
 <-3.4mm,-3.4mm>*{};<0mm,-6.8mm>*{}**@{-},
 %
 <3.4mm,-3.7mm>*{};<0mm,-6.8mm>*{}**@{-},
  <0mm,-6.8mm>*{\bullet};
  <0mm,-6.8mm>*{};<0mm,-10mm>*{\bullet}**@{-},
  <0mm,-6.8mm>*{};<3mm,-10mm>*{}**@{-},
<0mm,-6.8mm>*{};<-3mm,-10mm>*{}**@{-},
  <0.4mm,-8.5mm>*{};<3.8mm,-12mm>*{^2}**@{},
 <0.4mm,-6.5mm>*{};<-3.8mm,-12mm>*{^1}**@{},
 <0mm,0mm>*{};<0mm,4.5mm>*{^1}**@{},
 <0mm,0mm>*{};<3.6mm,4.5mm>*{^2}**@{},
%
(-0.39,0.39)*{}
   \ar@{->}@(ul,dl) (-3.6,-3.6)*{}
 \end{xy}}\Ea$
is not. Let $\fG^\uparrow(m,n)\subset \fG^\circlearrowright(m,n)$ denote the subset of oriented
$(m,n)$-graphs. We shall work from now on in this subsection with the set $\fG^\uparrow:= \sqcup_{m,n\geq 0}\fG^\uparrow(m,n)$ of oriented graphs though everything said below applies to the general case as well (giving us {\em wheeled}\, versions of the classical notions of prop, properad and operad, see \cite{Me2,MMS}).


\subsubsection{\bf Decorated oriented graphs}
Let $E$ be an $\bS$-{\em bimodule}, that is, a family, $\{E(m,n)\}_{m,n\geq 0}$, of vector spaces on
which the group
$\bS_m$ acts on the left and the group $\bS_n$ acts on the right, and both actions commute with each other.
We shall use elements of $E$ to decorate vertices of an arbitrary graph
$\Ga\in \fG^\uparrow$  as follows. First, for each vertex $v\in V(\Ga)$ with $q$ input edges and $p$ output edges we construct
a vector space
$$
E({\mathit O\mathit u\mathit t}_v, In_v):=  \langle {\mathit O\mathit u\mathit t}_v
\rangle \ot_{\bS_p} E(p,q) \ot_{\bS_q} \langle In_v\rangle,
$$
where  $\langle {\mathit O\mathit u\mathit t}_v\rangle$ (resp.,
$\langle In_v\rangle$) is the vector space spanned by all bijections
$[p]\rar {\mathit O\mathit u\mathit t}_v$
(resp., $In_v\rar [q])$.
It is (non-canonically) isomorphic to $E(p,q)$ as a vector space and carries  natural actions
of the automorphism groups of the sets ${\mathit O\mathit u\mathit t}_v$ and $In_v$. These
actions  make the following
{\em unordered tensor product}\, over the set $V(\Ga)$ (of cardinality, say, $k$),
$$
\bigotimes_{v\in V(\Ga)} E(Out_v, In_v):= \left(\bigoplus_{i:[k]\rar V(G) }
 E({\mathit O\mathit u\mathit t}_{i(1)}, In_{i(1)})
\ot\ldots \ot
 E({\mathit O\mathit u\mathit t}_{i(k)}, In_{i(k)})\right)_{\bS_k},
$$
into a representation space of the automorphism group, $Aut(\Ga)$, of the graph $\Ga$ which,
by definition, is the subgroup of the
 symmetry group of the 1-dimensional
$CW$-complex underlying the graph $\Ga$ which fixes its legs. Hence with an arbitrary
graph $\Ga\in \fG^\uparrow$
and an arbitrary $\bS$-bimodule $E$ one can associate a vector space,
$$
\Ga\langle E\rangle:= \left(
\otimes_{v\in V(G)} E(Out_v, In_v)\right)_{Aut \Ga},
$$
whose elements are called {\em decorated (by $E$) oriented graphs}. For example, the automorphism
group of the graph
$\Ga=\Ba{c}\resizebox{6mm}{!}{
\begin{xy}
 <0mm,0mm>*{\bullet};
<0mm,0.41mm>*{};<0mm,2.9mm>*{}**@{-},
 <0.39mm,-0.39mm>*{};<2.4mm,-2.4mm>*{}**@{-},
 <-0.35mm,-0.35mm>*{};<-2.4mm,-2.4mm>*{}**@{-},
 <-2.4mm,-2.4mm>*{};<-0.4mm,-4.5mm>*{}**@{-},
 %
 <2.4mm,-2.4mm>*{};<0.4mm,-4.5mm>*{}**@{-},
  <0mm,-5.1mm>*{\bullet};
  <0.4mm,-5.5mm>*{};<2mm,-7.7mm>*{}**@{-},
<-0.4mm,-5.5mm>*{};<-2mm,-7.7mm>*{}**@{-},
  <0.4mm,-5.5mm>*{};<2.9mm,-9.7mm>*{^2}**@{},
 <0.4mm,-5.5mm>*{};<-2.9mm,-9.7mm>*{^1}**@{},
 \end{xy}}
\Ea$
is $\Z_2$ so that $\Ga\langle E \rangle\cong E(1,2)\ot_{\Z_2} E(2,2)$. It is useful to think
of an element in $\Ga\langle E\rangle$  as of the graph $\Ga$
whose vertices are literarily decorated by some
elements $a\in E(1,2)$ and $b\in E(2,1)$ and are subject to the following
relations,
$
\Ba{c}\resizebox{6mm}{!}{
\begin{xy}
 <0mm,0mm>*{\bullet};
<2.5mm,0mm>*{^a};
<0mm,0.41mm>*{};<0mm,2.9mm>*{}**@{-},
 <0.39mm,-0.39mm>*{};<2.4mm,-2.4mm>*{}**@{-},
 <-0.35mm,-0.35mm>*{};<-2.4mm,-2.4mm>*{}**@{-},
 <-2.4mm,-2.4mm>*{};<-0.4mm,-4.5mm>*{}**@{-},
 %
 <2.4mm,-2.4mm>*{};<0.4mm,-4.5mm>*{}**@{-},
  <0mm,-5.1mm>*{\bullet};
 <2.5mm,-5.1mm>*{_b};
  <0.4mm,-5.5mm>*{};<2mm,-7.7mm>*{}**@{-},
<-0.4mm,-5.5mm>*{};<-2mm,-7.7mm>*{}**@{-},
  <0.4mm,-5.5mm>*{};<2.9mm,-9.7mm>*{^2}**@{},
 <0.4mm,-5.5mm>*{};<-2.9mm,-9.7mm>*{^1}**@{},
 \end{xy}}
\Ea
=
\Ba{c}\resizebox{9mm}{!}{
\begin{xy}
 <0mm,0mm>*{\bullet};
<5.2mm,0mm>*{^{a\sigma^{-1}}};
<0mm,0.41mm>*{};<0mm,2.9mm>*{}**@{-},
 <0.39mm,-0.39mm>*{};<2.4mm,-2.4mm>*{}**@{-},
 <-0.35mm,-0.35mm>*{};<-2.4mm,-2.4mm>*{}**@{-},
 <-2.4mm,-2.4mm>*{};<-0.4mm,-4.5mm>*{}**@{-},
 %
 <2.4mm,-2.4mm>*{};<0.4mm,-4.5mm>*{}**@{-},
  <0mm,-5.1mm>*{\bullet};
 <3mm,-5.1mm>*{_{\sigma b}};
  <0.4mm,-5.5mm>*{};<2mm,-7.7mm>*{}**@{-},
<-0.4mm,-5.5mm>*{};<-2mm,-7.7mm>*{}**@{-},
  <0.4mm,-5.5mm>*{};<2.9mm,-9.7mm>*{^2}**@{},
 <0.4mm,-5.5mm>*{};<-2.9mm,-9.7mm>*{^1}**@{},
 \end{xy}}
 \Ea$ for $\sigma\in \bS_2$,
$
\lambda\left(
\Ba{c}\resizebox{6mm}{!}{
\begin{xy}
 <0mm,0mm>*{\bullet};
<3mm,0mm>*{a};
<0mm,0.41mm>*{};<0mm,2.9mm>*{}**@{-},
 <0.39mm,-0.39mm>*{};<2.4mm,-2.4mm>*{}**@{-},
 <-0.35mm,-0.35mm>*{};<-2.4mm,-2.4mm>*{}**@{-},
 <-2.4mm,-2.4mm>*{};<-0.4mm,-4.5mm>*{}**@{-},
 %
 <2.4mm,-2.4mm>*{};<0.4mm,-4.5mm>*{}**@{-},
  <0mm,-5.1mm>*{\bullet};
 <3mm,-5.1mm>*{b};
  <0.4mm,-5.5mm>*{};<2mm,-7.7mm>*{}**@{-},
<-0.4mm,-5.5mm>*{};<-2mm,-7.7mm>*{}**@{-},
  <0.4mm,-5.5mm>*{};<2.9mm,-9.7mm>*{^2}**@{},
 <0.4mm,-5.5mm>*{};<-2.9mm,-9.7mm>*{^1}**@{},
 \end{xy}}
\Ea
\right)
=
\Ba{c}\resizebox{8mm}{!}{
\begin{xy}
 <0mm,0mm>*{\bullet};
<5mm,0mm>*{\lambda a};
<0mm,0.41mm>*{};<0mm,2.9mm>*{}**@{-},
 <0.39mm,-0.39mm>*{};<2.4mm,-2.4mm>*{}**@{-},
 <-0.35mm,-0.35mm>*{};<-2.4mm,-2.4mm>*{}**@{-},
 <-2.4mm,-2.4mm>*{};<-0.4mm,-4.5mm>*{}**@{-},
 %
 <2.4mm,-2.4mm>*{};<0.4mm,-4.5mm>*{}**@{-},
  <0mm,-5.1mm>*{\bullet};
 <3mm,-5.1mm>*{b};
  <0.4mm,-5.5mm>*{};<2mm,-7.7mm>*{}**@{-},
<-0.4mm,-5.5mm>*{};<-2mm,-7.7mm>*{}**@{-},
  <0.4mm,-5.5mm>*{};<2.9mm,-9.7mm>*{^2}**@{},
 <0.4mm,-5.5mm>*{};<-2.9mm,-9.7mm>*{^1}**@{},
 \end{xy}}
\Ea
=
\Ba{c}\resizebox{8mm}{!}{
\begin{xy}
 <0mm,0mm>*{\bullet};
<3mm,0mm>*{a};
<0mm,0.41mm>*{};<0mm,2.9mm>*{}**@{-},
 <0.39mm,-0.39mm>*{};<2.4mm,-2.4mm>*{}**@{-},
 <-0.35mm,-0.35mm>*{};<-2.4mm,-2.4mm>*{}**@{-},
 <-2.4mm,-2.4mm>*{};<-0.4mm,-4.5mm>*{}**@{-},
 %
 <2.4mm,-2.4mm>*{};<0.4mm,-4.5mm>*{}**@{-},
  <0mm,-5.1mm>*{\bullet};
 <5mm,-5.1mm>*{\lambda b};
  <0.4mm,-5.5mm>*{};<2mm,-7.7mm>*{}**@{-},
<-0.4mm,-5.5mm>*{};<-2mm,-7.7mm>*{}**@{-},
  <0.4mm,-5.5mm>*{};<2.9mm,-9.7mm>*{^2}**@{},
 <0.4mm,-5.5mm>*{};<-2.9mm,-9.7mm>*{^1}**@{},
 \end{xy}}
\Ea$ for any $\lambda \in \K$, and
$
\Ba{c}\resizebox{11mm}{!}{
\begin{xy}
 <0mm,0mm>*{\bullet};
<7mm,0mm>*{a_1\hspace{-0.5mm}+\hspace{-0.5mm} a_2};
<0mm,0.41mm>*{};<0mm,2.9mm>*{}**@{-},
 <0.39mm,-0.39mm>*{};<2.4mm,-2.4mm>*{}**@{-},
 <-0.35mm,-0.35mm>*{};<-2.4mm,-2.4mm>*{}**@{-},
 <-2.4mm,-2.4mm>*{};<-0.4mm,-4.5mm>*{}**@{-},
 %
 <2.4mm,-2.4mm>*{};<0.4mm,-4.5mm>*{}**@{-},
  <0mm,-5.1mm>*{\bullet};
 <3mm,-5.1mm>*{b};
  <0.4mm,-5.5mm>*{};<2mm,-7.7mm>*{}**@{-},
<-0.4mm,-5.5mm>*{};<-2mm,-7.7mm>*{}**@{-},
  <0.4mm,-5.5mm>*{};<2.9mm,-9.7mm>*{^2}**@{},
 <0.4mm,-5.5mm>*{};<-2.9mm,-9.7mm>*{^1}**@{},
 \end{xy}}
\Ea=
\Ba{c}\resizebox{6mm}{!}{
\begin{xy}
 <0mm,0mm>*{\bullet};
<3mm,0mm>*{a_1};
<0mm,0.41mm>*{};<0mm,2.9mm>*{}**@{-},
 <0.39mm,-0.39mm>*{};<2.4mm,-2.4mm>*{}**@{-},
 <-0.35mm,-0.35mm>*{};<-2.4mm,-2.4mm>*{}**@{-},
 <-2.4mm,-2.4mm>*{};<-0.4mm,-4.5mm>*{}**@{-},
 %
 <2.4mm,-2.4mm>*{};<0.4mm,-4.5mm>*{}**@{-},
  <0mm,-5.1mm>*{\bullet};
 <3mm,-5.1mm>*{b};
  <0.4mm,-5.5mm>*{};<2mm,-7.7mm>*{}**@{-},
<-0.4mm,-5.5mm>*{};<-2mm,-7.7mm>*{}**@{-},
  <0.4mm,-5.5mm>*{};<2.9mm,-9.7mm>*{^2}**@{},
 <0.4mm,-5.5mm>*{};<-2.9mm,-9.7mm>*{^1}**@{},
 \end{xy}}
\Ea
+
\Ba{c}\resizebox{6mm}{!}{
\begin{xy}
 <0mm,0mm>*{\bullet};
<3mm,0mm>*{a_2};
<0mm,0.41mm>*{};<0mm,2.9mm>*{}**@{-},
 <0.39mm,-0.39mm>*{};<2.4mm,-2.4mm>*{}**@{-},
 <-0.35mm,-0.35mm>*{};<-2.4mm,-2.4mm>*{}**@{-},
 <-2.4mm,-2.4mm>*{};<-0.4mm,-4.5mm>*{}**@{-},
 %
 <2.4mm,-2.4mm>*{};<0.4mm,-4.5mm>*{}**@{-},
  <0mm,-5.1mm>*{\bullet};
 <3mm,-5.1mm>*{b};
  <0.4mm,-5.5mm>*{};<2mm,-7.7mm>*{}**@{-},
<-0.4mm,-5.5mm>*{};<-2mm,-7.7mm>*{}**@{-},
  <0.4mm,-5.5mm>*{};<2.9mm,-9.7mm>*{^2}**@{},
 <0.4mm,-5.5mm>*{};<-2.9mm,-9.7mm>*{^1}**@{},
 \end{xy}}
\Ea$  and similarly for $b$.
It also follows from the definition that
$
\Ba{c}\resizebox{6mm}{!}{
\begin{xy}
 <0mm,0mm>*{\bullet};
<2.5mm,0mm>*{^a};
<0mm,0.41mm>*{};<0mm,2.9mm>*{}**@{-},
 <0.39mm,-0.39mm>*{};<2.4mm,-2.4mm>*{}**@{-},
 <-0.35mm,-0.35mm>*{};<-2.4mm,-2.4mm>*{}**@{-},
 <-2.4mm,-2.4mm>*{};<-0.4mm,-4.5mm>*{}**@{-},
 %
 <2.4mm,-2.4mm>*{};<0.4mm,-4.5mm>*{}**@{-},
  <0mm,-5.1mm>*{\bullet};
 <2.5mm,-5.1mm>*{_b};
  <0.4mm,-5.5mm>*{};<2mm,-7.7mm>*{}**@{-},
<-0.4mm,-5.5mm>*{};<-2mm,-7.7mm>*{}**@{-},
  <0.4mm,-5.5mm>*{};<2.9mm,-9.7mm>*{^2}**@{},
 <0.4mm,-5.5mm>*{};<-2.9mm,-9.7mm>*{^1}**@{},
 \end{xy}}
\Ea =
\Ba{c}\resizebox{9mm}{!}{
\begin{xy}
 <0mm,0mm>*{\bullet};
<2.5mm,0mm>*{^a};
<0mm,0.41mm>*{};<0mm,2.9mm>*{}**@{-},
 <0.39mm,-0.39mm>*{};<2.4mm,-2.4mm>*{}**@{-},
 <-0.35mm,-0.35mm>*{};<-2.4mm,-2.4mm>*{}**@{-},
 <-2.4mm,-2.4mm>*{};<-0.4mm,-4.5mm>*{}**@{-},
 %
 <2.4mm,-2.4mm>*{};<0.4mm,-4.5mm>*{}**@{-},
  <0mm,-5.1mm>*{\bullet};
 <5mm,-5.1mm>*{_{b(12)}};
  <0.4mm,-5.5mm>*{};<2mm,-7.7mm>*{}**@{-},
<-0.4mm,-5.5mm>*{};<-2mm,-7.7mm>*{}**@{-},
  <0.4mm,-5.5mm>*{};<2.9mm,-9.7mm>*{^1}**@{},
 <0.4mm,-5.5mm>*{};<-2.9mm,-9.7mm>*{^2}**@{},
 \end{xy}}
\Ea$,  $(12)\in \bS_2$.
%Thus one can define alternatively $G\langle E \rangle$ as the quotient space,
%$\prod_{v\in V(G)} E(Out_v, In_v)/\sim$,
%with respect to the equivalence relation generated by all the above pictures.


\sip

 If $E=\{E(m,n)\}$ is a {\em dg}\,
 $\bS$-bimodule, i.e. if each vector space $E(m,n)$ is a complex equipped with an $\bS_m\times \bS_n$-equivariant
differential $\delta$, then, for any graph $\Ga\in \fG^\circlearrowright(m,n)$,
the associated  graded
vector space
$\Ga\langle E \rangle$ comes equipped with an induced  $\bS_m\times \bS_n$-equivariant differential (which we denote by the same symbol $\delta$) so
that the collection, $\{\bigoplus_{G\in \fG^\circlearrowright(m,n)} G\langle E \rangle\}_{m,n\geq 0}$,
 is again  a {\em dg}\, $\bS$-bimodule.


\sip

 The one vertex graph
$
{\mathfrak C}_{m,n}:=
\Ba{c}
\begin{xy}
 <0mm,0mm>*{\bullet};
 <-0.5mm,0.2mm>*{};<-8mm,3mm>*{}**@{-},
 <-0.4mm,0.3mm>*{};<-4.5mm,3mm>*{}**@{-},
 <0mm,0mm>*{};<0mm,2.6mm>*{\ldots}**@{},
 <0.4mm,0.3mm>*{};<4.5mm,3mm>*{}**@{-},
 <0.5mm,0.2mm>*{};<8mm,3mm>*{}**@{-},
%
<-0.4mm,-0.2mm>*{};<-8mm,-3mm>*{}**@{-},
 <-0.5mm,-0.3mm>*{};<-4.5mm,-3mm>*{}**@{-},
 <0mm,0mm>*{};<0mm,-2.6mm>*{\ldots}**@{},
 <0.5mm,-0.3mm>*{};<4.5mm,-3mm>*{}**@{-},
 <0.4mm,-0.2mm>*{};<8mm,-3mm>*{}**@{-};
<0mm,5mm>*{\overbrace{\ \ \ \ \ \ \ \ \ \ \ \ \ \  }};
<0mm,-5mm>*{\underbrace{\ \ \ \ \ \ \ \ \ \ \ \ \ \ }};
<0mm,7mm>*{^{m\ \ output\ legs}};
<0mm,-7mm>*{_{n\ \ input\ legs}};
 \end{xy}\in \fG^\uparrow(m,n)\Ea
$
is often called {\em the $(m,n)$-corolla}. It is clear  that for any $\bS$-bimodule $E$ one has
${\mathfrak C}_{m,n}\langle E \rangle \cong E(m,n)$.





%%%%%%%%%%%%%%%%%%%%%%%%%%%%%%%%%%%%%%%%%%%%%%%%%%%%%%%%%
\subsubsection{\bf Props}\label{1: subsect props}
A {\em  prop}\,  is an $\bS$-bimodule $\cP=\{\cP(m,n)\}$ together with
a family of linear $\bS_m\times \bS_n$-equivariant maps,
$$
\left\{\mu_\Ga: \Ga\langle \cP\rangle\rar \cP(m,n)\right\}_{\Ga\in \fG^\uparrow(m,n)},\ \ m,n\geq 0,
$$
 which
satisfy the following ``associativity'' condition,
\Beq\label{graph-associativity}
\mu_\Ga=\mu_{\Ga/\ga}\circ \mu_\ga',
\Eeq
 for any subgraph $\ga\subset \Ga$ such that the quotient graph  $\Ga/\ga$ (which is obtained from $\Ga$ by shrinking
 all the vertices and internal edges of $\ga$ into a single internal vertex) is oriented, and
  $\mu_\ga': \Ga\langle E \rangle \rar (\Ga/\ga)\langle E\rangle$ stands for the map
which equals $\mu_\ga$ on the decorated vertices lying in $\ga$ and which is identity on all other vertices of $\Ga$.

\sip

If the $\bS$-bimodule $\cP$ underlying a  prop has a differential $\delta$ satisfying,
for any $\Ga\in \fG^\circlearrowright$, the condition
$\delta\circ \mu_\Ga=\mu_\Ga\circ \delta$, then the
 prop $\cP$
is called {\em differential}.

\sip

As ${\mathfrak C}_{m,n}\langle E \rangle = E(m,n)$, the values of the maps $\mu_\Ga$ can be identified with decorated corollas, and hence
the maps themselves can  be visually understood as  {\em contraction}\, maps,
$\mu_{\Ga\in \fG^\uparrow(m,n)}: \Ga\langle \cP\rangle\rar {\mathfrak C}_{m,n}\langle \cP \rangle$,
contracting all the  edges and vertices of $\Ga$ into a single vertex.

\sip

 Strictly speaking, the notion introduced just above should be called a
prop {\em without unit}. A  prop {\em with unit}\, can be defined as above
provided one  enlarges
$\fG^\uparrow$ by adding a family of graphs,
$\{
\uparrow \ \uparrow  \cdots \uparrow\}
$,  {\em without vertices}.

%%%%%%%%%%%%%%%%%%%%%%%%%%%
\subsubsection{\bf Props, properads, operads, etc.\ as $\fG$-algebras} Let $\fG=\sqcup_{m,n}\fG(m,n)$
be a subset of the set $\fG^\uparrow$, say, one of the subsets defined in Table 1 below.
 A subgraph $\ga$ of a graph $\Ga\in \fG$ is called
{\em admissible}\, if $\ga\in \fG$ and $\Ga/\ga\in \fG$.
A $\fG$-{\em algebra}\,  is, by definition,
an $\bS$-bimodule $\cP=\{\cP(m,n)\}$ together with
a family of linear $\bS_m\times \bS_n$-equivariant maps,
$\left\{\mu_\Ga: \Ga\langle \cP\rangle\rar \cP(m,n)\right\}_{G\in \fG^\circlearrowright(m,n)},$
parameterized by elements $\Ga\in \fG$, which
satisfy condition (\ref{graph-associativity}) for any admissible subgraph $H\subset \Ga$.
Applying this idea to the subfamilies
$\fG\subset \fG^\circlearrowright$ from Table 1 gives us, in the chronological order,
the notions of {\em prop, operad, dioperad, $\frac{1}{2}$-prop}\, and {\em properad}\, introduced, respectively,
in the papers \cite{Mc, May, G, Ko, Va}.



%%%%%%%%%%%%%%%%%%%%%%%%%%%%%%%%%%%%%%%
\subsubsection{\bf Basic examples of $\fG$-algebras}
{ ({\it i}\,)} For any $\fG$ and any vector space $V$ the $\bS$-bimodule
$\cE nd_V=\{ \Hom(V^{\ot n}, V^{\ot m})\}$ is naturally a
 $\fG$-algebra with contraction maps $\mu_{G\in \fG}$ being ordinary compositions of
 linear maps; it is
 called the {\em endomorphism
$\fG$-algebra
of $V$}.

\sip

({\it ii}\,) With any  $\bS$-bimodule,
 $E=\{E(m,n)\}$, there is associated  another $\bS$-bimodule,
$\cF ree^\fG\langle E\rangle=\{\cF^\fG\langle E\rangle (m,n)\}$ with
$
\cF ree^\fG\langle E\rangle (m,n):= \bigoplus_{\Ga\in \fG(m,n)} \Ga\langle E\rangle$,
which has a natural $\fG$-algebra structure with the contraction maps
$\mu_G$ being tautological. The $\fG$-algebra  $\cF ree^\fG \langle E\rangle$ is called {\em the free $\fG$-algebra
generated
by the $\bS$-bimodule $E$}. We often abbreviate notations by replacing $\cF ree^{\fG}$
by $\cF ree $.

\sip


({\it iii}\,) Definitions of $\fG$-{\em sub}algebras, $\mathcal Q\subset \cP$, of $\fG$-algebras, of their
ideals, $\cI\subset \cP$, and the associated quotient $\fG$-algebras, $\cP/\cI$, are
straightforward. We omit the details.

\mip


\begin{center}
%%%%%%%%%%%%%%%%%%%%%%%%%%%%%%%%%%%%%%%%%%%%%%%%%%%%%%%%%%%%%%%%%%%%%%%%%%%%%%%%%
\begin{tabular}{|c|c||c|c|c|}
%\multicolumn{4}{c}{}
\multicolumn{4}{c}{{\bf Table 1}\vspace{2 mm}: A list of $\fG$-algebras}\\
%\hline
\hline
&&&\vspace{-2mm}\\
 $\fG$ & {\bf Definition}
  &   $\fG${\bf-algebra}  & {\bf Typical examples} \\
  &&& \vspace{-2mm} \\
\hline
\hline


$\fG^\uparrow$ & $\Ba{c} \mbox{The set of all possible }
\\ \mbox{oriented graphs} \Ea$ &
  $\Ba{c}\mbox{Prop}\Ea $ &
    \begin{xy}
 <-0.4mm,0.0mm>*{};<-2.4mm,2.1mm>*{}**@{-},
 <0.38mm,-0.2mm>*{};<2.8mm,2.5mm>*{}**@{-},
<-0mm,-0.8mm>*{\bullet};
<-0mm,-1.0mm>*{};<-0mm,-3.6mm>*{}**@{-},
%
 <-2.96mm,2.4mm>*{\bullet};
 <-2.4mm,2.8mm>*{};<-0mm,5mm>*{}**@{-},
  <-3.4mm,3.1mm>*{};<-5.1mm,5mm>*{}**@{-},
 %<3.4mm,3.1mm>*{};<6.1mm,5.7mm>*{^3}**@{-},
%
<2.8mm,2.5mm>*{};<-0mm,5mm>*{\bullet}**@{},
<2.8mm,2.5mm>*{};<-0mm,5mm>*{}**@{-},
%
<-0mm,5mm>*{};<-0mm,8.6mm>*{}**@{-},
\end{xy}
    \begin{xy}
 <0.4mm,0.0mm>*{};<2.4mm,2.1mm>*{}**@{-},
 <-0.38mm,-0.2mm>*{};<-2.8mm,2.5mm>*{}**@{-},
<0mm,-0.8mm>*{\bullet};
<0mm,-1.0mm>*{};<0mm,-3.6mm>*{}**@{-},
%
 <2.96mm,2.4mm>*{\bullet};
 <2.4mm,2.8mm>*{};<0mm,5mm>*{}**@{-},
  <3.4mm,3.1mm>*{};<5.1mm,5mm>*{}**@{-},
 %<3.4mm,3.1mm>*{};<6.1mm,5.7mm>*{^3}**@{-},
%
<-2.8mm,2.5mm>*{};<0mm,5mm>*{\bullet}**@{},
<-2.8mm,2.5mm>*{};<0mm,5mm>*{}**@{-},
%
<0mm,5mm>*{};<0mm,8.6mm>*{}**@{-},
\end{xy}
  \\
&&& \vspace{-3mm}\\
\hline
&&&\vspace{-3mm} \\
$\fG^\uparrow_c$ & $\Ba{c} \mbox{A subset}\ \fG^\uparrow_c\subset \fG^\uparrow\
\mbox{consisting}
\\ \mbox{ of all {\em connected}\, graphs}\Ea $ &
  $\Ba{c}\mbox{Properad}\Ea $ &
   \begin{xy}
 <0.4mm,0.0mm>*{};<2.4mm,2.1mm>*{}**@{-},
 <-0.38mm,-0.2mm>*{};<-2.8mm,2.5mm>*{}**@{-},
<0mm,-0.8mm>*{\bullet};
<0mm,-1.0mm>*{};<0mm,-3.6mm>*{}**@{-},
%
 <2.96mm,2.4mm>*{\bullet};
 <2.4mm,2.8mm>*{};<0mm,5mm>*{}**@{-},
  <3.4mm,3.1mm>*{};<5.1mm,5mm>*{}**@{-},
 %<3.4mm,3.1mm>*{};<6.1mm,5.7mm>*{^3}**@{-},
%
<-2.8mm,2.5mm>*{};<0mm,5mm>*{\bullet}**@{},
<-2.8mm,2.5mm>*{};<0mm,5mm>*{}**@{-},
%
<0mm,5mm>*{};<0mm,8.6mm>*{}**@{-},
\end{xy}
  \\
&&& \vspace{-3mm}\\
\hline
&&& \vspace{-3mm}\\
$\fG^\uparrow_{c,0}$ & $\Ba{c} \mbox{A subset}\ \fG^\uparrow_{c,0}\subset \fG^\uparrow_c\
\mbox{consisting}
\\ \mbox{ of graphs of genus zero
}\Ea $ &
  $\Ba{c}\mbox{Dioperad}\Ea $ &
\begin{xy}
 <0mm,2.47mm>*{};<0mm,-0.5mm>*{}**@{-},
 <0.5mm,3.5mm>*{};<2.2mm,5.2mm>*{}**@{-},
 <-0.48mm,3.48mm>*{};<-2.2mm,5.2mm>*{}**@{-},
 <0mm,3mm>*{\bullet};<0mm,3mm>*{}**@{},
  <0mm,-0.8mm>*{\bullet};<0mm,-0.8mm>*{}**@{},
<0mm,-0.8mm>*{};<-2.2mm,-3.5mm>*{}**@{-},
<0mm,-0.8mm>*{};<2.2mm,-3.5mm>*{}**@{-},
   %
 <-2.5mm,5.7mm>*{\bullet};<0mm,0mm>*{}**@{},
<-2.5mm,5.7mm>*{};<-2.5mm,9.4mm>*{}**@{-},
<-2.5mm,5.7mm>*{};<-5mm,3mm>*{}**@{-},
%<-5mm,3mm>*{};<-5mm,-0.8mm>*{}**@{-},
%
 <-2.5mm,-4.2mm>*{\bullet};<0mm,3mm>*{}**@{},
 <-2.8mm,-3.6mm>*{};<-5mm,-0.8mm>*{}**@{-},
 <-2.5mm,-4.6mm>*{};<-2.5mm,-7.3mm>*{}**@{-},
%   %
\end{xy}
  \\
&&& \vspace{-3mm}\\
\hline
&&& \vspace{-3mm}\\
$\fG^{\frac{1}{2}}$ & $\Ba{c} \mbox{A subset}\ \fG^{\frac{1}{2}}\subset\fG^\uparrow_{c,0}\
\mbox{consisting of}
\\ \mbox{all $(m,n)$-graphs with the number }\\ \mbox{of directed paths from input legs}\\
 \mbox{to the output legs equal to $mn$}\\
  \mbox{and with at least trivalent vertices}\\
\Ea $ &
  $\Ba{c}\mbox{$\frac{1}{2}$-Prop}\Ea $ &
   \begin{xy}
 <0mm,0mm>*{\bullet};
 <0mm,0.69mm>*{};<0mm,3.5mm>*{}**@{-},
 <0.39mm,-0.39mm>*{};<2.4mm,-2.4mm>*{}**@{-},
 <-0.35mm,-0.35mm>*{};<-1.9mm,-1.9mm>*{}**@{-},
 <-2.4mm,-2.4mm>*{\bullet};
 <-2.0mm,-2.8mm>*{};<0mm,-4.9mm>*{}**@{-},
 <-2.8mm,-2.9mm>*{};<-4.7mm,-4.9mm>*{}**@{-},
 <0mm,3.5mm>*{\bullet};
 <0mm,3.5mm>*{};<-2.4mm,5.9mm>*{}**@{-},
 <0mm,3.5mm>*{};<2.4mm,5.9mm>*{}**@{-},
 \end{xy}
  \\
&&& \vspace{-3mm}\\
\hline
&&& \vspace{-3mm}\\
$\fG^\curlywedge$ & $\Ba{c} \mbox{A subset}\ \fG^\curlywedge\subset \fG^\uparrow_{c,0}\
\mbox{consisting of graphs}
\\
 \mbox{ whose vertices have precisely {one} output leg
}\Ea $ &
  $\Ba{c}\mbox{Operad}\Ea $ &

   \begin{xy}
 <0mm,0mm>*{\bullet};<0mm,0mm>*{}**@{},
 <0mm,0.69mm>*{};<0mm,4.0mm>*{}**@{-},
 <0.39mm,-0.39mm>*{};<2.4mm,-2.4mm>*{}**@{-},
 <-0.35mm,-0.35mm>*{};<-1.9mm,-1.9mm>*{}**@{-},
 <-2.4mm,-2.4mm>*{\bullet};<-2.4mm,-2.4mm>*{}**@{},
 <-2.0mm,-2.8mm>*{};<0mm,-4.9mm>*{}**@{-},
 <-2.8mm,-2.9mm>*{};<-4.7mm,-4.9mm>*{}**@{-},
 \end{xy}
  \\
&&& \vspace{-3mm}\\
\hline
&&& \vspace{-3mm}\\

$\fG^\mid$ & $\Ba{c} \mbox{A subset}\ \fG^\mid\subset \fG^\curlywedge
\\
 \mbox{consisting of graphs whose}\\
\mbox{vertices have precisely}\\ \mbox{ {one} input leg
}\Ea $ &
  $\Ba{c}\mbox{Associative}\\
  \mbox{algebra}\Ea $ &

   \begin{xy}
 <0mm,0mm>*{\bullet};
  <0mm,3.5mm>*{\bullet};
  <0mm,-3.5mm>*{\bullet};
 <0mm,-7mm>*{};<0mm,7.0mm>*{}**@{-},
 \end{xy}
\vspace{-3mm}  \\
&&&\\
\hline
\end{tabular}
\end{center}



\mip

%%%%%%%%%%%%%%%%%%%%%%%%
\subsubsection{\bf Morphisms and resolutions of $\fG$-algebras}\label{1: section morphisms and resolutions}
 A morphism of $\fG$-algebras, $\rho: \cP_1\rar \cP_2$,
is a morphism of the underlying $\bS$-bimodules such that, for any graph $G$,
 one has $\rho\circ \mu_G= \mu_G\circ (\rho^{\ot G})$,
where $\rho^{\ot G}$ is a map, $G\langle \cP_1\rangle \rar G\langle \cP_2\rangle$, which changes
decorations of each vertex
 in $G$ in accordance with $\rho$.
A morphism of $\fG$-algebras, $\cP\rar \cE nd_V$, is called a {\em representation}\, of the
$\fG$-algebra $\cP$ in a graded vector space $V$.


\sip

 A {\em free resolution}\, of a dg $\fG$-algebra
$\cP$ is, by definition, a dg free $\fG$-algebra, $(\cF^\fG\langle E \rangle, \delta)$,
together with a morphism,
$\pi: (\cF\langle E \rangle, \delta) \rar \cP$, which induces a cohomology isomorphism.
If the differential $\delta$ in $\cF\langle \cE \rangle$ is
decomposable with respect to compositions $\mu_G$, then it is
 called a {\em minimal model}\, of $\cP$ and  is often denoted by
$\cP_\infty$.









\subsection{Involutive Lie bialgebras} A  {\em Lie bialgebra}\, is a graded vector space
$\fg$,
equipped with degree zero linear maps,
$$
\vartriangle: \fg\rightarrow \fg\wedge \fg \ \ \ \mbox{and}\ \ \  [\ , \ ]: \wedge^2 \fg
\rightarrow \fg,
$$
such that
\Bi
\item the data $(\fg,\vartriangle)$ is a Lie coalgebra;
\item the data $(\fg, [\ ,\ ])$ is a Lie algebra;
\item the compatibility condition,
$$
\vartriangle [a, b] = \sum a'\otimes [a'', b] +  [a,
b']\otimes b'' - (-1)^{|a||b|}( [b, a']\otimes a''
+ b'\otimes [b'', a]),
$$
holds for any $a,b\in \fg$. Here $\vartriangle a=:\sum a'\otimes a''$, $\vartriangle
b=:\sum
b'\otimes b''$.
\Ei
 A Lie bialgebra $(\fg, [\ ,\ ], \vartriangle)$ is called {\em
involutive}\, if the composition map
$$
\Ba{ccccc}
V & \stackrel{\vartriangle}{\lon} & \Lambda^2V & \stackrel{[\, ,\, ]}{\lon} & V\\
a & \lon &    \sum a'\otimes a'' &\lon & \sum [a',a'']
\Ea
$$
vanishes. A dg (involutive) Lie bialgebra is a
complex $(\fg,d)$ equipped with the structure of an (involutive) Lie bialgebra such that
the maps $[\ ,\ ]$ and $\Delta$ are morphisms of complexes.


\subsubsection{\bf An example}\label{2: subsection on cyclic words}
Let $W$ be a finite dimensional graded vector space over a field $\K$ of
characteristic zero equipped with a degree $0$
skewsymmetric pairing,
$$
\Ba{rccl}
\om: & W\ot W & \lon & \K \\
  &  w_1\ot w_2 & \lon & \om(w_1,w_2)=-(-1)^{|w_1||w_2|}\om(w_2,w_1).
\Ea
$$
 Then the associated
vector space of ``cyclic words in $W$'',
$$
Cyc^\bu(W):=\bigoplus_{n\geq 0} (W^{\ot n})_{\Z_n},
$$
admits an involutive Lie bialgebra structure  given by \cite{Ch}
$$
[(w_1\ot...\ot w_n)_{\Z_n}, (v_1\ot ...\ot v_m)_{\Z_n}]:=\hspace{110mm}
$$
$$
\hspace{20mm} \sum_{i\in[n]\atop
j\in [m]}  \pm
\om(w_i,v_j) (w_1\ot ...\ot  w_{i-1}\ot v_{j+1}\ot ... \ot v_m\ot v_1\ot ... \ot v_{j-1}\ot w_{i+1}\ot\ldots\ot w_n)_{\Z_{n+m-2}}
$$
and
\Beqrn
\vartriangle (w_1\ot\ldots\ot w_n)_{\Z_n}:&=&\sum_{i\neq j}
\pm\om(w_i,w_j)(w_{i+1}\ot ...\ot w_{j-1})_{\Z_{j-i-1}}\bigotimes
(w_{j+1}\ot ...\ot w_{i-1})_{\Z_{n-j+i-1}}    \\
\Eeqrn

This example has many applications in various  areas of modern research (see, e.g., \cite{Ba1,Ch, CL, Ha}).


% especially in the class of Lie $2k$-bialgebras whose underlying
%vector space, $V$, is the space of cyclic words  (CHECK THE GRADING),
%$$
%V = \bigoplus_{m\geq 0} (\ot^m (X[1-k]))^{\Z_m}[k-1],
%$$
%built from some auxiliary vector space $X$ equipped with an arbitrary {\em degree $k$}\,
%scalar product.

\mip

%%%%%%%%%%%%%%%%%%%%%%%%%%%%%%%%%%%%%%%%%%%%%%
\subsection{Properad of involutive Lie bialgebras}
By definition, the properad, $\LoB$, of involutive Lie bialgebras is a quadratic properad
given as the quotient,
$$
\LoB:=\cF ree\langle E\rangle/<\cR>,
$$
of the free properad generated by an  $\bS$-bimodule $E=\{E(m,n)\}_{m,n\geq 1}$ with
 all $E(m,n)=0$ except
$$
E(2,1):=\id_1\ot \sgn_2=\mbox{span}\left\langle
\begin{xy}
 <0mm,-0.55mm>*{};<0mm,-2.5mm>*{}**@{-},
 <0.5mm,0.5mm>*{};<2.2mm,2.2mm>*{}**@{-},
 <-0.48mm,0.48mm>*{};<-2.2mm,2.2mm>*{}**@{-},
 <0mm,0mm>*{\circ};<0mm,0mm>*{}**@{},
 <0mm,-0.55mm>*{};<0mm,-3.8mm>*{_1}**@{},
 <0.5mm,0.5mm>*{};<2.7mm,2.8mm>*{^2}**@{},
 <-0.48mm,0.48mm>*{};<-2.7mm,2.8mm>*{^1}**@{},
 \end{xy}
=-
\begin{xy}
 <0mm,-0.55mm>*{};<0mm,-2.5mm>*{}**@{-},
 <0.5mm,0.5mm>*{};<2.2mm,2.2mm>*{}**@{-},
 <-0.48mm,0.48mm>*{};<-2.2mm,2.2mm>*{}**@{-},
 <0mm,0mm>*{\circ};<0mm,0mm>*{}**@{},
 <0mm,-0.55mm>*{};<0mm,-3.8mm>*{_1}**@{},
 <0.5mm,0.5mm>*{};<2.7mm,2.8mm>*{^1}**@{},
 <-0.48mm,0.48mm>*{};<-2.7mm,2.8mm>*{^2}**@{},
 \end{xy}
   \right\rangle
   %\equiv
%\left\{
%\Ba{rr}
%\sgn_2\ot \id_1[-k]  & \mbox{if}\ k\ \mbox{is even},\vspace{3mm}\\
%\id_2\ot \id_1 [-k]  & \mbox{if}\ k\ \mbox{is odd},\Ea
%\right.,
$$
$$
E(1,2):= \sgn_2\ot \id_1=\mbox{span}\left\langle
\begin{xy}
 <0mm,0.66mm>*{};<0mm,3mm>*{}**@{-},
 <0.39mm,-0.39mm>*{};<2.2mm,-2.2mm>*{}**@{-},
 <-0.35mm,-0.35mm>*{};<-2.2mm,-2.2mm>*{}**@{-},
 <0mm,0mm>*{\circ};<0mm,0mm>*{}**@{},
   <0mm,0.66mm>*{};<0mm,3.4mm>*{^1}**@{},
   <0.39mm,-0.39mm>*{};<2.9mm,-4mm>*{^2}**@{},
   <-0.35mm,-0.35mm>*{};<-2.8mm,-4mm>*{^1}**@{},
\end{xy}=-
\begin{xy}
 <0mm,0.66mm>*{};<0mm,3mm>*{}**@{-},
 <0.39mm,-0.39mm>*{};<2.2mm,-2.2mm>*{}**@{-},
 <-0.35mm,-0.35mm>*{};<-2.2mm,-2.2mm>*{}**@{-},
 <0mm,0mm>*{\circ};<0mm,0mm>*{}**@{},
   <0mm,0.66mm>*{};<0mm,3.4mm>*{^1}**@{},
   <0.39mm,-0.39mm>*{};<2.9mm,-4mm>*{^1}**@{},
   <-0.35mm,-0.35mm>*{};<-2.8mm,-4mm>*{^2}**@{},
\end{xy}
\right\rangle
$$
modulo the ideal generated by the following relations
\Beq\label{R for LieB}
\cR:\left\{
\Ba{c}
\begin{xy}
 <0mm,0mm>*{\circ};<0mm,0mm>*{}**@{},
 <0mm,-0.49mm>*{};<0mm,-3.0mm>*{}**@{-},
 <0.49mm,0.49mm>*{};<1.9mm,1.9mm>*{}**@{-},
 <-0.5mm,0.5mm>*{};<-1.9mm,1.9mm>*{}**@{-},
 <-2.3mm,2.3mm>*{\circ};<-2.3mm,2.3mm>*{}**@{},
 <-1.8mm,2.8mm>*{};<0mm,4.9mm>*{}**@{-},
 <-2.8mm,2.9mm>*{};<-4.6mm,4.9mm>*{}**@{-},
   <0.49mm,0.49mm>*{};<2.7mm,2.3mm>*{^3}**@{},
   <-1.8mm,2.8mm>*{};<0.4mm,5.3mm>*{^2}**@{},
   <-2.8mm,2.9mm>*{};<-5.1mm,5.3mm>*{^1}**@{},
 \end{xy}
\ + \
\begin{xy}
 <0mm,0mm>*{\circ};<0mm,0mm>*{}**@{},
 <0mm,-0.49mm>*{};<0mm,-3.0mm>*{}**@{-},
 <0.49mm,0.49mm>*{};<1.9mm,1.9mm>*{}**@{-},
 <-0.5mm,0.5mm>*{};<-1.9mm,1.9mm>*{}**@{-},
 <-2.3mm,2.3mm>*{\circ};<-2.3mm,2.3mm>*{}**@{},
 <-1.8mm,2.8mm>*{};<0mm,4.9mm>*{}**@{-},
 <-2.8mm,2.9mm>*{};<-4.6mm,4.9mm>*{}**@{-},
   <0.49mm,0.49mm>*{};<2.7mm,2.3mm>*{^2}**@{},
   <-1.8mm,2.8mm>*{};<0.4mm,5.3mm>*{^1}**@{},
   <-2.8mm,2.9mm>*{};<-5.1mm,5.3mm>*{^3}**@{},
 \end{xy}
\ + \
\begin{xy}
 <0mm,0mm>*{\circ};<0mm,0mm>*{}**@{},
 <0mm,-0.49mm>*{};<0mm,-3.0mm>*{}**@{-},
 <0.49mm,0.49mm>*{};<1.9mm,1.9mm>*{}**@{-},
 <-0.5mm,0.5mm>*{};<-1.9mm,1.9mm>*{}**@{-},
 <-2.3mm,2.3mm>*{\circ};<-2.3mm,2.3mm>*{}**@{},
 <-1.8mm,2.8mm>*{};<0mm,4.9mm>*{}**@{-},
 <-2.8mm,2.9mm>*{};<-4.6mm,4.9mm>*{}**@{-},
   <0.49mm,0.49mm>*{};<2.7mm,2.3mm>*{^1}**@{},
   <-1.8mm,2.8mm>*{};<0.4mm,5.3mm>*{^3}**@{},
   <-2.8mm,2.9mm>*{};<-5.1mm,5.3mm>*{^2}**@{},
 \end{xy}\ =\ 0,
 \vspace{3mm}\\
%%%%%%%%%%%%%% Lie %%%%%%%%%%%%%%%%%%%%%%%%
 \begin{xy}
 <0mm,0mm>*{\circ};<0mm,0mm>*{}**@{},
 <0mm,0.69mm>*{};<0mm,3.0mm>*{}**@{-},
 <0.39mm,-0.39mm>*{};<2.4mm,-2.4mm>*{}**@{-},
 <-0.35mm,-0.35mm>*{};<-1.9mm,-1.9mm>*{}**@{-},
 <-2.4mm,-2.4mm>*{\circ};<-2.4mm,-2.4mm>*{}**@{},
 <-2.0mm,-2.8mm>*{};<0mm,-4.9mm>*{}**@{-},
 <-2.8mm,-2.9mm>*{};<-4.7mm,-4.9mm>*{}**@{-},
    <0.39mm,-0.39mm>*{};<3.3mm,-4.0mm>*{^3}**@{},
    <-2.0mm,-2.8mm>*{};<0.5mm,-6.7mm>*{^2}**@{},
    <-2.8mm,-2.9mm>*{};<-5.2mm,-6.7mm>*{^1}**@{},
 \end{xy}
\ + \
 \begin{xy}
 <0mm,0mm>*{\circ};<0mm,0mm>*{}**@{},
 <0mm,0.69mm>*{};<0mm,3.0mm>*{}**@{-},
 <0.39mm,-0.39mm>*{};<2.4mm,-2.4mm>*{}**@{-},
 <-0.35mm,-0.35mm>*{};<-1.9mm,-1.9mm>*{}**@{-},
 <-2.4mm,-2.4mm>*{\circ};<-2.4mm,-2.4mm>*{}**@{},
 <-2.0mm,-2.8mm>*{};<0mm,-4.9mm>*{}**@{-},
 <-2.8mm,-2.9mm>*{};<-4.7mm,-4.9mm>*{}**@{-},
    <0.39mm,-0.39mm>*{};<3.3mm,-4.0mm>*{^2}**@{},
    <-2.0mm,-2.8mm>*{};<0.5mm,-6.7mm>*{^1}**@{},
    <-2.8mm,-2.9mm>*{};<-5.2mm,-6.7mm>*{^3}**@{},
 \end{xy}
\ + \
 \begin{xy}
 <0mm,0mm>*{\circ};<0mm,0mm>*{}**@{},
 <0mm,0.69mm>*{};<0mm,3.0mm>*{}**@{-},
 <0.39mm,-0.39mm>*{};<2.4mm,-2.4mm>*{}**@{-},
 <-0.35mm,-0.35mm>*{};<-1.9mm,-1.9mm>*{}**@{-},
 <-2.4mm,-2.4mm>*{\circ};<-2.4mm,-2.4mm>*{}**@{},
 <-2.0mm,-2.8mm>*{};<0mm,-4.9mm>*{}**@{-},
 <-2.8mm,-2.9mm>*{};<-4.7mm,-4.9mm>*{}**@{-},
    <0.39mm,-0.39mm>*{};<3.3mm,-4.0mm>*{^1}**@{},
    <-2.0mm,-2.8mm>*{};<0.5mm,-6.7mm>*{^3}**@{},
    <-2.8mm,-2.9mm>*{};<-5.2mm,-6.7mm>*{^2}**@{},
 \end{xy}\ =\ 0,
 \\
%%%%%%%%%%%%%%%%%%%%%%% Lie[1]Bi %%%%%%%%%%%%%%%
 \begin{xy}
 <0mm,2.47mm>*{};<0mm,0.12mm>*{}**@{-},
 <0.5mm,3.5mm>*{};<2.2mm,5.2mm>*{}**@{-},
 <-0.48mm,3.48mm>*{};<-2.2mm,5.2mm>*{}**@{-},
 <0mm,3mm>*{\circ};<0mm,3mm>*{}**@{},
  <0mm,-0.8mm>*{\circ};<0mm,-0.8mm>*{}**@{},
<-0.39mm,-1.2mm>*{};<-2.2mm,-3.5mm>*{}**@{-},
 <0.39mm,-1.2mm>*{};<2.2mm,-3.5mm>*{}**@{-},
     <0.5mm,3.5mm>*{};<2.8mm,5.7mm>*{^2}**@{},
     <-0.48mm,3.48mm>*{};<-2.8mm,5.7mm>*{^1}**@{},
   <0mm,-0.8mm>*{};<-2.7mm,-5.2mm>*{^1}**@{},
   <0mm,-0.8mm>*{};<2.7mm,-5.2mm>*{^2}**@{},
\end{xy}
\  - \
\begin{xy}
 <0mm,-1.3mm>*{};<0mm,-3.5mm>*{}**@{-},
 <0.38mm,-0.2mm>*{};<2.0mm,2.0mm>*{}**@{-},
 <-0.38mm,-0.2mm>*{};<-2.2mm,2.2mm>*{}**@{-},
<0mm,-0.8mm>*{\circ};<0mm,0.8mm>*{}**@{},
 <2.4mm,2.4mm>*{\circ};<2.4mm,2.4mm>*{}**@{},
 <2.77mm,2.0mm>*{};<4.4mm,-0.8mm>*{}**@{-},
 <2.4mm,3mm>*{};<2.4mm,5.2mm>*{}**@{-},
     <0mm,-1.3mm>*{};<0mm,-5.3mm>*{^1}**@{},
     <2.5mm,2.3mm>*{};<5.1mm,-2.6mm>*{^2}**@{},
    <2.4mm,2.5mm>*{};<2.4mm,5.7mm>*{^2}**@{},
    <-0.38mm,-0.2mm>*{};<-2.8mm,2.5mm>*{^1}**@{},
    \end{xy}
\  +\
\begin{xy}
 <0mm,-1.3mm>*{};<0mm,-3.5mm>*{}**@{-},
 <0.38mm,-0.2mm>*{};<2.0mm,2.0mm>*{}**@{-},
 <-0.38mm,-0.2mm>*{};<-2.2mm,2.2mm>*{}**@{-},
<0mm,-0.8mm>*{\circ};<0mm,0.8mm>*{}**@{},
 <2.4mm,2.4mm>*{\circ};<2.4mm,2.4mm>*{}**@{},
 <2.77mm,2.0mm>*{};<4.4mm,-0.8mm>*{}**@{-},
 <2.4mm,3mm>*{};<2.4mm,5.2mm>*{}**@{-},
     <0mm,-1.3mm>*{};<0mm,-5.3mm>*{^2}**@{},
     <2.5mm,2.3mm>*{};<5.1mm,-2.6mm>*{^1}**@{},
    <2.4mm,2.5mm>*{};<2.4mm,5.7mm>*{^2}**@{},
    <-0.38mm,-0.2mm>*{};<-2.8mm,2.5mm>*{^1}**@{},
    \end{xy}
\  - \
\begin{xy}
 <0mm,-1.3mm>*{};<0mm,-3.5mm>*{}**@{-},
 <0.38mm,-0.2mm>*{};<2.0mm,2.0mm>*{}**@{-},
 <-0.38mm,-0.2mm>*{};<-2.2mm,2.2mm>*{}**@{-},
<0mm,-0.8mm>*{\circ};<0mm,0.8mm>*{}**@{},
 <2.4mm,2.4mm>*{\circ};<2.4mm,2.4mm>*{}**@{},
 <2.77mm,2.0mm>*{};<4.4mm,-0.8mm>*{}**@{-},
 <2.4mm,3mm>*{};<2.4mm,5.2mm>*{}**@{-},
     <0mm,-1.3mm>*{};<0mm,-5.3mm>*{^2}**@{},
     <2.5mm,2.3mm>*{};<5.1mm,-2.6mm>*{^1}**@{},
    <2.4mm,2.5mm>*{};<2.4mm,5.7mm>*{^1}**@{},
    <-0.38mm,-0.2mm>*{};<-2.8mm,2.5mm>*{^2}**@{},
    \end{xy}
\ + \
\begin{xy}
 <0mm,-1.3mm>*{};<0mm,-3.5mm>*{}**@{-},
 <0.38mm,-0.2mm>*{};<2.0mm,2.0mm>*{}**@{-},
 <-0.38mm,-0.2mm>*{};<-2.2mm,2.2mm>*{}**@{-},
<0mm,-0.8mm>*{\circ};<0mm,0.8mm>*{}**@{},
 <2.4mm,2.4mm>*{\circ};<2.4mm,2.4mm>*{}**@{},
 <2.77mm,2.0mm>*{};<4.4mm,-0.8mm>*{}**@{-},
 <2.4mm,3mm>*{};<2.4mm,5.2mm>*{}**@{-},
     <0mm,-1.3mm>*{};<0mm,-5.3mm>*{^1}**@{},
     <2.5mm,2.3mm>*{};<5.1mm,-2.6mm>*{^2}**@{},
    <2.4mm,2.5mm>*{};<2.4mm,5.7mm>*{^1}**@{},
    <-0.38mm,-0.2mm>*{};<-2.8mm,2.5mm>*{^2}**@{},
    \end{xy}\ =\ 0,\\
\Ba{c}
\xy
 (0,0)*{\circ}="a",
(0,6)*{\circ}="b",
(3,3)*{}="c",
(-3,3)*{}="d",
 (0,9)*{}="b'",
(0,-3)*{}="a'",
%
\ar@{-} "a";"c" <0pt>
\ar @{-} "a";"d" <0pt>
\ar @{-} "a";"a'" <0pt>
\ar @{-} "b";"c" <0pt>
\ar @{-} "b";"d" <0pt>
\ar @{-} "b";"b'" <0pt>
\endxy
\Ea\ =\ 0.
\Ea
\right.
\Eeq

The properad governing Lie bialgebras $\LieBi$ is defined in the same manner, except that the last relation of \eqref{R for LieB} is omitted.

\sip

Recall \cite{Va} that any quadratic properad $\cP$ has an associated
Koszul dual coproperad $\cP^\Koz$ such that its cobar construction,
$$
\Omega(\cP^\Koz)=\cF ree\langle\bar\cP^\Koz[-1]\rangle,
$$
comes equipped with a differential $d$ and with a canonical surjective map of dg
properads,
$$
(\Omega(\cP^\Koz), d) \lon (\cP, 0).
$$
This map always induces an isomorphism in cohomology in degree 0. If, additionally, the map is a quasi-isomorphism, then the properad $\cP$ is called {\em Koszul}. In
this
case the cobar construction $\Omega(\cP^\Koz)$ gives us a minimal resolution of $\cP$
and is denoted by $\cP_\infty$. It is well known, for example, that the properad governing Lie bialgebras $\LieBi$ is Koszul \cite{MaVo}.

\sip

We shall study below the Koszul dual properad $\LoB^\Koz$, its
cobar construction $\Omega(\LoB^\Koz)$ and prove that the natural surjection
 $\Omega(\LoB^\Koz)\rar \LoB$ is a quasi-isomorphism. Anticipating this conclusion, we
 often use the symbol $\LoB_\infty$ as a shorthand for  $\Omega(\LoB^\Koz)$.


\subsection{An explicit description of the dg properad $\LoB_\infty$}\label{sec:explicit Koszul dual of Lob}
The Koszul dual of $\LoB$ is a coproperad $\LoB^\Koz$ whose (genus-)graded dual, $(\LoB^\Koz)^*$,
 is the properad generated by  degree $1$ corollas,
$$
 \begin{xy}
 <0mm,-0.55mm>*{};<0mm,-2.5mm>*{}**@{-},
 <0.5mm,0.5mm>*{};<2.2mm,2.2mm>*{}**@{-},
 <-0.48mm,0.48mm>*{};<-2.2mm,2.2mm>*{}**@{-},
 <0mm,0mm>*{\bu};<0mm,0mm>*{}**@{},
 <0mm,-0.55mm>*{};<0mm,-3.8mm>*{_1}**@{},
 <0.5mm,0.5mm>*{};<2.7mm,2.8mm>*{^2}**@{},
 <-0.48mm,0.48mm>*{};<-2.7mm,2.8mm>*{^1}**@{},
 \end{xy}
=-
\begin{xy}
 <0mm,-0.55mm>*{};<0mm,-2.5mm>*{}**@{-},
 <0.5mm,0.5mm>*{};<2.2mm,2.2mm>*{}**@{-},
 <-0.48mm,0.48mm>*{};<-2.2mm,2.2mm>*{}**@{-},
 <0mm,0mm>*{\bu};<0mm,0mm>*{}**@{},
 <0mm,-0.55mm>*{};<0mm,-3.8mm>*{_1}**@{},
 <0.5mm,0.5mm>*{};<2.7mm,2.8mm>*{^1}**@{},
 <-0.48mm,0.48mm>*{};<-2.7mm,2.8mm>*{^2}**@{},
 \end{xy}\ \ \ , \ \ \
 \begin{xy}
 <0mm,0.66mm>*{};<0mm,3mm>*{}**@{-},
 <0.39mm,-0.39mm>*{};<2.2mm,-2.2mm>*{}**@{-},
 <-0.35mm,-0.35mm>*{};<-2.2mm,-2.2mm>*{}**@{-},
 <0mm,0mm>*{\bu};<0mm,0mm>*{}**@{},
   <0mm,0.66mm>*{};<0mm,3.4mm>*{^1}**@{},
   <0.39mm,-0.39mm>*{};<2.9mm,-4mm>*{^2}**@{},
   <-0.35mm,-0.35mm>*{};<-2.8mm,-4mm>*{^1}**@{},
\end{xy}=-
\begin{xy}
 <0mm,0.66mm>*{};<0mm,3mm>*{}**@{-},
 <0.39mm,-0.39mm>*{};<2.2mm,-2.2mm>*{}**@{-},
 <-0.35mm,-0.35mm>*{};<-2.2mm,-2.2mm>*{}**@{-},
 <0mm,0mm>*{\bu};<0mm,0mm>*{}**@{},
   <0mm,0.66mm>*{};<0mm,3.4mm>*{^1}**@{},
   <0.39mm,-0.39mm>*{};<2.9mm,-4mm>*{^1}**@{},
   <-0.35mm,-0.35mm>*{};<-2.8mm,-4mm>*{^2}**@{},
\end{xy}
$$
with the following relations,
$$
\Ba{c}
\begin{xy}
 <0mm,0mm>*{\bu};<0mm,0mm>*{}**@{},
 <0mm,-0.49mm>*{};<0mm,-3.0mm>*{}**@{-},
 <0.49mm,0.49mm>*{};<1.9mm,1.9mm>*{}**@{-},
 <-0.5mm,0.5mm>*{};<-1.9mm,1.9mm>*{}**@{-},
 <-2.3mm,2.3mm>*{\bu};<-2.3mm,2.3mm>*{}**@{},
 <-1.8mm,2.8mm>*{};<0mm,4.9mm>*{}**@{-},
 <-2.8mm,2.9mm>*{};<-4.6mm,4.9mm>*{}**@{-},
   <0.49mm,0.49mm>*{};<2.7mm,2.3mm>*{^3}**@{},
   <-1.8mm,2.8mm>*{};<0.4mm,5.3mm>*{^2}**@{},
   <-2.8mm,2.9mm>*{};<-5.1mm,5.3mm>*{^1}**@{},
 \end{xy}\Ea
\ =- \
\Ba{c}
\begin{xy}
 <0mm,0mm>*{\bu};<0mm,0mm>*{}**@{},
 <0mm,-0.49mm>*{};<0mm,-3.0mm>*{}**@{-},
 <0.49mm,0.49mm>*{};<1.9mm,1.9mm>*{}**@{-},
 <-0.5mm,0.5mm>*{};<-1.9mm,1.9mm>*{}**@{-},
 <2.3mm,2.3mm>*{\bu};<-2.3mm,2.3mm>*{}**@{},
 <1.8mm,2.8mm>*{};<0mm,4.9mm>*{}**@{-},
 <2.8mm,2.9mm>*{};<4.6mm,4.9mm>*{}**@{-},
   <0.49mm,0.49mm>*{};<-2.7mm,2.3mm>*{^1}**@{},
   <-1.8mm,2.8mm>*{};<0mm,5.3mm>*{^2}**@{},
   <-2.8mm,2.9mm>*{};<5.1mm,5.3mm>*{^3}**@{},
 \end{xy}\Ea, \ \ \ \ \
%%%%%%%%%%%%%%%%%%%%%%%%%%%%%%%%%%%%%%
 \Ba{c}\begin{xy}
 <0mm,0mm>*{\bu};<0mm,0mm>*{}**@{},
 <0mm,0.69mm>*{};<0mm,3.0mm>*{}**@{-},
 <0.39mm,-0.39mm>*{};<2.4mm,-2.4mm>*{}**@{-},
 <-0.35mm,-0.35mm>*{};<-1.9mm,-1.9mm>*{}**@{-},
 <-2.4mm,-2.4mm>*{\bu};<-2.4mm,-2.4mm>*{}**@{},
 <-2.0mm,-2.8mm>*{};<0mm,-4.9mm>*{}**@{-},
 <-2.8mm,-2.9mm>*{};<-4.7mm,-4.9mm>*{}**@{-},
    <0.39mm,-0.39mm>*{};<3.3mm,-4.0mm>*{^3}**@{},
    <-2.0mm,-2.8mm>*{};<0.5mm,-6.7mm>*{^2}**@{},
    <-2.8mm,-2.9mm>*{};<-5.2mm,-6.7mm>*{^1}**@{},
 \end{xy}\Ea
\ =- \
 \Ba{c}\begin{xy}
 <0mm,0mm>*{\bu};<0mm,0mm>*{}**@{},
 <0mm,0.69mm>*{};<0mm,3.0mm>*{}**@{-},
 <0.39mm,-0.39mm>*{};<2.4mm,-2.4mm>*{}**@{-},
 <-0.35mm,-0.35mm>*{};<-1.9mm,-1.9mm>*{}**@{-},
 <2.4mm,-2.4mm>*{\bu};<-2.4mm,-2.4mm>*{}**@{},
 <2.0mm,-2.8mm>*{};<0mm,-4.9mm>*{}**@{-},
 <2.8mm,-2.9mm>*{};<4.7mm,-4.9mm>*{}**@{-},
    <0.39mm,-0.39mm>*{};<-3mm,-4.0mm>*{^1}**@{},
    <-2.0mm,-2.8mm>*{};<0mm,-6.7mm>*{^2}**@{},
    <-2.8mm,-2.9mm>*{};<5.2mm,-6.7mm>*{^3}**@{},
 \end{xy}\Ea,\ \ \ \ \ \
 %
 \begin{xy}
 <0mm,2.47mm>*{};<0mm,0.12mm>*{}**@{-},
 <0.5mm,3.5mm>*{};<2.2mm,5.2mm>*{}**@{-},
 <-0.48mm,3.48mm>*{};<-2.2mm,5.2mm>*{}**@{-},
 <0mm,3mm>*{\bu};<0mm,3mm>*{}**@{},
  <0mm,-0.8mm>*{\bu};<0mm,-0.8mm>*{}**@{},
<-0.39mm,-1.2mm>*{};<-2.2mm,-3.5mm>*{}**@{-},
 <0.39mm,-1.2mm>*{};<2.2mm,-3.5mm>*{}**@{-},
     <0.5mm,3.5mm>*{};<2.8mm,5.7mm>*{^2}**@{},
     <-0.48mm,3.48mm>*{};<-2.8mm,5.7mm>*{^1}**@{},
   <0mm,-0.8mm>*{};<-2.7mm,-5.2mm>*{^1}**@{},
   <0mm,-0.8mm>*{};<2.7mm,-5.2mm>*{^2}**@{},
\end{xy}
\  =- \
\begin{xy}
 <0mm,-1.3mm>*{};<0mm,-3.5mm>*{}**@{-},
 <0.38mm,-0.2mm>*{};<2.0mm,2.0mm>*{}**@{-},
 <-0.38mm,-0.2mm>*{};<-2.2mm,2.2mm>*{}**@{-},
<0mm,-0.8mm>*{\bu};<0mm,0.8mm>*{}**@{},
 <2.4mm,2.4mm>*{\bu};<2.4mm,2.4mm>*{}**@{},
 <2.77mm,2.0mm>*{};<4.4mm,-0.8mm>*{}**@{-},
 <2.4mm,3mm>*{};<2.4mm,5.2mm>*{}**@{-},
     <0mm,-1.3mm>*{};<0mm,-5.3mm>*{^1}**@{},
     <2.5mm,2.3mm>*{};<5.1mm,-2.6mm>*{^2}**@{},
    <2.4mm,2.5mm>*{};<2.4mm,5.7mm>*{^2}**@{},
    <-0.38mm,-0.2mm>*{};<-2.8mm,2.5mm>*{^1}**@{},
    \end{xy}.
$$
Hence the following graphs
\Beq\label{2: basis of inv Frob}
\resizebox{10mm}{!}{
\xy
(0,-6)*{};
(0,0)*+{a}*\cir{}
**\dir{-};
(0,6)*{};
(0,0)*+{a}*\cir{}
**\dir{-};
(0,6)*-{\bu}="1",
(-3,9)*-{\bu}="2",
(3,9)*{}="2'",
(-6,12)*{}="3",
(-0,12)*{}="3'",
(-6.7,12.7)*{\cdot};
(-7.7,13.7)*{\cdot};
(-9,15)*-{\bu}="4",
(-12,18)*{}="4'",
(-6,18)*{}="4''",
%
(0,-6)*-{\bu}="-1",
(-3,-9)*-{\bu}="-2",
(3,-9)*{}="-2'",
(-6,-12)*{}="-3",
(-0,-12)*{}="-3'",
(-6.7,-12.7)*{\cdot};
(-7.7,-13.7)*{\cdot};
(-9,-15)*-{\bu}="-4",
(-12,-18)*{}="-4'",
(-6,-18)*{}="-4''",
%
\ar @{-} "1";"2" <0pt>
\ar @{-} "1";"2'" <0pt>
\ar @{-} "2";"3" <0pt>
\ar @{-} "2";"3'" <0pt>
\ar @{-} "4";"4'" <0pt>
\ar @{-} "4";"4''" <0pt>
%
\ar @{-} "-1";"-2" <0pt>
\ar @{-} "-1";"-2'" <0pt>
\ar @{-} "-2";"-3" <0pt>
\ar @{-} "-2";"-3'" <0pt>
\ar @{-} "-4";"-4'" <0pt>
\ar @{-} "-4";"-4''" <0pt>
\endxy
}
\Eeq
where
\Beq\label{2: a weight in LieB-Koszul}
\resizebox{3.1mm}{!}{\xy
(0,-5)*{};
(0,0)*+{_a}*\cir{}
**\dir{-};
(0,5)*{};
(0,0)*+{_a}*\cir{}
**\dir{-};
\endxy}:=\Ba{c}\resizebox{4mm}{!}{\xy
 (0,0)*-{\bu}="a",
(0,6)*-{\bu}="b",
(3,3)*{}="c",
(-3,3)*{}="d",
 (0,9)*{}="b'",
(0,-3)*{}="a'",
\ar @{-} "a";"c" <0pt>
\ar @{-} "a";"d" <0pt>
\ar @{-} "a";"a'" <0pt>
\ar @{-} "b";"c" <0pt>
\ar @{-} "b";"d" <0pt>
\ar @{-} "b";"b'" <0pt>
\endxy}\\
\cdot\vspace{-2mm}\\
\cdot\\
\resizebox{4mm}{!}{\xy
 (0,0)*-{\bu}="a",
(0,6)*-{\bu}="b",
(3,3)*{}="c",
(-3,3)*{}="d",
 (0,9)*{}="b'",
(0,-3)*{}="a'",
\ar @{-} "a";"c" <0pt>
\ar @{-} "a";"d" <0pt>
\ar @{-} "a";"a'" <0pt>
\ar @{-} "b";"c" <0pt>
\ar @{-} "b";"d" <0pt>
\ar @{-} "b";"b'" <0pt>
\endxy}
\Ea
\Eeq
is the composition of $a=0,1,2,\dots$ graphs of the form $\resizebox{4mm}{!}{\xy
 (0,0)*-{\bu}="a",
(0,6)*-{\bu}="b",
(3,3)*{}="c",
(-3,3)*{}="d",
 (0,9)*{}="b'",
(0,-3)*{}="a'",
\ar @{-} "a";"c" <0pt>
\ar @{-} "a";"d" <0pt>
\ar @{-} "a";"a'" <0pt>
\ar @{-} "b";"c" <0pt>
\ar @{-} "b";"d" <0pt>
\ar @{-} "b";"b'" <0pt>
\endxy}$\, ,
form  a basis of $(\LoB^\Koz)^*$.
If the graph (\ref{2: basis of inv Frob}) has $n$ input legs and $m$ output legs,
then it has $m+n+2a-2$ vertices and its degree is equal to
$m+n+2a-2$. Hence the properad
$\Omega((\LoB)^\Koz)=\cF ree\langle \overline{(\LoB)^\Koz}[-1]\rangle$ is a free properad generated
by the following skewsymmetric corollas of degree $3-m-n-2a$,
\Beq\label{2: generating corollas of LoB infty}
\resizebox{15mm}{!}{
\xy
(-9,-6)*{};
(0,0)*+{a}*\cir{}
**\dir{-};
(-5,-6)*{};
(0,0)*+{a}*\cir{}
**\dir{-};
(9,-6)*{};
(0,0)*+{a}*\cir{}
**\dir{-};
(5,-6)*{};
(0,0)*+{a}*\cir{}
**\dir{-};
(0,-6)*{\ldots};
(-10,-8)*{_1};
(-6,-8)*{_2};
(10,-8)*{_n};
%
(-9,6)*{};
(0,0)*+{a}*\cir{}
**\dir{-};
(-5,6)*{};
(0,0)*+{a}*\cir{}
**\dir{-};
(9,6)*{};
(0,0)*+{a}*\cir{}
**\dir{-};
(5,6)*{};
(0,0)*+{a}*\cir{}
**\dir{-};
(0,6)*{\ldots};
(-10,8)*{_1};
(-6,8)*{_2};
(10,8)*{_m};
\endxy}
=(-1)^{\sigma+\tau}
\resizebox{18mm}{!}{
\xy
(-9,-6)*{};
(0,0)*+{a}*\cir{}
**\dir{-};
(-5,-6)*{};
(0,0)*+{a}*\cir{}
**\dir{-};
(9,-6)*{};
(0,0)*+{a}*\cir{}
**\dir{-};
(5,-6)*{};
(0,0)*+{a}*\cir{}
**\dir{-};
(0,-6)*{\ldots};
(-12,-8)*{_{\tau(1)}};
(-6,-8)*{_{\tau(2)}};
(12,-8)*{_{\tau(n)}};
%
(-9,6)*{};
(0,0)*+{a}*\cir{}
**\dir{-};
(-5,6)*{};
(0,0)*+{a}*\cir{}
**\dir{-};
(9,6)*{};
(0,0)*+{a}*\cir{}
**\dir{-};
(5,6)*{};
(0,0)*+{a}*\cir{}
**\dir{-};
(0,6)*{\ldots};
(-12,8)*{_{\sigma(1)}};
(-6,8)*{_{\sigma(2)}};
(12,8)*{_{\sigma(m)}};
\endxy}\ \ \ \forall \sigma\in \bS_m, \forall \tau\in \bS_n,
\Eeq
where $m+n+ a\geq 3$, $m\geq 1$, $n\geq 1$, $a\geq 0$. The non-negative number $a$ is
called the {\em weight}\, of the generating corolla (\ref{2: generating corollas of LoB
infty}). The differential in
$\Omega((\LoB)^\Koz)$ is given by\footnote{The precise sign factors in this formula can be
determined via a usual trick: the analogous differential in the degree shifted properad
$\Omega((\LoB)^\Koz)\{1\}$ must be given by the same formula but with all sign factors equal to
$+1$.}
\Beq\label{2: d on Lie inv infty}
\delta\Ba{c}
\resizebox{15mm}{!}{\xy
(-9,-6)*{};
(0,0)*+{a}*\cir{}
**\dir{-};
(-5,-6)*{};
(0,0)*+{a}*\cir{}
**\dir{-};
(9,-6)*{};
(0,0)*+{a}*\cir{}
**\dir{-};
(5,-6)*{};
(0,0)*+{a}*\cir{}
**\dir{-};
(0,-6)*{\ldots};
(-10,-8)*{_1};
(-6,-8)*{_2};
(10,-8)*{_n};
%
(-9,6)*{};
(0,0)*+{a}*\cir{}
**\dir{-};
(-5,6)*{};
(0,0)*+{a}*\cir{}
**\dir{-};
(9,6)*{};
(0,0)*+{a}*\cir{}
**\dir{-};
(5,6)*{};
(0,0)*+{a}*\cir{}
**\dir{-};
(0,6)*{\ldots};
(-10,8)*{_1};
(-6,8)*{_2};
(10,8)*{_m};
\endxy}\Ea=
\sum_{a=b+c+l-1}\sum_{[m]=I_1\sqcup I_2\atop
[n]=J_1\sqcup J_2} \pm
\Ba{c}
%
%
%%%%%%%%%%%%%%%% two vertex graph with l internal edges %%%%%%%%%%
\resizebox{20mm}{!}{\xy
(0,0)*+{b}*\cir{}="b",
(10,10)*+{c}*\cir{}="c",
%
%%%%%%%%%% edges to b %%%%%%%%%%%%
(-9,6)*{}="1",
(-7,6)*{}="2",
(-2,6)*{}="3",
(-3.5,5)*{...},
(-4,-6)*{}="-1",
(-2,-6)*{}="-2",
(4,-6)*{}="-3",
(1,-5)*{...},
(0,-8)*{\underbrace{\ \ \ \ \ \ \ \ }},
(0,-11)*{_{J_1}},
(-6,8)*{\overbrace{ \ \ \ \ \ \ }},
(-6,11)*{_{I_1}},
%%%%%%%%%% edges to c %%%%%%%%%%%%
(6,16)*{}="1'",
(8,16)*{}="2'",
(14,16)*{}="3'",
(11,15)*{...},
(11,6)*{}="-1'",
(16,6)*{}="-2'",
(18,6)*{}="-3'",
(13.5,6)*{...},
(15,4)*{\underbrace{\ \ \ \ \ \ \ }},
(15,1)*{_{J_2}},
(10,18)*{\overbrace{ \ \ \ \ \ \ \ \ }},
(10,21)*{_{I_2}},
%
%%%%%%%%%%% internal curved edges %%%%%%%%%%%
(0,2)*-{};(8.0,10.0)*-{}
**\crv{(0,10)};
(0.5,1.8)*-{};(8.5,9.0)*-{}
**\crv{(0.4,7)};
%
(1.5,0.5)*-{};(9.1,8.5)*-{}
**\crv{(5,1)};
(1.7,0.0)*-{};(9.5,8.6)*-{}
**\crv{(6,-1)};
(5,5)*+{...};
%
\ar @{-} "b";"1" <0pt>
\ar @{-} "b";"2" <0pt>
\ar @{-} "b";"3" <0pt>
\ar @{-} "b";"-1" <0pt>
\ar @{-} "b";"-2" <0pt>
\ar @{-} "b";"-3" <0pt>
%
\ar @{-} "c";"1'" <0pt>
\ar @{-} "c";"2'" <0pt>
\ar @{-} "c";"3'" <0pt>
\ar @{-} "c";"-1'" <0pt>
\ar @{-} "c";"-2'" <0pt>
\ar @{-} "c";"-3'" <0pt>
\endxy}
%%%%%%%%%%%%%%%%%%%%%%%%%%%%%%%%%%%%%%%%%%%
\Ea
\Eeq
where  the parameter $l$ counts the number of internal edges connecting
the two vertices
on the right-hand side. We have, in particular,
$$
\delta\Ba{c}
\resizebox{5mm}{!}{\xy
(-3,-4)*{};
(0,0)*+{_0}*\cir{}
**\dir{-};
(3,-4)*{};
(0,0)*+{_0}*\cir{}
**\dir{-};
%
(0,5)*{};
(0,0)*+{_0}*\cir{}
**\dir{-};
\endxy}\Ea
=0, \ \ \ \ \ \delta\Ba{c} \resizebox{5mm}{!}{\xy
(-3,4)*{};
(0,0)*+{_0}*\cir{}
**\dir{-};
(3,4)*{};
(0,0)*+{_0}*\cir{}
**\dir{-};
%
(0,-5)*{};
(0,0)*+{_0}*\cir{}
**\dir{-};
\endxy}\Ea=0,\ \ \ \
\delta\Ba{c} \resizebox{3mm}{!}{\xy
(0,5)*{};
(0,0)*+{_1}*\cir{}
**\dir{-};
%
(0,-5)*{};
(0,0)*+{_1}*\cir{}
**\dir{-};
\endxy}\Ea=
\Ba{c}\resizebox{4mm}{!}{\xy
(-3,0)*{};
(0,4)*+{_0}*\cir{}
**\dir{-};
(3,0)*{};
(0,4)*+{_0}*\cir{}
**\dir{-};
%
(0,9)*{};
(0,4)*+{_0}*\cir{}
%
**\dir{-};
(-3,0)*{};
(0,-4)*+{_0}*\cir{}
**\dir{-};
(3,0)*{};
(0,-4)*+{_0}*\cir{}
**\dir{-};
%
(0,-9)*{};
(0,-4)*+{_0}*\cir{}
**\dir{-};
\endxy}\Ea
$$
so that the map
\Beq\label{2: pi quasi-iso}
\pi: \LoB_\infty \lon \LoB,
\Eeq
which sends to zero all generators of $\LoB_\infty$ except the following ones,
$$
\pi\left(\Ba{c}\resizebox{5mm}{!}{ \xy
(-3,-4)*{};
(0,0)*+{_0}*\cir{}
**\dir{-};
(3,-4)*{};
(0,0)*+{_0}*\cir{}
**\dir{-};
%
(0,5)*{};
(0,0)*+{_0}*\cir{}
**\dir{-};
\endxy}\Ea\right)
=\Ba{c}\resizebox{3.6mm}{!}{\begin{xy}
 <0mm,0.66mm>*{};<0mm,4mm>*{}**@{-},
 <0.39mm,-0.39mm>*{};<2.2mm,-3.2mm>*{}**@{-},
 <-0.35mm,-0.35mm>*{};<-2.2mm,-3.2mm>*{}**@{-},
 <0mm,0mm>*{\circ};<0mm,0mm>*{}**@{},
\end{xy}}\Ea\ ,\ \ \ \ \ \
\pi\left(\Ba{c}\resizebox{5mm}{!}{ \xy
(-3,4)*{};
(0,0)*+{_0}*\cir{}
**\dir{-};
(3,4)*{};
(0,0)*+{_0}*\cir{}
**\dir{-};
%
(0,-5)*{};
(0,0)*+{_0}*\cir{}
**\dir{-};
\endxy}\Ea\right)
=\Ba{c}\resizebox{3.6mm}{!}{\begin{xy}
 <0mm,-0.66mm>*{};<0mm,-4mm>*{}**@{-},
 <0.4mm,0.4mm>*{};<2.2mm,3.2mm>*{}**@{-},
 <-0.4mm,0.4mm>*{};<-2.2mm,3.2mm>*{}**@{-},
 <0mm,-0.1mm>*{\circ};<0mm,0mm>*{}**@{},
\end{xy}}\Ea\ ,
$$
is a morphism of dg properads, as expected. Showing that the properad $\LoB$ is Koszul is
equivalent to showing that the map $\pi$ is a quasi-isomorphism. As $\LoB_\infty$ is non-positively
graded, the map $\pi$ is a quasi-isomorphism if and only if the cohomology of the dg
properad
$\LoB_\infty$ is concentrated in degree zero. We shall prove this property below in \S
{\ref{2: Theorem on Koszulness}}
with the help of several auxiliary constructions which we discuss next.

\subsection{A decomposition of the complex  $\LoB_\infty$}\label{sec:decomposition}
As a vector space the properad  $\LoB_\infty=\{\LoB_\infty(m,n)\}_{m,n\geq 1}$ is spanned
by oriented graphs built from corollas (\ref{2: generating corollas of LoB infty}). For
such a
graph $\Ga \in  \LoB_\infty$ we set
$$
||\Ga||:= g(\Ga)+
w(\Ga)\in \N
$$
where  $g(\Ga)$ is its genus and  $w(\Ga)$ is its total weight defined as the sum of
weights of its vertices (corollas). It is obvious that the differential $\delta$ in
$\LoB_\infty$ respects this total grading,
$$
|| \delta \Ga||=||\Ga||.
$$
Therefore each complex $(\LoB_\infty(m,n),\delta)$ decomposes into a direct sum of
subcomplexes,
$$
\LoB_\infty(m,n)=\sum_{s\geq 0}\LoB_\infty(m,n)^{(s)}
$$
where $\LoB_\infty(m,n)^{(s)}\subset \LoB_\infty(m,n)$ is spanned by graphs $\Gamma$ with $||\Ga||=s$.

\subsubsection{\bf Lemma}
{\em For any fixed  $m,n\geq 1$ and $s\geq 0$ the subcomplex $\LoB_\infty(m,n)^{(s)}$ is
finite-dimensional}.
\begin{proof}
The number of bivalent vertices in every graph $\Ga$ with $||\Ga||=s$ is finite.
As the genus of the graph $\Ga$ is also finite, it must have a finite number of vertices of
valence
$\geq 3$ as well.
\end{proof}

 This lemma guarantees convergence of all spectral sequences which we consider below in the
 context of computing the cohomology of $\LoB_\infty$  and which, for general dg free
 properads, can be ill-behaved.


\subsection{An auxiliary graph complex}\label{2 subsect on aux graph comlxes} Let us
consider a graph complex,
$$
C =\bigoplus_{n\geq 1} C^n.
$$
where  $C^n$ is spanned by graphs of the form,
$\Ba{c}\resizebox{27mm}{!}{\xy
(-8,2)*+{};
(0,2)*+{_{a_1}}*\cir{}
**\dir{-};
(10,2)*+{_{a_2}}*\cir{}
**\dir{-};
(20,2)*+{...}
**\dir{-};
(30,2)*+{_{a_{n}}}*\cir{};
**\dir{-};
(31.5,2)*+{};(37,2)
**\dir{-};
\endxy}\Ea$\ , \ \ with $a_1,\ldots, a_n\in \N$.
The differential is given on the generators of the graphs (viewed as
elements of a $\frac{1}{2}$-prop) by
$$
d\Ba{c}\resizebox{8.0
mm}{!}{\xy(0,1)*+{_{a}}*\cir{};
(5,1)
**\dir{-};(0,1)*+{_{a}}*\cir{};
(-5,1)
**\dir{-};
\endxy}\Ea = \sum_{a=b+c\atop b\geq 1, c\geq 1}\Ba{c}\resizebox{15mm}{!}{\xy(0,1)*+{_{b}}*\cir{};
(8,1)*+{_{c}}*\cir{}
**\dir{-};
(0,1)*+{_{b}}*\cir{};
(-5,1)**\dir{-};
(8,1)*+{_{c}}*\cir{};
(13,1)**\dir{-};
\endxy}\Ea.
$$


\subsubsection{\bf Proposition}\label{2: propos on aux graph complexes} {\em One has}
$H^\bu(C)= \mbox{span} \langle \xy(0,0)*+{_{1}}*\cir{};
(5,0)
**\dir{-};(0,0)*+{_1}*\cir{};
(-5,0)
**\dir{-};
\endxy   \rangle$.

\begin{proof} It is well-known that the cohomology of the cobar construction
  $\Omega(T^c(V))$ of the  tensor coalgebra $T^c(V)$ generated by any vector space $V$
  over a field $\K$ is equal to $\K\oplus V$, so that the cohomology of the reduced cobar
  construction, $\overline{\Omega}(T^c(V))$, equals $V$. The complex $C$ is
  isomorphic to  $\overline{\Omega}(T^c(V))$ for a one-dimensional vector space $V$ via the
  following identification
  $$
\xy(0,0)*+{_{a}}*\cir{};
\endxy \cong  V^{\ot a}.
  $$
  Hence the claim.
\end{proof}



\subsection{An auxiliary dg properad}\label{2: subsection on P} Let $\cP$ be a dg properad
generated by a degree $-1$ corolla
$\begin{xy}
 <0mm,-0.55mm>*{};<0mm,-3mm>*{}**@{-},
 <0mm,0.5mm>*{};<0mm,3mm>*{}**@{-},
 <0mm,0mm>*{\bu};<0mm,0mm>*{}**@{},
 \end{xy}$\, and degree zero corollas,
$
\Ba{c}
 \begin{xy}
 <0mm,-0.55mm>*{};<0mm,-2.5mm>*{}**@{-},
 <0.5mm,0.5mm>*{};<2.2mm,2.2mm>*{}**@{-},
 <-0.48mm,0.48mm>*{};<-2.2mm,2.2mm>*{}**@{-},
 <0mm,0mm>*{\circ};<0mm,0mm>*{}**@{},
 %<0mm,-0.55mm>*{};<0mm,-3.8mm>*{_1}**@{},
 <0.5mm,0.5mm>*{};<2.7mm,2.8mm>*{^2}**@{},
 <-0.48mm,0.48mm>*{};<-2.7mm,2.8mm>*{^1}**@{},
 \end{xy}
=-
\begin{xy}
 <0mm,-0.55mm>*{};<0mm,-2.5mm>*{}**@{-},
 <0.5mm,0.5mm>*{};<2.2mm,2.2mm>*{}**@{-},
 <-0.48mm,0.48mm>*{};<-2.2mm,2.2mm>*{}**@{-},
 <0mm,0mm>*{\circ};<0mm,0mm>*{}**@{},
 %<0mm,-0.55mm>*{};<0mm,-3.8mm>*{_1}**@{},
 <0.5mm,0.5mm>*{};<2.7mm,2.8mm>*{^1}**@{},
 <-0.48mm,0.48mm>*{};<-2.7mm,2.8mm>*{^2}**@{},
 \end{xy}\Ea
 $ and $\Ba{c}\begin{xy}
 <0mm,0.66mm>*{};<0mm,3mm>*{}**@{-},
 <0.39mm,-0.39mm>*{};<2.2mm,-2.2mm>*{}**@{-},
 <-0.35mm,-0.35mm>*{};<-2.2mm,-2.2mm>*{}**@{-},
 <0mm,0mm>*{\circ};<0mm,0mm>*{}**@{},
   %<0mm,0.66mm>*{};<0mm,3.4mm>*{^1}**@{},
   <0.39mm,-0.39mm>*{};<2.9mm,-4mm>*{^2}**@{},
   <-0.35mm,-0.35mm>*{};<-2.8mm,-4mm>*{^1}**@{},
\end{xy}=-
\begin{xy}
 <0mm,0.66mm>*{};<0mm,3mm>*{}**@{-},
 <0.39mm,-0.39mm>*{};<2.2mm,-2.2mm>*{}**@{-},
 <-0.35mm,-0.35mm>*{};<-2.2mm,-2.2mm>*{}**@{-},
 <0mm,0mm>*{\circ};<0mm,0mm>*{}**@{},
   %<0mm,0.66mm>*{};<0mm,3.4mm>*{^1}**@{},
   <0.39mm,-0.39mm>*{};<2.9mm,-4mm>*{^1}**@{},
   <-0.35mm,-0.35mm>*{};<-2.8mm,-4mm>*{^2}**@{},
\end{xy}\Ea$,
 modulo relations \Beq\label{2: relation in P}
\xy
%
(0,-1.9)*{\bu}="0",
 (0,1.9)*{\bu}="1",
(0,-5)*{}="d",
(0,5)*{}="u",
%
\ar @{-} "0";"u" <0pt>
\ar @{-} "0";"1" <0pt>
\ar @{-} "1";"d" <0pt>
\endxy=0, \ \ \ \
\xy
%
(0,-1.9)*{\circ}="0",
 (0,1.9)*{\bu}="1",
(-2.5,-5)*{}="d1",
(2.5,-5)*{}="d2",
(0,5)*{}="u",
%
\ar @{-} "1";"u" <0pt>
\ar @{-} "0";"1" <0pt>
\ar @{-} "0";"d1" <0pt>
\ar @{-} "0";"d2" <0pt>
\endxy  -
\xy
%
(-2.5,-1.9)*{\bu}="0",
 (0,1.9)*{\circ}="1",
(-2.5,-1.9)*{}="d1",
(2.5,-1.9)*{}="d2",
(-2.5,-5)*{}="d",
(0,5)*{}="u",
%
\ar @{-} "1";"u" <0pt>
\ar @{-} "0";"1" <0pt>
\ar @{-} "0";"d" <0pt>
\ar @{-} "1";"d1" <0pt>
\ar @{-} "1";"d2" <0pt>
\endxy
  -
\xy
%
(2.5,-1.9)*{\bu}="0",
 (0,1.9)*{\circ}="1",
(-2.5,-1.9)*{}="d1",
(2.5,-1.9)*{}="d2",
(2.5,-5)*{}="d",
(0,5)*{}="u",
%
\ar @{-} "1";"u" <0pt>
\ar @{-} "0";"1" <0pt>
\ar @{-} "0";"d" <0pt>
\ar @{-} "1";"d1" <0pt>
\ar @{-} "1";"d2" <0pt>
\endxy
=0\ , \ \ \ \
\xy
%
(0,1.9)*{\circ}="0",
 (0,-1.9)*{\bu}="1",
(-2.5,5)*{}="d1",
(2.5,5)*{}="d2",
(0,-5)*{}="u",
%
\ar @{-} "1";"u" <0pt>
\ar @{-} "0";"1" <0pt>
\ar @{-} "0";"d1" <0pt>
\ar @{-} "0";"d2" <0pt>
\endxy  -
\xy
%
(-2.5,1.9)*{\bu}="0",
 (0,-1.9)*{\circ}="1",
(-2.5,1.9)*{}="d1",
(2.5,1.9)*{}="d2",
(-2.5,5)*{}="d",
(0,-5)*{}="u",
%
\ar @{-} "1";"u" <0pt>
\ar @{-} "0";"1" <0pt>
\ar @{-} "0";"d" <0pt>
\ar @{-} "1";"d1" <0pt>
\ar @{-} "1";"d2" <0pt>
\endxy
  -
\xy
%
(2.5,1.9)*{\bu}="0",
 (0,-1.9)*{\circ}="1",
(-2.5,1.9)*{}="d1",
(2.5,1.9)*{}="d2",
(2.5,5)*{}="d",
(0,-5)*{}="u",
%
\ar @{-} "1";"u" <0pt>
\ar @{-} "0";"1" <0pt>
\ar @{-} "0";"d" <0pt>
\ar @{-} "1";"d1" <0pt>
\ar @{-} "1";"d2" <0pt>
\endxy
=0.
\Eeq and the first three relations in (\ref{R for
 LieB}).
 The differential in $\cP$ is given on the generators by
\Beq\label{2: differential in P}
d\, \begin{xy}
 <0mm,0.66mm>*{};<0mm,3mm>*{}**@{-},
 <0.39mm,-0.39mm>*{};<2.2mm,-2.2mm>*{}**@{-},
 <-0.35mm,-0.35mm>*{};<-2.2mm,-2.2mm>*{}**@{-},
 <0mm,0mm>*{\circ};<0mm,0mm>*{}**@{},
\end{xy}=0, \ \ \ \
d\, \begin{xy}
 <0mm,-0.55mm>*{};<0mm,-2.5mm>*{}**@{-},
 <0.5mm,0.5mm>*{};<2.2mm,2.2mm>*{}**@{-},
 <-0.48mm,0.48mm>*{};<-2.2mm,2.2mm>*{}**@{-},
 <0mm,0mm>*{\circ};<0mm,0mm>*{}**@{},
 \end{xy} =0, \ \ \ \
d\, \begin{xy}
 <0mm,-0.55mm>*{};<0mm,-3mm>*{}**@{-},
 <0mm,0.5mm>*{};<0mm,3mm>*{}**@{-},
 <0mm,0mm>*{\bu};<0mm,0mm>*{}**@{},
 \end{xy}=\Ba{c}\xy
 (0,0)*{\circ}="a",
(0,6)*{\circ}="b",
(3,3)*{}="c",
(-3,3)*{}="d",
 (0,9)*{}="b'",
(0,-3)*{}="a'",
%"a";"c"**\dir{-};
\ar@{-} "a";"c" <0pt>
\ar @{-} "a";"d" <0pt>
\ar @{-} "a";"a'" <0pt>
\ar @{-} "b";"c" <0pt>
\ar @{-} "b";"d" <0pt>
\ar @{-} "b";"b'" <0pt>
\endxy\Ea.
\Eeq
%%%%%%%%%%%%%%%%%%%%%%%%%%%%%%%%%%%%%%%%%%%%%%%%%
\begin{theorem}\label{2: proposition on nu quasi-iso} { The surjective morphism of
dg properads,
\Beq\label{2: nu quasi-iso to P}
\nu: \LoB_\infty \lon \cP,
\Eeq
which sends all generators to zero except for the following ones
\Beq\label{2: map from P to P}
\nu\left( \xy
(-3,-4)*{};
(0,0)*+{_0}*\cir{}
**\dir{-};
(3,-4)*{};
(0,0)*+{_0}*\cir{}
**\dir{-};
%
(0,5)*{};
(0,0)*+{_0}*\cir{}
**\dir{-};
\endxy\right)
=\begin{xy}
 <0mm,0.66mm>*{};<0mm,4mm>*{}**@{-},
 <0.39mm,-0.39mm>*{};<2.2mm,-3.2mm>*{}**@{-},
 <-0.35mm,-0.35mm>*{};<-2.2mm,-3.2mm>*{}**@{-},
 <0mm,0mm>*{\circ};<0mm,0mm>*{}**@{},
\end{xy}\ ,\ \ \ \ \ \
\nu\left( \xy
(-3,4)*{};
(0,0)*+{_0}*\cir{}
**\dir{-};
(3,4)*{};
(0,0)*+{_0}*\cir{}
**\dir{-};
%
(0,-5)*{};
(0,0)*+{_0}*\cir{}
**\dir{-};
\endxy\right)
=\begin{xy}
 <0mm,-0.66mm>*{};<0mm,-4mm>*{}**@{-},
 <0.4mm,0.4mm>*{};<2.2mm,3.2mm>*{}**@{-},
 <-0.4mm,0.4mm>*{};<-2.2mm,3.2mm>*{}**@{-},
 <0mm,-0.1mm>*{\circ};<0mm,0mm>*{}**@{},
\end{xy}\ \ , \ \
\nu\left(
\xy
(0,5)*{};
(0,0)*+{_1}*\cir{}
**\dir{-};
%
(0,-5)*{};
(0,0)*+{_1}*\cir{}
**\dir{-};
\endxy\right)=\begin{xy}
 <0mm,-0.55mm>*{};<0mm,-3mm>*{}**@{-},
 <0mm,0.5mm>*{};<0mm,3mm>*{}**@{-},
 <0mm,0mm>*{\bu};<0mm,0mm>*{}**@{},
 \end{xy}
\Eeq
is a quasi-isomorphism.}
\end{theorem}


\begin{proof} The argument is based on several converging spectral sequences.
\mip



{\em Step 1: An exact functor}.

We define the following functor:
$$
F : \mbox{category of dg}\ \frac{1}{2}\mbox{-props} \lon \mbox{category of dg properads},
$$
by
$$
F(s)(m,n) = \bigoplus_{\Gamma \in \overline{\text{Gr}}(m,n)} \left(\bigotimes_{v\in v(\Gamma)} s(Out(v),In(v)) \otimes \odot H^1(\Gamma, \partial \Gamma)\right) _{Aut(\Gamma)},
$$
where $\overline{\text{Gr}}(m,n)$ represents the set of all (isomorphism classes of) oriented graphs with $n$ output legs and $m$ input legs that are irreducible in the sense that they do not allow any $\frac{1}{2}\mbox{-prop}$ic contractions. We consider the relative cohomology $H^1(\Gamma,\partial \Gamma)$ to live in cohomological degree $1$. In particular the graded symmetric product $\odot H^1(\Gamma, \partial \Gamma)$ is finite dimensional, and the square of any relative cohomology class vanishes. The differential acts trivially on the $H^1(\Gamma, \partial \Gamma)$ part.
Our functor $F$ is similar to the Kontsevich functor $F$ discussed in full details
in \cite{MaVo} except the tensor factor $\odot H^1(\Gamma, \partial \Gamma)$. One defines the properadic compositions on $F(s)$ as in the Kontsevich case, with the tensor factors handled as follows: Suppose we compose elements corresponding to graphs $\Gamma_1,\dots,\Gamma_n$ to an element corresponding to a graph $\Gamma$. Then we first map the tensor factors using the natural maps $H^1(\Gamma_j, \partial \Gamma_j)\to H^1(\Gamma, \partial \Gamma)$, and then multiply them. (To this end, note that if $\Gamma'\subset \Gamma$ is a subgraph, then one has a natural map $H^1(\Gamma', \partial \Gamma')\to H^1(\Gamma, \partial \Gamma)$, and contractions of a graph do not change $H^1(\Gamma, \partial \Gamma)$.)

Our modification of the Kontsevich functor allows treatment of properads $\cP$ which might have $\cP(1,1)$ non-zero (which is strictly prohibited in the original Kontsevich approach), see Steps 3 and 4 below.

%\subsubsection{\bf Lemma}
\begin{lemma}\label{2: exact functor}
$F$ is an exact functor, i.e.\ it preserves cohomology.
\end{lemma}
\begin{proof}
Since the differential preserves the underlying graph, we get
\begin{eqnarray}\label{functor1}
H_\bullet(F(s)(m,n)) = \bigoplus_{\Gamma \in \overline{\text{Gr}}(m,n)} H_\bullet \left(\left(\bigotimes_{v\in v(\Gamma)} s(Out(v),In(v)) \otimes \odot H^1(\Gamma, \partial \Gamma)\right) _{Aut(\Gamma)} \right)
\end{eqnarray}
Since the differential commutes with elements of $Aut(\Gamma)$, $Aut(\Gamma)$ is finite and $\K$ is a field of characteristic zero, by Maschke's Theorem we have
\begin{eqnarray}\label{functor2}
(\ref{functor1}) = \bigoplus_{\Gamma \in \overline{\text{Gr}}(m,n)} \left( H_\bullet \left(\bigotimes_{v\in v(\Gamma)} s(Out(v),In(v)) \otimes \odot H^1(\Gamma, \partial \Gamma)\right) \right) _{Aut(\Gamma)}
\end{eqnarray}
Applying the K\"{u}nneth formula twice, together with the fact that the differential is trivial on $H^1(\Gamma, \partial \Gamma)$  we get
\begin{eqnarray*}
(\ref{functor2}) = \bigoplus_{\Gamma \in \overline{\text{Gr}}(m,n)}  \left(\bigotimes_{v\in v(\Gamma)} H_\bullet(s(Out(v),In(v))) \otimes \odot H^1(\Gamma, \partial \Gamma)\right)  _{Aut(\Gamma)} = F(H_\bullet(s))(m,n).
\end{eqnarray*}
\end{proof}

\sip

{\em Step 2: A genus filtration}.

 Consider the genus filtration of  $(\LoB_\infty, \delta)$,
and denote by
 $(\gr\LoB_\infty, \delta^{gen})$ the associated  graded properad.   The
 differential $\delta^{gen}$ in the
 complex $\gr\LoB_\infty$ is given by the formula,
 \Beq\label{2: delta_0 in E_0 diamond}
\delta^{gen}
\resizebox{14mm}{!}{\xy
(-9,-6)*{};
(0,0)*+{_a}*\cir{}
**\dir{-};
(-5,-6)*{};
(0,0)*+{_a}*\cir{}
**\dir{-};
(9,-6)*{};
(0,0)*+{_a}*\cir{}
**\dir{-};
(5,-6)*{};
(0,0)*+{_a}*\cir{}
**\dir{-};
(0,-6)*{\ldots};
(-10,-8)*{_1};
(-6,-8)*{_2};
(10,-8)*{_n};
%
(-9,6)*{};
(0,0)*+{_a}*\cir{}
**\dir{-};
(-5,6)*{};
(0,0)*+{_a}*\cir{}
**\dir{-};
(9,6)*{};
(0,0)*+{_a}*\cir{}
**\dir{-};
(5,6)*{};
(0,0)*+{_a}*\cir{}
**\dir{-};
(0,6)*{\ldots};
(-10,8)*{_1};
(-6,8)*{_2};
(10,8)*{_m};
\endxy}=
\sum_{\substack{b,c \\ b+c=a}}\sum_{[m]=I_1\sqcup I_2\atop
[n]=J_1\sqcup J_2} \pm
\Ba{c}
%
%
%%%%%%%%%%%%%%%% two vertex graph with 1 internal edge %%%%%%%%%%
\resizebox{18mm}{!}{\xy
(0,0)*+{_b}*\cir{}="b",
(10,10)*+{_c}*\cir{}="c",
%
%%%%%%%%%% edges to b %%%%%%%%%%%%
(-9,6)*{}="1",
(-7,6)*{}="2",
(-2,6)*{}="3",
(-3.5,5)*{...},
(-4,-6)*{}="-1",
(-2,-6)*{}="-2",
(4,-6)*{}="-3",
(1,-5)*{...},
(0,-8)*{\underbrace{\ \ \ \ \ \ \ \ }},
(0,-11)*{_{J_1}},
(-6,8)*{\overbrace{ \ \ \ \ \ \ }},
(-6,11)*{_{I_1}},
%%%%%%%%%% edges to c %%%%%%%%%%%%
(6,16)*{}="1'",
(8,16)*{}="2'",
(14,16)*{}="3'",
(11,15)*{...},
(11,6)*{}="-1'",
(16,6)*{}="-2'",
(18,6)*{}="-3'",
(13.5,6)*{...},
(15,4)*{\underbrace{\ \ \ \ \ \ \ }},
(15,1)*{_{J_2}},
(10,18)*{\overbrace{ \ \ \ \ \ \ \ \ }},
(10,21)*{_{I_2}},
%
%%%%%%%%%%% internal curved edges %%%%%%%%%%%
\ar @{-} "b";"c" <0pt>
%
\ar @{-} "b";"1" <0pt>
\ar @{-} "b";"2" <0pt>
\ar @{-} "b";"3" <0pt>
\ar @{-} "b";"-1" <0pt>
\ar @{-} "b";"-2" <0pt>
\ar @{-} "b";"-3" <0pt>
%
\ar @{-} "c";"1'" <0pt>
\ar @{-} "c";"2'" <0pt>
\ar @{-} "c";"3'" <0pt>
\ar @{-} "c";"-1'" <0pt>
\ar @{-} "c";"-2'" <0pt>
\ar @{-} "c";"-3'" <0pt>
\endxy}
%%%%%%%%%%%%%%%%%%%%%%%%%%%%%%%%%%%%%%%%%%%
\Ea
\Eeq
Consider also the genus filtration of the dg properad $\cP$ and denote by
$(\gr\cP, 0)$ the associated graded. The morphism (\ref{2: nu quasi-iso
to P}) of filtered complexes
induces a sequence of morphisms of the associated graded complexes,
\Beq\label{2: morphism of 1st terms of genus spectral seq}
\nu: \gr\LoB_\infty \lon \gr\cP,
\Eeq
Thanks to the Spectral Sequences Comparison Theorem (see p.126 in \cite{We}), Theorem {\ref{2: proposition
on nu quasi-iso}} will be proven if we  show that the map $\nu$ is a quasi-isomorphism of complexes. We shall compute below
 the cohomology $H^\bu(\gr \LoB_\infty,\delta^{gen})$ which will make it evident
 that the map
 $
 \nu
 $ is a quasi-isomorphism indeed.
\bip


{\em Step 3: An auxiliary prop}.
Let us consider a properad $\cQ = F( \Omega_{\frac{1}{2}} ( \LB^{\text{!`}}_{\frac{1}{2}}))$, where $\LB_{\frac{1}{2}}$ is the $\frac{1}{2}$-prop governing Lie bialgebras and $\LB^{\text{!`}}_{\frac{1}{2}}$ its Koszul dual. Explicitly, $\cQ$ can be understood as generated by corollas as in \eqref{2: generating corollas of LoB infty} with either $a=m=n=1$ or $a=0$ and $m+n\geq 3$ subject to the relations
$$
%%%%%%%%%%%%%%%%%
\Ba{c}
\resizebox{3mm}{!}{\xy
(0,0)*+{_1}*\cir{}="b",
(0,6)*+{_1}*\cir{}="c",
%
(0,-4)*{}="-1",
(0,10)*{}="1'",
\ar @{-} "b";"c" <0pt>
%
\ar @{-} "b";"-1" <0pt>
\ar @{-} "c";"1'" <0pt>
\endxy}
\Ea=0\ \ \ , \ \ \
%%%%
\sum_{i=1}^m
\Ba{c}
\resizebox{17mm}{!}{\xy
(-9,12.5)*{^{1}},
(-4,12.5)*{^{^{i-1}}},
(0,21)*{^{^{i}}},
(5,12.5)*{^{^{i+1}}},
(9.9,12.5)*{^{^m}},
%
(-9,-1.5)*{_{_1}},
(-4.5,-1.5)*{_{_{2}}},
(4.5,-1.5)*{_{_{n-1}}},
(9.9,-1.5)*{_{_n}},
%
(-5,11)*{...},
(5,11)*{...},
(0,1)*{...},
%
(0,6)*+{_0}*\cir{}="b",
(0,15)*+{_1}*\cir{}="c",
%
(0,-4)*{}="-1",
(-9,12)*{}="1'",
(-3.5,12)*{}="2'",
(3.5,12)*{}="3'",
(9,12)*{}="4'",
(0,21)*{}="u",
%
(-9,0)*{}="-1",
(-4.5,0)*{}="-2",
(4.5,0)*{}="-3",
(9,0)*{}="-4",
%
\ar @{-} "u";"c" <0pt>
\ar @{-} "b";"c" <0pt>
\ar @{-} "b";"-1" <0pt>
\ar @{-} "b";"1'" <0pt>
\ar @{-} "b";"2'" <0pt>
\ar @{-} "b";"3'" <0pt>
\ar @{-} "b";"4'" <0pt>
%
\ar @{-} "b";"-1" <0pt>
\ar @{-} "b";"-2" <0pt>
\ar @{-} "b";"-3" <0pt>
\ar @{-} "b";"-4" <0pt>
\endxy}
\Ea
-
%%%%%%%%%%%%%%%%%%%%%%%%%%%%%
\sum_{i=1}^n
\Ba{c}
\resizebox{17mm}{!}{\xy
(-9,-13.5)*{_{_1}},
(-4,-13.5)*{_{_{i-1}}},
(0,-22)*{_{_{i}}},
(5,-13.5)*{_{_{i+1}}},
(9.9,-13.5)*{_{_n}},
%
(-9,1.5)*{_{_1}},
(-4.5,1.5)*{_{_{2}}},
(4.5,1.5)*{^{_{m-1}}},
(9.9,1.5)*{_{_m}},
%
(-5,-11)*{...},
(5,-11)*{...},
(0,-1)*{...},
%
(0,-6)*+{_0}*\cir{}="b",
(0,-15)*+{_1}*\cir{}="c",
%
(0,4)*{}="-1",
(-9,-12)*{}="1'",
(-3.5,-12)*{}="2'",
(3.5,-12)*{}="3'",
(9,-12)*{}="4'",
(0,-21)*{}="u",
%
(-9,0)*{}="-1",
(-4.5,0)*{}="-2",
(4.5,0)*{}="-3",
(9,0)*{}="-4",
%
\ar @{-} "u";"c" <0pt>
\ar @{-} "b";"c" <0pt>
\ar @{-} "b";"-1" <0pt>
\ar @{-} "b";"1'" <0pt>
\ar @{-} "b";"2'" <0pt>
\ar @{-} "b";"3'" <0pt>
\ar @{-} "b";"4'" <0pt>
%
\ar @{-} "b";"-1" <0pt>
\ar @{-} "b";"-2" <0pt>
\ar @{-} "b";"-3" <0pt>
\ar @{-} "b";"-4" <0pt>
\endxy}
\Ea
=0
$$
To see this, note that for any graph $\Gamma$, $H^1(\Gamma,\partial \Gamma)$ may be identified with the space of formal linear combinations of edges of $\Gamma$, modulo the relations that the sum of incoming edges at any vertex equals the sum of outgoing edges.
Similarly, $\odot^k H^1(\Gamma,\partial \Gamma)$ may be identified with formal linear combinations of $k$-fold (``wedge'') products of edges, modulo similar relations.
Of course, such a product of $k$ edges may be represented combinatorially by putting a marking on those $k$ edges. In our case, this marking is the corolla
$\vcenter{\vbox{\xy
(0,5)*{};
(0,0)*+{_1}*\cir{}
**\dir{-};
%
(0,-5)*{};
(0,0)*+{_1}*\cir{}
**\dir{-};
\endxy}}$,
which we may put on edges.

With the above combinatorial description of $\cQ$ we see that the map $\nu$ above factors naturally through $\cQ$, say $\nu:\gr\LoB_\infty \overset{p}\twoheadrightarrow \cQ \overset{q}\twoheadrightarrow  \gr\cP$.
Furthermore we claim that the right-hand map $q$ is a quasi-isomorphism.
First notice that  $\gr\cP = F(\LB_{\frac{1}{2}})$.
The $\frac{1}{2}$-prop $\LB_{\frac{1}{2}}$ is Koszul, i.~e., the natural projection
$\Omega_{\frac{1}{2}} ( \LB^{\text{!`}}_{\frac{1}{2}}) \twoheadrightarrow \LB_{\frac{1}{2}}$ is a quasi-isomorphism.
The result follows by applying the functor $F$ to this map and by Lemma {\ref{2: exact functor}}.
\bip

{\em Step 4: The map $p:\gr\LoB_\infty\to \cQ$ is a quasi-isomorphism}.
% TODO: verify 2
Consider a filtration of $\gr\LoB_\infty$ given, for any graph $\Ga$, by the the difference $a(\Ga)- n(\Ga)$,
where $a(\Ga)$ is the sum of all decorations of non-bivalent vertices and $n(\Ga)$ is the sum of valences of non-bivalent vertices.
On the $0$-th page of this spectral sequence the differential acts only by splitting bivalent vertices. Then Proposition $\ref{2: propos on aux graph complexes}$ tells us that the first page of this spectral sequence consists of graphs with no bivalent vertices such that every vertex is decorated by a number $a\in \mathbb Z^+$ and every edge has either a decoration $\xy(0,0)*+{_{1}}*\cir{};
\endxy$ or no decoration.
The differential acts by
\begin{equation}\label{equ:diffspecs}
\resizebox{15mm}{!}{\xy
(-9,-6)*{};
(0,0)*+{_a}*\cir{}
**\dir{-};
(-5,-6)*{};
(0,0)*+{_a}*\cir{}
**\dir{-};
(9,-6)*{};
(0,0)*+{_a}*\cir{}
**\dir{-};
(5,-6)*{_a};
(0,0)*+{a}*\cir{}
**\dir{-};
(0,-6)*{\ldots};
(-10,-8)*{_1};
(-6,-8)*{_2};
(10,-8)*{_n};
%
(-9,6)*{};
(0,0)*+{_a}*\cir{}
**\dir{-};
(-5,6)*{};
(0,0)*+{_a}*\cir{}
**\dir{-};
(9,6)*{};
(0,0)*+{_a}*\cir{}
**\dir{-};
(5,6)*{};
(0,0)*+{_a}*\cir{}
**\dir{-};
(0,6)*{\ldots};
(-10,8)*{_1};
(-6,8)*{_2};
(10,8)*{_m};
\endxy}
\lon
%%%%%%%%%%%%%%%%%%%%%%%%%%%%
\sum_{i=1}^m
\Ba{c}
\resizebox{17mm}{!}{\xy
(-9,12.5)*{^{1}},
(-4,12.5)*{^{^{i-1}}},
(0,21)*{^{^{i}}},
(5,12.5)*{^{^{i+1}}},
(9.9,12.5)*{^{^m}},
%
(-9,-1.5)*{_{_1}},
(-4.5,-1.5)*{_{_{2}}},
(4.5,-1.5)*{_{_{n-1}}},
(9.9,-1.5)*{_{_n}},
%
(-5,11)*{...},
(5,11)*{...},
(0,1)*{...},
%
(0,6)*+{_{a-1}}*\cir{}="b",
(0,15)*+{_1}*\cir{}="c",
%
(0,-4)*{}="-1",
(-9,12)*{}="1'",
(-3.5,12)*{}="2'",
(3.5,12)*{}="3'",
(9,12)*{}="4'",
(0,21)*{}="u",
%
(-9,0)*{}="-1",
(-4.5,0)*{}="-2",
(4.5,0)*{}="-3",
(9,0)*{}="-4",
%
\ar @{-} "u";"c" <0pt>
\ar @{-} "b";"c" <0pt>
\ar @{-} "b";"-1" <0pt>
\ar @{-} "b";"1'" <0pt>
\ar @{-} "b";"2'" <0pt>
\ar @{-} "b";"3'" <0pt>
\ar @{-} "b";"4'" <0pt>
%
\ar @{-} "b";"-1" <0pt>
\ar @{-} "b";"-2" <0pt>
\ar @{-} "b";"-3" <0pt>
\ar @{-} "b";"-4" <0pt>
\endxy}
\Ea
-
%%%%%%%%%%%%%%%%%%%%%%%%%%%%%
\sum_{i=1}^n
\Ba{c}
\resizebox{17mm}{!}{\xy
(-9,-13.5)*{_{_1}},
(-4,-13.5)*{_{_{i-1}}},
(0,-22)*{_{_{i}}},
(5,-13.5)*{_{_{i+1}}},
(9.9,-13.5)*{_{_n}},
%
(-9,1.5)*{_{_1}},
(-4.5,1.5)*{_{_{2}}},
(4.5,1.5)*{^{_{m-1}}},
(9.9,1.5)*{_{_m}},
%
(-5,-11)*{...},
(5,-11)*{...},
(0,-1)*{...},
%
(0,-6)*+{_{a-1}}*\cir{}="b",
(0,-15)*+{_1}*\cir{}="c",
%
(0,4)*{}="-1",
(-9,-12)*{}="1'",
(-3.5,-12)*{}="2'",
(3.5,-12)*{}="3'",
(9,-12)*{}="4'",
(0,-21)*{}="u",
%
(-9,0)*{}="-1",
(-4.5,0)*{}="-2",
(4.5,0)*{}="-3",
(9,0)*{}="-4",
%
\ar @{-} "u";"c" <0pt>
\ar @{-} "b";"c" <0pt>
\ar @{-} "b";"-1" <0pt>
\ar @{-} "b";"1'" <0pt>
\ar @{-} "b";"2'" <0pt>
\ar @{-} "b";"3'" <0pt>
\ar @{-} "b";"4'" <0pt>
%
\ar @{-} "b";"-1" <0pt>
\ar @{-} "b";"-2" <0pt>
\ar @{-} "b";"-3" <0pt>
\ar @{-} "b";"-4" <0pt>
\endxy}
\Ea
\end{equation}

The complex we obtain is precisely
$$
 \bigoplus_{\Gamma \in \overline{\text{Gr}}(m,n)} \left(\bigotimes_{v\in v(\Gamma)} \Omega_{\frac{1}{2}} ( \LB^{\text{!`}}_{\frac{1}{2}})(Out(v),In(v)) \otimes \odot C^*(\Gamma, \partial \Gamma)\right) _{Aut(\Gamma)},
$$
%$\displaystyle\bigoplus_{\Gamma\in \overline{\text{Gr}}(m,n)} \odot (C^*(\Gamma, \partial %\Gamma))_{Aut(\Gamma)}$,
where $C^*(\Gamma, \partial \Gamma)$ are the simplicial co-chains of $\Gamma$ relative to its boundary; the differential in this complex is given by the standard differential in  $C^*(\Gamma, \partial \Gamma)$.
Indeed, we may identify $C^0(\Gamma, \partial \Gamma)\cong \K[V(\Ga)]$ and $C^1(\Gamma, \partial \Gamma)\cong \K[E(\Ga)]$.
A vertex $v=\xy(0,0)*+{_{a_v}}*\cir{};
\endxy$ with weight $a_v$ corresponds to the $a_v$-th power of the cochain representing the vertex, and an edge decorated
with the symbol $\xy(0,0)*+{_{1}}*\cir{};
\endxy$ corresponds to the cochain representing the edge.
The differential $d$ on $C^1(\Gamma, \partial \Gamma)$ is the map dual to the standard boundary map $\p: C_1(\Gamma, \partial \Gamma)\cong \K[E(\Ga)]
\lon C_0(\Gamma, \partial \Gamma)\cong \K[V(\Ga)]$. It is given, on a vertex $v\in V(\Ga)$, by
$$
dv =\sum_{e_v'\in Out(v)} e_v' -
 \sum_{e_v'\in In(v)} e_v'',
$$
where $Out(v)$ is the set of edges outgoing from $v$ and $In(v)$ is the set of edges ingoing to $v$. This exactly matches the
differential \eqref{equ:diffspecs} on the first page of the spectral sequence.
As $H^0(\Ga,\p \Ga)=0$ %, using an argument similar to Lemma {\ref{2: exact functor}},
%together with the and
and since the symmetric product functor $\odot$ is exact we obtain $\cQ$ on the second page of the spectral sequence,
 $$
H^\bu(\gr\LoB_\infty)\cong
 \bigoplus_{\Gamma \in \overline{\text{Gr}}(m,n)} \left(\bigotimes_{v\in v(\Gamma)} \Omega_{\frac{1}{2}} ( \LB^{\text{!`}}_{\frac{1}{2}})(Out(v),In(v)) \otimes \odot H^1(\Gamma, \partial \Gamma)\right)_{Aut(\Gamma)}=
  F( \Omega_{\frac{1}{2}} ( \LB^{\text{!`}}_{\frac{1}{2}}))
 %\displaystyle\bigoplus_{\Gamma\in \overline{\text{Gr}}(m,n)} \odot (H^1(\Gamma, \partial %\Gamma))_{Aut(\Gamma)}
  \cong Q
$$
 thus showing that $p$ is a quasi-isomorphism. Hence so is the map $\nu$ from \eqref{2: morphism of 1st terms of genus spectral seq}, and hence Theorem \ref{2: proposition on nu quasi-iso} is shown.
\end{proof}


\subsection{Auxiliary complexes} \label{sec:extracomplexes}
Let $\cA_n$ be a quadratic algebra generated by $x_1,\dots, x_n$ with relations $x_ix_{i+1}=x_{i+1}x_i$ for $i=1,\dots, n-1$.
We denote by $D_n=\cA_n^{\text{!`}}$ the Koszul dual coalgebra. Notice that $\cA_n$ and $D_n$ are weight graded and the weight $k$ component of $D_n$, $D_n^{(k)}$ is zero if $k\geq 3$, while $D_n^{(1)}=\vecspan \{x_1,\dots, x_n\}$ and
$D_n^{(2)}=\vecspan\{u_{1,2}=x_1x_2-x_2x_1,u_{2,3}=x_2x_3-x_3x_2, \dots, u_{n-1,n}=x_{n-1}x_n-x_nx_{n-1}\}$.

\begin{proposition}\label{2: toy problem}
The algebra $\cA_n$ is Koszul. In particular, the canonical projection map
$$A_n:=\Omega(D_n)\to \cA_n$$
from the cobar construction of $D_n$ is a quasi-isomorphism.
\end{proposition}

The proof of this proposition is given in Appendix \ref{app:koszulnessproof}.

%Let $B_n$ be the co-algebra generated by $x_1,...,x_n$ and co-relations $u_{i,i+1} = x_ix_{i+1}-x_{i+1}x_i$.

\sip

Proposition {\ref{2: toy problem}} in particular implies that the homology of the $A_n$ vanishes in positive degree. The complex
$A_n$ is naturally multigraded by the amount of times each index $j$ appears on each word and the differential respects this multigrading.
We will be interested in particular in the subcomplex $A_n^{1,1,\dots,1}$ of $A_n$ that is spanned by words in $x_j$ and $u_{i,i+1}$ such that each index occurs exactly once. Since $A_n^{1,1,\dots,1}$ is a direct summand of $A_n$, its homology also vanishes in positive degree.


Let us define a Lie algebra $\caL_n = \caL ie(x_1,\dots,x_n)/[x_i,x_{i+1}]$ and a complex $L_n=\caL ie(x_1,\dots,x_n, u_{1,2},\dots u_{n-1,n})$, with $dx_i=0$ and $d(u_{i,i+1})= [x_i,x_{i+1}]$. Here $\caL ie$ stands for the free Lie algebra functor.

\begin{lemma}\label{2: Ln quasi-iso}
The projection map $L_n \twoheadrightarrow \caL_n$ is a quasi-isomorphism.
\end{lemma}
\begin{proof}
It is clear that $H^0(L_n) = \caL_n$, therefore it is enough to see that the homology of $L_n$ vanishes in positive degree.
The Poincar\' e-Birkhoff-Witt Theorem\footnote{
   Let us describe in more detail how the Poincar\'e-Birkhoff-Witt Theorem applies here. In general, this Theorem states that for any graded Lie algebra $\alg g$ there is an isomorphism (of coalgebras) $\odot(\alg g)\to \mathcal U \alg g$ between the symmetric coalgebra in $\alg g$ and the universal enveloping algebra of $\alg g$. In particular, if $\alg g$ is a free Lie algebra, then the universal enveloping algebra $\mathcal U \alg g$ is the free associative algebra in the same generators. This is precisely our situation.
}
gives us an isomorphism $$\odot(\caL  ie(x_1,\dots,x_n, u_{1,2},\dots u_{n-1,n}))=\odot (L_n) \overset{\sim}\longrightarrow \cA ss(x_1,\dots,x_n, u_{1,2},\dots u_{n-1,n})= A_n.$$
This map commutes with with the differentials, therefore we have an isomorphism in homology $H_{\bullet}(\odot (L_n))= H_{\bullet}(A_n)$. Since $\odot$ is an exact functor it commutes with taking homology and since the homology of $A_n$ vanishes in positive degree by Proposition {\ref{2: toy problem}} the result follows.
\end{proof}


Let us define $A_{n_1,\dots, n_r}$ as the  coproduct of $A_{n_1}, \dots, A_{n_r}$ in the category of associative algebras; $A_{n_1,\dots, n_k}$ consists of words in $x_{1}^1,x_2^1,\dots x_{n_1}^1, x_{1}^2,\dots, x_{n_2}^2,\dots ,x_1^r,\dots, x_{n_r}^r,
u_{1,2}^1,\dots, u_{n_1-1,n_1}^1,u_{1,2}^2,\dots , u_{n_r-1,n_r}^r.$ We define similarly $L_{n_1,\dots, n_r}$ and $\caL_{n_1,\dots, n_r}$.

\begin{lemma}
The homology of $A_{n_1,\dots, n_r}$ vanishes in positive degree.
\end{lemma}
\begin{proof}
Let $\bar A_n\subset A_n$ be the kernel of the natural augmentation such that $A_n \cong \K\oplus \bar A_n$ as complexes. Similarly, define $\bar A_{n_1,\dots, n_r}$. The complex $\bar A_{n_1,\dots, n_r}$ splits as
\[
 \bar A_{n_1,\dots, n_r}
 = \bigoplus_{k\geq 1} \bigoplus_{j_1,\dots,j_k} \bar A_{n_{j_1}}\otimes \cdots \otimes \bar A_{n_{j_r}}
\]
where the second sum runs over all strings $(j_1,\dots,j_k)\in [r]^{\times k}$ such that no adjacent indices $j_i$, $j_{i+1}$ are equal.
Since no $\bar A_{n_{j}}$ has homology in positive degrees by Proposition \ref{2: toy problem}, neither has $A_{n_1,\dots, n_r}$.
\end{proof}

\begin{lemma}
The map $L_{n_1,\dots, n_r} \twoheadrightarrow \caL_{n_1,\dots, n_r}$ is a quasi-isomorphism.
\end{lemma}
\begin{proof}
The same argument from Lemma {\ref{2: Ln quasi-iso}} holds.
\end{proof}

We define the subcomplexes $L_{n_1,\dots, n_r}^{1,\dots 1}\subset L_{n_1,\dots, n_r}$ and  $\caL_{n_1,\dots, n_r}^{1,\dots 1}\subset \caL_{n_1,\dots, n_r}$ spanned by Lie words in which each index occurs exactly once.

\begin{corollary}\label{2: auxiliary corollary}
The map $L_{n_1,\dots, n_r}^{1,\dots 1} \twoheadrightarrow \caL_{n_1,\dots, n_r}^{1,\dots 1}$ is a quasi-isomorphism.
\end{corollary}

\subsection{Main Theorem}\label{2: Theorem on Koszulness} {\em The properad $\LoB$ is
Koszul, i.e.\ the natural surjection \eqref{2: pi quasi-iso}
is a quasi-isomorphism.}


\begin{proof} The surjection  \eqref{2: pi quasi-iso} factors through the surjection
\eqref{2: nu quasi-iso to P},
$$
\pi: \LoB_\infty \stackrel{\nu}{\lon} \cP \stackrel{\rho}{\lon} \LoB.
$$
In view of Theorem {\ref{2: proposition on nu quasi-iso}}, the Main theorem is
proven once it is shown that the morphism $\rho$ is a quasi-isomorphism. The latter
statement is, in turn, proven once it is shown that the cohomology of the
non-negatively graded dg properad $\cP$ is concentrated in degree zero. For notation reasons
it is suitable to work with the dg prop, $\cP \cP$, generated by the properad $\cP$. We also
denote by $\caL ie P$ the prop governing Lie algebras and by $\caL ie CP$ the prop
governing Lie coalgebras.


\sip


%It follows from the proofs of Proposition~{\ref{2: propos on aux 1/2 prop S}} and  Theorem {\ref{2: proposition on nu quasi-iso}} (see steps
%3 and 4 there) that the
It is easy to see that the dg prop  $\cP \cP=\{\cP\cP(m,n)\}$ is isomorphic, as a graded $\mathbb{S}$-bimodule, to
the graded prop generated by a degree 1 corolla
$\begin{xy}
 <0mm,-0.55mm>*{};<0mm,-3mm>*{}**@{-},
 <0mm,0.5mm>*{};<0mm,3mm>*{}**@{-},
 <0mm,0mm>*{\bu};<0mm,0mm>*{}**@{},
 \end{xy}$\, , degree zero corollas
$
 \begin{xy}
 <0mm,-0.55mm>*{};<0mm,-2.5mm>*{}**@{-},
 <0.5mm,0.5mm>*{};<2.2mm,2.2mm>*{}**@{-},
 <-0.48mm,0.48mm>*{};<-2.2mm,2.2mm>*{}**@{-},
 <0mm,0mm>*{\circ};<0mm,0mm>*{}**@{},
 %<0mm,-0.55mm>*{};<0mm,-3.8mm>*{_1}**@{},
 <0.5mm,0.5mm>*{};<2.7mm,2.8mm>*{^{_2}}**@{},
 <-0.48mm,0.48mm>*{};<-2.7mm,2.8mm>*{^{_1}}**@{},
 \end{xy}
=-
\begin{xy}
 <0mm,-0.55mm>*{};<0mm,-2.5mm>*{}**@{-},
 <0.5mm,0.5mm>*{};<2.2mm,2.2mm>*{}**@{-},
 <-0.48mm,0.48mm>*{};<-2.2mm,2.2mm>*{}**@{-},
 <0mm,0mm>*{\circ};<0mm,0mm>*{}**@{},
 %<0mm,-0.55mm>*{};<0mm,-3.8mm>*{_1}**@{},
 <0.5mm,0.5mm>*{};<2.7mm,2.8mm>*{^{_1}}**@{},
 <-0.48mm,0.48mm>*{};<-2.7mm,2.8mm>*{^{_2}}**@{},
 \end{xy}
 $ and $\begin{xy}
 <0mm,0.66mm>*{};<0mm,3mm>*{}**@{-},
 <0.39mm,-0.39mm>*{};<2.2mm,-2.2mm>*{}**@{-},
 <-0.35mm,-0.35mm>*{};<-2.2mm,-2.2mm>*{}**@{-},
 <0mm,0mm>*{\circ};<0mm,0mm>*{}**@{},
   %<0mm,0.66mm>*{};<0mm,3.4mm>*{^1}**@{},
   <0.39mm,-0.39mm>*{};<2.9mm,-4mm>*{^{_2}}**@{},
   <-0.35mm,-0.35mm>*{};<-2.8mm,-4mm>*{^{_1}}**@{},
\end{xy}=-
\begin{xy}
 <0mm,0.66mm>*{};<0mm,3mm>*{}**@{-},
 <0.39mm,-0.39mm>*{};<2.2mm,-2.2mm>*{}**@{-},
 <-0.35mm,-0.35mm>*{};<-2.2mm,-2.2mm>*{}**@{-},
 <0mm,0mm>*{\circ};<0mm,0mm>*{}**@{},
   %<0mm,0.66mm>*{};<0mm,3.4mm>*{^1}**@{},
   <0.39mm,-0.39mm>*{};<2.9mm,-4mm>*{^{_1}}**@{},
   <-0.35mm,-0.35mm>*{};<-2.8mm,-4mm>*{^{_2}}**@{},
\end{xy}$,
 modulo the first three relations of \eqref{R for LieB}
 %relations (\ref{relations in P first half})
and the following ones,
\Beq\label{2: relations for EP}
\xy
%
(0,-1.9)*{\bu}="0",
 (0,1.9)*{\bu}="1",
(0,-5)*{}="d",
(0,5)*{}="u",
%
\ar @{-} "0";"u" <0pt>
\ar @{-} "0";"1" <0pt>
\ar @{-} "1";"d" <0pt>
\endxy=0, \ \ \ \
\xy
%
(0,-1.9)*{\circ}="0",
 (0,1.9)*{\bu}="1",
(-2.5,-5)*{}="d1",
(2.5,-5)*{}="d2",
(0,5)*{}="u",
%
\ar @{-} "1";"u" <0pt>
\ar @{-} "0";"1" <0pt>
\ar @{-} "0";"d1" <0pt>
\ar @{-} "0";"d2" <0pt>
\endxy
=0\ , \ \ \ \
\xy
%
(0,1.9)*{\circ}="0",
 (0,-1.9)*{\bu}="1",
(-2.5,5)*{}="d1",
(2.5,5)*{}="d2",
(0,-5)*{}="u",
%
\ar @{-} "1";"u" <0pt>
\ar @{-} "0";"1" <0pt>
\ar @{-} "0";"d1" <0pt>
\ar @{-} "0";"d2" <0pt>
\endxy\ =\ 0\ .
\Eeq
 The latter prop can in turn be identified with the following
collection of graded vector spaces,
\begin{equation}\label{2: basis in PP}
   W(n,m):= \bigoplus_N \left( \caL ieP(n,N) \otimes V^{\otimes N} \otimes \caL ieCP(N,m)
   \right)_{\bS_N}
 \end{equation}
where $V$ is a two-dimensional vector space $V_0 \oplus V_1$, where $V_0 = \vecspan\langle  \ \xy
%
(0,0)*{}="0",
(0,-4)*{}="d",
(0,4)*{}="u",
%
\ar @{-} "0";"u" <0pt>
\ar @{-} "0";"d" <0pt>
\endxy \ \rangle$ and
$V_1 = \vecspan\langle \xy
%
(0,0)*{\bu}="0",
(0,-4)*{}="d",
(0,4)*{}="u",
%
\ar @{-} "0";"u" <0pt>
\ar @{-} "0";"d" <0pt>
\endxy \rangle$.
The isomorphism $W(n,m)\to \cP\cP(n,m)$ is realized in the more or less obvious way, by mapping $\caL ieP(n,N)\to \cP\cP(n,N)$, $\caL ieCP(N,m)\to \cP\cP(N,m)$ and composing "in the middle" with either the identity or $\xy
%
(0,0)*{\bu}="0",
(0,-4)*{}="d",
(0,4)*{}="u",
%
\ar @{-} "0";"u" <0pt>
\ar @{-} "0";"d" <0pt>
\endxy$.
The differential on $W(n,m)$ we define to be that induced by the differential on $\cP\cP(n,m)$, given by the formula (\ref{2: differential in P}).\\

Let us consider a slightly different complex
\Beq \label{equ:V_mn}
\begin{aligned}
V_{n,m}
&=
\bigoplus_N \left( \caL ieP(n,N) \otimes V^{\otimes N} \otimes \cA ssCP(N,m)
\right)_{\bS_N}\\
&\cong  \bigoplus_N \bigoplus_{N=n_1+...+n_m} \left( \caL ieP(n,N) \otimes
V^{\otimes N}
\otimes \cA ssC(n_1) \ot \dots \ot \cA ssC(n_m)\right)_{\bS_{n_1}\times \dots \times \bS_{n_m}},
\end{aligned}
\Eeq
where  $\cA ssCP$ is the prop governing coassociative coalgebras and $\cA ssC(n_j)=\cA ssCP(n_j,1)\cong \K[\bS_{n_j}]$.

The operad $\cA ss = \cC om \circ \caL ie = \bigoplus_k (\cC om \circ \caL ie)^{(k)}$ is, as an $\bS$-module, naturally graded with respect to the arity in $\cC om $.
This decomposition induces a multigrading in $V_{n,m} = \displaystyle\bigoplus_{(k_1,\dots, k_m)} V_{n;k_1,\dots, k_m}$. It is clear that this decomposition is actually a splitting of complexes and in fact the direct summand $V_{n;1,\dots,1}$ is just $W(n,m)$, therefore to show the theorem it suffices to show that the cohomology of $V_{n,m}$ is zero in positive degree.\\


There is a natural identification $\caL ieP(n,N) \cong \left(\left(\caL ie(y_1,\cdots, y_N)\right)^{\ot n}\right)^{1,\dots,1}$ where, as before, we use the notation $1,\dots,1$ to represent the subspace spanned by tensor products of words such that each index appears exactly once.

$\left(\caL ieP(n,N) \ot \cA ssC(n_1) \ot \cdots \cA ssC(n_m)\right)_{S_{n_1}\times \dots \times S_{n_k}}$ is isomorphic to  $\caL ieP(n,N) $ but there is a more natural identification than the one above, namely the $y_j$ can be gathered by blocks of size $n_j$, according to the action of $S_{1}\times\dots\times S_{n_m}$ on $\caL ieP(n,N)$ and can be relabeled accordingly:
\begin{eqnarray*}
y_1, y_2, \dots ,y_{N} \leadsto y_1^1,\dots , y_{n_1}^1, y_1^2,\dots , y_{n_2}^2 ,\dots , y_1^m,\dots , y_{n_m}^m.
\end{eqnarray*}

Since $V = V_0 \oplus V_1$, there is a natural decomposition of
\[
V^{\ot N} = \displaystyle\bigoplus_{\epsilon} V_{\epsilon_1^1} \ot \dots V_{\epsilon_{n_1}^1}\ot V_{\epsilon_{1}^2}\ot \dots V_{\epsilon_{n_2}^2}\ot \dots  V_{\epsilon_{1}^m}\ot \dots V_{\epsilon_{n_m}^m},
\]
where $\epsilon= (\epsilon_1^1,\dots,\epsilon_{n_m}^m)$ runs through all strings of $0$'s and $1$ of length $N$.
Then,
\begin{align*}
&\left(\caL ieP(n,N) \otimes
V^{\otimes N}
\otimes \cA ssC(n_1) \ot \dots \ot \cA ssC(n_m)\right)_{\bS_{n_1}\times \dots \times \bS_{n_m}}
\\
&=
 \bigoplus_{\epsilon}\left( \caL ieP(n,N) \otimes
V_{\epsilon_1^1} \ot \dots \ot V_{\epsilon_{n_m}^m}
\otimes \cA ssC(n_1) \ot \dots \ot \cA ssC(n_m)\right)_{\bS_{n_1}\times \dots \times \bS_{n_m}}
\\
&= \bigoplus_{\epsilon}\left( \caL ieP(n,N) \otimes
V_{\epsilon_1^1} \ot \dots \ot V_{\epsilon_{n_m}^m}\right)
\end{align*}

Given a fixed a string $\epsilon$ we look at the corresponding summand in the above direct sum individually. Our goal is to realize each summand as a subspace of a product of free Lie algebras
$$\left(\left(\caL ie(x_1^1,\dots , x_{\tilde n_1}^1,\dots , x_1^m,\dots , x_{\tilde n_m}^m, u_{1,2}^1,\dots, u_{\tilde n_1-1,\tilde n_1}^1,u_{1,2}^2,\dots , u_{\tilde n_m-1,\tilde n_m}^m )\right)^{\ot n}\right)^{1,\dots,1},$$
where $\tilde n_i = n_i + \displaystyle\sum_{j=1}^{n_i} \epsilon_j^i$ and where the superscript shall indicate that each index occurs exactly once.
We assume that bases of the one-dimensional spaces $V_0$, $V_1$ have been fixed.
Then, using the bases we may identify
$$
\caL ieP(n,N) \otimes V_{\epsilon_1^1} \ot \dots \ot V_{\epsilon_{n_m}^m }= \caL ieP(n,N)
$$
Now, an element $X \in \caL ieP(n,N)$ describes a way of taking an $n$-fold product of Lie words in $N$ generators (or linear combinations thereof), say
\[
X(\underbrace{-,\dots,-}_{n \text{ "slots"}})
\]

Our map
\begin{multline}\label{equ:poly_id}
\caL ieP(n,N) \cong \caL ieP(n,N) \otimes V_{\epsilon_1^1} \ot \dots \ot V_{\epsilon_{n_m}^m }
\\
\to
\left(\left(\caL ie(x_1^1,\dots , x_{\tilde n_1}^1,\dots , x_1^m,\dots , x_{\tilde n_m}^m, u_{1,2}^1,\dots, u_{\tilde n_1-1,\tilde n_1}^1,u_{1,2}^2,\dots , u_{\tilde n_m-1,\tilde n_m}^m )\right)^{\ot n}\right)^{1,\dots,1}
\end{multline}
is then realized by sending $X\in \caL ieP(n,N)$ to $X(y_1^1,\dots, y_{n_1}^1,\dots, y_{1}^m, \dots, y_{n_m}^m)$, where
\[
y_j^i
=
\begin{cases}
x^i_{\tilde j} & \text{if $\epsilon_j^i=0$, with $\tilde j=j+\sum_{k=1}^{j-1}\epsilon_k^i$} \\
u_{\tilde j,\tilde j+1}^i & \text{if $\epsilon_j^i=1$, with same $\tilde j$}
\end{cases}.
\]

%
%If $\epsilon_k$ is zero we wish to correspond the generator of $V_{\epsilon_k}$ to an element of the form $x_j^i$ and if $\epsilon_k=1$ we wish to correspond the generator of $V_{\epsilon_k}$ to an element of the form $u_{j,j+1}^i$. We do this recursively, to determine the correct indices:\\
%
%\Bi
%\item[-] For $k=1$: $\left\{\Ba{c} \mathrm{if}\ \epsilon_1 = 0, \mathrm{then}\ V_{\epsilon_1}= \text{span } \langle x_{1}^1\rangle;\\
% \mathrm{if}\ \epsilon_1 = 1\  \mathrm{then}\ V_{\epsilon_1}= \text{span } \langle u_{1,2}^1\rangle.
%\Ea\right.$
%
%\item[-] For $k>1$:
%\Bi
%\item[]	if $\epsilon_k = 0$ then
%\Bi
%\item[]		if $V_{\epsilon_{k-1}}= \text{span } \langle x_{j}^i\rangle$ or $V_{\epsilon_{k-1}}= \text{span } \langle u_{j-1,j}^i\rangle$ and the total number of variables with upper script $i$ is still smaller then $n_i$, then $V_{\epsilon_{k}} = \text{span }\langle x_{j+1}^i\rangle$.
%\item[]
%    Otherwise $V_{\epsilon_{k}} = \text{span }\langle x_{1}^{i+1}\rangle$.
%\Ei
%
%\item[]	if $\epsilon_k = 1$ then
%\Bi
%\item[]		if $V_{\epsilon_{k-1}}= \text{span } \langle x_{j}^i\rangle$ or $V_{\epsilon_{k-1}}= \text{span } \langle u_{j-1,j}^i\rangle$ and the total number of variables with upper script $i$ is still smaller then $n_i$, then $V_{\epsilon_{k}} = \text{span } \langle u_{j+1,j+2}^i \rangle$.
%    \item[] Otherwise $V_{\epsilon_{k}} = \text{span } \langle u_{1,2}^{i+1}\rangle$.
%\Ei
%\Ei
%\Ei
%With this notation, the total space
%
%$$V_{n,m} = \bigoplus_N \bigoplus_\epsilon \bigoplus_{N=n_1+...+n_m} \left( \caL ieP(n,N) \otimes
%V_{\epsilon_1} \ot \dots \ot V_{\epsilon_N}
%\otimes \cA ssC(n_1) \ot \dots \ot \cA ssC(n_m)\right)_{\bS_{n_1}\times \dots \times \bS_{n_k}}$$
%
%can be identified with a sum of spaces of the form
%$$\left(\left(\caL ie(x_1^1,\dots , x_{\tilde n_1}^1,\dots , x_1^m,\dots , x_{\tilde n_m}^m, u_{1,2}^1,\dots, u_{\tilde n_1-1,\tilde n_1}^1,u_{1,2}^2,\dots , u_{\tilde n_m-1,\tilde n_m}^m )\right)^{\ot n}\right)^{1,\dots,1},$$
%where $\tilde n_i$ is the biggest index produced by the algorithm described above.
%
% Explicitly $\tilde n_i = n_i + \displaystyle\sum_{j=n_1+...+n_{i-1}+1}^{n_1+...+n_{i}} \epsilon_j$.

   For example, consider the following element of $\left(\caL ieP(2,5)\otimes V^{\otimes 5}\ot \cA ssC(3)\ot \cA ssC(2)\right)_{\bS_3\times \bS_2}$:
   $$
\resizebox{40mm}{!}{   \xy
(-18,0)*{}="L",
(15,0)*{}="R",
%
(-5,15)*{}="u1",
(-5,10)*{\circ}="a1",
(-9,5)*{\circ}="a2",
(-13,0)*{}="L1",
(-5,0)*{\bu}="L2",
%
(5,14)*{}="u2",
(5,9)*{\circ}="b1",
(10,0)*{\bu}="L3",
%
(-5,-12)*{}="-u1",
(-5,-7)*{\circ}="-a1",
%
(5,-11)*{}="-u2",
(5,-6)*{\circ}="-b1",
%
(25,0)*{V^{\otimes 5}},
(25,8)*{\caL ieP(2,5)},
(25,-8)*{\cA ssC(3)\ot \cA ssC(2)},
%
\ar @{-} "u1";"a1" <0pt>
\ar @{-} "a1";"a2" <0pt>
\ar @{-} "a2";"L1" <0pt>
\ar @{-} "a2";"L2" <0pt>
\ar @{-} "u2";"b1" <0pt>
\ar @{-} "b1";"L3" <0pt>
\ar @{-} "-u1";"-a1" <0pt>
\ar @{-} "L1";"-a1" <0pt>
\ar @{-} "L2";"-a1" <0pt>
\ar @{-} "b1";"-a1" <0pt>
\ar @{-} "-b1";"-u2" <0pt>
\ar @{-} "-b1";"L3" <0pt>
\ar @{-} "-b1";"a1" <0pt>
\ar @{--} "L";"R" <0pt>
\endxy}
   $$
In the picture, we understand that the two corollas in the lower half correspond to a triple and a double (co-)product in $\cA ssC(3)$ and $\cA ssC(2)$, (co-)multiplying factors from left to right.
Then the element in the picture is mapped to the expression $[[x_1^1,u^1_{2,3}],x^2_1] \otimes [x^1_4,u^2_{2,3}] $.

With the map \eqref{equ:poly_id}, the total space

$$V_{n,m} = \bigoplus_N \bigoplus_\epsilon \bigoplus_{N=n_1+...+n_m} \left( \caL ieP(n,N) \otimes
V_{\epsilon_1} \ot \dots \ot V_{\epsilon_N}
\otimes \cA ssC(n_1) \ot \dots \ot \cA ssC(n_m)\right)_{\bS_{n_1}\times \dots \times \bS_{n_k}}$$

can be seen as a sum of spaces of the form
$$\left(\left(\caL ie(x_1^1,\dots , x_{\tilde n_1}^1,\dots , x_1^m,\dots , x_{\tilde n_m}^m, u_{1,2}^1,\dots, u_{\tilde n_1-1,\tilde n_1}^1,u_{1,2}^2,\dots , u_{\tilde n_m-1,\tilde n_m}^m )\right)^{\ot n}\right)^{1,\dots,1},$$

Under this identification the differential sends the elements $u^i_{j,j+1}$ to $[x_j^i,x_{j+1}^i]$ and it is zero on the elements $x_j^i$.
Then the differential preserves the $\tilde n_i$'s therefore it preserves this direct sum.

\sip

We conclude that the complex $V_{n,m}$ splits as a sum of tensor products of complexes of the form
$L_{p_1,\dots, p_k}^{1,\dots, 1}$, so from Corollary {\ref{2: auxiliary corollary}} we obtain that its cohomology is concentrated in degree zero.
The proof of the main theorem is completed.
\end{proof}


\subsection{Remark}\label{rem:degreeshifted} In applications of the theory of involutive Lie bialgebras to string topology, contact topology and quantum $\cA ss_\infty$ algebras  one is often interested in a version of the properad $\LoB$  in which degrees of Lie and coLie operations differ by an even number,
$$
|[\ ,\ ]| - |\vartriangle|=2d,\ \ \ \ d\in \N.
$$
The arguments proving Koszulness of $\LoB$ work also for such degree shifted versions of $\LoB$.
%(which we denote by the same symbol $\LoB$ from now on).
The same remark applies to the Koszul dual properads below.

One may also consider versions of the properad $\LieBi$ where the Lie bracket and cobracket have degrees differing by an odd number, and have opposite symmetry. However, in this case the involutivity is trivially satisfied (by symmetry) and does not pose an additional relation. The Koszulness of the corresponding properad is hence much simpler to show, analogously to the Koszulness of $\LieBi$.

\subsection{Properads of Frobenius algebras}
The properad of non-unital Frobenius algebras $\cF rob_d$ in dimension $d$ is the properad generated by operations
$
 \begin{xy}
 <0mm,-0.55mm>*{};<0mm,-2.5mm>*{}**@{-},
 <0.5mm,0.5mm>*{};<2.2mm,2.2mm>*{}**@{-},
 <-0.48mm,0.48mm>*{};<-2.2mm,2.2mm>*{}**@{-},
 <0mm,0mm>*{\circ};<0mm,0mm>*{}**@{},
 %<0mm,-0.55mm>*{};<0mm,-3.8mm>*{_1}**@{},
 <0.5mm,0.5mm>*{};<2.7mm,2.8mm>*{^{_2}}**@{},
 <-0.48mm,0.48mm>*{};<-2.7mm,2.8mm>*{^{_1}}**@{},
 \end{xy}
=
(-1)^d
\begin{xy}
 <0mm,-0.55mm>*{};<0mm,-2.5mm>*{}**@{-},
 <0.5mm,0.5mm>*{};<2.2mm,2.2mm>*{}**@{-},
 <-0.48mm,0.48mm>*{};<-2.2mm,2.2mm>*{}**@{-},
 <0mm,0mm>*{\circ};<0mm,0mm>*{}**@{},
 %<0mm,-0.55mm>*{};<0mm,-3.8mm>*{_1}**@{},
 <0.5mm,0.5mm>*{};<2.7mm,2.8mm>*{^{_1}}**@{},
 <-0.48mm,0.48mm>*{};<-2.7mm,2.8mm>*{^{_2}}**@{},
 \end{xy}
 $ (graded co-commutative comultiplication) of degree $d$
 and
 $
 \begin{xy}
 <0mm,0.66mm>*{};<0mm,3mm>*{}**@{-},
 <0.39mm,-0.39mm>*{};<2.2mm,-2.2mm>*{}**@{-},
 <-0.35mm,-0.35mm>*{};<-2.2mm,-2.2mm>*{}**@{-},
 <0mm,0mm>*{\circ};<0mm,0mm>*{}**@{},
   %<0mm,0.66mm>*{};<0mm,3.4mm>*{^1}**@{},
   <0.39mm,-0.39mm>*{};<2.9mm,-4mm>*{^{_2}}**@{},
   <-0.35mm,-0.35mm>*{};<-2.8mm,-4mm>*{^{_1}}**@{},
\end{xy}=
\begin{xy}
 <0mm,0.66mm>*{};<0mm,3mm>*{}**@{-},
 <0.39mm,-0.39mm>*{};<2.2mm,-2.2mm>*{}**@{-},
 <-0.35mm,-0.35mm>*{};<-2.2mm,-2.2mm>*{}**@{-},
 <0mm,0mm>*{\circ};<0mm,0mm>*{}**@{},
   %<0mm,0.66mm>*{};<0mm,3.4mm>*{^1}**@{},
   <0.39mm,-0.39mm>*{};<2.9mm,-4mm>*{^{_1}}**@{},
   <-0.35mm,-0.35mm>*{};<-2.8mm,-4mm>*{^{_2}}**@{},
\end{xy}
$ (graded commutative multiplication) of degree 0,
modulo the ideal generated by the following relations,
\begin{equation}\label{equ:Frobrelations}
\Ba{c}
\begin{xy}
 <0mm,0mm>*{\circ};<0mm,0mm>*{}**@{},
 <0mm,-0.49mm>*{};<0mm,-3.0mm>*{}**@{-},
 <0.49mm,0.49mm>*{};<1.9mm,1.9mm>*{}**@{-},
 <-0.5mm,0.5mm>*{};<-1.9mm,1.9mm>*{}**@{-},
 <-2.3mm,2.3mm>*{\circ};<-2.3mm,2.3mm>*{}**@{},
 <-1.8mm,2.8mm>*{};<0mm,4.9mm>*{}**@{-},
 <-2.8mm,2.9mm>*{};<-4.6mm,4.9mm>*{}**@{-},
   <0.49mm,0.49mm>*{};<2.7mm,2.3mm>*{^3}**@{},
   <-1.8mm,2.8mm>*{};<0.4mm,5.3mm>*{^2}**@{},
   <-2.8mm,2.9mm>*{};<-5.1mm,5.3mm>*{^1}**@{},
 \end{xy}\Ea
\ = \
\Ba{c}
\begin{xy}
 <0mm,0mm>*{\circ};<0mm,0mm>*{}**@{},
 <0mm,-0.49mm>*{};<0mm,-3.0mm>*{}**@{-},
 <0.49mm,0.49mm>*{};<1.9mm,1.9mm>*{}**@{-},
 <-0.5mm,0.5mm>*{};<-1.9mm,1.9mm>*{}**@{-},
 <2.3mm,2.3mm>*{\circ};<-2.3mm,2.3mm>*{}**@{},
 <1.8mm,2.8mm>*{};<0mm,4.9mm>*{}**@{-},
 <2.8mm,2.9mm>*{};<4.6mm,4.9mm>*{}**@{-},
   <0.49mm,0.49mm>*{};<-2.7mm,2.3mm>*{^1}**@{},
   <-1.8mm,2.8mm>*{};<0mm,5.3mm>*{^2}**@{},
   <-2.8mm,2.9mm>*{};<5.1mm,5.3mm>*{^3}**@{},
 \end{xy}\Ea, \ \ \ \ \
%%%%%%%%%%%%%%%%%%%%%%%%%%%%%%%%%%%%%%
 \Ba{c}\begin{xy}
 <0mm,0mm>*{\circ};<0mm,0mm>*{}**@{},
 <0mm,0.69mm>*{};<0mm,3.0mm>*{}**@{-},
 <0.39mm,-0.39mm>*{};<2.4mm,-2.4mm>*{}**@{-},
 <-0.35mm,-0.35mm>*{};<-1.9mm,-1.9mm>*{}**@{-},
 <-2.4mm,-2.4mm>*{\circ};<-2.4mm,-2.4mm>*{}**@{},
 <-2.0mm,-2.8mm>*{};<0mm,-4.9mm>*{}**@{-},
 <-2.8mm,-2.9mm>*{};<-4.7mm,-4.9mm>*{}**@{-},
    <0.39mm,-0.39mm>*{};<3.3mm,-4.0mm>*{^3}**@{},
    <-2.0mm,-2.8mm>*{};<0.5mm,-6.7mm>*{^2}**@{},
    <-2.8mm,-2.9mm>*{};<-5.2mm,-6.7mm>*{^1}**@{},
 \end{xy}\Ea
\ = \
 \Ba{c}\begin{xy}
 <0mm,0mm>*{\circ};<0mm,0mm>*{}**@{},
 <0mm,0.69mm>*{};<0mm,3.0mm>*{}**@{-},
 <0.39mm,-0.39mm>*{};<1.9mm,-1.9mm>*{}**@{-},
 <-0.35mm,-0.35mm>*{};<-1.9mm,-1.9mm>*{}**@{-},
 <2.4mm,-2.4mm>*{\circ};<-2.4mm,-2.4mm>*{}**@{},
 <2.0mm,-2.8mm>*{};<0mm,-4.9mm>*{}**@{-},
 <2.8mm,-2.9mm>*{};<4.7mm,-4.9mm>*{}**@{-},
    <0.39mm,-0.39mm>*{};<-3mm,-4.0mm>*{^1}**@{},
    <-2.0mm,-2.8mm>*{};<0mm,-6.7mm>*{^2}**@{},
    <-2.8mm,-2.9mm>*{};<5.2mm,-6.7mm>*{^3}**@{},
 \end{xy}\Ea,\ \ \ \ \ \
 %
 \begin{xy}
 <0mm,2.47mm>*{};<0mm,0.12mm>*{}**@{-},
 <0.5mm,3.5mm>*{};<2.2mm,5.2mm>*{}**@{-},
 <-0.48mm,3.48mm>*{};<-2.2mm,5.2mm>*{}**@{-},
 <0mm,3mm>*{\circ};<0mm,3mm>*{}**@{},
  <0mm,-0.8mm>*{\circ};<0mm,-0.8mm>*{}**@{},
<-0.39mm,-1.2mm>*{};<-2.2mm,-3.5mm>*{}**@{-},
 <0.39mm,-1.2mm>*{};<2.2mm,-3.5mm>*{}**@{-},
     <0.5mm,3.5mm>*{};<2.8mm,5.7mm>*{^2}**@{},
     <-0.48mm,3.48mm>*{};<-2.8mm,5.7mm>*{^1}**@{},
   <0mm,-0.8mm>*{};<-2.7mm,-5.2mm>*{^1}**@{},
   <0mm,-0.8mm>*{};<2.7mm,-5.2mm>*{^2}**@{},
\end{xy}
\  = \
\begin{xy}
 <0mm,-1.3mm>*{};<0mm,-3.5mm>*{}**@{-},
 <0.38mm,-0.2mm>*{};<2.0mm,2.0mm>*{}**@{-},
 <-0.38mm,-0.2mm>*{};<-2.2mm,2.2mm>*{}**@{-},
<0mm,-0.8mm>*{\circ};<0mm,0.8mm>*{}**@{},
 <2.4mm,2.4mm>*{\circ};<2.4mm,2.4mm>*{}**@{},
 <2.77mm,2.0mm>*{};<4.4mm,-0.8mm>*{}**@{-},
 <2.4mm,3mm>*{};<2.4mm,5.2mm>*{}**@{-},
     <0mm,-1.3mm>*{};<0mm,-5.3mm>*{^1}**@{},
     <2.5mm,2.3mm>*{};<5.1mm,-2.6mm>*{^2}**@{},
    <2.4mm,2.5mm>*{};<2.4mm,5.7mm>*{^2}**@{},
    <-0.38mm,-0.2mm>*{};<-2.8mm,2.5mm>*{^1}**@{},
    \end{xy}.
\end{equation}

For the purposes of this paper we will define the properad of non-unital Frobenius algebras to be
\[
 \cF rob :=\cF rob_2.
\]
For example, the cohomology $H(\Sigma)$ of any closed Riemann surface $\Sigma$ is a Frobenius algebra in this sense.
Comparing with section \ref{sec:explicit Koszul dual of Lob} we see that the properad $\cF rob$ is isomorphic to the Koszul dual properad of $\LoB$, up to a degree shift
\[
 \LoB^\Koz \cong \cF rob^*\{1\}.
\]

By Koszul duality theory of properads \cite{Va}, one hence obtains from Theorem {\ref{2: Theorem on Koszulness}} the following result.

 \begin{corollary}\label{2: corollary on Frob}
 The properad of non-unital (symmetric) Frobenius algebras $\cF rob$ is Koszul.
 \end{corollary}

 By adding the additional relation
 \[
 \vcenter{\vbox{
 \xy
 (0,0)*{\circ}="a",
(0,6)*{\circ}="b",
(3,3)*{}="c",
(-3,3)*{}="d",
 (0,9)*{}="b'",
(0,-3)*{}="a'",
%"a";"c"**\dir{-};
\ar@{-} "a";"c" <0pt>
\ar @{-} "a";"d" <0pt>
\ar @{-} "a";"a'" <0pt>
\ar @{-} "b";"c" <0pt>
\ar @{-} "b";"d" <0pt>
\ar @{-} "b";"b'" <0pt>
\endxy }}
=0
 \]
(which is automatic for $d$ odd) to the presentation of $\Frob_d$ we obtain the properad(s) of involutive Frobenius algebras $\invFrob_d$. They are Koszul dual to the operads governing degree shifted Lie bialgebras (cf.\ Remark \ref{rem:degreeshifted}), and in particular $\LieBi^\Koz\cong \invcoFrobtwo\{1\}$. It then follows from the Koszulness of $\LieBi$ (and its degree shifted relatives) that the properads $\invFrob_d$ are Koszul, as noted in \cite{JF, JF2}.

 \sip

The properad $uc\cF rob$ of unital-counital Frobenius algebras is, by definition, a quotient of the free properad
generated by degree zero corollas $\Ba{c}
\xy
 <0mm,-2mm>*{\circ};<0mm,2mm>*{}**@{-},
 \endxy\Ea
$ (unit),
 $\Ba{c}
\xy
 <0mm,2mm>*{\circ};<0mm,-2mm>*{}**@{-},
 \endxy\Ea
$ (counit),
$
 \begin{xy}
 <0mm,-0.55mm>*{};<0mm,-2.5mm>*{}**@{-},
 <0.5mm,0.5mm>*{};<2.2mm,2.2mm>*{}**@{-},
 <-0.48mm,0.48mm>*{};<-2.2mm,2.2mm>*{}**@{-},
 <0mm,0mm>*{\circ};<0mm,0mm>*{}**@{},
 %<0mm,-0.55mm>*{};<0mm,-3.8mm>*{_1}**@{},
 <0.5mm,0.5mm>*{};<2.7mm,2.8mm>*{^{_2}}**@{},
 <-0.48mm,0.48mm>*{};<-2.7mm,2.8mm>*{^{_1}}**@{},
 \end{xy}
=
\begin{xy}
 <0mm,-0.55mm>*{};<0mm,-2.5mm>*{}**@{-},
 <0.5mm,0.5mm>*{};<2.2mm,2.2mm>*{}**@{-},
 <-0.48mm,0.48mm>*{};<-2.2mm,2.2mm>*{}**@{-},
 <0mm,0mm>*{\circ};<0mm,0mm>*{}**@{},
 %<0mm,-0.55mm>*{};<0mm,-3.8mm>*{_1}**@{},
 <0.5mm,0.5mm>*{};<2.7mm,2.8mm>*{^{_1}}**@{},
 <-0.48mm,0.48mm>*{};<-2.7mm,2.8mm>*{^{_2}}**@{},
 \end{xy}
 $ (graded co-commutative comultiplication)
 and
 $
 \begin{xy}
 <0mm,0.66mm>*{};<0mm,3mm>*{}**@{-},
 <0.39mm,-0.39mm>*{};<2.2mm,-2.2mm>*{}**@{-},
 <-0.35mm,-0.35mm>*{};<-2.2mm,-2.2mm>*{}**@{-},
 <0mm,0mm>*{\circ};<0mm,0mm>*{}**@{},
   %<0mm,0.66mm>*{};<0mm,3.4mm>*{^1}**@{},
   <0.39mm,-0.39mm>*{};<2.9mm,-4mm>*{^{_2}}**@{},
   <-0.35mm,-0.35mm>*{};<-2.8mm,-4mm>*{^{_1}}**@{},
\end{xy}=
\begin{xy}
 <0mm,0.66mm>*{};<0mm,3mm>*{}**@{-},
 <0.39mm,-0.39mm>*{};<2.2mm,-2.2mm>*{}**@{-},
 <-0.35mm,-0.35mm>*{};<-2.2mm,-2.2mm>*{}**@{-},
 <0mm,0mm>*{\circ};<0mm,0mm>*{}**@{},
   %<0mm,0.66mm>*{};<0mm,3.4mm>*{^1}**@{},
   <0.39mm,-0.39mm>*{};<2.9mm,-4mm>*{^{_1}}**@{},
   <-0.35mm,-0.35mm>*{};<-2.8mm,-4mm>*{^{_2}}**@{},
\end{xy}
$ (graded commutative multiplication)
modulo the ideal generated by the relations \eqref{equ:Frobrelations}
% $$
% \Ba{c}
% \begin{xy}
%  <0mm,0mm>*{\circ};<0mm,0mm>*{}**@{},
%  <0mm,-0.49mm>*{};<0mm,-3.0mm>*{}**@{-},
%  <0.49mm,0.49mm>*{};<1.9mm,1.9mm>*{}**@{-},
%  <-0.5mm,0.5mm>*{};<-1.9mm,1.9mm>*{}**@{-},
%  <-2.3mm,2.3mm>*{\circ};<-2.3mm,2.3mm>*{}**@{},
%  <-1.8mm,2.8mm>*{};<0mm,4.9mm>*{}**@{-},
%  <-2.8mm,2.9mm>*{};<-4.6mm,4.9mm>*{}**@{-},
%    <0.49mm,0.49mm>*{};<2.7mm,2.3mm>*{^3}**@{},
%    <-1.8mm,2.8mm>*{};<0.4mm,5.3mm>*{^2}**@{},
%    <-2.8mm,2.9mm>*{};<-5.1mm,5.3mm>*{^1}**@{},
%  \end{xy}\Ea
% \ = \
% \Ba{c}
% \begin{xy}
%  <0mm,0mm>*{\circ};<0mm,0mm>*{}**@{},
%  <0mm,-0.49mm>*{};<0mm,-3.0mm>*{}**@{-},
%  <0.49mm,0.49mm>*{};<1.9mm,1.9mm>*{}**@{-},
%  <-0.5mm,0.5mm>*{};<-1.9mm,1.9mm>*{}**@{-},
%  <2.3mm,2.3mm>*{\circ};<-2.3mm,2.3mm>*{}**@{},
%  <1.8mm,2.8mm>*{};<0mm,4.9mm>*{}**@{-},
%  <2.8mm,2.9mm>*{};<4.6mm,4.9mm>*{}**@{-},
%    <0.49mm,0.49mm>*{};<-2.7mm,2.3mm>*{^1}**@{},
%    <-1.8mm,2.8mm>*{};<0mm,5.3mm>*{^2}**@{},
%    <-2.8mm,2.9mm>*{};<5.1mm,5.3mm>*{^3}**@{},
%  \end{xy}\Ea, \ \ \ \ \
% %%%%%%%%%%%%%%%%%%%%%%%%%%%%%%%%%%%%%%
%  \Ba{c}\begin{xy}
%  <0mm,0mm>*{\circ};<0mm,0mm>*{}**@{},
%  <0mm,0.69mm>*{};<0mm,3.0mm>*{}**@{-},
%  <0.39mm,-0.39mm>*{};<2.4mm,-2.4mm>*{}**@{-},
%  <-0.35mm,-0.35mm>*{};<-1.9mm,-1.9mm>*{}**@{-},
%  <-2.4mm,-2.4mm>*{\circ};<-2.4mm,-2.4mm>*{}**@{},
%  <-2.0mm,-2.8mm>*{};<0mm,-4.9mm>*{}**@{-},
%  <-2.8mm,-2.9mm>*{};<-4.7mm,-4.9mm>*{}**@{-},
%     <0.39mm,-0.39mm>*{};<3.3mm,-4.0mm>*{^3}**@{},
%     <-2.0mm,-2.8mm>*{};<0.5mm,-6.7mm>*{^2}**@{},
%     <-2.8mm,-2.9mm>*{};<-5.2mm,-6.7mm>*{^1}**@{},
%  \end{xy}\Ea
% \ = \
%  \Ba{c}\begin{xy}
%  <0mm,0mm>*{\circ};<0mm,0mm>*{}**@{},
%  <0mm,0.69mm>*{};<0mm,3.0mm>*{}**@{-},
%  <0.39mm,-0.39mm>*{};<2.4mm,-2.4mm>*{}**@{-},
%  <-0.35mm,-0.35mm>*{};<-1.9mm,-1.9mm>*{}**@{-},
%  <2.4mm,-2.4mm>*{\circ};<-2.4mm,-2.4mm>*{}**@{},
%  <2.0mm,-2.8mm>*{};<0mm,-4.9mm>*{}**@{-},
%  <2.8mm,-2.9mm>*{};<4.7mm,-4.9mm>*{}**@{-},
%     <0.39mm,-0.39mm>*{};<-3mm,-4.0mm>*{^1}**@{},
%     <-2.0mm,-2.8mm>*{};<0mm,-6.7mm>*{^2}**@{},
%     <-2.8mm,-2.9mm>*{};<5.2mm,-6.7mm>*{^3}**@{},
%  \end{xy}\Ea,\ \ \ \ \ \
%  %
%  \begin{xy}
%  <0mm,2.47mm>*{};<0mm,0.12mm>*{}**@{-},
%  <0.5mm,3.5mm>*{};<2.2mm,5.2mm>*{}**@{-},
%  <-0.48mm,3.48mm>*{};<-2.2mm,5.2mm>*{}**@{-},
%  <0mm,3mm>*{\circ};<0mm,3mm>*{}**@{},
%   <0mm,-0.8mm>*{\circ};<0mm,-0.8mm>*{}**@{},
% <-0.39mm,-1.2mm>*{};<-2.2mm,-3.5mm>*{}**@{-},
%  <0.39mm,-1.2mm>*{};<2.2mm,-3.5mm>*{}**@{-},
%      <0.5mm,3.5mm>*{};<2.8mm,5.7mm>*{^2}**@{},
%      <-0.48mm,3.48mm>*{};<-2.8mm,5.7mm>*{^1}**@{},
%    <0mm,-0.8mm>*{};<-2.7mm,-5.2mm>*{^1}**@{},
%    <0mm,-0.8mm>*{};<2.7mm,-5.2mm>*{^2}**@{},
% \end{xy}
% \  = \
% \begin{xy}
%  <0mm,-1.3mm>*{};<0mm,-3.5mm>*{}**@{-},
%  <0.38mm,-0.2mm>*{};<2.0mm,2.0mm>*{}**@{-},
%  <-0.38mm,-0.2mm>*{};<-2.2mm,2.2mm>*{}**@{-},
% <0mm,-0.8mm>*{\circ};<0mm,0.8mm>*{}**@{},
%  <2.4mm,2.4mm>*{\circ};<2.4mm,2.4mm>*{}**@{},
%  <2.77mm,2.0mm>*{};<4.4mm,-0.8mm>*{}**@{-},
%  <2.4mm,3mm>*{};<2.4mm,5.2mm>*{}**@{-},
%      <0mm,-1.3mm>*{};<0mm,-5.3mm>*{^1}**@{},
%      <2.5mm,2.3mm>*{};<5.1mm,-2.6mm>*{^2}**@{},
%     <2.4mm,2.5mm>*{};<2.4mm,5.7mm>*{^2}**@{},
%     <-0.38mm,-0.2mm>*{};<-2.8mm,2.5mm>*{^1}**@{},
%     \end{xy}.
% $$
and the additional relations
\Beq\label{2: uFrob relations}
\begin{xy}
<0mm,-0.55mm>*{};<0mm,-2.5mm>*{}**@{-},
<0.5mm,0.5mm>*{};<2.2mm,2.2mm>*{}**@{-},
<-0.48mm,0.48mm>*{};<-2.5mm,2.5mm>*{}**@{-},
<0mm,0mm>*{\circ};
<-2.78mm,2.78mm>*{\circ};
 \end{xy} \ - \
 \begin{xy}
<0mm,-2.2mm>*{};<0mm,2.2mm>*{}**@{-},
 \end{xy}\ =0 \ \ \ , \ \ \
 \begin{xy}
<0mm,0.55mm>*{};<0mm,2.5mm>*{}**@{-},
<0.5mm,-0.5mm>*{};<2.2mm,-2.2mm>*{}**@{-},
<-0.48mm,-0.48mm>*{};<-2.5mm,-2.5mm>*{}**@{-},
<0mm,0mm>*{\circ};
<-2.78mm,-2.78mm>*{\circ};
 \end{xy} \ - \
 \begin{xy}
<0mm,-2.2mm>*{};<0mm,2.2mm>*{}**@{-}
 \end{xy}\ =0 \ \ \ .
\Eeq
where the vertical line\ $\begin{xy}
<0mm,-2.2mm>*{};<0mm,2.2mm>*{}**@{-}
 \end{xy}$ \ stands for the unit in the properad $\cF rob$.
Similarly one defines a properad $u\cF rob$ of unital Frobenius algebras, and a properad
$c\cF rob$ of counital algebras. Clearly, $u\cF rob$ and $c\cF rob$ are subproperads of $uc\cF rob$.



 \begin{theorem}
 The properads\ \, $u\cF rob$, $c\cF rob$ and $uc\cF rob$ are Koszul.
 \end{theorem}
 \begin{proof} By curved Koszul duality theory \cite{HM}, it is enough to prove Koszulness
 of the associated quadratic properads, $qu\cF rob$, $qc\cF rob$ and $quc\cF rob$,
obtained from $u\cF rob$, $c\cF rob$ and, respectively, $uc\cF rob$ by replacing  inhomogeneous relations
(\ref{2: uFrob relations}) by the following ones \cite{HM},
$$
\begin{xy}
<0mm,-0.55mm>*{};<0mm,-2.5mm>*{}**@{-},
<0.5mm,0.5mm>*{};<2.2mm,2.2mm>*{}**@{-},
<-0.48mm,0.48mm>*{};<-2.5mm,2.5mm>*{}**@{-},
<0mm,0mm>*{\circ};
<-2.78mm,2.78mm>*{\circ};
 \end{xy} \ = \ 0 \ \ \ , \ \ \
 \begin{xy}
<0mm,0.55mm>*{};<0mm,2.5mm>*{}**@{-},
<0.5mm,-0.5mm>*{};<2.2mm,-2.2mm>*{}**@{-},
<-0.48mm,-0.48mm>*{};<-2.5mm,-2.5mm>*{}**@{-},
<0mm,0mm>*{\circ};
<-2.78mm,-2.78mm>*{\circ};
 \end{xy} \ = \ 0
$$
so that we have  decompositions into direct sums of $\bS$-bimodules,
$$
qu\cF rob= \mathrm{span}\left\langle \hspace{-1.0mm} \Ba{c}
\resizebox{7mm}{!}{\xy
(0,-5)*{\circ};
(0,0)*+{a}*\cir{}
**\dir{-};
(0,6)*{_{co\cC om}};
(0,0)*+{a}*\cir{}
**\dir{-};
\endxy}
\Ea \hspace{-1.0mm} \right\rangle \  \oplus\  \cF rob
 \ \ \ \ , \ \ \ \
 qc\cF rob= \mathrm{span}\left\langle \hspace{-1.0mm}\Ba{c}
\resizebox{5mm}{!}{\xy
(0,5)*{\circ};
(0,0)*+{_a}*\cir{}
**\dir{-};
(0,-6)*{_{\cC om}};
(0,0)*+{_a}*\cir{}
**\dir{-};
\endxy}
\Ea  \hspace{-1.0mm} \right\rangle\ \oplus\   \cF rob
$$
and
$$
 quc\cF rob=\mathrm{span}\left\langle \hspace{-1.0mm} \Ba{c}
\resizebox{3.5mm}{!}{\xy
(0,-5)*{\circ};
(0,0)*+{_a}*\cir{}
**\dir{-};
(0,5)*{\circ};
(0,0)*+{_a}*\cir{}
**\dir{-};
\endxy}
\Ea \hspace{-1.0mm} \right\rangle \
 \oplus \
  \mathrm{span}\left\langle \hspace{-1.0mm} \Ba{c}
\resizebox{7mm}{!}{\xy
(0,-5)*{\circ};
(0,0)*+{_a}*\cir{}
**\dir{-};
(0,6)*{_{co\cC om}};
(0,0)*+{_a}*\cir{}
**\dir{-};
\endxy}
\Ea \hspace{-1.0mm} \right\rangle \
 \oplus \ \mathrm{span}\left\langle \hspace{-1.0mm}\Ba{c}
\resizebox{5mm}{!}{\xy
(0,5)*{\circ};
(0,0)*+{_a}*\cir{}
**\dir{-};
(0,-6)*{_{\cC om}};
(0,0)*+{_a}*\cir{}
**\dir{-};
\endxy}
\Ea  \hspace{-1.0mm} \right\rangle\ \ \oplus \ \
 \cF rob.
$$
where $\resizebox{4mm}{!}{
\xy
(0,-4)*{};
(0,0)*+{_a}*\cir{}
**\dir{-};
(0,4)*{};
(0,0)*+{_a}*\cir{}
**\dir{-};
\endxy}
$ stands for the graph given in (\ref{2: a weight in LieB-Koszul}).

\sip

Consider, for example, the properad  $qc\cF rob$ (proofs of Koszulness of properads $qu\cF rob$ and $quc\cF rob$ can be given by a similar argument).
 %((Koszulness of $qu\cF rob$ and $quc\cF rob$ can be established by a completely analogous %argument).
 Its Koszul dual properad $qc\cF rob^!=:qc \LoB$
is generated by the properads $\LoB$ and   $\left\langle\hspace{-1.8mm}  \Ba{c}\xy
 <0mm,2mm>*{\circ};<0mm,-2mm>*{}**@{-},
 \endxy\Ea \hspace{-1.8mm} \right\rangle$  modulo the following relation,
 $$
  \begin{xy}
<0mm,0.55mm>*{};<0mm,3.5mm>*{\circ}**@{-},
<0.5mm,-0.5mm>*{};<2.2mm,-2.2mm>*{}**@{-},
<-0.48mm,-0.48mm>*{};<-2.5mm,-2.5mm>*{}**@{-},
<0mm,0mm>*{\circ};
<-2.78mm,-2.78mm>*{};
 \end{xy} \ = \ 0.
 $$
We have,
$$
qc\LoB^\Koz= (qc\cF rob)^*\{1\} \cong
 \mathrm{span}\left\langle \hspace{-1.0mm}\Ba{c}
\resizebox{4mm}{!}{\xy
(0,5)*{\circ};
(0,0)*+{_a}*\cir{}
**\dir{-};
(0,-6)*{_{\caL ie^\Koz}};
(0,0)*+{_a}*\cir{}
**\dir{-};
\endxy}
\Ea  \hspace{-1.0mm} \right\rangle\
\ \oplus \ \LoB^\Koz  ,
 $$
 where $(\cdots)^*$ denotes the genus graded dual.
%To prove its Koszulness is the same as to prove Koszulness of the Koszul dual properad $qc\LoB$.
It will suffice to show that the properad $qc\LoB$ is Koszul. To this end consider the dg properad $\Omega(qc\LoB^\Koz)$ which is a free properad generated by corollas
 (\ref{2: generating corollas of LoB infty}) and the following ones,
\Beq\label{2: 0-generators of qucLoB}
\Ba{c}
\resizebox{10mm}{!}{\xy
(-7,-8.5)*{_{_1}},
(-3.5,-8.5)*{_{_2}},
(7.5,-8.5)*{_{_n}},
%
(0,0)*+{_a}*\cir{}="o",
(-7,-7)*{}="d1",
(-3.5,-7)*{}="d2",
(1.6,-5.5)*{...},
(7,-7)*{}="d4",
%
\ar @{-} "o";"d1" <0pt>
\ar @{-} "o";"d2" <0pt>
\ar @{-} "o";"d4" <0pt>
\endxy}
\Ea=(-1)^\sigma
\Ba{c}
\resizebox{14mm}{!}{\xy
(-8.5,-8.5)*{_{_{\sigma(1)}}},
(-3,-8.5)*{_{_{\sigma(2)}}},
(7.9,-8.5)*{_{_{\sigma(n)}}},
%
(0,0)*+{_a}*\cir{}="o",
(-7,-7)*{}="d1",
(-3.5,-7)*{}="d2",
(1.6,-5.5)*{...},
(7,-7)*{}="d4",
%
\ar @{-} "o";"d1" <0pt>
\ar @{-} "o";"d2" <0pt>
\ar @{-} "o";"d4" <0pt>
\endxy}
\Ea
\Eeq
where $a\geq 0$, $n\geq 1$ and $\sigma\in \bS_n$ is an arbitrary permutation. The differential is given
on corollas  (\ref{2: generating corollas of LoB infty}) by the standard formula (\ref{2: d on Lie inv infty}) and on $(0,n)$-generators by
$$
d\Ba{c}
\resizebox{10mm}{!}{\xy
(-7,-8.5)*{_{_1}},
(-3.5,-8.5)*{_{_2}},
(7.5,-8.5)*{_{_n}},
%
(0,0)*+{_a}*\cir{}="o",
(-7,-7)*{}="d1",
(-3.5,-7)*{}="d2",
(1.6,-5.5)*{...},
(7,-7)*{}="d4",
%
\ar @{-} "o";"d1" <0pt>
\ar @{-} "o";"d2" <0pt>
\ar @{-} "o";"d4" <0pt>
\endxy}
\Ea
=
\sum_{a=b+c+l-1}\sum_{
[n]=J_1\sqcup J_2\atop \# J_1\geq 1} \pm \hspace{-6mm}
\Ba{c}
%
%
%%%%%%%%%%%%%%%% two vertex graph with l internal edges %%%%%%%%%%
\resizebox{18mm}{!}{\xy
(0,0)*+{_b}*\cir{}="b",
(10,10)*+{_c}*\cir{}="c",
%
%%%%%%%%%% edges to b %%%%%%%%%%%%
(-9,6)*{}="1",
(-7,6)*{}="2",
(-2,6)*{}="3",
(-4,-6)*{}="-1",
(-2,-6)*{}="-2",
(4,-6)*{}="-3",
(1,-5)*{...},
(0,-8)*{\underbrace{\ \ \ \ \ \ \ \ }},
(0,-11)*{_{J_1}},
%%%%%%%%%% edges to c %%%%%%%%%%%%
(6,16)*{}="1'",
(8,16)*{}="2'",
(14,16)*{}="3'",
(11,6)*{}="-1'",
(16,6)*{}="-2'",
(18,6)*{}="-3'",
(13.5,6)*{...},
(15,4)*{\underbrace{\ \ \ \ \ \ \ }},
(15,1)*{_{J_2}},
%
%%%%%%%%%%% internal curved edges %%%%%%%%%%%
(0,2)*-{};(8.0,10.0)*-{}
**\crv{(0,10)};
(0.5,1.8)*-{};(8.5,9.0)*-{}
**\crv{(0.4,7)};
%
(1.5,0.5)*-{};(9.1,8.5)*-{}
**\crv{(5,1)};
(1.7,0.0)*-{};(9.5,8.6)*-{}
**\crv{(6,-1)};
(5,5)*+{...};
%
\ar @{-} "b";"-1" <0pt>
\ar @{-} "b";"-2" <0pt>
\ar @{-} "b";"-3" <0pt>
%
\ar @{-} "c";"-1'" <0pt>
\ar @{-} "c";"-2'" <0pt>
\ar @{-} "c";"-3'" <0pt>
\endxy}
%%%%%%%%%%%%%%%%%%%%%%%%%%%%%%%%%%%%%%%%%%%
\Ea
$$
where $l$ counts the number of internal edges connecting
the two vertices
on the right-hand side. There is a natural morphism of properads
$$
\Omega(qc\LoB^\Koz) \lon qc\LoB,
$$
which is a quasi-isomorphism if and only if  $qc\LoB$ is Koszul. Thus to prove Koszulness
of $qc\LoB$ it is enough to establish an isomorphism $H^\bu(\Omega(qc\LoB^\Koz)) \cong qc\LoB$ of $\bS$-bimodules.


\sip

To do this, one may closely follow the proof of Theorem {\ref{2: Theorem on Koszulness}}, adjusting it slightly so as to allow for the additional $(0,n)$-ary generators.
First, we define a properad $\tilde \cP$ which is generated by the properad $\cP$ of section \ref{2: subsection on P}, together with an additional generator of arity $(0,1)$, in pictures $\hspace{-1.8mm}  \Ba{c}\xy
 <0mm,2mm>*{\circ};<0mm,-2mm>*{}**@{-},
 \endxy\Ea \hspace{-1.8mm}$, with the additional relations
\begin{align*}
   \begin{xy}
<0mm,0.55mm>*{};<0mm,3.5mm>*{\circ}**@{-},
<0.5mm,-0.5mm>*{};<2.2mm,-2.2mm>*{}**@{-},
<-0.48mm,-0.48mm>*{};<-2.5mm,-2.5mm>*{}**@{-},
<0mm,0mm>*{\circ};
<-2.78mm,-2.78mm>*{};
 \end{xy} &= \ 0
 &
 \xy
%
(0,-1.4)*{\bu}="0",
 (0,2.4)*{\circ}="1",
(0,-4.5)*{}="d2",
%
\ar @{-} "0";"1" <0pt>
\ar @{-} "0";"d2" <0pt>
\endxy
&=
0\,.
\end{align*}

The map $\Omega(qc\LoB^\Koz)\to qc\LoB$ clearly factors through $\tilde\cP$
\Beq\label{equ:Koszul factoring}
 \Omega(qc\LoB^\Koz) \to \tilde\cP \to qc\LoB
\Eeq
and it suffices to show that both of the above maps are quasi-isomorphisms.
Consider first the left-hand map.
The fact that this map is a quasi-isomorphism may be proven by copying the proof of Theorem \ref{2: proposition on nu quasi-iso}, except that now the functor $F$ (as in section \ref{2: subsection on P}) is applied not to the
cobar construction, $\Omega_{\frac{1}{2}} ( \LB^{\text{!`}}_{\frac{1}{2}})$
but to $\Omega_{\frac{1}{2}} ( qc\LB^{\text{!`}}_{\frac{1}{2}})$. Here
\[
 qc\LB_{\frac{1}{2}}(m,n)=
 \begin{cases}
  \LB_{\frac{1}{2}}(m,n) & \text{if $m\neq 0$} \\
  \K & \text{if $m= 0$}
 \end{cases}
\]
is the $\frac 1 2$-prop governing Lie bialgebras with a counit operation killed by the cobracket.
More concretely, $qc\LB^{\text{!`}}_{\frac{1}{2}}(m,n)$ is the same as $\LB^{\text{!`}}_{\frac{1}{2}}(m,n)$ in all arities $(m,n)$ with $m,n>0$, but $qc\LB^{\text{!`}}_{\frac{1}{2}}(0,n)$ is one-dimensional, the extra operations corresponding to corollas
\[
 \Ba{c}
\resizebox{10mm}{!}{\xy
(-7,-8.5)*{_{}},
(-3.5,-8.5)*{_{}},
(7.5,-8.5)*{_{}},
%
(0,0)*+{}*\cir{}="o",
(-7,-7)*{}="d1",
(-3.5,-7)*{}="d2",
(1.6,-5.5)*{...},
(7,-7)*{}="d4",
%
\ar @{-} "o";"d1" <0pt>
\ar @{-} "o";"d2" <0pt>
\ar @{-} "o";"d4" <0pt>
\endxy}
\Ea\, .
\]
One can check\footnote{The piece of the $\frac 1 2$-prop $\Omega_{\frac{1}{2}} ( qc\LB^{\text{!`}}_{\frac{1}{2}}))$ involving the additional generators is isomorphic to the complex $(E_1^{\mathsf{Lie}+},d_1^{\mathsf{Lie}+})$ from \cite{Me2} (see page 344). According to loc. cit. its cohomology is one-dimensional.} that the $\frac{1}{2}$-prop $qc\LB^{\text{!`}}_{\frac{1}{2}}$ is Koszul, i.~e., that
\[
 H(\Omega_{\frac{1}{2}}( qc\LB^{\text{!`}}_{\frac{1}{2}}))\cong qc\LB_{\frac{1}{2}}.
\]
The properad $\tilde\cP$ is obtained by applying the exact functor $F$ to this $\frac 1 2$-prop, and hence, by essentially the same arguments as in the proof of Theorem \ref{2: nu quasi-iso to P} the left-hand map of \eqref{equ:Koszul factoring} is a quasi-isomorphism.

\sip

Next consider the right hand map of \eqref{equ:Koszul factoring}. It can be shown to be a quasi-isomorphism along the lines of the proof of Theorem \ref{2: Theorem on Koszulness}.
Again, it is clear that the degree zero cohomology of $\tilde\cP$ is $qc\LoB$, so it will suffice to show that $H^{>0}(\tilde\cP)=0$.
First, let $\widetilde{\cP\cP}$ be the prop generated by the properad $\tilde\cP$. As a dg $\bS$-bimodule it is isomorphic to (cf. \eqref{2: basis in PP})
\begin{equation*}
   \tilde W(n,m):= \bigoplus_{N,M} \left( \caL ieP(n,N) \otimes V^{\otimes N} \otimes \K^{\otimes M}\otimes \caL ieCP(N+M,m)
   \right)_{\bS_N\times \bS_M}
\end{equation*}
where $V$ is as in \eqref{2: basis in PP}.
The above complex $\tilde W(n,m)$ is a direct summand of the complex (cf. \eqref{equ:V_mn})
\[
\tilde V_{n,m}
:=
\bigoplus_{N,M} \left( \caL ieP(n,N) \otimes V^{\otimes N} \otimes \K^{\otimes M} \otimes \cA ssCP(N+M,m)
\right)_{\bS_N\times \bS_M}
\]
by arguments similar to those following \eqref{equ:V_mn}.
Then again by the Koszulness results of section \ref{sec:extracomplexes} it follows that the above complex has no cohomology in positive degrees, hence neither can $\widetilde{\cP\cP}$ have cohomology in positive degrees. Hence we can conclude that the properad $qc\LoB$ is Koszul.



\end{proof}

\bip

{\large
\section{\bf Deformation complexes}\label{3: Def complexes}
}

As one application of the Koszulness of $\LoB$ and $\Frob$ we obtain minimal models $\LoB_\infty=\Omega(\LoB^\Koz)=\Omega(\coFrob\{1\})$ and $\Frob_\infty=\Omega(\invcoLieBi\{1\})$ of these properads and hence minimal models for their deformation complexes and for the deformation complexes of their algebras.

\sip

\subsection{\bf A deformation complex of an involutive Lie bialgebra}\label{3: Section on Def complex}

%\subsection{From deformation complex  to Moyal type brackets}
According to the general theory \cite{MV}, $\LoB_\infty$-algebra structures on a dg
vector space $(\fg, d)$ can be identified with Maurer-Cartan elements,
  $$
  \mathcal{MC}\left({\mathsf{InvLieB}}(\fg)\right):= \left\{\Ga\in \mathsf{InvLieB}(\fg)
  : |\Ga|=3\ \mbox{and}\  [\Ga,\Ga]_{CE}=0\right\},
  $$
  of a  graded Lie algebra,\footnote{More precisely, $\mathsf{InvLieB}(\fg)$ is a $\Lie\{2\}$-algebra, not a Lie algebra, i.~e., the Lie bracket has degree $-2$. We will abuse notation and still call $\mathsf{InvLieB}(\fg)$ a Lie algebra.}
\Beq\label{3:  InvLieB(g)}
{\mathsf{InvLieB}}(\fg):=\Def\left(\LoB_\infty \stackrel{0}{\lon} \cE nd_\fg\right)[-2],
\Eeq
which controls deformations of the zero morphism from $\LoB_\infty$ to the endomorphism
properad $\cE nd_\fg=\{\Hom(\fg^{\ot n}, \fg^{\ot m})\}$.
 As a $\Z$-graded vector space ${\mathsf{InvLieB}}(\fg)$ can be identified with the vector
space of homomorphisms of $\bS$-bimodules,
\Beqrn
{\mathsf{InvLieB}}(\fg)&=&\Hom_\bS\left((\LoB)^\Koz,  \cE nd_\fg\right)[-2]\\
&=&\prod_{a\geq0, m,n\geq 1\atop
m+n+a\geq 3} \Hom_{\bS_m\times \bS_n}\left(\sgn_n\ot\sgn_m[m+n+2a-2],
\Hom(\fg^{\ot  n}, \fg^{\ot m}) \right)[-2]\\
&=& \prod_{a\geq 0, m,n\geq 1\atop
m+n+a\geq 3} \Hom(\odot^n(\fg[-1]),  \odot^m(\fg[-1]) )[-2a]\\
&\subset& \displaystyle  \widehat{\odot^\bu}\left(\fg[-1]\oplus \fg^*[-1]\oplus
\K[-2]\right)\simeq \K[[\eta^i,\psi_i,\hbar]]
\Eeqrn
% If $\fg$ is finite dimensional, we may further understand
% \Beqrn
% {\mathsf{InvLieB}}(\fg) \subset \displaystyle  \widehat{\odot^\bu}\left(\fg[-1]\oplus \fg^*[-1]\oplus
% \K[-2]\right)\simeq \K[[\eta^i,\psi_i,\hbar]]
%\Eeqrn

where $ \hbar$ is a formal  parameter of degree $2$  (a basis vector of the summand $\K[-2]$  above), and, for a basis
$(e_1, e_2,\ldots, e_i,\ldots  )$ in $\fg$ and the associated dual basis
$(e^1, e^2,\ldots, e^i,\ldots  )$ in $\fg^*$ we set $\eta^i:=s\ e^i$,
$\psi_i:=s\ e_i$,
where $s: V\rar V[-1]$ is the suspension map. Therefore the Lie algebra ${\mathsf{InvLieB}}(\fg)$
has a canonical structure of a module over the algebra $\K[[\hbar]]$;
moreover, for finite dimensional $\alg g$ its elements can be identified with
formal power series\footnote{In fact, this is true for a class of infinite-dimensional vector spaces. Consider a
category
of graded vector spaces which are inverse limits of finite dimensional ones (with the corresponding topology and  with the completed tensor product), and also a category
of graded vector spaces which are direct limits of finite dimensional ones. If $\alg g$  belongs to one of these categories, then ${\alg g}^*$ belongs (almost by definition) to the other, and we have isomorphisms of the type $({\alg g} \ot {\alg g})^*= {\alg g}^* \ot {\alg g }^*$ which are required for the ``local coordinate" formulae to work.}, $f$, in variables $\psi_i$, $\eta^i$ and $\hbar$, which  satisfy the ``boundary"
conditions,
\Beq\label{Appendix: boundary conditions}
f(\psi,\eta,\hbar)|_{\psi_i=0}=0,\ \ \ \ f(\psi,\eta,\hbar)|_{\eta^i=0}=0,\ \ \ \
f(\psi,\eta,\hbar)|_{\hbar=0}\in I^3
\Eeq
where $I$ is the maximal ideal in $\K[[\psi_i,\eta^i]]$. The Lie brackets
in  ${\mathsf{InvLieB}}(\fg)$ can be read off either from the coproperad structure in
$(\LoB)^\Koz$
or directly from the formula (\ref{2: d on Lie inv infty}) for the differential,
and are given explicitly by (cf.\ \cite{DCTT}),
\Beq\label{3: brackets in InvLieB(g)}
[f,g]_\hbar:=f *_\hbar g - (-1)^{|f||g|} g *_\hbar f
\Eeq
where (up to  Koszul signs),
$$
f *_\hbar g:=\sum_{k=0}^\infty \frac{\hbar^{k-1}}{k!}\sum_{i_1,\ldots, i_k}
\pm \frac{\p^ k f}{\p \eta^{i_1}\cdots \eta^{i_k}}\frac{\p^ k g}{\p\psi_{i_1}\cdots
\p\psi_{i_k}}
$$
is an associative product. Note that the differential
$d_\fg$ in $\fg$ gives rise to a quadratic element, $D_\fg=\sum_{i,j}\pm d_j^i
\psi_i\eta^j$, of homological degree $3$ in $\K[[\eta^i,\psi_i,\hbar]]$, where $d_j^i$ are the
structure constants of $d_\fg$ in the chosen basis,
 $d_\fg(e_i)=:\sum_{j} d_i^j e_j$.


\bip

Finally, we can identify $\LoB_\infty$ structures in a finite dimensional dg vector space $(\fg,d_\fg)$ with
a homogenous formal power series,
$$
\Ga:= D_\fg + f \in  \K[[\eta^i,\psi_i,\hbar]],
$$
of homological degree 3 such that
\Beq\label{Appendix: equation for Gamma-h}
\Ga *_\hbar \Ga=\sum_{k=0}^\infty \frac{\hbar^{k-1}}{k!}\sum_{i_1,\ldots, i_k}
\pm \frac{\p^ k \Ga}{\p \eta^{i_1}\cdots \eta^{i_k}}\frac{\p^ k \Ga}{\p\psi_{i_1}\cdots
\p\psi_{i_k}}=0,
\Eeq
 and the summand
 $f$ satisfies boundary conditions (\ref{Appendix: boundary conditions}).

\bip

For example, let
$$
\left(\bigtriangleup:V\rar \wedge^2 V,\ \ \  [\ ,\ ]:\wedge^2V \rar V\right)
$$
be a Lie bialgebra structure in a  vector space $V$ which we assume for simplicity to be concentrated in degree $0$. Let
$C_{ij}^k$ and $\Phi_k^{ij}$ be the associated structure constants,
$$
[x_i,x_j]=:\sum_{k\in I} C_{ij}^k x_k,\ \ \ \ \bigtriangleup(x_k)=:\sum_{i,j\in I} \Phi_k^{ij} x_i\wedge x_j.
$$
Then it is easy to check that all the involutive Lie bialgebra axioms (\ref{R for LieB}) get encoded into a single equation $\Ga *_\hbar \Ga=0$ for
$
\Ga:=\sum_{i,j,k\in I}\left( C_{ij}^k \psi_k\eta^i\eta^j + \Phi_k^{ij}\eta^k \psi_i\psi_j
\right).
$
\sip

Note that all the above formulae taken modulo the ideal generated by the formal variable $\hbar$
give us a Lie algebra,
\Beq\label{3:  LieB(g)}
{\mathsf{LieB}}(\fg):=\Def\left(\caL ie\cB_\infty \stackrel{0}{\lon} \cE nd_\fg\right)[-2]\cong \K[[\psi_i,\eta^i]]
\Eeq
controlling the deformation theory of (not-necessarily involutive) Lie bialgebra structures in a dg space $\fg$.  Lie brackets in (\ref{3:  LieB(g)}) are given in coordinates by the standard Poisson formula,
\Beq\label{3: Poisson brackets in LieB(g)}
\{f, g\}=\sum_{i\in I} (-1)^{|f||\eta^i|} \frac{\p f}{\p  \psi_i} \frac{\p g}{\p \eta^i} -
 (-1)^{|f||\psi_i|} \frac{\p f}{\p \eta^i} \frac{\p g}{\p \psi_i}
\Eeq
for any $f,g\in \K[[\psi_i,\eta^i]]$. Formal power series, $f\in  \K[[\psi_i,\eta^i]]$, which have homological degree $3$ and satisfy the equations,
$$
\{f,f\}=0, \ \ \ \ f(\psi,\eta)|_{\psi_i=0}=0,\ \ \ \ f(\psi,\eta)|_{\eta^i=0}=0,
$$
are in one-to-one correspondence with strongly homotopy Lie bialgebra structures in a finite dimensional dg vector space $\fg$.




\subsection{Deformation complexes of properads} \label{sec:defcomplexes}
The deformation complex of a properad $\cP$ is by definition the dg Lie algebra $\Der(\tilde \cP)$ of derivations of a cofibrant resolution $\tilde \cP\stackrel{\sim}{\to}\cP$. (See the remarks at the end of the introduction for our slightly non-standard definition of $\Der(\dots)$, and \S 5.1 in \cite{Ta} for similar considerations in the operadic setting.). It may be identified as a complex with the deformation complex of the identity map $\tilde \cP\to \tilde \cP$ (which controls deformations of $\cP$-algebras) up to a degree shift:
\[
 \Der(\tilde \cP) \cong \Def(\tilde \cP\to \tilde \cP)[1].
\]
Note however that both $\Der(\cP)$ and $\Def(\tilde \cP\to \tilde \cP)$ have natural dg Lie (or $\caL ie_\infty$) algebra structures that are \emph{not} preserved by the above map. Furthermore, there is a quasi-isomorphism of dg Lie algebras
\begin{equation}\label{equ:Defsimpl}
 \Def(\tilde \cP\to \tilde \cP)\to \Def(\tilde \cP\to \cP)
\end{equation}

The zeroth cohomology $H^0(\Der(\tilde \cP))$ is of particular importance. It is a differential graded Lie algebra whose elements act on the space of $\tilde \cP$ algebra structures on any vector space. We shall see that in the examples we are interested in this dg Lie algebra is very rich, and that it acts non-trivially in general.


\sip
Using the Koszulness of the properads $\LieBi$, $\invFrob$ from \cite{MaVo, Ko} and the Koszulness of $\invLieBi$ and $\Frob$ from Theorem \ref{2: Theorem on Koszulness} and Corollary \ref{2: corollary on Frob} we can write down the following models for the deformation complexes.
\begin{align*}
\Der(\LieBi_\infty) &=
\prod_{n,m\geq 1} \Hom_{\bS_n\times \bS_m}((\cF rob^\diamond_2)^*\{1\}(n,m), \LieBi_\infty(n,m))[1]
\\
&\cong \prod_{n,m\geq 1} (\LieBi_\infty(n,m) \otimes \sgn_n\otimes \sgn_m)^{\bS_n\times \bS_m}[3-n-m]
\\
\Der(\LoB_\infty) &=  \prod_{n,m\geq 1} \Hom_{\bS_n\times \bS_m}((\cF rob_2)^*\{1\}(n,m), \LoB_\infty(n,m))[1]
\\
&\cong \prod_{n,m\geq 1} (\LoB_\infty(n,m)\otimes \sgn_n\otimes \sgn_m )^{\bS_n\times \bS_m} [3-n-m][[\hbar]]
 \\
\Der(\Frob_\infty) &=\prod_{n,m\geq 1} \Hom_{\bS_n\times \bS_m}((\caL ie^\diamond \cB_2)^*\{1\}(n,m), \Frob_\infty(n,m))[1]
\\
\Der(\invFrob_\infty) &=\prod_{n,m\geq 1} \Hom_{\bS_n\times \bS_m}((\caL ie\cB_2)^*\{1\}(n,m), \invFrob_\infty(n,m))[1]
\end{align*}
Here $\hbar$ is a formal variable of degree 2, $\cF rob^\diamond_2/\caL ie^\diamond \cB_2$ are $\cF rob^\diamond_2/\caL ie\cB_2$
are analogues of (involutive) Frobenius/Lie bialgebras properads with $(2,1)$ generator placed in degree zero the $(1,2)$-generator placed in degree 2.
Each of the models on the right has a natural combinatorial interpretation as a graph complex, cf. also \cite[section 1.7 ]{MaVo}.
For example $\Der(\LieBi_\infty)$ may be interpreted as a complex of directed  graphs which have incoming and outgoing legs but have no closed paths of directed edges. The differential is obtained by splitting vertices and by attaching new vertices at one of the external legs, see Figure \ref{fig:hairyGC}.






\begin{figure}
 \centering
 \begin{align*} &
\resizebox{15mm}{!}{ \xy
%(0,0)*+{_a}*\cir{}="o",
(0,0)*{\bu}="d1",
(10,0)*{\bu}="d2",
(-5,-5)*{}="dl",
(5,-5)*{}="dc",
(15,-5)*{}="dr",
(0,10)*{\bu}="u1",
(10,10)*{\bu}="u2",
(5,15)*{}="uc",
(15,15)*{}="ur",
(0,15)*{}="ul",
%
\ar @{->} "d1";"d2" <0pt>
\ar @{->} "d1";"dl" <0pt>
\ar @{->} "d1";"dc" <0pt>
\ar @{->} "d2";"dc" <0pt>
\ar @{->} "d2";"dr" <0pt>
\ar @{->} "u1";"d1" <0pt>
\ar @{->} "u1";"d2" <0pt>
\ar @{->} "u2";"d2" <0pt>
\ar @{->} "u2";"d1" <0pt>
\ar @{->} "uc";"u2" <0pt>
\ar @{->} "ur";"u2" <0pt>
\ar @{->} "ul";"u1" <0pt>
\endxy}
 %
 % &\begin{tikzpicture}
 %  \node[int] (v1)  at (0,0) {};
 %  \node[int] (v2)  at (1,0) {};
 %  \node[int] (v3)  at (0,1) {};
 %  \node[int] (v4)  at (1,1) {};
 %  \draw[-latex] (v4) edge (v1) edge (v2)
 %                (v3) edge (v1) edge (v2)
 %                (v1) edge (v2) edge +(-.5,-.5) edge +(.5,-.5)
 %                (v2) edge +(-.5,-.5) edge +(.5,-.5)
 %                (v4) +(.5,.5) edge (v4)  +(-.5,.5) edge (v4)
 %                (v3) +(0,.5) edge (v3);
 % \end{tikzpicture}
 &
 \delta \Gamma
 &=
 \delta_{\LieBi_\infty }\Gamma
 \pm
 \sum\Ba{c}
 \resizebox{9mm}{!}{ \xy
 (0,0)*+{\Ga}="Ga",
(-5,5)*{\bu}="0",
(-8,2)*{}="-1",
(-8,8)*{}="1",
(-5,8)*{}="2",
(-2,8)*{}="3",
%
\ar @{-} "0";"Ga" <0pt>
\ar @{-} "0";"-1" <0pt>
\ar @{-} "0";"1" <0pt>
\ar @{-} "0";"2" <0pt>
\ar @{-} "0";"3" <0pt>
 \endxy}\Ea
 %
 %\begin{tikzpicture}[baseline=-.65ex, scale=.5]
 % \node (v) at (0,0) {$\Gamma$};
 % \node[int] (w) at (-.8,1) {};
 % \draw (v) edge (w) (w) edge +(-.5,.5) edge +(0,.5) edge +(.5,.5) edge +(-.5,-.5);
 %\end{tikzpicture}
 %
  \pm
 \sum\Ba{c}
\resizebox{9mm}{!}{  \xy
 (0,0)*+{\Ga}="Ga",
(-5,-5)*{\bu}="0",
(-8,-2)*{}="-1",
(-8,-8)*{}="1",
(-5,-2)*{}="2",
(-2,-8)*{}="3",
%
\ar @{-} "0";"Ga" <0pt>
\ar @{-} "0";"-1" <0pt>
\ar @{-} "0";"1" <0pt>
\ar @{-} "0";"2" <0pt>
\ar @{-} "0";"3" <0pt>
 \endxy}\Ea
 &
 %\begin{tikzpicture}[baseline=-.65ex, scale=.5]
 % \node (v) at (0,0) {$\Gamma$};
 % \node[int] (w) at (-.8,-1) {};
 % \draw (v) edge (w) (w) edge +(-.5,.5) edge +(0,.5) edge +(.5,-.5) edge +(-.5,-.5);
 %\end{tikzpicture}
 %
 \end{align*}
 \caption{\label{fig:hairyGC} A graph interpretation of an example element of $\Der(\LieBi_\infty)$ (left), and the pictorial description of the differential (right). For the two right-most terms, one sums over all possible ways of attaching an additional vertex to an external leg of $\Gamma$, as is indicated by the picture.
 }
\end{figure}

\sip

Similarly, $\Der(\LoB_\infty)$ may be interpreted as a complex of $\hbar$-power series of graphs with weighted vertices. The differential is obtained by splitting vertices and attaching vertices at external legs as indicated in Figure~ \ref{fig:hairyweightedGC}.

\begin{figure}
 \centering
 \begin{align*}
  &
\resizebox{15mm}{!}{  \xy
%(0,0)*+{_a}*\cir{}="o",
(0,0)*+{_3}*\cir{}="d1",
(10,0)*+{_2}*\cir{}="d2",
(-5,-5)*{}="dl",
(5,-5)*{}="dc",
(15,-5)*{}="dr",
(0,10)*+{_0}*\cir{}="u1",
(10,10)*+{_3}*\cir{}="u2",
(5,15)*{}="uc",
(15,15)*{}="ur",
(0,15)*{}="ul",
%
\ar @{->} "d1";"d2" <0pt>
\ar @{->} "d1";"dl" <0pt>
\ar @{->} "d1";"dc" <0pt>
\ar @{->} "d2";"dc" <0pt>
\ar @{->} "d2";"dr" <0pt>
\ar @{->} "u1";"d1" <0pt>
\ar @{->} "u1";"d2" <0pt>
\ar @{->} "u2";"d2" <0pt>
\ar @{->} "u2";"d1" <0pt>
\ar @{->} "uc";"u2" <0pt>
\ar @{->} "ur";"u2" <0pt>
\ar @{->} "ul";"u1" <0pt>
\endxy}
%
%  \begin{tikzpicture}
%   \node[ext] (v1)  at (0,0) {3};
%   \node[ext] (v2)  at (1,0) {2};
%   \node[ext] (v3)  at (0,1) {0};
%   \node[ext] (v4)  at (1,1) {2};
%   \draw[-latex] (v4) edge (v1) edge (v2)
%                 (v3) edge (v1) edge (v2)
%                 (v1) edge (v2) edge +(-.5,-.5) edge +(.5,-.5)
%                 (v2) edge +(-.5,-.5) edge +(.5,-.5)
%                 (v4) +(.5,.5) edge (v4)  +(-.5,.5) edge (v4)
%                 (v3) +(0,.5) edge (v3);
%  \end{tikzpicture}
 &
 \delta \Gamma
 &=
 \delta_{\LoB_\infty }\Gamma
\pm
 \sum \hbar^{p+k-1}\Ba{c}
\resizebox{10mm}{!}{  \xy
  (-3,4)*+{_k},
 (0,0)*+{\Ga}="Ga",
(-7,7)*+{_p}*\cir{}="0",
(-10,4)*{}="-1",
(-10,11)*{}="1",
(-7,11)*{}="2",
(-4,11)*{}="3",
%%%%%%%%%%% internal curved edges
(-5.3,6.8)*-{};(-0.2,1.5)*-{}
**\crv{(0.4,5)};
%
(-6.9,5.1)*-{};(-1.4,0.3)*-{}
**\crv{(-5,-1.5)};
%
(-6.0,5.4)*-{};(-1.4,0.7)*-{}
**\crv{(-4,-0.5)};
%
%\ar @{-} "0";"Ga" <0pt>
\ar @{-} "0";"-1" <0pt>
\ar @{-} "0";"1" <0pt>
\ar @{-} "0";"2" <0pt>
\ar @{-} "0";"3" <0pt>
%
 \endxy}\Ea
 %
 %\begin{tikzpicture}[baseline=-.65ex, scale=.7]
 % \node (v) at (0,0) {$\Gamma$};
 % \node[ext] (w) at (-.8,1) {$\scriptstyle p$};
 % \draw (v) edge[bend left] node[right]{{$\scriptstyle k$}} (w) edge[bend right] (w) (w) edge %+(-.5,.5) edge +(0,.5) edge +(.5,.5) edge +(-.5,-.5);
 %\end{tikzpicture}
  \pm
 \sum \hbar^{p+k-1}\Ba{c}
 \resizebox{10mm}{!}{  \xy
  (-3,-4)*+{_k},
 (0,0)*+{\Ga}="Ga",
(-7,-7)*+{_p}*\cir{}="0",
(-10,-4)*{}="-1",
(-10,-11)*{}="1",
(-7,-11)*{}="2",
(-4,-11)*{}="3",
%%%%%%%%%%% internal curved edges
(-5.3,-6.8)*-{};(-0.2,-1.5)*-{}
**\crv{(0.4,-5)};
%
(-6.9,-5.1)*-{};(-1.4,-0.3)*-{}
**\crv{(-5,1.5)};
%
(-6.0,-5.4)*-{};(-1.4,-0.7)*-{}
**\crv{(-4,-0.5)};
%
%\ar @{-} "0";"Ga" <0pt>
\ar @{-} "0";"-1" <0pt>
\ar @{-} "0";"1" <0pt>
\ar @{-} "0";"2" <0pt>
\ar @{-} "0";"3" <0pt>
%
 \endxy}\Ea
 %
 %\begin{tikzpicture}[baseline=-.65ex, scale=.7]
 % \node (v) at (0,0) {$\Gamma$};
 % \node[ext] (w) at (-.8,-1) {$\scriptstyle p$};
 % \draw (v) edge[bend left] (w) edge[bend right] (w) (w) edge +(-.5,.5) edge +(0,.5) edge %+(.5,-.5) edge +(-.5,-.5);
 %\end{tikzpicture}
 \end{align*}
 \caption{\label{fig:hairyweightedGC} A graph interpretation of an element of $\Der(\LoB_\infty)$, and the pictorial description of the differential. In the two terms on the right one sums over all ways of attaching a new vertex to some subset of the incoming or outgoing legs ($k$ many), and sums over all possible decorations $p$ of the added vertex, with an appropriate power of $\hbar$ as prefactor. Note that the power of $\hbar$ counts the number of loops added to the graph, if we count a vertex decorated by $p$ as contributing $p$ loops.}
\end{figure}

The Lie bracket is combinatorially obtained by inserting graphs into vertices of another.
We leave it to the reader to work out the structure of the graph complexes and the differentials for the complexes $\Der(\Frob_\infty)$ and $\Der(\invFrob_\infty)$.

\sip

The cohomology of all these graph complexes is hard to compute. We may however simplify the computation by using formula (\ref{equ:Defsimpl}) and equivalently compute instead
\begin{align*}
 \Def(\LieBi_\infty\to \LieBi) &= \prod_{n,m} \Hom_{\bS_n\times \bS_m}((\cF rob^\diamond_2)^*\{1\}(n,m), \LieBi(n,m))
 \\
 &\cong \prod_{n,m}  (\LieBi(n,m)\otimes \sgn_n\otimes \sgn_m)^{\bS_n\times \bS_m}[2-n-m]
  \\
  \Def(\LoB_\infty\to \LoB) &=  \prod_{n,m} \Hom_{\bS_n\times \bS_m}((\cF rob_2)^*\{1\}(n,m), \LoB(n,m))
 \\
 &\cong \prod_{n,m} (\LoB(n,m)\otimes \sgn_n\otimes \sgn_m)^{\bS_n\times \bS_m}[2-n-m] [[\hbar]]
 \\
 \Def(\Frob_\infty\to \Frob) &=\prod_{n,m} \Hom_{\bS_m\times \bS_n}((\caL ie^\diamond \cB_2)^*\{1\}(n,m), \Frob(n,m))
 \\
 \Def(\invFrob_\infty\to \invFrob) &=\prod_{n,m} \Hom_{\bS_n\times \bS_m}((\caL ie \cB_2)^*\{1\}(n,m), \invFrob(n,m))
 \, .
\end{align*}
Note however that in passing from $\Der(\dots)$ to the (quasi-isomorphic) simpler complexes $\Def(\dots)$ above we lose the dg Lie algebra structure, or rather  there is a different Lie algebra structure on the above complexes.
The above complexes may again be interpreted as graph complexes. For example $\Def(\LieBi_\infty\to \LieBi)$ consists of oriented trivalent  graphs with incoming and outgoing legs, modulo the Jacobi and Drinfeld five term relations. The differential is obtained by attaching a trivalent vertex at one external leg in all possible ways.

% We also note for later use that the deformation complexes of the natural maps $\LieBi\to\LoB$ and $\Frob\to \invFrob$ may similarly be interpereted as graph complexes.
% \begin{align*}
%  \Def(\LieBi_\infty\to \LoB) &=  \prod_{N,M\geq 1} \Hom_{S_N\times S_M}(\invcoFrob(N,M), \LoB(N,M))
% = \prod_{N,M} \LoB(N,M)^{S_N\times S_M}
%  \\
%  \Def(\Frob_\infty\to \invFrob) &=\prod_{N,M} \Hom_{S_N\times S_M\geq 1}(\invcoLieBi(N,M), \invFrob(N,M))
% \end{align*}


\sip
Finally we note that of the above four deformation complexes only two are essentially different.
For example, note that $\Hom_{\bS_n\times \bS_m}(\coLieBi\{1\}(n,m), \invFrob(n,m))$ is just a completion of
\[
\Hom_{\bS_n\times \bS_m}(\invcoFrob\{1\}(n,m), \LieBi(n,m))\cong (\invFrob(n,m)\otimes \sgn_n\otimes \sgn_m) \otimes_{\bS_n\times \bS_n}\LieBi(n,m)[n-m]
\]
Concretely, the completion is with respect to the genus grading of $\LieBi$, and the differential preserves the genus grading. Hence the cohomology of one complex is just the completion of the cohomology of the other with respect to the genus grading.

\sip

Similar arguments show that the cohomologies $\Def(\LoB_\infty\to \LoB)$ and $\Def(\Frob_\infty\to \Frob)$ are the same up to completion issues. Here the differential does not preserve the genus but preserves the quantity (genus)-($\hbar$-degree).
Hence it suffices to discuss one of each pair of deformation complexes. We will discuss $\Def(\LieBi_\infty\to \LieBi)$ and $\Def(\LoB_\infty\to \LoB)$ in the next section.

\bip



%%%%%%%%%%%%%%%%%%%%%%%%%%%%%%%%%%%%%%%%%%%%%%%%%%

{\large
\section{\bf Oriented graph complexes and the $\grt_1$ action}
}


The goal of this section is to reduce the computation of the above deformation complexes to the computation of the cohomology of M. Kontsevich's graph complex.
By a result of one of the authors \cite{Wi1} the degree zero cohomology of this graph complex agrees with the Grothendieck-Teichm\"uller Lie algebra $\grt_1$.
This will allow us to conclude that the Grothendieck-Teichm\"uller group universally acts on $\LoB_\infty$ structures. This extends the well known result that the Grothendieck-Teichm\"uller group acts on Lie bialgebra structures.

\bip

\subsection{Grothendieck-Teichm\"uller group} The profinite and prounipotent Grothendieck-Teichm\"uller groups were introduced by Vladimir Drinfeld in his
study of braid groups and quasi-Hopf algebras. They turned out to be one of the
most interesting and mysterious objects in modern mathematics. The profinite Grothendieck-Teichm\"uller group $\widehat{GT}$ plays an important role in number theory
and algebraic geometry. The pro-unipotent Grothendieck-Teichm\"uller group $GT$ (and its graded version $GRT$) over a field of characteristic zero appeared in Pavel Etingof and David Kazhdan's solution of Drinfeld's quantization conjecture for Lie bialgebras. Maxim Kontsevich's and Dmitry Tamarkin's formality theory
unravels the role of the group ${GRT}$  in the deformation quantization of Poisson structures.  Later Anton Alekseev and Charles Torossian applied $GRT$ to the Kashiwara-Vergne problem in Lie theory. The Grothendieck-Teichm\"uller group unifies different fields, and every time this group appears
in a mathematical theory, there follows a breakthrough in that theory. We refer to Hidekazu Furusho's lecture note \cite{Fu} for precise definitions and references.


\sip

In this paper we consider the Grothendieck-Teichm\"uller group $GRT_1$ which is the kernel of the canonical morphism of groups $GRT\rar \K^*$.   As $GRT_1$  is prounipotent, it is of the form $\exp(\alg{grt}_1)$ for some Lie algebra $\alg{grt}_1$ whose definition can be found, for example, in \S 6 of \cite{Wi1}. Therefore to understand representations of $GRT_1$ is the same as to understand representations of the  Grothendieck-Teichm\"uller Lie algebra $\alg{grt}_1$.

\subsection{Completed versions of $\LoB$ and $\LieBi$}\label{sec:completed versions}
The properads $\LieBi$ and $\LoB$ are naturally graded by the genus of the graphs describing the operations.
We will denote by $\hLoB$ and $\hLieBi$ the completions with respect to this grading.
Similarly, we denote by $\hLieBi_\infty$ the completion of $\LieBi_\infty$ with respect to the genus grading.
The natural map $\hLieBi_\infty\to\hLieBi$ is a quasi-isomorphism. Furthermore, we denote by $\hLoB_\infty$ the completion of $\LoB_\infty$ with respect to the genus plus the total weight-grading, i.~e., with respect to the grading $||\cdot||$ described in section {\ref{sec:decomposition}}. Then the map $\hLoB_\infty\to \hLoB$ is a quasi-isomorphism.

We will call a continuous representation of $\hLieBi$ (respectively of $\hLoB$) a \emph{genus complete} (involutive) Lie bialgebra.
Here the topology on $\hLieBi$ (respectively on $\hLoB$) is the one induced by the genus filtration (respectively the filtration $||\cdot||$).
For example, the involutive Lie bialgebra discussed in section {\ref{2: subsection on cyclic words}} is clearly genus complete since both the cobracket and the bracket reduce the lengths of the cyclic words.

Abusing notation slightly we will denote by $\Der(\hLieBi_\infty)$ (respectively by $\Der(\hLoB_\infty)$) the complex of \emph{continuous} derivations. % with respect to the topology induced by the genus filtration (respectively the filtration $||\cdot||$).
The sub-properads $\LieBi_\infty\subset \hLieBi_\infty$ and $\LoB_\infty \subset \hLoB_\infty$ are dense by definition and hence any continuous derivation is determined by its restriction to these sub-properads. It also follows that the above complexes of derivations are isomorphic as complexes to $\Def(\LieBi_\infty\to \hLieBi_\infty)[1]$ and  $\Def(\LoB_\infty\to \hLoB_\infty)[1]$.
Finally we note that the cohomology of these complexes is merely the completion of the cohomology of the complexes $\Der(\LieBi_\infty)$ and $\Der(\LoB_\infty)$, since the differential respects the gradings.


\subsection{An operad of graphs $\cG ra^\uparrow$} A graph is called {\em directed}\, if its edges are equipped with directions as in the following examples,
$$
\Ba{c}
\resizebox{15mm}{!}{ \xy
%
(0,17)*{\bu}="u",
(-5,12)*{\bu}="0",
 (0,7)*{\bu}="a",
(-10,7)*{\bu}="L",
(10,7)*{\bu}="R",
(-5,2)*{\bu}="b_1",
(5,2)*{\bu}="b_2",
%
\ar @{<-} "a";"0" <0pt>
\ar @{->} "a";"b_1" <0pt>
\ar @{<-} "a";"b_2" <0pt>
\ar @{->} "b_1";"L" <0pt>
\ar @{<-} "0";"L" <0pt>
\ar @{->} "b_2";"R" <0pt>
\ar @{->} "R";"u" <0pt>
\ar @{->} "0";"u" <0pt>
\endxy}
\Ea
\ \ \ ,\ \ \
\Ba{c}
\resizebox{15mm}{!}{ \xy
%
(0,17)*{\bu}="u",
(-5,12)*{\bu}="0",
 (0,7)*{\bu}="a",
(-10,7)*{\bu}="L",
(10,7)*{\bu}="R",
(-5,2)*{\bu}="b_1",
(5,2)*{\bu}="b_2",
%
\ar @{->} "a";"0" <0pt>
\ar @{<-} "a";"b_1" <0pt>
\ar @{<-} "a";"b_2" <0pt>
\ar @{->} "b_1";"L" <0pt>
\ar @{<-} "0";"L" <0pt>
\ar @{->} "b_2";"R" <0pt>
\ar @{->} "R";"u" <0pt>
\ar @{->} "0";"u" <0pt>
\endxy}
\Ea
$$
A directed graph is called {\em oriented}\, or {\em acyclic}\, if it contains no  directed {\em closed}\, paths of edges.
%(which are sometimes called {\em wheels}).
For example, the second graph above is oriented while the first one is not.
 For arbitrary integers $n\geq 1$ and $l\geq 0$ let ${\sG}^\uparrow_{n,l}$ stand for the
set of connected oriented graphs, $\{\Ga\}$, with $n$ vertices and $l$ edges
such that the vertices of $\Ga$ are labelled by elements of $[n]:=\{1,\ldots, n\}$, i.e.\ an isomorphism $V(\Ga)\rar [n]$ is fixed. We allow graphs with multiple edges between two vertices throughout.
%


\sip

Let
 $\K\langle \sG_{n,l}^\uparrow\rangle$  be the vector space over a  field $\K$ of characteristic zero which is  spanned by graphs from
$\sG_{n,l}^\uparrow$,
and consider a  $\Z$-graded $\bS_n$-module,
$$
\cG ra^\uparrow (n):=\bigoplus_{l=0}^\infty \K\langle \sG^\uparrow_{n,l}\rangle[2l].
$$
For example,  $\xy
(0,2)*{_{1}},
(7,2)*{_{2}},
%
 (0,0)*{\bullet}="a",
(7,0)*{\bu}="b",
%
\ar @{->} "a";"b" <0pt>
\endxy$ is a degree $-2$ element in  $\cG ra^\uparrow(2)$.
 The $\bS$-module, $\cG ra^\uparrow :=\{\cG ra (n)^\uparrow\}_{n\geq 1}$, is naturally an operad with the
 operadic compositions given by
\Beq\label{3: operad comp in Gra}
\Ba{rccc}
\circ_i: & \cG ra^\uparrow (n)\ot \cG ra^\uparrow (m) & \lon &  \cG ra^\uparrow (m+n-1)\\
&  \Ga_1 \ot \Ga_2   &\lon & \sum_{\Ga\in \sG_{\Ga_1, \Ga_2}^i}  \Ga
\Ea
\Eeq
where $ \sG_{\Ga_1, \Ga_2}^i$ is the subset of $\sG^\uparrow_{n+m-1, \# E(\Ga_1) + \#E(\Ga_2)}$ consisting
of graphs, $\Ga$, satisfying the condition: the full subgraph of $\Ga$ spanned by the vertices labeled by
the set $\{i,i+1, \ldots, i+m-1\}$ is isomorphic to $\Ga_2$ and the quotient graph $\Ga/\Ga_2$ (which is obtained from $\Ga$
obtained by contracting that subgraph $\Ga_2$ to a single vertex) is isomorphic to $\Ga_1$, see, e.g.,  \S 7 in \cite{Me} or \S 2 in
\cite{Wi1}  for explicit examples of this kind of operadic compositions.
The unique element in $\sG_{1,0}^\uparrow$ serves as the unit in the operad  $\cG ra^\uparrow$.

\subsubsection{\bf A representation of $\cG ra^\uparrow$ in $\mathsf{LieB}(\fg)$}
\label{3: subsec on canonical repr of Gra}
For any graded vector space $\fg$ the operad  $\cG ra^\uparrow$ has a natural representation in the associated graded vector space $\mathsf{LieB}(\fg)$ (see (\ref{3:  LieB(g)})),
\Beq\label{3: Gra representation in g_V}
\Ba{rccc}
\rho: & \cG ra^\uparrow(n) & \lon & \cE  nd_{\mathsf{LieB}(\fg)}(n)=
\Hom( \mathsf{LieB}(\fg)^{\ot n},\mathsf{LieB}(\fg))\\
      & \Ga &\lon & \Phi_\Ga
\Ea
\Eeq
given by the formula,
$$
\Ba{rccc}
\Phi_\Ga: & \ot^n \mathsf{LieB}(\fg)   & \lon & \mathsf{LieB}(\fg)\\
& \ga_1\ot \ldots \ot \ga_n   &\lon &
\Phi_\Ga(\ga_1,\ldots, \ga_n) :=\mu\left(\left(\prod_{e\in E(\Ga)}
\Delta_e\right) \ga_1(\psi,\eta)\ot \ga_2(\psi,\eta)\ot \ldots\ot
\ga_n(\psi,\eta) \right)
\Ea
$$
where, for an edge $e=\Ba{c}\xy
(0,2)*{_{a}},
(6,2)*{_{b}},
%
 (0,0)*{\bullet}="a",
(6,0)*{\bu}="b",
%
\ar @{->} "a";"b" <0pt>
\endxy \Ea$ connecting a vertex labeled by $a\in [n]$ and to a vertex labelled by $b\in [n]$, we set
$$
\Delta_e \left(\ga_1\ot \ga_2\ot \ldots\ot
\ga_n \right)=
\left\{\Ba{cc}\underset{i\in I}{\sum}(-1)^{|\eta^i|(|\ga_a| + |\ga_{a+1}|+\ldots+ |\ga_{b-1}|)} \ga_1\ot ...\ot \frac{\p\ga_a}{\p \psi_i}\ot ...
\ot \frac{\p\ga_b}{\p \eta^i}\ot ... \ot
\ga_n & \mbox{for} \ a< b   \\
\underset{i\in I}{\sum}(-1)^{|\psi_i|(|\ga_b| + |\ga_{b+1}|+\ldots+ |\ga_{a-1}| + |\eta^i| )} \ga_1\ot ... \ot \frac{\p\ga_b}{\p \eta^i}\ot ...
\ot \frac{\p\ga_a}{\p \psi_i}\ot ... \ot
\ga_n & \mbox{for} \ b< a
\Ea\right.
$$
and where  $\mu$ is the standard multiplication map in the ring  $\mathsf{LieB}(\fg)\subset \K[[\psi_i,\eta^i]]$,
$$
\Ba{rccc}
\mu:&   \mathsf{LieB}(\fg)^{\ot n} & \lon & \mathsf{LieB}(\fg)\\
   & \ga_1\ot \ga_2\ot \ldots \ot \ga_n &\lon & \ga_1 \ga_2 \cdots
   \ga_n.
\Ea
$$
Note that this representation makes sense for both finite- and {\em infinite}\, dimensional vector spaces $\fg$ as graphs from $\cG ra^\uparrow$ do not contain oriented cycles.

 \begin{remark}
 The above action of $\cG ra^\uparrow$ on $\Def\left(\caL ie\cB_\infty \stackrel{0}{\lon} \cE nd_\fg \right)[-2]$ only uses
 the properadic compositions in $\cE nd_\fg$ and no further data. It follows that the same formulas may in fact be used to define an action of $\cG ra^\uparrow$ on the deformation complex
 \[
  \Def\left(\caL ie\cB_\infty \stackrel{0}{\lon} \cP \right)[-2]
  \cong \prod_{m,n}(\cP(m,n)\otimes \sgn_{m}\otimes \sgn_{n})^{\bS_m\times \bS_n}[-m-n]
 \]
 for any properad $\cP$. To give a more concrete description of the action, let us identify $\bS_m\times \bS_n$-coinvariants with invariants by symmetrization, and let us describe an action on the space of coinvariants
 \[
 \prod_{m,n}(\cP(m,n)\otimes \sgn_{m}\otimes \sgn_{n})_{\bS_m\times \bS_n}[-m-n]
 \]
 instead.
 Concretely, let $\Gamma\in \cG ra^\uparrow(n)$ be a graph with $n$ vertices and let
\[
x_j\in (\cP(m_j,n_j)\otimes \sgn_{m_j}\otimes \sgn_{n_j})_{\bS_m\times \bS_n}
\]
for $j=1,\dots, n$. If some vertex $j$ of $\Gamma$ has more then $n_j$ outgoing or more then $m_j$ incoming edges, then we define the action to be trivial: $\Gamma(x_1,\dots,x_n)=0$. Otherwise, we want to interpret the directed graph $\Gamma$ as a properadic composition pattern. For notational simplicity, we assume that the $x_j$ are actual elements of $\cP(m_j,n_j)$, representing the corresponding elements of the coinvariant space.
Suppose that for each vertex $j$ of $\Gamma$ an injective map from the set of the (say $k_j$ many) incoming half-edges at $j$ to $\{1,\dots, m_j\}$, and an injective map from the set of (say $l_j$ many) outgoing half-edges at $j$ to $\{1,\dots, n_j\}$ is fixed. Denote the collection of those maps (for all $j$) by $f$ for concreteness. Then we may define
\[
\Gamma_f(x_1,\dots, x_n) \in \cP(\sum_{j=1}^n (m_j-k_j), \sum_{j=1}^n (n_j-l_j))
\]
obtained by using the appropriate properadic composition. (The overall inputs and outputs are to be ordered according to the numbering of the vertices).
Then we define our desired action to be
\[
\Gamma(x_1,\dots, x_n) := \sum_f \pm \Gamma_f(x_1,\dots, x_n)
\]
where the $f$ in the sum runs over assignments of half-edges to inputs/outputs as above. The sign can be determined by considering half-edges and inputs/outputs of the $x_j$ as odd objects. The sign is then the sign of the permutation bringing each half-edge "to the left of" the input/output it is assigned to via $f$.
\end{remark}



\subsection{An oriented graph complex}
Let $\caL ie\{2\}$ be a (degree shifted) operad of Lie algebras, and let
 $\caL ie_\infty\{2\}$ be its minimal resolution. Thus $\caL ie\{2\}$ is a quadratic operad generated by degree $-2$ skewsymmetric binary operation,
$$
\begin{xy}
 <0mm,0.66mm>*{};<0mm,3mm>*{}**@{-},
 <0.39mm,-0.39mm>*{};<2.2mm,-2.2mm>*{}**@{-},
 <-0.35mm,-0.35mm>*{};<-2.2mm,-2.2mm>*{}**@{-},
 <0mm,0mm>*{\bu};<0mm,0mm>*{}**@{},
   <0.39mm,-0.39mm>*{};<2.9mm,-4mm>*{^{_2}}**@{},
   <-0.35mm,-0.35mm>*{};<-2.8mm,-4mm>*{^{_1}}**@{},
\end{xy}=-
\begin{xy}
 <0mm,0.66mm>*{};<0mm,3mm>*{}**@{-},
 <0.39mm,-0.39mm>*{};<2.2mm,-2.2mm>*{}**@{-},
 <-0.35mm,-0.35mm>*{};<-2.2mm,-2.2mm>*{}**@{-},
 <0mm,0mm>*{\bu};<0mm,0mm>*{}**@{},
   <0.39mm,-0.39mm>*{};<2.9mm,-4mm>*{^{_1}}**@{},
   <-0.35mm,-0.35mm>*{};<-2.8mm,-4mm>*{^{_2}}**@{},
\end{xy}
$$
modulo the Jacobi relations,
\Beq\label{3: Jacobi relation}
\Ba{c}
\begin{xy}
 <0mm,0mm>*{\bu};<0mm,0mm>*{}**@{},
 <0mm,0.69mm>*{};<0mm,3.0mm>*{}**@{-},
 <0.39mm,-0.39mm>*{};<2.4mm,-2.4mm>*{}**@{-},
 <-0.35mm,-0.35mm>*{};<-1.9mm,-1.9mm>*{}**@{-},
 <-2.4mm,-2.4mm>*{\bu};<-2.4mm,-2.4mm>*{}**@{},
 <-2.0mm,-2.8mm>*{};<0mm,-4.9mm>*{}**@{-},
 <-2.8mm,-2.9mm>*{};<-4.7mm,-4.9mm>*{}**@{-},
    <0.39mm,-0.39mm>*{};<3.3mm,-4.0mm>*{^{_3}}**@{},
    <-2.0mm,-2.8mm>*{};<0.5mm,-6.7mm>*{^{_2}}**@{},
    <-2.8mm,-2.9mm>*{};<-5.2mm,-6.7mm>*{^{_1}}**@{},
 \end{xy}
\ + \
 \begin{xy}
 <0mm,0mm>*{\bu};<0mm,0mm>*{}**@{},
 <0mm,0.69mm>*{};<0mm,3.0mm>*{}**@{-},
 <0.39mm,-0.39mm>*{};<2.4mm,-2.4mm>*{}**@{-},
 <-0.35mm,-0.35mm>*{};<-1.9mm,-1.9mm>*{}**@{-},
 <-2.4mm,-2.4mm>*{\bu};<-2.4mm,-2.4mm>*{}**@{},
 <-2.0mm,-2.8mm>*{};<0mm,-4.9mm>*{}**@{-},
 <-2.8mm,-2.9mm>*{};<-4.7mm,-4.9mm>*{}**@{-},
    <0.39mm,-0.39mm>*{};<3.3mm,-4.0mm>*{^{_2}}**@{},
    <-2.0mm,-2.8mm>*{};<0.5mm,-6.7mm>*{^{_1}}**@{},
    <-2.8mm,-2.9mm>*{};<-5.2mm,-6.7mm>*{^{_3}}**@{},
 \end{xy}
\ + \
 \begin{xy}
 <0mm,0mm>*{\bu};<0mm,0mm>*{}**@{},
 <0mm,0.69mm>*{};<0mm,3.0mm>*{}**@{-},
 <0.39mm,-0.39mm>*{};<2.4mm,-2.4mm>*{}**@{-},
 <-0.35mm,-0.35mm>*{};<-1.9mm,-1.9mm>*{}**@{-},
 <-2.4mm,-2.4mm>*{\bu};<-2.4mm,-2.4mm>*{}**@{},
 <-2.0mm,-2.8mm>*{};<0mm,-4.9mm>*{}**@{-},
 <-2.8mm,-2.9mm>*{};<-4.7mm,-4.9mm>*{}**@{-},
    <0.39mm,-0.39mm>*{};<3.3mm,-4.0mm>*{^{_1}}**@{},
    <-2.0mm,-2.8mm>*{};<0.5mm,-6.7mm>*{^{_3}}**@{},
    <-2.8mm,-2.9mm>*{};<-5.2mm,-6.7mm>*{^{_2}}**@{},
 \end{xy}\Ea=0
\Eeq
while  $\caL ie_\infty\{2\}$ is the free operad generated by an
$\bS$-module $E=\{E(n)\}_{n\geq 2}$,
%\Beq\label{3: generators of Lie_infty2}
$$
E(n):=sgn_n[3n-4]=\left\langle\Ba{c}
 \resizebox{19mm}{!}  {\xy
(1,-5)*{\ldots},
(-13,-7)*{_1},
(-8,-7)*{_2},
(-3,-7)*{_3},
(7,-7)*{_{n-1}},
(13,-7)*{_n},
%
 (0,0)*{\bu}="a",
(0,5)*{}="0",
(-12,-5)*{}="b_1",
(-8,-5)*{}="b_2",
(-3,-5)*{}="b_3",
(8,-5)*{}="b_4",
(12,-5)*{}="b_5",
%
\ar @{-} "a";"0" <0pt>
\ar @{-} "a";"b_2" <0pt>
\ar @{-} "a";"b_3" <0pt>
\ar @{-} "a";"b_1" <0pt>
\ar @{-} "a";"b_4" <0pt>
\ar @{-} "a";"b_5" <0pt>
\endxy}=(-1)^{\sigma}
\resizebox{19mm}{!}  {\xy
(1,-6)*{\ldots},
(-13,-7)*{_{\sigma(1)}},
(-6.7,-7)*{_{\sigma(2)}},
%(-3,-7)*{_{\sigma(3)}},
%(7,-8)*{_{n-1}},
(13,-7)*{_{\sigma(n)}},
%
 (0,0)*{\bu}="a",
(0,5)*{}="0",
(-12,-5)*{}="b_1",
(-8,-5)*{}="b_2",
(-3,-5)*{}="b_3",
(8,-5)*{}="b_4",
(12,-5)*{}="b_5",
%
\ar @{-} "a";"0" <0pt>
\ar @{-} "a";"b_2" <0pt>
\ar @{-} "a";"b_3" <0pt>
\ar @{-} "a";"b_1" <0pt>
\ar @{-} "a";"b_4" <0pt>
\ar @{-} "a";"b_5" <0pt>
\endxy}\Ea
\right\rangle_{\sigma\in \bS_n}
%\Eeq
$$
and equipped with the following differential,
\Beq\label{3: Lie_infty differential}
\p\hspace{-3mm}
\resizebox{19mm}{!}
{ \xy
(1,-5)*{\ldots},
(-13,-7)*{_1},
(-8,-7)*{_2},
(-3,-7)*{_3},
(7,-7)*{_{n-1}},
(13,-7)*{_n},
%
 (0,0)*{\bu}="a",
(0,5)*{}="0",
(-12,-5)*{}="b_1",
(-8,-5)*{}="b_2",
(-3,-5)*{}="b_3",
(8,-5)*{}="b_4",
(12,-5)*{}="b_5",
%
\ar @{-} "a";"0" <0pt>
\ar @{-} "a";"b_2" <0pt>
\ar @{-} "a";"b_3" <0pt>
\ar @{-} "a";"b_1" <0pt>
\ar @{-} "a";"b_4" <0pt>
\ar @{-} "a";"b_5" <0pt>
\endxy}
=
\sum_{ [n]=I_1\sqcup I_2\atop
\# I_1\geq 2, \# I_2\geq 1}(-1)^{\sigma(I_1\sqcup I_2) +|I_1||I_2|}
\Ba{c}
\resizebox{21mm}{!}{
\begin{xy}
<10mm,0mm>*{\bu},
<10mm,0.8mm>*{};<10mm,5mm>*{}**@{-},
<0mm,-10mm>*{...},
<14mm,-5mm>*{\ldots},
<13mm,-7mm>*{\underbrace{\ \ \ \ \ \ \ \ \ \ \ \ \  }},
<14mm,-10mm>*{_{I_2}};
<10.3mm,0.1mm>*{};<20mm,-5mm>*{}**@{-},
<9.7mm,-0.5mm>*{};<6mm,-5mm>*{}**@{-},
<9.9mm,-0.5mm>*{};<10mm,-5mm>*{}**@{-},
<9.6mm,0.1mm>*{};<0mm,-4.4mm>*{}**@{-},
<0mm,-5mm>*{\bu};
<-5mm,-10mm>*{}**@{-},
<-2.7mm,-10mm>*{}**@{-},
<2.7mm,-10mm>*{}**@{-},
<5mm,-10mm>*{}**@{-},
<0mm,-12mm>*{\underbrace{\ \ \ \ \ \ \ \ \ \ }},
<0mm,-15mm>*{_{I_1}},
\end{xy}}
\Ea
\Eeq
where $\sigma(I_1\sqcup I_2)$ is the sign of the shuffle $[n]\rar [I_1\sqcup I_2]$.

\subsubsection{\bf Proposition \cite{Wi2}}\label{3: Prop on map from Lie2 to Gra}
{\em There is a morphism of operads
$$
\varphi: \caL ie\{2\} \lon \cG ra^\uparrow
$$
given on the generators  by
%\Beq\label{3: Lie to Gra}
$$
\Ba{c}
\xy
 <0mm,0.55mm>*{};<0mm,3.5mm>*{}**@{-},
 <0.5mm,-0.5mm>*{};<2.2mm,-2.2mm>*{}**@{-},
 <-0.48mm,-0.48mm>*{};<-2.2mm,-2.2mm>*{}**@{-},
 <0mm,0mm>*{\bu};<0mm,0mm>*{}**@{},
 <0.5mm,-0.5mm>*{};<2.7mm,-3.2mm>*{_2}**@{},
 <-0.48mm,-0.48mm>*{};<-2.7mm,-3.2mm>*{_1}**@{},
 \endxy\Ea
   \ \ \ \ \lon \ \ \ \ \xy
(0,2)*{_{1}},
(7,2)*{_{2}},
%
 (0,0)*{\bullet}="a",
(7,0)*{\bu}="b",
%
\ar @{->} "a";"b" <0pt>
\endxy\
- \
\xy
(0,2)*{_{2}},
(7,2)*{_{1}},
%
 (0,0)*{\bullet}="a",
(7,0)*{\bu}="b",
%
\ar @{->} "a";"b" <0pt>
\endxy
%\Eeq
$$
}

\begin{proof} Using the definition of the operadic composition in $\cG ra^\uparrow$ we get
\Beqr
\varphi\left(\Ba{c}
 \begin{xy}
 <0mm,0mm>*{\bu};<0mm,0mm>*{}**@{},
 <0mm,0.69mm>*{};<0mm,3.0mm>*{}**@{-},
 <0.39mm,-0.39mm>*{};<2.4mm,-2.4mm>*{}**@{-},
 <-0.35mm,-0.35mm>*{};<-1.9mm,-1.9mm>*{}**@{-},
 <-2.4mm,-2.4mm>*{\bu};<-2.4mm,-2.4mm>*{}**@{},
 <-2.0mm,-2.8mm>*{};<0mm,-4.9mm>*{}**@{-},
 <-2.8mm,-2.9mm>*{};<-4.7mm,-4.9mm>*{}**@{-},
    <0.39mm,-0.39mm>*{};<3.3mm,-4.0mm>*{^3}**@{},
    <-2.0mm,-2.8mm>*{};<0.5mm,-6.7mm>*{^2}**@{},
    <-2.8mm,-2.9mm>*{};<-5.2mm,-6.7mm>*{^1}**@{},
 \end{xy}\Ea\right)&=&
\Ba{c}\xy
(0,2)*{_{1}},
(6,2)*{_{2}},
(12,2)*{_{3}},
%
 (0,0)*{\bullet}="a",
(6,0)*{\bu}="b",
(12,0)*{\bu}="c",
%
\ar @{->} "a";"b" <0pt>
\ar @{->} "b";"c" <0pt>
\endxy\Ea
\ - \
\Ba{c}\xy
(0,2)*{_{2}},
(6,2)*{_{1}},
(12,2)*{_{3}},
%
 (0,0)*{\bullet}="a",
(6,0)*{\bu}="b",
(12,0)*{\bu}="c",
%
\ar @{->} "a";"b" <0pt>
\ar @{->} "b";"c" <0pt>
\endxy\Ea
\ + \
\Ba{c}\xy
(0,2)*{_{2}},
(6,2)*{_{1}},
(12,2)*{_{3}},
%
 (0,0)*{\bullet}="a",
(6,0)*{\bu}="b",
(12,0)*{\bu}="c",
%
\ar @{<-} "a";"b" <0pt>
\ar @{->} "b";"c" <0pt>
\endxy\Ea
-
\Ba{c}\xy
(0,2)*{_{1}},
(6,2)*{_{2}},
(12,2)*{_{3}},
%
 (0,0)*{\bullet}="a",
(6,0)*{\bu}="b",
(12,0)*{\bu}="c",
%
\ar @{<-} "a";"b" <0pt>
\ar @{->} "b";"c" <0pt>
\endxy\Ea \label{3: morhism f into Gra}
\\
&&+
\Ba{c}\xy
(0,2)*{_{1}},
(6,2)*{_{2}},
(12,2)*{_{3}},
%
 (0,0)*{\bullet}="a",
(6,0)*{\bu}="b",
(12,0)*{\bu}="c",
%
\ar @{<-} "a";"b" <0pt>
\ar @{<-} "b";"c" <0pt>
\endxy\Ea
\ - \
\Ba{c}\xy
(0,2)*{_{2}},
(6,2)*{_{1}},
(12,2)*{_{3}},
%
 (0,0)*{\bullet}="a",
(6,0)*{\bu}="b",
(12,0)*{\bu}="c",
%
\ar @{<-} "a";"b" <0pt>
\ar @{<-} "b";"c" <0pt>
\endxy\Ea
\ + \
\Ba{c}\xy
(0,2)*{_{2}},
(6,2)*{_{1}},
(12,2)*{_{3}},
%
 (0,0)*{\bullet}="a",
(6,0)*{\bu}="b",
(12,0)*{\bu}="c",
%
\ar @{->} "a";"b" <0pt>
\ar @{<-} "b";"c" <0pt>
\endxy\Ea
-
\Ba{c}\xy
(0,2)*{_{1}},
(6,2)*{_{2}},
(12,2)*{_{3}},
%
 (0,0)*{\bullet}="a",
(6,0)*{\bu}="b",
(12,0)*{\bu}="c",
%
\ar @{->} "a";"b" <0pt>
\ar @{<-} "b";"c" <0pt>
\endxy\Ea \nonumber
\Eeqr
which implies
$$
\varphi\left(\Ba{c}
 \begin{xy}
 <0mm,0mm>*{\bu};<0mm,0mm>*{}**@{},
 <0mm,0.69mm>*{};<0mm,3.0mm>*{}**@{-},
 <0.39mm,-0.39mm>*{};<2.4mm,-2.4mm>*{}**@{-},
 <-0.35mm,-0.35mm>*{};<-1.9mm,-1.9mm>*{}**@{-},
 <-2.4mm,-2.4mm>*{\bu};<-2.4mm,-2.4mm>*{}**@{},
 <-2.0mm,-2.8mm>*{};<0mm,-4.9mm>*{}**@{-},
 <-2.8mm,-2.9mm>*{};<-4.7mm,-4.9mm>*{}**@{-},
    <0.39mm,-0.39mm>*{};<3.3mm,-4.0mm>*{^3}**@{},
    <-2.0mm,-2.8mm>*{};<0.5mm,-6.7mm>*{^2}**@{},
    <-2.8mm,-2.9mm>*{};<-5.2mm,-6.7mm>*{^1}**@{},
 \end{xy}
\ + \
 \begin{xy}
 <0mm,0mm>*{\bu};<0mm,0mm>*{}**@{},
 <0mm,0.69mm>*{};<0mm,3.0mm>*{}**@{-},
 <0.39mm,-0.39mm>*{};<2.4mm,-2.4mm>*{}**@{-},
 <-0.35mm,-0.35mm>*{};<-1.9mm,-1.9mm>*{}**@{-},
 <-2.4mm,-2.4mm>*{\bu};<-2.4mm,-2.4mm>*{}**@{},
 <-2.0mm,-2.8mm>*{};<0mm,-4.9mm>*{}**@{-},
 <-2.8mm,-2.9mm>*{};<-4.7mm,-4.9mm>*{}**@{-},
    <0.39mm,-0.39mm>*{};<3.3mm,-4.0mm>*{^2}**@{},
    <-2.0mm,-2.8mm>*{};<0.5mm,-6.7mm>*{^1}**@{},
    <-2.8mm,-2.9mm>*{};<-5.2mm,-6.7mm>*{^3}**@{},
 \end{xy}
\ + \
 \begin{xy}
 <0mm,0mm>*{\bu};<0mm,0mm>*{}**@{},
 <0mm,0.69mm>*{};<0mm,3.0mm>*{}**@{-},
 <0.39mm,-0.39mm>*{};<2.4mm,-2.4mm>*{}**@{-},
 <-0.35mm,-0.35mm>*{};<-1.9mm,-1.9mm>*{}**@{-},
 <-2.4mm,-2.4mm>*{\bu};<-2.4mm,-2.4mm>*{}**@{},
 <-2.0mm,-2.8mm>*{};<0mm,-4.9mm>*{}**@{-},
 <-2.8mm,-2.9mm>*{};<-4.7mm,-4.9mm>*{}**@{-},
    <0.39mm,-0.39mm>*{};<3.3mm,-4.0mm>*{^1}**@{},
    <-2.0mm,-2.8mm>*{};<0.5mm,-6.7mm>*{^3}**@{},
    <-2.8mm,-2.9mm>*{};<-5.2mm,-6.7mm>*{^2}**@{},
 \end{xy}\Ea\right)=0.
$$
\end{proof}

All possible morphisms of dg operads,  $\caL ie_\infty\{2\} \lon \cG ra^\uparrow$, can be usefully encoded as Maurer-Cartan elements in the graded Lie algebra,
$$
\mathsf{f} \sG\sC_3^{or}:= \Def(\caL ie_\infty\{2\} \stackrel{0}{\lon} \cG ra^\uparrow),
$$
which controls deformation theory of the zero morphism (cf.\ \cite{MV}). As a graded vector space,
\Beq \label{equ:GCordef}
\mathsf{f} \sG\sC_3^{or}\cong \prod_{n\geq 2} \Hom_{\bS_n}(E(n), \cG ra^\uparrow(n))[-1]=  \prod_{n\geq 2} \cG ra^\uparrow(n)^{\bS_n}[3-3n],
\Eeq
so that its elements can be understood as ($\K$-linear series of) graphs $\Ga$ from $\cG ra^\uparrow$ whose vertex labels are skewsymmetrized (so that we can often forget numerical labels of vertices in our pictures), and which are assigned the homological degree
$$
|\Ga|= 3\# V(\Ga) -3 - 2\# E(\Ga),
$$
where $V(\Ga)$ (resp.\, $E(\Ga)$) stands for the set of vertices (resp., edges) of $\Ga$.

\mip

The Lie brackets, $[\ , \ ]_{\mathsf{gra}}$, in $\mathsf{f} \sG\sC_3^{or}$  can be either read from the differential (\ref{3: Lie_infty differential}), or, equivalently, from the following explicit Lie algebra structure \cite{KM} associated with the degree shifted operad $\cG ra_3^\uparrow\{3\}$ (and which makes sense for any operad),
$$
\Ba{rccc}
[\ ,\ ]:&  \sP \ot \sP & \lon & \sP\\
& (a\in \cP(n), b\in \cP(m)) & \lon &
[a, b]:= \sum_{i=1}^n a\circ_i b - (-1)^{|a||b|}\sum_{i=1}^m b\circ_i a
\Ea
$$
where
$
\sP:= \prod_{n\geq 1}\cG ra^\uparrow(n)[3-3n]$. These Lie brackets in $\sP$ induce
Lie brackets, $[\ ,\ ]_{\mathsf{gra}}$, in the subspace of $\bS$-coinvariants \cite{KM}. By the isomorphism of invariants and co-invariants we obtain a Lie bracket on the space of invariants
$$
\sP^\bS:=  \prod_{n\geq 1}\cG ra^\uparrow(n)[3-3n]^{\bS_n}= \fGCor_3. %\sG\sC_3^\uparrow
$$
via the standard symmetrization map $\sP\rar \sP^\bS$.


\mip

The graph
$$
\xy
%
 (0,0)*{\bullet}="a",
(7,0)*{\bu}="b",
%
\ar @{->} "a";"b" <0pt>
\endxy:=
\xy
(0,2)*{_{1}},
(7,2)*{_{2}},
%
 (0,0)*{\bullet}="a",
(7,0)*{\bu}="b",
%
\ar @{->} "a";"b" <0pt>
\endxy\
- \
\xy
(0,2)*{_{2}},
(7,2)*{_{1}},
%
 (0,0)*{\bullet}="a",
(7,0)*{\bu}="b",
%
\ar @{->} "a";"b" <0pt>
\endxy
$$
is a degree $2\cdot 3-3-2=1$ element in $\mathsf{fGC}_3^{or}$, which, in fact, is a Maurer-Cartan element,
$$
\left[\xy
%
 (0,0)*{\bullet}="a",
(7,0)*{\bu}="b",
%
\ar @{->} "a";"b" <0pt>
\endxy, \xy
%
 (0,0)*{\bullet}="a",
(7,0)*{\bu}="b",
%
\ar @{->} "a";"b" <0pt>
\endxy \right]= \mathrm{skewsymmetrization\ of\ the\ r.h.s.\ in}\ (\ref{3: morhism f into Gra})=0,
$$
which represents the above morphism $\varphi$ in the Lie algebra $\fGCor_3$. This element makes, therefore, $\mathsf{f}\sG\sC_3^{or}$ into a {\em differential}\, graded Lie algebra
with the differential
\Beq\label{3: d in GC_3}
d\Ga:= \left[\xy
%
 (0,0)*{\bullet}="a",
(7,0)*{\bu}="b",
%
\ar @{->} "a";"b" <0pt>
\endxy, \Ga\right]_{\mathsf{gra}}.
\Eeq
Let $\sG\sC_3^{or}$  be a subspace of $\mathsf{f}\sG\sC_3^{or}$ spanned by connected graphs whose vertices are at least bivalent, and if bivalent do not have one incoming and one outgoing edge. It is easy to see that this is a dg Lie subalgebra.

\begin{remark}
 The definition of $\sG\sC_3^{or}$ in \cite{Wi1} differs slightly from the present one as all bivalent vertices are allowed in loc. cit. However, it is easy to check that this extra condition does not change the cohomology.
\end{remark}

The cohomology of the  {\em oriented graph complex}\, $(\sG\sC_3^{or}, d)$ was partially computed in \cite{Wi1, Wi2}.



\subsubsection{\bf Theorem \cite{Wi2}}\label{3: Willwacher theorem on GC_3} (i) {\em  $H^0(\sG\sC_3^{or}, d)=\fg\fr\ft_1$, where
$\fg\fr\ft_1$ is the Lie algebra of the prounipotent Grothendieck-Teichm\"uller group $GRT_1$
introduced by Drinfeld in \cite{D2}.  }

(ii) {\em $H^{-1}(\sG\sC_3^{or}, d)\cong  \K$. The single class is represented by the graph
$
\Ba{c}\resizebox{4mm}{!}{   \xy
   %\ar@/^1pc/(0,0)*{\bullet};(0,-10)*{\bullet}
   %\ar@/^{-1pc}/(0,0)*{\bullet};(0,-10)*{\bullet}
   \ar@/^0.6pc/(0,5)*{\bullet};(0,-5)*{\bullet}
   \ar@/^{-0.6pc}/(0,5)*{\bullet};(0,-5)*{\bullet}
 \endxy}\Ea
$.}

(iii) {\em $H^i(\sG\sC_3^{or}, d)=0$ for all $i\leq-2$.}


\mip

\subsection{Action on $\hLieBi_\infty$}
%The above action of degree 0 cocyles in $\sG\sC_3^{or}$ on Lie bialgebra structures in fact comes from a right action of
There is a natural action of $\sG\sC_3^{or}$ on the properad $\hLieBi_\infty$ by properadic derivations.
Concretely, for any graph $\Gamma$ we define the derivation $F(\Gamma)\in \Der(\hLieBi_\infty)$ sending the generator $\mu_{m,n}$ of $\hLieBi_\infty$ to the linear combination of graphs
\Beq \label{equ:def GC action 1}
\mu_{m,n}\cdot \Gamma=
 \sum
    \overbrace{
 \underbrace{ \Ba{c}\resizebox{9mm}{!}  {\xy
(0,4.5)*+{...},
(0,-4.5)*+{...},
%
(0,0)*+{\Ga}="o",
(-5,6)*{}="1",
(-3,6)*{}="2",
(3,6)*{}="3",
(5,6)*{}="4",
(-3,-6)*{}="5",
(3,-6)*{}="6",
(5,-6)*{}="7",
(-5,-6)*{}="8",
%
\ar @{-} "o";"1" <0pt>
\ar @{-} "o";"2" <0pt>
\ar @{-} "o";"3" <0pt>
\ar @{-} "o";"4" <0pt>
\ar @{-} "o";"5" <0pt>
\ar @{-} "o";"6" <0pt>
\ar @{-} "o";"7" <0pt>
\ar @{-} "o";"8" <0pt>
\endxy}\Ea
 }_{n\times}
 }^{m\times}
% \xy
%(0,4.5)*+{\ldots},
%(0,-4.5)*+{\ldots},
%
%(0,0)*+{\Ga}="o",
%(-5,5)*{}="1",
%(5,5)*{}="2",
%(5,-5)*{}="3",
%(-5,-5)*{}="4",
%
%\ar @{-} "o";"1" <0pt>
%\ar @{-} "o";"2" <0pt>
%\ar @{-} "o";"3" <0pt>
%\ar @{-} "o";"4" <0pt>
%\endxy
%
%\begin{tikzpicture}[baseline=-.65ex]
%  \node (v) at (0,0) {$\Gamma$};
%  \node (v1) at (-1,1) {$1$};
%  \node (v2) at (0,1) {$\cdots$};
%  \node (v3) at (1,1) {$n$};
%  \node (w1) at (-1,-1) {$1$};
%  \node (w2) at (0,-1) {$\cdots$};
%  \node (w3) at (1,-1) {$m$};
%  \draw (v) edge (v1) edge (v3) edge (w1) edge (w3);
% \end{tikzpicture}
\Eeq
where the sum is taken over all ways of attaching the incoming and outgoing legs such that all vertices are at least trivalent and have at least one incoming and one outgoing edge.

\begin{lemma}\label{lem:GC3 action on LieBi}
 The above formula defines a right action of $\GC_3^{or}$ on $\hLieBi_\infty$.
\end{lemma}
\begin{proof}[Proof sketch]
 We denote by $\bullet$ the pre-Lie product on the deformation complex $\Def(\caL ie_\infty\{2\} \stackrel{0}{\lon} \cG ra^\uparrow)\supset \GC_3^{or}$, so that the Lie bracket on $\GC_3^{or}$ may be written as $[\Gamma, \Gamma']=\Gamma\bullet \Gamma' \pm \Gamma' \bullet \Gamma$ for $\Gamma,\Gamma'\in \GC_3^{or}$.
 Note that
 \[
\mu_{m,n} \cdot (\Gamma \bullet \Gamma')=(\mu_{m,n}^k \cdot \Gamma) \cdot \Gamma' ,
 \]
where it is important that we excluded graphs with bivalent vertices with one incoming and one outgoing edge from the definition of $\sG\sC_3^{or}$. It follows that the formula above defines an action of the graded Lie algebra $\GC_3^{or}$. We leave it to the reader to check that this action also commutes with the differential.
%However, the differential on $\GC_3^{or}$ is given by \eqref{3: d in GC_3} while the differential in $\LieBi_\infty$ is such that (abusing notation slightly)
% \newcommand{\MCgraph}{\xy
%  (0,0)*{\bullet}="a",
% (7,0)*{\bu}="b",
% \ar @{->} "a";"b" <0pt>
% \endxy}
% \[
% d\mu_{m,n} = \mu_{m,n}\cdot \MCgraph.
% \]
% Hence one can compute that
% \begin{align*}
% d(\mu_{m,n}\cdot \Gamma) &= \pm (\mu_{m,n}\cdot \Gamma) \cdot \MCgraph
% =
% \pm \mu_{m,n}\cdot (\Gamma) \bullet \MCgraph)
% =
% \pm \mu_{m,n}\cdot (d\Gamma)
% \pm \mu_{m,n}\cdot (\MCgraph \bullet \Gamma)
% \\&=
% \pm \mu_{m,n}\cdot (d\Gamma)
% \pm (\mu_{m,n}\cdot \MCgraph) \cdot \Gamma
% =\pm \mu_{m,n}\cdot (d\Gamma) \pm (d\mu_{m,n}) \cdot \Gamma.
% \end{align*}
\end{proof}

Of course, by a change of sign the right action may be transformed into a left action and hence we obtain a map of Lie algebras
\[
 F\colon \sG\sC_3^{or}\to \Der(\hLieBi_\infty)\, .
\]
Interpreting the right hand side as a graph complex as in section \ref{sec:defcomplexes}, the map $F$ sends a graph $\Gamma\in \GC_3^{or}$ to the series of graphs
\[
\pm
\sum_{m,n}
 \sum
    \overbrace{
 \underbrace{\Ba{c}\resizebox{10mm}{!}  { \xy
(0,4.5)*+{...},
(0,-4.5)*+{...},
%
(0,0)*+{\Ga}="o",
(-5,6)*{}="1",
(-3,6)*{}="2",
(3,6)*{}="3",
(5,6)*{}="4",
(-3,-6)*{}="5",
(3,-6)*{}="6",
(5,-6)*{}="7",
(-5,-6)*{}="8",
%
\ar @{-} "o";"1" <0pt>
\ar @{-} "o";"2" <0pt>
\ar @{-} "o";"3" <0pt>
\ar @{-} "o";"4" <0pt>
\ar @{-} "o";"5" <0pt>
\ar @{-} "o";"6" <0pt>
\ar @{-} "o";"7" <0pt>
\ar @{-} "o";"8" <0pt>
\endxy}\Ea
 }_{n\times}
 }^{m\times}
%
% \xy
%(0,4.5)*+{\ldots},
%(0,-4.5)*+{\ldots},
%
%(0,0)*+{\Ga}="o",
%(-5,5)*{}="1",
%(5,5)*{}="2",
%(5,-5)*{}="3",
%(-5,-5)*{}="4",
%
%\ar @{-} "o";"1" <0pt>
%\ar @{-} "o";"2" <0pt>
%\ar @{-} "o";"3" <0pt>
%\ar @{-} "o";"4" <0pt>
%\endxy
 %
 %\begin{tikzpicture}[baseline=-.65ex]
 % \node (v) at (0,0) {$\Gamma$};
 % \node (v1) at (-1,1) {};
 % \node (v2) at (0,1) {$\cdots$};
 % \node (v3) at (1,1) {};
 % \node (w1) at (-1,-1) {};
 % \node (w2) at (0,-1) {$\cdots$};
 % \node (w3) at (1,-1) {};
 % \draw (v) edge (v1) edge (v3) edge (w1) edge (w3);
 %\end{tikzpicture}
\]

\begin{remark} \label{rem:Fqiso}
 It can be shown that the map $F : \GC_3^{or}\to \Der(\hLieBi_\infty)$ is a quasi-isomorphism, up to one class in  $\Der(\hLieBi_\infty)$ represented by the series
 \[
  \sum_{m,n}(m+n-2)
  \overbrace{
  \underbrace{
 \Ba{c}\resizebox{10mm}{!}  {\xy
(0,4.5)*+{...},
(0,-4.5)*+{...},
%
(0,0)*{\bu}="o",
(-5,5)*{}="1",
(-3,5)*{}="2",
(3,5)*{}="3",
(5,5)*{}="4",
(-3,-5)*{}="5",
(3,-5)*{}="6",
(5,-5)*{}="7",
(-5,-5)*{}="8",
%
\ar @{-} "o";"1" <0pt>
\ar @{-} "o";"2" <0pt>
\ar @{-} "o";"3" <0pt>
\ar @{-} "o";"4" <0pt>
\ar @{-} "o";"5" <0pt>
\ar @{-} "o";"6" <0pt>
\ar @{-} "o";"7" <0pt>
\ar @{-} "o";"8" <0pt>
\endxy}\Ea
 %
%
%  \begin{tikzpicture}[baseline=-2.5ex, scale=.7]
%  \node[int] (v) at (0,0) {};
% \coordinate (v0) at (-.7,-1);
%  \coordinate (v1) at (-.3,-1);
%  \coordinate (v2) at (.3,-1);
%  \coordinate (v3) at (.7,-1);
%   \coordinate (w0) at (-.7,1);
%  \coordinate (w1) at (-.3,1);
%  \coordinate (w2) at (.3,1);
%  \coordinate (w3) at (.7,1);
%  \draw[-latex] (v) edge (v0) edge (v1) edge (v2) edge (v3)
%     (w1) edge (v) (w0) edge (v) (w2) edge (v) (w3) edge (v);
% \end{tikzpicture}
 }_{n\times}
 }^{m\times}.
 \]
 The result will not be used directly in this paper. The proof is an adaptation of the proof of \cite[Proposition 3]{Wi2} and is given in \cite{CMW}.\footnote{\label{footn:refremoval} A sketch of the proof was contained as an Appendix in the preprint version of this article, but removed following the suggestion of a referee.}
\end{remark}


\mip

\subsection{$GRT_1$ action on Lie bialgebra structures}
The action of the Lie algebra of closed degree zero cocycles $\GCor_{3,cl}\subset \GCor_3$ on $\hLieBi_\infty$ by derivations may be integrated
to an action of the exponential group $\Exp\GCor_{3,cl}$ on $\hLieBi_\infty$ by (continuous) automorphisms.
Hence this exponential group acts on the set of $\hLieBi_\infty$ algebra structures on any dg vector space $\fg$, i.~e., on the set of morphisms of properads
\[
 \hLieBi_\infty \to \End_\fg
\]
by precomposition.
Furthermore, it follows that the cohomology Lie algebra $H^0(\GCor_3)\cong \grt_1$ maps into the Lie algebra of continuous derivations up to homotopy $H^0(\Der(\hLieBi_\infty))$ and the the exponential group $\Exp H^0(\GCor_3)\cong GRT_1$ maps into the set of homotopy classes of continuous automorphisms of $\hLieBi_\infty$.


\begin{remark}\label{rem:incomplete}
Note also that one may define a non-complete version $\GCor_{3,inc}$ of the graph complex $\GCor_3$ by merely replacing the direct product by a direct sum in \eqref{equ:GCordef}. The zeroth cohomology of $\GCor_{3,inc}$ is a non-complete version of the Grothendieck-Teichm\"uller group. Furthermore $\GCor_{3,inc}$ acts on the non-completed operad $\LieBi_\infty$
by derivations, using the formulas \eqref{equ:def GC action 1} of the previous subsection, and hence also on $\LieBi_\infty$ algebra structures. However, these actions can in general not be integrated, whence we work with the completed properad $\hLieBi_\infty$ above.
\end{remark}

%This theorem implies a universal action, up to homotopy, of the group $GRT_1$ on the set of
%genus complete strongly homotopy Lie bialgebra structures in any dg vector space $\fg$ \cite{Wi2}.
Finally, let us describe the action $\GCor_{3,cl}$ on Lie bialgebra structures in yet another form.
We have a sequence of morphisms of dg Lie algebras,
$$
\sG\sC_3^{or}\  \ \lon \ \ \Def(\caL ie_\infty\{2\} \rar \cG ra^\uparrow) \ \ {\lon} \ \
 \Def(\caL ie_\infty\{2\} \xrightarrow{\scriptstyle \{\, ,\, \}} \cE nd_{\mathsf{LieB}(\fg)})
$$
where the first arrow is just the inclusion, and the second arrow is induced by the canonical representation (\ref{3: Gra representation in g_V}) and which obviously satisfies
$$
\rho\circ \varphi \left(\Ba{c}
\xy
 <0mm,0.55mm>*{};<0mm,3.5mm>*{}**@{-},
 <0.5mm,-0.5mm>*{};<2.2mm,-2.2mm>*{}**@{-},
 <-0.48mm,-0.48mm>*{};<-2.2mm,-2.2mm>*{}**@{-},
 <0mm,0mm>*{\bu};<0mm,0mm>*{}**@{},
 <0.5mm,-0.5mm>*{};<2.7mm,-3.2mm>*{_2}**@{},
 <-0.48mm,-0.48mm>*{};<-2.7mm,-3.2mm>*{_1}**@{},
 \endxy\Ea\right)= \{\ ,\ \}\in \Hom\left(\wedge^2 \mathsf{LieB}, \mathsf{LieB}[-2]\right)\subset \Def(\caL ie_\infty\{2\} \xrightarrow{\scriptstyle  \{\, , \, \}} \cE nd_{\mathsf{LieB}(\fg)}).
$$


The dg Lie algebra,
$$
\Def(\caL ie_\infty\{2\} \xrightarrow{\scriptstyle \{\, ,\, \}} \cE nd_{\mathsf{LieB}(\fg)}) =: CE^\bu(\mathsf{LieB}(\fg)),
$$
is nothing but the classical Chevalley-Eilenberg complex controlling deformations of the Poisson brackets (\ref{3: Poisson brackets in LieB(g)}) in $\mathsf{LieB}(\fg)$.
In particular for any closed degree zero element $g\in \sG\sC_3^{or}$, and in particular for representatives of elements of $\fg\fr\ft_1$ in $\sG\sC_3^{or}$, we obtain a $\Lie_\infty$ derivation of $\mathsf{LieB}(\fg)$. This derivation may be integrated into a $\Lie_\infty$ automorphism $\exp(ad_g)$.
For a Maurer-Cartan element $\ga$ in $\mathsf{LieB}(\fg)$ corresponding to a $\hLieBi_\infty$ structure on $\fg$ the series
$$
\ga \lon exp(ad_g) \ga
$$
converges and defines again a $\hLieBi_\infty$ structure on $\fg$.

% on Maurer-Cartan elements of the Lie algebra  $CE^\bu(\mathsf{LieB}(\fg))$
%
% which sends elements of $\fg\fr\ft_1$ into degree zero cyclic elements, $g$, in  $CE^\bu(\mathsf{LieB}(\fg))$, whose gauge action,
%
% preserve the particular element $\ga=\{\ ,\ \}$. Such cycles $g$ give, therefore, rise to
% universal $\caL ie_\infty$-automorphisms of the Lie algebra, $(\mathsf{LieB}(\fg), \{\ ,\ \})$; such an  automorphism acts, in turn, on Maurer-Cartan elements of the latter Lie algebra, that is,
% on the set of $\caL ie\cB_\infty$-algebra structures in $\fg$.


\begin{remark}
 By degree reasons the above action on $\hLieBi_\infty$ structures maps $\hLieBi$ structures again to $\hLieBi$ structures. In other words, no higher homotopies are created if there were none before.
\end{remark}




\subsection{Another oriented graph complex}\label{sec:GChbar}
We shall introduce next a new oriented graph complex and then use Theorem {\ref{3: Willwacher theorem on GC_3}} to partially compute its cohomology and then deduce formulae for an action
of $GRT_1$ on {\em involutive}\, Lie bialgebra structures.


Let $\hbar$ be a formal variable of homological degree $2$. The Lie brackets $[\ ,\
]_{\mathrm{gra}}$ in $\sG\sC_3^{or}$ extend $\hbar$-linearly to the topological
vector space $\sG\sC_3^{or}[[\hbar]]$. %:= \sG\sC_3^{or}\ot_\K \K[[\hbar]]$.



\subsection{Proposition} \label{prop:phihbar} {\em The element
$$
\Phi_\hbar:= \sum_{k=1}^\infty \hbar^{k-1} \underbrace{
\Ba{c}\resizebox{6mm}{!}  {\xy
(0,-5)*{...},
   \ar@/^1pc/(0,0)*{\bullet};(0,-10)*{\bullet}
   \ar@/^{-1pc}/(0,0)*{\bullet};(0,-10)*{\bullet}
   \ar@/^0.6pc/(0,0)*{\bullet};(0,-10)*{\bullet}
   \ar@/^{-0.6pc}/(0,0)*{\bullet};(0,-10)*{\bullet}
 \endxy}
 \Ea}_{k\ \mathrm{edges}}
$$
is a Maurer-Cartan element in the Lie algebra $(\mathsf{f}\sG\sC_3^{or}[[\hbar]], [\ ,\ ]_{\mathsf{gra}})$}.

\smallskip

{\bf Convention:} Here we adopt the convention that a picture of an unlabeled graph with black vertices shall stand for the element of $\fGC_3^{or}/\sG\sC_3^{or}$ (i.e., a symmetrically labelled graph) by summing over all labelings of vertices, and dividing by the order of the symmetry group of the graph. In particular this means that the $k$-th term in the above formula for $\Phi_\hbar$ carries an implicit prefactor $\frac 1 {k!}$. This convention will kill many many prefactors arising in calculations.

\smallskip


\begin{proof}[Proof of Proposition \ref{prop:phihbar}]
\Beqrn
\frac 1 2 [\Phi_\hbar, \Phi_\hbar] &=& %\mathrm{skewsymmetrization\ of}
\sum_{k=1}^\infty \sum_{l=1}^\infty \hbar^{k+l-2}\sum_{k=k'+k''}
\Ba{c}\resizebox{12mm}{!}  {\xy
(0,-5)*{\stackrel{l}{...}},
(0,5)*{\stackrel{k'}{...}},
(8,0)*{\stackrel{k''}{...}},
   %\ar@/^1pc/(0,0)*{\bullet};(0,-10)*{\bullet}
   \ar@/^{-1pc}/(0,0)*{\bullet};(0,-10)*{\bullet}
   \ar@/^0.6pc/(0,0)*{\bullet};(0,-10)*{\bullet}
   \ar@/^{-0.6pc}/(0,0)*{\bullet};(0,-10)*{\bullet}
   %\ar@/^1pc/(0,0)*{\bullet};(0,-10)*{\bullet}
   \ar@/^{-1pc}/(0,10)*{\bullet};(0,0)*{\bullet}
   \ar@/^0.6pc/(0,10)*{\bullet};(0,0)*{\bullet}
   \ar@/^{-0.6pc}/(0,10)*{\bullet};(0,0)*{\bullet}
   %
   \ar@/^{2.4pc}/(0,10)*{\bullet};(0,-10)*{\bullet}
   \ar@/^{1.3pc}/(0,10)*{\bullet};(0,-10)*{\bullet}
 \endxy}
 \Ea -
 \sum_{k=1}^\infty \sum_{l=1}^\infty \hbar^{k+l-2}\sum_{k=k'+k''}
 \Ba{c}\resizebox{12.5mm}{!}  {
 \xy
(0,-5)*{\stackrel{k'}{...}},
(0,5)*{\stackrel{l}{...}},
(8,0)*{\stackrel{k''}{...}},
   %\ar@/^1pc/(0,0)*{\bullet};(0,-10)*{\bullet}
   \ar@/^{-1pc}/(0,0)*{\bullet};(0,-10)*{\bullet}
   \ar@/^0.6pc/(0,0)*{\bullet};(0,-10)*{\bullet}
   \ar@/^{-0.6pc}/(0,0)*{\bullet};(0,-10)*{\bullet}
   %\ar@/^1pc/(0,0)*{\bullet};(0,-10)*{\bullet}
   \ar@/^{-1pc}/(0,10)*{\bullet};(0,0)*{\bullet}
   \ar@/^0.6pc/(0,10)*{\bullet};(0,0)*{\bullet}
   \ar@/^{-0.6pc}/(0,10)*{\bullet};(0,0)*{\bullet}
   %
   \ar@/^{2.4pc}/(0,10)*{\bullet};(0,-10)*{\bullet}
   \ar@/^{1.3pc}/(0,10)*{\bullet};(0,-10)*{\bullet}
 \endxy}
 \Ea \\
 &=& 0.
\Eeqrn
\end{proof}



Hence the degree one continuous map
$$
\Ba{rccc}
d_\hbar: & \sG\sC_3^{or}[[\hbar]] &\lon & \sG\sC_3^{or}[[\hbar]]\\
         &  \Ga &  \lon & d_\hbar\Ga:= [\Phi_\hbar,\Ga]_{\mathrm{gra}}
\Ea
$$
is a differential in $\sG\sC_3^{or}[[\hbar]]$. The induced differential, $d$,  in
$\sG\sC_3^{or}=\sG\sC_3^{or}[[\hbar]]/ \hbar \sG\sC_3^{or}[[\hbar]]$ is precisely the
original
differential (\ref{3: d in GC_3}).


\subsection{Action on $\hLoB_\infty$}\label{sec:action on LoB}
The dg Lie algebra $(\GC_3^{or}[[\hbar]], d_\hbar)$
acts naturally on the properad $\hLoB_\infty$ by continuous properadic derivations. More precisely, let $\Gamma\in \GC_3^{or}$ be a graph.
Then to the element $\hbar^N\Gamma\in \GC_3^{or}[[\hbar]]$ we assign the derivation of $\hLoB_\infty$ that sends the generator $\mu_{m,n}^k$ to zero if $k<N$ and to
\Beq \label{equ:def GC action 2}
\mu_{m,n}^k \cdot (\hbar^N\Gamma)
=
\mathrm{mark}_{k-N}(\mu_{m,n}\cdot \Gamma)
%=
% \sum
%\Ba{c}\resizebox{13mm}{!}{ \xy
% (-5,7)*{^{_1}},
% (-3,7)*{^{_2}},
% (5.5,7)*{^{_m}},
% (-5,-7.9)*{^{_1}},
% (-3,-7.9)*{^{_2}},
% (5.5,-7.9)*{^{_n}},
% %
%(1,4.5)*+{...},
%(1,-4.5)*+{...},
%%
%(0,0)*+{\Ga}="o",
%(-5,6)*{}="1",
%(-3,6)*{}="2",
%(5,6)*{}="3",
%(5,-6)*{}="4",
%(-3,-6)*{}="5",
%(-5,-6)*{}="6",
%%
%\ar @{-} "o";"1" <0pt>
%\ar @{-} "o";"2" <0pt>
%\ar @{-} "o";"3" <0pt>
%\ar @{-} "o";"4" <0pt>
%\ar @{-} "o";"5" <0pt>
%\ar @{-} "o";"6" <0pt>
%\endxy}\Ea
 %
%
% \begin{tikzpicture}[baseline=-.65ex]
%  \node (v) at (0,0) {$\Gamma$};
%  \node (v1) at (-1,1) {$1$};
%  \node (v2) at (0,1) {$\cdots$};
%  \node (v3) at (1,1) {$n$};
%  \node (w1) at (-1,-1) {$1$};
%  \node (w2) at (0,-1) {$\cdots$};
%  \node (w3) at (1,-1) {$m$};
%  \draw (v) edge (v1) edge (v3) edge (w1) edge (w3);
% \end{tikzpicture}
\Eeq
where $\mu_{m,n}\cdot \Gamma$ is a series of graphs obtained attaching external legs to $\Gamma$ in all possible ways as in \eqref{equ:def GC action 1} and the operation $\mathrm{mark}_{k-N}$ assigns weights to the vertices in all possible ways such that the weights sum to $k-N$.

\begin{lemma}\label{lem:action of GC on LoB}
 The above formula defines a right action of $(\GC_3^{or}[[\hbar]], d_\hbar)$ on $\hLoB_\infty$.
\end{lemma}
\begin{proof}[Proof sketch.]
The proof is similar to that of Lemma \ref{lem:GC3 action on LieBi} after noting that
\[
\mu_{m,n}^k \cdot (\hbar^N \Gamma \bullet \hbar^M \Gamma')=(\mu_{m,n}^k \cdot \hbar^N\Gamma) \cdot \hbar^M \Gamma'
\]
for all $M,N$ and $\Gamma,\Gamma'\in \GC_3^{or}$.
\end{proof}


Again, by a change of sign the right action may be transformed into a left action and hence we obtain a map of Lie algebras
\[
 F_\hbar \colon \sG\sC_3^{or}[[\hbar]]\to \Der(\hLoB_\infty)\, .
\]
% Interpreting the right hand side as a graph complex as in section \ref{sec:defcomplexes}, the map $F$ sends a graph $\Gamma$ to the series of graphs
% \[
% \sum_{m,n}
%   \sum
%  \begin{tikzpicture}[baseline=-.65ex]
%   \node (v) at (0,0) {$\Gamma$};
%   \node (v1) at (-1,1) {$1$};
%   \node (v2) at (0,1) {$\cdots$};
%   \node (v3) at (1,1) {$n$};
%   \node (w1) at (-1,-1) {$1$};
%   \node (w2) at (0,-1) {$\cdots$};
%   \node (w3) at (1,-1) {$m$};
%   \draw (v) edge (v1) edge (v3) edge (w1) edge (w3);
%  \end{tikzpicture}
% \]

\begin{remark}\label{rem:Fhbarqiso}
 It can be seen that the map $F_\hbar$ is a quasi-isomorphism, up to classes $T\K[[\hbar]]\subset \Der(\hLoB_\infty)$ where
 \[
  T=
  \sum_{m,n,p}(m+n+2p-2) \hbar^{p}
   \overbrace{
 \underbrace{
 \xy
(0,4.5)*+{...},
(0,-4.5)*+{...},
%
(0,0)*+{_p}*\cir{}="o",
(-5,5)*{}="1",
(-3,5)*{}="2",
(3,5)*{}="3",
(5,5)*{}="4",
(-3,-5)*{}="5",
(3,-5)*{}="6",
(5,-5)*{}="7",
(-5,-5)*{}="8",
%
\ar @{-} "o";"1" <0pt>
\ar @{-} "o";"2" <0pt>
\ar @{-} "o";"3" <0pt>
\ar @{-} "o";"4" <0pt>
\ar @{-} "o";"5" <0pt>
\ar @{-} "o";"6" <0pt>
\ar @{-} "o";"7" <0pt>
\ar @{-} "o";"8" <0pt>
\endxy
%
% \begin{tikzpicture}[baseline=-2.5ex, scale=.7]
%  \node[draw, circle, inner sep=1pt] (v) at (0,0) {$p$};
% \coordinate (v0) at (-.7,-1);
%  \coordinate (v1) at (-.3,-1);
%  \coordinate (v2) at (.3,-1);
%  \coordinate (v3) at (.7,-1);
%   \coordinate (w0) at (-.7,1);
%  \coordinate (w1) at (-.3,1);
%  \coordinate (w2) at (.3,1);
%  \coordinate (w3) at (.7,1);
%  \draw[-latex] (v) edge (v0) edge (v1) edge (v2) edge (v3)
%     (w1) edge (v)
%     (w0) edge (v)
%     (w2) edge (v)
%     (w3) edge (v);
% \end{tikzpicture}
 }_{n\times}
 }^{m\times}.
 \]
 The result will not be used in this paper, and the proof will appear elsewhere \cite{CMW}.\footnotemark[\ref{footn:refremoval}]
\end{remark}



\subsection{Action on involutive Lie bialgebra structures}
By the previous subsection the Lie algebra of degree 0 cocycles in $\sG\sC_3^{or}[[\hbar]]$ acts on the properad $\hLoB_\infty$ by derivations. The action may be integrated to an action of the corresponding exponential group on $\hLoB_\infty$ by continuous automorphisms, and hence also on the set of $\hLoB_\infty$ algebra structures on some dg vector space by precomposition.
Furthermore, the cohomology Lie algebra $H^0(\sG\sC_3^{or}[[\hbar]])$ maps into the the Lie algebra of continuous derivations up to homotopy $H^0(\Der(\hLoB_\infty))$, while the
the exponential group $\exp H^0(\GCor_3[[\hbar]])$ maps into the set of homotopy classes of continuous automorphisms of $\hLoB_\infty$.
By precomposition, we also have a map of Lie algebras $H^0(\sG\sC_3^{or}[[\hbar]])\to H^0(\Def(\hLoB_\infty\to \End_\fg))$ for any $\hLoB_\infty$ algebra $\fg$ and an action of $\exp H^0(\GCor_3[[\hbar]])$ on the set of such algebra structures up to homotopy.
Let us encode these findings in the following corollary.

\begin{corollary}  The Lie algebra $H^0(\GC_3^{or}[[\hbar]])$ and its exponential group $\exp( H^0(\GC_3^{or}[[\hbar]]))$ canonically act on the set of graded complete strong homotopy involutive Lie bialgebra structures on the dg vector space $\alg g$ up to homotopy.
\end{corollary}

\begin{remark}
Note again that, analogously to Remark \ref{rem:incomplete}, we may define a non-complete version of the graph complex $\sG\sC_3^{or}[[\hbar]]$ which acts on the non-complete properad $\LoB_\infty$ by derivations, and hence also on ordinary Lie bialgebra structures. However, these actions can in general not be integrated, whence we prefer to work with the complete version of the graph complex and $\hLoB_\infty$.
\end{remark}

Finally let us give another yet another description of the action of $\GCor_3[[\hbar]]$ on involutive Lie bialgebra structures, strengthening the above result a little.
As usual, the deformation complex of the zero morphism of dg props,
$\mathsf{InvLieB}(\fg):=
\Def(\LoB_\infty\stackrel{0}{\rar} \cE nd_\fg)$
has a canonical dg Lie algebra structure, with the Lie bracket $[\ ,\ ]_{\hbar}$ given explicitly by (\ref{3: brackets in InvLieB(g)}), such that the
Maurer-Cartan
elements are in 1-to-1 correspondence with $\LoB_\infty$-structures on the dg vector space $\fg$. The Maurer-Cartan element $\Phi_\hbar$ in $\mathsf{f}\sG\sC_3^{or}[[\hbar]]$ corresponds to a continuous morphism of operads,
$$
\varphi_\hbar: \caL ie\{2\}[[\hbar]] \lon \cG ra^\uparrow[[\hbar]],
$$
given on the generator of  $\caL ie\{2\}$ by the formula
$$
\varphi_\hbar \left(\Ba{c}
\xy
 <0mm,0.55mm>*{};<0mm,3.5mm>*{}**@{-},
 <0.5mm,-0.5mm>*{};<2.2mm,-2.2mm>*{}**@{-},
 <-0.48mm,-0.48mm>*{};<-2.2mm,-2.2mm>*{}**@{-},
 <0mm,0mm>*{\bu};<0mm,0mm>*{}**@{},
 <0.5mm,-0.5mm>*{};<2.7mm,-3.2mm>*{_2}**@{},
 <-0.48mm,-0.48mm>*{};<-2.7mm,-3.2mm>*{_1}**@{},
 \endxy\Ea\right)=
 \sum_{k=1}^\infty \hbar^{k-1}\left(
 \Ba{c}\resizebox{7mm}{!}{
\xy
(0,2.2)*{_1},
(0,-12.2)*{_2},
(0,-3.5)*{_k},
(0,-5)*{...},
   \ar@/^1pc/(0,0)*{\bullet};(0,-10)*{\bullet}
   \ar@/^{-1pc}/(0,0)*{\bullet};(0,-10)*{\bullet}
   \ar@/^0.6pc/(0,0)*{\bullet};(0,-10)*{\bullet}
   \ar@/^{-0.6pc}/(0,0)*{\bullet};(0,-10)*{\bullet}
 \endxy}
 \Ea -
  \Ba{c}\resizebox{6mm}{!} {\xy
(0,2.2)*{_2},
(0,-12.2)*{_1},
(0,-3.5)*{_k},
(0,-5)*{...},
   \ar@/^1pc/(0,0)*{\bullet};(0,-10)*{\bullet}
   \ar@/^{-1pc}/(0,0)*{\bullet};(0,-10)*{\bullet}
   \ar@/^0.6pc/(0,0)*{\bullet};(0,-10)*{\bullet}
   \ar@/^{-0.6pc}/(0,0)*{\bullet};(0,-10)*{\bullet}
 \endxy}
 \Ea
 \right)\, .
$$
The representation (\ref{3: Gra representation in g_V}) of the operad $\cG ra^\uparrow$ in the deformation complex $\Def(\LieBi_\infty\stackrel{0}{\to} \cP)[-2]$ extends $\hbar$-linearly to a representation $\rho_\hbar$ of $\cG ra^\uparrow[[\hbar]]$ in $\Def(\LoB_\infty\stackrel{0}{\to} \cP)[-2]$ for any properad $\cP$.
Furthermore it is almost immediate to see that the action of $\Lie\{2\}$ on the latter deformation complex factors through the map $\Lie\{2\}\to \cG ra^\uparrow[[\hbar]]$.

%
%
%
%
% $\mathsf{LieB}(\fg)\subset \K[[\psi_i,\eta^i]]$ extends $\hbar$-linearly to a continuous  representation,
% $\rho_\hbar$, of  $\cG ra^\uparrow[[\hbar]]$ in $\mathsf{InvLieB}(\fg)\subset \K[[\psi_i,\eta^i,\hbar]]$. It is almost immediate to see that
% $$
% \rho_\hbar\circ \phi_\hbar \left( \Ba{c}\xy
%  <0mm,0.55mm>*{};<0mm,3.5mm>*{}**@{-},
%  <0.5mm,-0.5mm>*{};<2.2mm,-2.2mm>*{}**@{-},
%  <-0.48mm,-0.48mm>*{};<-2.2mm,-2.2mm>*{}**@{-},
%  <0mm,0mm>*{\bu};<0mm,0mm>*{}**@{},
%  <0.5mm,-0.5mm>*{};<2.7mm,-3.2mm>*{_2}**@{},
%  <-0.48mm,-0.48mm>*{};<-2.7mm,-3.2mm>*{_1}**@{},
%  \endxy\Ea\right)=[\ ,\ ]_\hbar \in \Hom_{cont}\left(\wedge^2 \mathsf{InvLieB(\fg)}, \mathsf{InvLieB(\fg)}[-2]\right)\, .
% $$
It follows that one has a morphism of dg Lie algebras induced by $\varphi_\hbar$
$$
\left(\sG\sC_3^{or}[[\hbar]], d_\hbar\right)\ \  \lon \ \  \Def\left(\caL ie_\infty\{2\}[[\hbar]] \xrightarrow{\varphi_\hbar} \cG ra^\uparrow[[\hbar]]\right)\ \ \stackrel{\rho_\hbar}{\lon}
\ \
 \Def\left(\caL ie_\infty\{2\}[[\hbar]] \xrightarrow{[\ ,\ ]_\hbar} \cE nd_{D}\right)=:CE^\bu\left(D\right)
$$
from the graph complex $\left(\sG\sC_3^{or}[[\hbar]], d_\hbar\right)$ into the Chevalley-Eilenberg dg Lie algebra of $D:=\Def(\LieBi_\infty\stackrel{0}{\to} \cP)[-2]$.
In particular this implies  the following:

% \begin{proposition} { There is a map of
%
% Lie algebra $H^0(GC_3^{or}[[\hbar]],
% d_\hbar)$
% acts (up to homotopy) as derivations of the Chevalley-Eilenberg dg Lie algebra,
% $$
%  CE^\bu\left(\mathsf{InvLieB}(\fg)\right):= \Def\left(\caL ie\{2\}[[\hbar]] \stackrel{[\ ,\ ]_\hbar}{\lon} \cE nd_{\mathsf{InvLieB}(\fg)}\right),
% $$
% controlling deformations of the Lie brackets $[\ ,\ ]_\hbar$ in $\mathsf{InvLieB}(\fg)$ for any dg space $\fg$.}
% \end{proposition}

% \begin{corollary}  The Lie algebra $H^0(\GC_3^{or}[[\hbar]])$ and its exponential group $\exp( H^0(\GC_3^{or}[[\hbar]]))$ canonically act on the space of Maurer-Cartan elements in $\mathsf{InvLieB}(\fg)$ corresponding to graded complete $\LoB_\infty$ structures up to gauge, or equivalently on the space of graded complete $\LoB_\infty$ structures on $\alg g$ up to homotopy.
% \end{corollary}

% \begin{remark}
%  Note that this action is highly non-trivial in general.
%  Consider for example the $\Frob$ algebra $A=H^\bullet(S^2)\cong \K\oplus \K[-2]$.
%  %Note that by degree reasons all $\Frob$ operations
%  Consider an element of $\grt_1$ that corresponds to the cocyle
%  \[
%   \psi\in \Def\left(\Frob_\infty \to \Frob \right)\cong \prod_{m,n} \Hom_{S_m\times S_n}(\Frob^\Koz(m,n), \Frob(m,n))
%   \cong \prod_{m,n} (\Frob^\Koz(m,n)^*)^{S_m\times S_n}[\hbar][-2n+2].
%  \]
% Note that by degree reasons $\psi$ lives in the part of the product with $m=1$, $n=2$ or $n=1$, $m=2$, and in $\hbar$-degree 0.
% Let us write
% \[
%  \psi=
% \]
%
%  Consider now the image of $\psi$ in the deformation complex of $A$,
% \[
%  \tilde \psi\in \Def\left(\Frob_\infty \to \cE nd_A \right)\cong \prod_{m,n} \Hom_{S_m\times S_n}(\Frob^\Koz(m,n), \cE nd_A(m,n) ).
% \]
% If
% \end{remark}



\subsection{\bf Theorem}\label{4: Theorem on cohomology of GC-hbar} {\em  $H^0( \GC_3^{or}[[\hbar]], d_\hbar)\simeq
H^0(\GC_3^{or},d_0)\simeq \grt_1$ as Lie algebras. Moreover, $H^i( \GC_3^{or}[[\hbar]], d_\hbar)=0$
for all $i\leq -2$ and $H^{-1}( \GC_3^{or}[[\hbar]], d_\hbar)\cong \K$, with the single class being represented by
\[
\sum_{k=2}^\infty (k-1)\hbar^{k-2} \underbrace{
\Ba{c}\resizebox{7mm}{!}{ \xy
(0,0)*{...},
   \ar@/^1pc/(0,5)*{\bullet};(0,-5)*{\bullet}
   \ar@/^{-1pc}/(0,5)*{\bullet};(0,-5)*{\bullet}
   \ar@/^0.6pc/(0,5)*{\bullet};(0,-5)*{\bullet}
   \ar@/^{-0.6pc}/(0,5)*{\bullet};(0,-5)*{\bullet}
 \endxy}
 \Ea}_{k\ \mathrm{edges}}.
\]
}
\begin{proof}
First note that the element above is exactly $\frac d {d\hbar} \Phi_\hbar$ and the fact that it is closed follows easily by differentiating the Maurer-Cartan equation
\[
 0 = \frac d {d\hbar} [\Phi_\hbar,\Phi_\hbar]_{\mathrm{gra}} = 2 [\Phi_\hbar,\frac d {d\hbar} \Phi_\hbar]_{\mathrm{gra}}
 =
 2d_\hbar \left( \frac d {d\hbar} \Phi_\hbar \right).
\]
It is easy to see that the cocyle $\frac d {d\hbar} \Phi_\hbar$ cannot be exact, by just considering the leading term in $\hbar$, which is given by the following graph.
\begin{equation}\label{equ:oriented biedge}
\xy
   %\ar@/^1pc/(0,5)*{\bullet};(0,-5)*{\bullet}
   %\ar@/^{-1pc}/(0,5)*{\bullet};(0,-5)*{\bullet}
   \ar@/^0.6pc/(0,5)*{\bullet};(0,-5)*{\bullet}
   \ar@/^{-0.6pc}/(0,5)*{\bullet};(0,-5)*{\bullet}
 \endxy
\end{equation}

Let us write
$$
d_\hbar= \sum_{k=1}^\infty \hbar^{k-1} d_k.
$$
%and let
%$$
%\Ga_\hbar=\sum_{k=1}^\infty \hbar^{k-1} \Ga_k
%$$
%be a degree zero cycle with respect to $d_\hbar$,
%$$
%d_\hbar \Ga_\hbar=0.
%$$
Consider a decreasing filtration of $\sG\sC_3^{or}[[\hbar]]$ by the powers in $\hbar$. The
first term of the associated spectral sequence is
$$
\cE_1= \bigoplus_{i\in \Z} \cE_1^i,\ \ \ \ \cE_1^i=\bigoplus_{p\geq 0}
H^{i-2p}(\sG\sC_3^{or}, d_0) \hbar^p
$$
with the differential equal to $\hbar d_1$.
The main result of \cite{Wi2} states that $H^0(\sG\sC_3^{or}, d_0)\simeq \fg\fr\ft_1$, $H^{\leq -2}(\sG\sC_3^{or}, d_0)=0$ and $H^{-1}(\sG\sC_3^{or}, d_0)\cong \K$, with the single class being represented by \eqref{equ:oriented biedge}. The desired results follow by degree reasons: First, there is clearly no cohomology in $\cE_1$ in degrees $\leq -2$, so that there can be no such cohomology in $\sG\sC_3^{or}[[\hbar]]$. The single class in $\cE_1$ of degree $-1$ may, as we just saw above, be extended to a cocycle in $\sG\sC_3^{or}[[\hbar]]$, and hence will be killed by all further differentials in the spectral sequence. Hence no elements of degree $0$ in $\cE_1$ can be rendered exact on later pages of the spectral sequence. Hence the only thing that remains to be shown is that the degree 0 elements in $\cE_1$ can be extended to cocycles, i.e., that they are closed on all further pages of the spectral sequence. However, the differential on later pages will necessarily increase the number of $\hbar$'s occurring. Hence the differential on later pages will map the degree 0 part of $\cE_1$ (i.e., $\grt_1$) into (subquotients of) $H^{1-2p}(\sG\sC_3^{or}, d_0)$ for $p\geq 1$. But by the aforementioned vanishing result of \cite{Wi2}, there are no such classes, except possibly for $p=1$, when $H^{-1}(\sG\sC_3^{or}, d_0)\cong \K$. However, the relevant cocycle is represented by $\hbar$ times the two-vertex graph \eqref{equ:oriented biedge}, which cannot be "hit" because the differential increases the number of vertices by one, and all elements of $\grt_1$ are represented by graphs with more than one (in fact, more than 6) vertices. (See also the following section.)
\end{proof}


\subsubsection{\bf Remark} {\em The above result in particular provides us with an action of the group $GRT_1$ on the set of
$\widehat{\LoB}_\infty$-structures up to homotopy on an arbitrary differential graded vector space $\fg$}.


\subsubsection{\bf Iterative construction of graph representatives
of elements of $\fg\fr\ft_1$}\label{4: subsubsection on iterated construction}
The above Theorem {\ref{4: Theorem on cohomology of GC-hbar}} says that any degree zero graph $\Ga\in \sG\sC_3^{or}$
satisfying the cocycle condition, $d_0\Gamma_0=0$, can be extended to a formal power
series,
$$
\Ga_\hbar=\Ga_0+\hbar \Gamma_1 + \hbar^2 \Gamma_2+\ldots,
$$
satisfying the  cocycle condition
$
d_\hbar\Ga_\hbar=0.
$
Let us show how this inductive extension works in detail.
The equation $d_\hbar^2=0$ implies, for any $n\geq 0$,
$
\sum_{n=i+j \atop i,j\geq 0} d_id_j=0,
$
which in turn reads,
\Beqrn
d_0^2&=&0\\
d_0d_1+ d_1d_0&=&0\\
d_0d_2+ d_2d_0 + d_1^2&=&0\ \ \ \mbox{etc}.\\
\Eeqrn
Thus the equation $d_0\Ga_0=0$ implies
$$
0=d_1d_0 \Ga_0=- d_0d_1\Ga_0
$$
The oriented graph $d_1\Ga_0\in \sG\sC_3^{or} $ has degree $-1$ and $H^{-1}(
\sG\sC_3^{or}, d_0)\cong \K$.
Since the one cohomology class cannot be hit (its leading term has necessarily only two vertices), there exists a degree $-2$ graph $\Gamma_1$ such that
$
d_1\Ga_0=-d_0\Ga_1
$
so that
$$
d_\hbar (\Gamma_0 + \hbar \Ga_1)=0 \bmod \hbar^2
$$
Assume by induction that we constructed a degree zero polynomial,
$$
\Ga_0+\hbar \Ga_1 +\ldots + \hbar^n \Ga_n \in \sG\sC_3^{or}[[\hbar]]
$$
such that
\Beq\label{induction}
d_\hbar(\Ga_0+\hbar \Ga_1 +\ldots + \hbar^n \Ga_n)=0\bmod \hbar^{n+1}.
\Eeq
Let us show that there exists an oriented graph $\Ga_{n+1}$ of degree $-2n-2$
such that
$$
d_\hbar(\Ga_0+\hbar \Ga_1 +\ldots + \hbar^n \Ga_n + \hbar^{n+1}\Ga_{n+1})=0\bmod
\hbar^{n+2}.
$$
or, equivalently, such that
\Beq\label{n+1 induction step}
d_0\Ga_{n+1} + d_{n+1}\Ga_0 + \sum_{n+1=i+j\atop i,j\geq 1} d_i\Ga_j=0.
\Eeq


Equation (\ref{induction}) implies, for any $j\leq n$,
$$
d_0\Ga_j + d_j\Ga_0 + \sum_{j=p+q\atop p,q\geq 1} d_{p}\Ga_q=0.
$$
We have
\Beqrn
0&=& d_{n+1}d_0\Ga_0
=-d_0d_{n+1}\Ga_0 -  \sum_{n+1=i+j\atop i,j\geq 1} d_id_j\Ga_0\\
&=& -d_0d_{n+1}\Ga_0 +
 \sum_{n+1=i+j\atop i,j\geq 1} d_i d_0\Ga_j +  \sum_{n+1=i+p+q\atop i,p,q\geq 1} d_i
 d_{p}\Ga_q\\
 &=& -d_0d_{n+1}\Ga_0 -  \sum_{n+1=i+j\atop i,j\geq 1} d_0 d_i\Ga_j
 -   \sum_{n+1=i+p+q\atop i,p,q\geq 1} d_i d_{p}\Ga_q +  \sum_{n+1=i+p+q\atop i,p,q\geq 1}
 d_i d_{p}\Ga_q\\
 &=& -d_0\left(d_{n+1}\Ga_0 +  \sum_{n+1=i+j\atop i,j\geq 1}  d_i\Ga_j\right)
\Eeqrn
As $H^{-1-2n}(\sG\sC_3^{or}, d_0)=0$ for all $n\geq 1$, there exists a degree $-2-2n$ graph $\Gamma_{n+1}$
such that the required equation (\ref{n+1 induction step}) is satisfied. This completes an
inductive construction of $\Ga_\hbar$ from $\Ga_0$.


% \subsection{A twisted operad} To any dg operad $\cP$ and any morphism of operads $f:\caL ie\{k\}\rar \cP$,
% one can associate a {\em twisted}\, dg operad $Tw\cP$ on which the dg Lie algebra
% $\Def(\caL ie\{k\}\stackrel{f}{\lon}\cP)$ acts as derivations \cite{Wi1}. Let   $Tw\cG ra_3^{or}=\{Tw\cG ra_3^{or}(n)\}$ be the twisted dg operad
% associated to the data $\phi: \caL ie\{2\}\rar \cG ra_3^{or}$; as a grade vector space $Tw\cG ra_3^{or}(n)$  is spanned by oriented graphs $\Ga$ whose set of vertices, $V(\Ga)$,
% splits into a disjoint union,
% $$
% V(\Ga)= V_\circ(\Ga) \sqcup V_\bu(\Ga),
% $$
% such that the cardinality of the subset of white vertices, $V_\circ(\Ga)$, equals $n$ while
% the cardinality of the subset of  black vertices, $V_\bu(\Ga)$, can be zero or any natural number; note also that an isomorphism $V_\circ(\Ga)\rar [n]$ is fixed
% (so that  $Tw \cG ra_3^{or}(n)$ is naturally an $\bS_n$-module). Following \cite{Wi2}, we denote by
% $\cG raphs_3^{or}$  the quotient of $Tw\cG ra_3^{or}$ by the ideal generated by oriented graphs which have at least one black vertex with no outgoing edges, and/or at least white
% vertex with an outgoing edge. By construction,  $\cG \mathit{raphs}_3^{or}$ is spanned by graphs $\Ga$ such that (i) every black vertex of $\Ga$  is connected by a directed path of edges to a white vertex, (ii) white vertices have only incoming edges, and (iii) an isomorphism, . The homological degree is defined by the formula,
% $$
% |\Ga|= 3\# V_\bu(\Ga) - 2 \#E(\Ga)
% $$
% The operadic composition,
% $i\in [n]$,
% $$
% \Ba{rccc}
% \circ_i: & \cG raphs_3^{or}(n)\ot \cG raphs_3^{or}(n) & \lon & \cG raphs_3^{or}(n+m-1) \\
%          &     \Ga_1 \circ \Ga_2 & \lon &  \Ga_1\circ_i \Ga_2
% \Ea
% $$
% is given by substitution of the graph $\Ga_2$ into the $i$-th labelled white vertex of
% $\Ga_1$ and then taking a sum over all possible gluings  of hanging half-edges to both white and black vertices of $\Ga_2$. The differential, $\delta$, in $\cG raphs_3^{or}$ is completely determined
% by the differential $d$ in $\sC \sG_3^{or}$ (see (\ref{3: d in GC_3}))
% $$
% \delta \Ga:= d\Ga +
%  \xy(0,2)*{_{1}},
% %
%  (0,0)*{\circ}="a",
% (6,0)*{\bu}="b",
% %
% \ar @{<-} "a";"b" <0pt>
% \endxy \circ_1 \Ga\ \
% - \ \ (-1)^{|\Ga|} \sum_{v\in V_\circ(\Ga)} \Ga \circ_v
% \xy(0,2)*{_{1}},
% %
%  (0,0)*{\circ}="a",
% (6,0)*{\bu}="b",
% %
% \ar @{<-} "a";"b" <0pt>
% \endxy
% $$
% where $d$ acts only on the black vertices of $\Ga$. It was
% proven in \cite{Wi2} that the cohomology operad
% $H^\bu({\cG raphs}_3^{or}, \delta)$ equals  operad, $\cG$, of Gerstenhaber
% algebras.
%
%
%
%
% \bip
%
% Similarly one can define a twisted dg operad corresponding to the
% morphism $\phi_\hbar: \caL ie\{2\}[[\hbar]] \rar \cG ra^\uparrow[[\hbar]]$. As an operad,
% it is isomorphic to $\cG raphs_3^{or}[[\hbar]]$, while the differential is given by the formula
% $$
%  \delta_\hbar\Ga=  \sum_{k=1}^\infty \hbar^{k-1} (-1)^{|\Ga|}\sum_{v\in V_\bu(\Ga)} \Ga \bu_v
%  \Ba{c}\xy
% (0,-12.2)*{_1},
% (0,-3.5)*{_k},
% (0,-5)*{...},
%    \ar@/^1pc/(0,0)*{\bullet};(0,-10)*{\bu}
%    \ar@/^{-1pc}/(0,0)*{\bullet};(0,-10)*{\circ}
%    \ar@/^0.6pc/(0,0)*{\bullet};(0,-10)*{\circ}
%    \ar@/^{-0.6pc}/(0,0)*{\bullet};(0,-10)*{\circ}
%  \endxy
%  \Ea
%  \
%   +\
%     \sum_{k=1}^\infty \hbar^{k-1} \left( \Ba{c}\xy
% (0,-12.2)*{_1},
% (0,-3.5)*{_k},
% (0,-5)*{...},
%    \ar@/^1pc/(0,0)*{\bullet};(0,-10)*{\circ}
%    \ar@/^{-1pc}/(0,0)*{\bullet};(0,-10)*{\circ}
%    \ar@/^0.6pc/(0,0)*{\bullet};(0,-10)*{\circ}
%    \ar@/^{-0.6pc}/(0,0)*{\bullet};(0,-10)*{\circ}
%  \endxy
%  \Ea
%  \ \circ_1   \Ga    \ - \ \    (-1)^{|\Ga|} \sum_{v\in V_\circ(\Ga)} \Ga \circ_v
%   \Ba{c}\xy
% (0,-12.2)*{_1},
% (0,-3.5)*{_k},
% (0,-5)*{...},
%    \ar@/^1pc/(0,0)*{\bullet};(0,-10)*{\circ}
%    \ar@/^{-1pc}/(0,0)*{\bullet};(0,-10)*{\circ}
%    \ar@/^0.6pc/(0,0)*{\bullet};(0,-10)*{\circ}
%    \ar@/^{-0.6pc}/(0,0)*{\bullet};(0,-10)*{\circ}
%  \endxy
%  \Ea
%   \right)
% $$
% where $\bu_v \ga$ means insertion of a graph $\ga$ into the black vertex $v$ of $\Ga$.
% We can decompose differential as a formal power series in $\hbar$: $\delta_\hbar
% =\delta+ \hbar \delta_1 + \ldots $.
%
%
% \subsubsection{\bf Proposition} {\em The cohomology operad $H^\bu(\cG raphs_3^{or}[[\hbar]],
% \delta_\hbar)$ is isomorphic to $\caL ie\{1\}$}.
%
% \begin{proof} Consider a filtration of the dg operad $\cG raphs_3^{or}[[\hbar]]$ by powers
% of $\hbar$. The zero-th term, $(\cE_0, d_0)$ of the associated spectral sequence is
% $(\cG raphs_3^{or}[[\hbar]], d_0)$ so that the next term
%  first term, $\cE^1=H^\bu(\cE_0,d_0)$,  of the spectral sequence is equal, according
%  to \cite{Wi2}, to the Gerstenhaber operad with coefficients in $\K[[\hbar]]$,
% $$
% \cE_1= \cG[[\hbar]]
% $$
% The induced differential in $\cE_1$ is given by
% $$
% d_1\left(\mathrm{generator\ of}\ \cC omm\right)= \hbar \left(
% \mathrm{generator\ of}\ \caL ie\{1\}\right).
% $$
% Hence the spectral sequence generates at $\cE_2=\caL ie\{1\}$ implying the claim.
% \end{proof}
%
%
% Therefore, contrary to the case of $(\cG raphs_3^{or}, \delta)$, the dg operad $(\cG raphs_3^{or}[[\hbar]], \delta_\hbar)$ is not much interesting. This fact prompts us to consider a ``weighted" version of the operad $\cG ra^\uparrow_3$ and then its twisted operad; it will be shown in the next Section  that the latter is closely related not only to the theory of involutive Lie bialgebras but also to the theory of (strongly homotopy) Batalin-Vilkovisky algebras.
%

%
%
% \subsubsection{\bf A twisted operad $\cG raphs^\varpi_3$} (REWRITE, WRONG GRADINGS) As the operad $\cG ra_3^\varpi$
% comes equipped with a morphism (\ref{4: morphism phi-varpi}),
% it can be twisted, $\cG ra_3^\varpi\rar Tw \cG ra_3^\varpi$ along the general principles described in \cite{Wi1}; let $\cG raphs_3$ the quotient of  $Tw \cG ra_3^\varpi$ by the ideal generated by oriented graphs which have at least one ``black" vertex with no outgoing edges, and/or at least ``white"
% vertex with an outgoing edge.
%  \sip
%
%
%  The dg operad $\cG raphs_3$  contains a suboperad $\Ga\in \cG raphs_3^\varpi\subset \cG raphs_3$ spanned by graphs such that all white vertices have weight $0$, and the output weight is $0$.
% % differential  ideal $I$ generated by graphs $\Ga$ such that at least one
% %``white" vertex of $\Ga$ has weight $\geq 1$. The quotient dg operad, $\cG raphs^\varpi_3:= \cG raphs_3/I$,
% It is spanned thereby by graphs $\Ga\in \cG raphs_3^\varpi(n)$ with $n$ unweighted (but labelled by elements of $[n]$) ``white" vertices and any number of weighted (but unlabelled) ``black" vertices (which we show in our pictures as big circles rather than as black bullets), e.g.
% $$
% \Ba{c}\xy
%  (0,7)*+{_{1}}*\frm{o};
% (0,0)*+{_0}*\frm{o}
% **\dir{-};
%    \ar@/^0.6pc/(0,0)*+{_{0}}*\frm{o};(0,-10)*{\circ}*\frm{}
%    \ar@/^{-0.6pc}/(0,0)*+{_{0}}*\frm{o};(0,-10)*{\circ}*\frm{}
%  \endxy
%  \Ea  \in \cG raphs_3^\varpi(1).
% $$
% The homological degree of a graph $\Ga\in \cG raphs_3^\varpi$ is given by the formula
% $$
% |\Ga|=3\# V_\bu(\Ga) - 2 \# E(\Ga) - 2 \varpi(\Ga),
% $$
% where $\varpi(\Ga)=\sum_{v\in V_\bu(\Ga)} \varpi(v)$ is the total weight of $\Ga$.
% The differential in  $ \cG raphs_3^\varpi$ is given by the formula,
% $$
% \delta_\varpi\Ga=
% (-1)^{|\Ga|}\sum_{k=1}^\infty \sum_{k-1=a+b\atop a,b\geq 0}  \sum_{v\in V_\bu(\Ga)} \Ga \bu_v
% \left(
% \Ba{c}\xy
% (0,-3.5)*{_k},
% (0,-5)*{...},
%    \ar@/^1pc/(0,0)*+{_{a}}*\frm{o};(0,-10)*+{_{b}}*\frm{o}
%    \ar@/^{-1pc}/(0,0)*+{_{a}}*\frm{o};(0,-10)*+{_{b}}*\frm{o}
%    \ar@/^0.6pc/(0,0)*+{_{a}}*\frm{o};(0,-10)*+{_{b}}*\frm{o}
%    \ar@/^{-0.6pc}/(0,0)*+{_{a}}*\frm{o};(0,-10)*+{_{b}}*\frm{o}
%  \endxy
%  \Ea\right)
% \ +  \
%   \Ba{c}\xy
% (0.0,-12.2)*{_1},
%    \ar@/^{-0.0pc}/(0,0)*+{_{0}}*\frm{o};(0,-10)*{\circ}*\frm{}
%  \endxy
%  \Ea
%  \circ_1 \Ga
%  - (-1)^{|\Ga|} \sum_{v\in V_\circ(\Ga)}  \Ga\circ_v
%   \Ba{c}\xy
% (0.0,-12.2)*{_1},
%    \ar@/^{-0.0pc}/(0,0)*+{_{0}}*\frm{o};(0,-10)*{\circ}*\frm{}
%  \endxy
%  \Ea
% $$
%  The cohomology of the operad $(\cG raphs_3^\varpi, \delta_\varpi)$ turns out to an important operad of Batalin-Vilkovisky algebras which we discuss next.
%
%
%
% \bip




\subsection{Deformations of Frobenius algebra structures}
Note that the complexes $\Def(\invFrob_\infty\to \invFrob)$ and $\Def(\LieBi_\infty\to \hLieBi)$ are isomorphic. We hence have a zigzag of (quasi-)isomorphisms of complexes
\[
 \Der(\invFrob_\infty) \to \Def(\invFrob_\infty\to \invFrob)[1]\cong \Def(\LieBi_\infty\to \hLieBi)[1]\leftarrow \Der(\hLieBi_\infty).
\]
In particular we obtain a map\footnote{In fact, the first arrow is an injection and almost an isomorphism by Remark {\ref{rem:Fqiso}}.}
\[
 \grt_1 \to H^0(\Der(\hLieBi_\infty))\cong H^0(\Der(\invFrob_\infty)).
\]
Hence we obtain a large class of homotopy non-trivial derivations of the properad $\invFrob_\infty$ and accordingly a large class of potentially homotopy non-trivial universal deformations of any $\invFrob_\infty$ algebra.

\begin{remark}
From the above map $\grt_1\to H^0(\Der(\invFrob_\infty))$ we obtain a map $\grt_1\to H^1(\Def(\invFrob_\infty\to \End_A))$ for any $\invFrob_\infty$ algebra $A$, and hence a large class of universal deformations of $\invFrob_\infty$ structures on $A$.
\end{remark}

Next consider the Frobenius properad $\Frob$ and let $\hFrob$ be its genus completion.
Analogously to section {\ref{sec:completed versions}} let $\hFrob_\infty$ be the completion of $\Frob_\infty$ with respect to the total genus and let $\Der(\hFrob_\infty)$ be the continuous derivations.
Note that the complex $\Def(\Frob_\infty\to \hFrob)$ is isomorphic to the complex $\Def(\LoB_\infty\to \hLoB)$.
We hence obtain a zigzag of quasi-isomorphisms
\[
  \Der(\hFrob_\infty) \to \Def(\Frob_\infty\to \hFrob)[1]\cong \Def(\LoB_\infty\to \hLoB)[1]\leftarrow \Der(\hLoB_\infty).
\]
In particular we obtain a map
\[
 \grt_1 \to H^0(\Der(\hLoB_\infty))\cong H^0(\Der(\hFrob_\infty)).
\]
Consider the explicit construction of representatives of $\grt_1$-elements of section {\ref{4: subsubsection on iterated construction}}.
The $\hbar^n$-correction term $\Gamma_n$ to some graph cohomology class $\Gamma$ of genus $g$ has genus $g+n$. It follows that the map $\grt_1\to H^0(\Der(\hFrob_\infty))$ in fact factors through $H^0(\Der(\Frob_\infty))$ and in particular we have a map
\[
\grt_1\to  H^0(\Der(\Frob_\infty))
\]
and hence a map from $\grt_1$ into the deformation complex of any $\Frob_\infty$ algebra.

\begin{remark}
From the map $\grt_1\to H^0(\Der(\Frob_\infty))$ we obtain a map $\grt_1\to H^1(\Def(\Frob_\infty\to \End_A))$ for any $\Frob_\infty$ algebra $A$, and hence a large class of universal deformations of $\Frob_\infty$ structures on $A$.
\end{remark}

\bip



{\large
\section{\bf Involutive Lie bialgebras as homotopy Batalin-Vilkovisky algebras}
}


Let $\fg$ be a Lie bialgebra. Then it is a well known fact the Chevalley-Eilenberg complex of $\fg$ (as a Lie coalgebra) $CE(\fg)=\odot^\bu\fg[-1]$ carries a Gerstenhaber algebra structure.
Concretely, the commutative algebra structure is the obvious one.
To define the Lie bracket (of degree -1) it is sufficient to define it on the generators $\fg[-1]$, where it is given by the Lie bracket on $\fg$.

\sip

Similarly, if $\fg$ is an $\LoB$ algebra, then $CE(\fg)=\odot^{\bu}\fg[-1]$ carries a natural Batalin-Vilkovisky (BV) algebra structure.
The product and Lie bracket are as before. The BV operator $\Delta$ is defined on a word $x_1\cdots x_n$ as
\[
 \Delta (x_1\cdots x_n) = -\sum_{i<j} (-1)^{i+j} [x_i,x_j]x_1\cdots\hat x_i \cdots\hat x_j \cdots x_n.
\]
The involutivity condition is needed for the BV operator to be compatible with the differential.
Now suppose that $\fg$ is a $\LieBi_\infty$ algebra.
We call it \emph{good} if for any fixed $m$ only finitely many of the generating operations $\mu_{m,n}\in \Hom(\fg^{\otimes n}, \fg^{\otimes m})$ are non-zero.
Then one may define the Chevalley complex $CE(\fg)=\odot^{\bu}\fg[-1]$ of $\fg$ as a $\caL ie_\infty$ coalgebra.
It is known (see, e.~g. Remark 1 of \cite{Wi2}) that $CE(\fg)=\odot^{\bu}\fg[-1]$ carries a natural homotopy Gerstenhaber structure.
In this section we show that similarly, if $\fg$ is a good $\LoB_\infty$ algebra, then the Chevalley-Eilenberg complex $CE(\fg)$ carries a natural homotopy BV algebra structure.

\bip


\subsection{The order of an operator}\label{5: subsec on order of operators} Let $V$ be a graded commutative algebra. For a linear operator $D: V\rar V$ define a collection,
$$
\Ba{rccc}
F_n^D: & \ot^n V &\lon & V\\
 & v_1\ot\ldots\ot v_n & \lon & F_n^D(v_1,\ldots,v_n)
\Ea
$$
of linear maps by induction: $F_1^D=D$,
\Beqrn
F_{n+1}^D(v_1,\ldots,v_{n-1},v_n, v_{n+1})&=& F_n^D(v_1,\ldots,v_{n-1}, v_n\cdot v_{n+1}) -
F_n^D(v_1,\ldots,v_n)\cdot v_{n+1} \\
&& - (-1)^{|v_n||v_{n+1}|} F_n^D(v_1,\ldots,v_{n-1},v_{n+1})\cdot v_{n}.
\Eeqrn
The operator $D$ is said to have {\em order}\, $\leq n$ if $F_{n+1}^D=0$.

\mip

 The operators
$F_n^{D}$ are in fact graded symmetric; moreover, if $D$ is a differential in $V$ (that is, $|D|=1$ and $D^2=0$), then the collection, $\{F_n^D: \odot^n V\rar  V\}_{n\geq 1}$  defines a  $\caL ie_\infty$-structure on the space $V[-1]$ (see \cite{Kr}). Indeed, consider
a graded Lie algebra,
$$
\mathrm{CoDer}(\otimes^{\bu \geq 1} V)\cong \prod_{n\geq 1} \Hom_\K(\otimes^n V,V),
$$
of coderivations of the tensor coalgebra $\otimes^{\bu\geq 1} V$. As the differential $D:V\rar V$ is a Maurer-Cartan element in this Lie algebra and the multiplication $\mu:\odot^2 V\rar V$
is its degree zero element, we can gauge transform $D$,
$$
D\lon F^D:=e^{-\mu} D e^{\mu}=\sum_{n=0}^\infty \frac{1}{n!}[\ldots[[D,\mu],\mu], \ldots, \mu],
$$
into a less trivial codifferential whose components
the associated components,
$$
 F^D=\left\{F^D_{n+1}=\frac{1}{n!}[\ldots[[D,\mu],\mu], \ldots, \mu]: \odot^{n+1} V\rar V\right\}_{n\geq 0}
$$
coincide precisely with the defined above tensors $F^D_{n+1}$ which measure a failure of $D$ to respect the multiplication operation in $V$. There is a standard symmetrization functor which associates
to any $\cA_\infty$ algebra an associated $\caL ie_\infty$ algebra; as tensors  $F^D_{n+1}$ are already
graded symmetric \cite{Kr}, the collection  $\{F^D_{n+1}\}_{n\geq 1}$ gives us a $\caL ie_\infty$ structure in $V[-1]$ as required.

\subsection{Batalin-Vilkovisky algebras} A {\em Batalin-Vilkovisky algebra}\, is, by definition, a graded commutative algebra $V$ equipped with a degree $-1$ operator
$\Delta:V \rar V$ of order $\leq 2$ such that $\Delta^2=0$. Denote by $\cB\cV$ the operad whose representations
are Batalin-Vilkovisky algebras. This is, therefore,  a graded operad generated by corollas,
$$
\Ba{c}\resizebox{3.6mm}{!}{ \xy
(0,5)*{};
(0,0)*+{_1}*\cir{}
**\dir{-};
(0,-5)*{};
(0,0)*+{_1}*\cir{}
**\dir{-};
\endxy}\Ea\ \ \ \ \mbox{and}\ \ \ \
\begin{xy}
 <0mm,0.66mm>*{};<0mm,3mm>*{}**@{-},
 <0.39mm,-0.39mm>*{};<2.2mm,-2.2mm>*{}**@{-},
 <-0.35mm,-0.35mm>*{};<-2.2mm,-2.2mm>*{}**@{-},
 <0mm,0mm>*{\circ};<0mm,0mm>*{}**@{},
   <0.39mm,-0.39mm>*{};<2.9mm,-4mm>*{^2}**@{},
   <-0.35mm,-0.35mm>*{};<-2.8mm,-4mm>*{^1}**@{},
\end{xy}=
\begin{xy}
 <0mm,0.66mm>*{};<0mm,3mm>*{}**@{-},
 <0.39mm,-0.39mm>*{};<2.2mm,-2.2mm>*{}**@{-},
 <-0.35mm,-0.35mm>*{};<-2.2mm,-2.2mm>*{}**@{-},
 <0mm,0mm>*{\circ};<0mm,0mm>*{}**@{},
   <0.39mm,-0.39mm>*{};<2.9mm,-4mm>*{^1}**@{},
   <-0.35mm,-0.35mm>*{};<-2.8mm,-4mm>*{^2}**@{},
\end{xy}
$$
of homological degrees $-1$ and $0$ respectively, modulo the following relations,
$$
\Ba{c}\resizebox{5mm}{!}{ \xy
(0,0)*+{_1}*\cir{}="b",
(0,6)*+{_1}*\cir{}="c",
%
%%%%%%%%%% edges to b %%%%%%%%%%%%
(0,-4)*{}="-1",
%%%%%%%%%% edges to c %%%%%%%%%%%%
(0,10)*{}="1'",
%%%%%%%%%%% internal curved edges %%%%%%%%%%%
\ar @{-} "b";"c" <0pt>
%
\ar @{-} "b";"-1" <0pt>
\ar @{-} "c";"1'" <0pt>
\endxy}
\Ea=0\ \ \ ,\ \ \ \Ba{c}\begin{xy}
 <0mm,0mm>*{\circ};<0mm,0mm>*{}**@{},
 <0mm,0.69mm>*{};<0mm,3.0mm>*{}**@{-},
 <0.39mm,-0.39mm>*{};<2.4mm,-2.4mm>*{}**@{-},
 <-0.35mm,-0.35mm>*{};<-1.9mm,-1.9mm>*{}**@{-},
 <-2.4mm,-2.4mm>*{\circ};<-2.4mm,-2.4mm>*{}**@{},
 <-2.0mm,-2.8mm>*{};<0mm,-4.9mm>*{}**@{-},
 <-2.8mm,-2.9mm>*{};<-4.7mm,-4.9mm>*{}**@{-},
    <0.39mm,-0.39mm>*{};<3.3mm,-4.0mm>*{^3}**@{},
    <-2.0mm,-2.8mm>*{};<0.5mm,-6.7mm>*{^2}**@{},
    <-2.8mm,-2.9mm>*{};<-5.2mm,-6.7mm>*{^1}**@{},
 \end{xy}\Ea
\ = \
 \Ba{c}\begin{xy}
 <0mm,0mm>*{\circ};<0mm,0mm>*{}**@{},
 <0mm,0.69mm>*{};<0mm,3.0mm>*{}**@{-},
 <0.39mm,-0.39mm>*{};<2.4mm,-2.4mm>*{}**@{-},
 <-0.35mm,-0.35mm>*{};<-1.9mm,-1.9mm>*{}**@{-},
 <2.4mm,-2.4mm>*{\circ};<-2.4mm,-2.4mm>*{}**@{},
 <2.0mm,-2.8mm>*{};<0mm,-4.9mm>*{}**@{-},
 <2.8mm,-2.9mm>*{};<4.7mm,-4.9mm>*{}**@{-},
    <0.39mm,-0.39mm>*{};<-3mm,-4.0mm>*{^1}**@{},
    <-2.0mm,-2.8mm>*{};<0mm,-6.7mm>*{^2}**@{},
    <-2.8mm,-2.9mm>*{};<5.2mm,-6.7mm>*{^3}**@{},
 \end{xy}\Ea,\ \ \
 %%%%%%%%%%%%%%%%%%%%%%%%%%%%%%
 \Ba{c}\resizebox{10mm}{!}{ \xy
(-6,-15.6)*+{_1};
(0,-15.6)*+{_2};
(3,-11.6)*+{_3};
 %
(0,-1)*+{_1}*\cir{}="b",
(0,-6)*{\circ}="c1",
(-3,-10)*{\circ}="c2",
%
(3,-10)*{}="-3'",
(-6,-14)*{}="-1'",
(0,-14)*{}="-2'",
(0,5)*{}="1'",
%%%%
\ar @{-} "b";"1'" <0pt>
%
\ar @{-} "b";"c1" <0pt>
\ar @{-} "c1";"c2" <0pt>
\ar @{-} "c1";"-3'" <0pt>
\ar @{-} "c2";"-1'" <0pt>
\ar @{-} "c2";"-2'" <0pt>
\endxy}\Ea
=
%%%%%%%%%%%%%%%%%%%%%%%%%
\sum_{n=0}^2\Ba{c}\resizebox{16mm}{!}{
\xy
(-6.9,-15.9)*+{_{_{\zeta^n(1)}}};
(2,-15.9)*+{_{_{\zeta^n(2)}}};
(7,-7.9)*+{_{\zeta^n(3)}};
 %
(0,-6)*+{_1}*\cir{}="c1",
(3,-1)*{\circ}="b",
(-3,-10)*{\circ}="c2",
%
(6,-6)*{}="-3'",
(-6,-14)*{}="-1'",
(0,-14)*{}="-2'",
(3,4)*{}="1'",
%%%%
\ar @{-} "b";"1'" <0pt>
%
\ar @{-} "b";"c1" <0pt>
\ar @{-} "c1";"c2" <0pt>
\ar @{-} "b";"-3'" <0pt>
\ar @{-} "c2";"-1'" <0pt>
\ar @{-} "c2";"-2'" <0pt>
\endxy}\Ea
-
%%%%%%%%%%%%%
\sum_{n=0}^2
 \Ba{c}\resizebox{16mm}{!}{ \xy
(-6,-20.6)*+{_{_{\zeta^n(1)}}};
(1,-15.6)*+{_{_{\zeta^n(2)}}};
(5,-11.6)*+{_{_{\zeta^n(3)}}};
 %
(-6,-14)*+{_1}*\cir{}="b",
(0,-6)*{\circ}="c1",
(-3,-10)*{\circ}="c2",
%
(3,-10)*{}="-3'",
(-6,-19)*{}="-1'",
(0,-14)*{}="-2'",
(0,-1)*{}="1'",
%%%%
\ar @{-} "c1";"1'" <0pt>
%
\ar @{-} "b";"c2" <0pt>
\ar @{-} "c1";"c2" <0pt>
\ar @{-} "c1";"-3'" <0pt>
\ar @{-} "b";"-1'" <0pt>
\ar @{-} "c2";"-2'" <0pt>
\endxy}\Ea
$$
where $\zeta$ is the cyclic permutation $(123)$.
 A nice non-minimal cofibrant resolution of the operad $\cB\cV$ has been constructed in \cite{GTV}. We denote this resolution by $\cB\cV^K_\infty$ in this paper, $K$ standing for {\em Koszul}. The minimal resolution, $\cB\cV_\infty$, has been constructed in \cite{DV}.
  %We shall need some results of
 %\cite{GTV} below and hence remind next an equivalent definition of the operad $\cB\cV$ which %played a crucial in \cite{GTV} for the
 %construction of the Koszul resolution $\cB\cV^K_\infty$.



\subsection{An operad ${\cB\cV}_\infty^{com}$}
A ${\BV}_\infty^{com}$-algebra is, by definition \cite{Kr}, a differential graded commutative algebra  $(V,d)$
equipped with a countable collections of homogeneous linear maps, $\{\Delta_a: V\rar V,\
    |\Delta_a|=1-2a\}_{a\geq 1}$, such that each $\Delta_a$ is of order $\leq a+1$ and the equations,
    \Beq\label{5: BV_comm equation for Delta}
    \sum_{a=0}^n \Delta_a \circ \Delta_{n-a}=0,
    \Eeq
hold for any $n\in \N$,
    where $\Delta_0:=-d$.

\mip

Let $\BV_\infty^{com}$ be a dg operad of $\BV_\infty^{com}$-algebras. This operad is a quotient
of the free operad generated by one binary operation in degree zero,
$
\begin{xy}
 <0mm,0.66mm>*{};<0mm,3mm>*{}**@{-},
 <0.39mm,-0.39mm>*{};<2.2mm,-2.2mm>*{}**@{-},
 <-0.35mm,-0.35mm>*{};<-2.2mm,-2.2mm>*{}**@{-},
 <0mm,0mm>*{\circ};<0mm,0mm>*{}**@{},
   <0.39mm,-0.39mm>*{};<2.9mm,-4mm>*{^2}**@{},
   <-0.35mm,-0.35mm>*{};<-2.8mm,-4mm>*{^1}**@{},
\end{xy}=
\begin{xy}
 <0mm,0.66mm>*{};<0mm,3mm>*{}**@{-},
 <0.39mm,-0.39mm>*{};<2.2mm,-2.2mm>*{}**@{-},
 <-0.35mm,-0.35mm>*{};<-2.2mm,-2.2mm>*{}**@{-},
 <0mm,0mm>*{\circ};<0mm,0mm>*{}**@{},
   <0.39mm,-0.39mm>*{};<2.9mm,-4mm>*{^1}**@{},
   <-0.35mm,-0.35mm>*{};<-2.8mm,-4mm>*{^2}**@{},
\end{xy}
$,
and a countable family of unary operations,
$
\left\{ \Ba{c}\resizebox{3.1mm}{!}{  \xy
(0,5)*{};
(0,0)*+{_a}*\cir{}
**\dir{-};
(0,-5)*{};
(0,0)*+{_a}*\cir{}
**\dir{-};
\endxy}\Ea \right\}_{a\geq 1}
$
(of homological degree $1-2a$), modulo the ideal $I$ generated by the associativity relations
for the binary operation $\begin{xy}
 <0mm,0.66mm>*{};<0mm,3mm>*{}**@{-},
 <0.39mm,-0.39mm>*{};<2.2mm,-2.2mm>*{}**@{-},
 <-0.35mm,-0.35mm>*{};<-2.2mm,-2.2mm>*{}**@{-},
 <0mm,0mm>*{\circ};<0mm,0mm>*{}**@{},
\end{xy}$ and the compatibility relations between the latter and unary operations
encoding the requirement that each unary operation $\Ba{c}\resizebox{2.9mm}{!}{\xy
(0,5)*{};
(0,0)*+{_a}*\cir{}
**\dir{-};
(0,-5)*{};
(0,0)*+{_a}*\cir{}
**\dir{-};
\endxy}\Ea$  is of order $\leq a+1$ with
respect to the multiplication operation.
The differential $\delta$ in the operad $\BV_\infty^{com}$ is given by
\Beq\label{5: diff in BV comm}
\delta\begin{xy}
 <0mm,0.66mm>*{};<0mm,3mm>*{}**@{-},
 <0.39mm,-0.39mm>*{};<2.2mm,-2.2mm>*{}**@{-},
 <-0.35mm,-0.35mm>*{};<-2.2mm,-2.2mm>*{}**@{-},
 <0mm,0mm>*{\circ};<0mm,0mm>*{}**@{},
   <0.39mm,-0.39mm>*{};<2.9mm,-4mm>*{^2}**@{},
   <-0.35mm,-0.35mm>*{};<-2.8mm,-4mm>*{^1}**@{},
\end{xy}=0, \ \ \ \
\delta\ \Ba{c}\resizebox{4mm}{!}{ \xy
(0,5)*{};
(0,0)*+{_a}*\cir{}
**\dir{-};
(0,-5)*{};
(0,0)*+{_a}*\cir{}
**\dir{-};
\endxy}\Ea: =\sum_{a=b+c\atop b,c\geq 1}\Ba{c} \resizebox{4mm}{!}{
\xy
(0,6,3)*{};
(0,0)*+{_c}*\cir{}
**\dir{-};
(0,-5)*{};
(0,0)*+{_c}*\cir{}
**\dir{-};
(0,13)*{};
(0,8)*+{_b}*\cir{}
**\dir{-};
\endxy}
\Ea
\Eeq
There is an explicit morphism of dg operads (see  Proposition 23 in \cite{GTV})\footnote{We are grateful to Bruno Vallette for pointing out this result to us.},
$$
 \cB\cV_\infty^K \lon \cB\cV^{com}_\infty,
$$
which implies existence of a morphism of dg operads $\cB\cV_\infty \rar \cB\cV^{com}_\infty$. The existence of such a morphism  follows also from the following Theorem
whose proof is given in Appendix \ref{app:BV_com proof}.
%{\ref{5: Theorem on cohom BV_com}} below.


\begin{theorem}\label{5: Theorem on cohom BV_com} { The dg operad $\BV_\infty^{com}$ is formal with the cohomology
operad $H^\bu(\BV_\infty^{com})$ isomorphic to the operad, $\BV$, of
Batalin-Vilkovisky algebras, i.e.\ there is a canonical surjective quasi-isomorphism of
operads,
$$
\pi: \BV_\infty^{com} \lon \BV
$$
which sends to zero all generators $\Ba{c}\resizebox{4mm}{!}{ \xy
(0,5)*{};
(0,0)*+{_a}*\cir{}
**\dir{-};
(0,-5)*{};
(0,0)*+{_a}*\cir{}
**\dir{-};
\endxy}\Ea$ with $a\geq 2$.
}
\end{theorem}


\subsection{From strongly homotopy involutive  Lie bialgebras to $\cB\cV_\infty$-algebras}

We call a $\LoB_\infty$ algebra $\fg$ \emph{good} if for any fixed $m$ and $k$ only finitely many of the operations $\mu_{m,n}^k\in \Hom(\fg^{\otimes n}, \fg^{\otimes m})$ are non-zero.
In this case we define the Chevalley-Eilenberg complex $CE(\fg) = {\odot}^\bu(\fg[-1])$ of $\fg$ as an $\caL ie_\infty$ coalgebra.
More concretely, for a finite dimensional $\fg$ we may understand the $\LoB_\infty$ algebra structure as a formal power series
$\Ga_\hbar=\Ga_\hbar(\psi_i,\eta^i,\hbar)$ as explained in section \ref{3: Section on Def complex}.
Using similar notation, we may understand the space $CE(\fg)$ as the space of polynomials in the variables $\psi_i$.
Then the differential $\Delta_0$ on $CE(\fg)$ is given by the formula
\[
\Delta_0:=\sum_{i}\frac{\p \Ga_\hbar}{\p \eta^{i}}|_{\hbar=\eta^i=0}\frac{\p}{\p \psi_i} .
\]

\begin{proposition}
Let $\fg$ be a good $\LoB_\infty$ algebra, with the $\LoB_\infty$ algebra structure being defined a power series $\Ga_\hbar=\Ga_\hbar(\psi_i,\eta^i,\hbar)$ as explained in section \ref{3: Section on Def complex}. Then there is a natural $\BV_\infty^{com}$ algebra structure $\rho$ on the complex $CE(\fg)$ given by the formulas:

$$
\rho\left(\begin{xy}
 <0mm,0.66mm>*{};<0mm,3mm>*{}**@{-},
 <0.39mm,-0.39mm>*{};<2.2mm,-2.2mm>*{}**@{-},
 <-0.35mm,-0.35mm>*{};<-2.2mm,-2.2mm>*{}**@{-},
 <0mm,0mm>*{\circ};<0mm,0mm>*{}**@{},
\end{xy}\right):=\mathrm{the\ standard\ multiplication\ in}\
{\odot}^\bu(\fg[-1])[[\hbar]]
$$
and, for any $a\geq 1$,
$$
\rho\left(\Ba{c}\resizebox{4mm}{!}{ \xy
(0,5)*{};
(0,0)*+{_a}*\cir{}
**\dir{-};
(0,-5)*{};
(0,0)*+{_a}*\cir{}
**\dir{-};
\endxy}\Ea\right):= \sum_{p+k=a+1} \frac{1}{p!k!}\sum_{i_1,\ldots,i_{k}} \frac{\p^{a+1} \Ga_\hbar}{\p^p\hbar \p \eta^{i_1}\cdots
\p \eta^{i_{k}}}|_{\hbar=\eta^i=0}\frac{\p^{k} }{\p \psi_{i_1}\cdots
\p \psi_{i_{k}}}
$$
%For infinite dimensional $\fg$ these formulas extend accordingly.
\end{proposition}

% \begin{proposition} { Given a a good $\LoB_\infty$ algebra structure on a dg vector
% space
% $(\fg,d_\fg)$ (understood as a formal power series
% $\Ga_\hbar=\Ga_\hbar(\psi_i,\eta^i,\hbar)$ as explained in \S ). There is an
% associated continuous representation,
% $$
% \rho: \left(\BV_\infty^{comm}[[\hbar]], \delta\right) \lon \cE
% nd_{({\odot}^\bu(\fg[-1])[[\hbar]], \Delta_0)}
% $$
%  of the dg operad
% $\BV_\infty^{com}[[\hbar]]:= \BV_\infty^{com}\ot_\K \K[[\hbar]]$ in the complex
% \Beq\label{5: odot g[-1] complex}
% \left({\odot}^\bu(\fg[-1])[[\hbar]]\simeq \K[[\psi_i,\hbar]],\ \ \ \ \
% \Delta_0:=\sum_{i}\frac{\p \Ga_\hbar}{\p \eta^{i}}|_{\eta^i=0}\frac{\p}{\p \psi_i}
% \right)
% \Eeq
% given by
% $$
% \rho\left(\begin{xy}
%  <0mm,0.66mm>*{};<0mm,3mm>*{}**@{-},
%  <0.39mm,-0.39mm>*{};<2.2mm,-2.2mm>*{}**@{-},
%  <-0.35mm,-0.35mm>*{};<-2.2mm,-2.2mm>*{}**@{-},
%  <0mm,0mm>*{\circ};<0mm,0mm>*{}**@{},
% \end{xy}\right):=\mathrm{the\ standard\ multiplication\ in}\
% {\odot}^\bu(\fg[-1])[[\hbar]]
% $$
% and, for any $a\geq 1$,
% $$
% \rho\left(\xy
% (0,5)*{};
% (0,0)*+{a}*\cir{}
% **\dir{-};
% (0,-5)*{};
% (0,0)*+{a}*\cir{}
% **\dir{-};
% \endxy\right):= \frac{1}{(a+1)!}\sum_{i_1,\ldots,i_{k+1}} \frac{\p^{a+1} \Ga_\hbar}{\p \eta^{i_1}\cdots
% \p \eta^{i_{a+1}}}|_{\eta^i=0}\frac{\p^{a+1} }{\p \psi_{i_1}\cdots
% \p \psi_{i_{a+1}}}
% $$
% }
% \end{proposition}

\begin{proof} It is clear that $\Delta_{a}:=\rho\left(\Ba{c}\resizebox{4mm}{!}{ \xy
(0,5)*{};
(0,0)*+{a}*\cir{}
**\dir{-};
(0,-5)*{};
(0,0)*+{a}*\cir{}
**\dir{-};
\endxy}\Ea\right)$ is an operator of order $\leq a+1$ with respect to the standard multiplication in
the graded commutative algebra
${\odot}^\bu(\fg[-1])$. The verification that the operators $\{\Delta_a\}_{a\geq 0}$ satisfy identities \eqref{5: BV_comm equation for Delta}
is best done pictorially. We represent the expression on the right hand side by the picture
$$
\rho\left(\Ba{c}\resizebox{4mm}{!}{ \xy
(0,5)*{};
(0,0)*+{a}*\cir{}
**\dir{-};
(0,-5)*{};
(0,0)*+{a}*\cir{}
**\dir{-};
\endxy}\Ea\right)=  \sum_{a+1=p+k\atop k\geq 1,p\geq 0}
\underbrace{
\Ba{c}\resizebox{7mm}{!}{ \xy
(0.0,-12.2)*{_1},
%(0,-3.5)*{_k},
(0,-5)*{...},
   \ar@/^1pc/(0,0)*+{_{p}}*\frm{o};(0,-10)*{\circ}*\frm{}
   \ar@/^{-1pc}/(0,0)*+{_{p}}*\frm{o};(0,-10)*{\circ}*\frm{}
   \ar@/^0.6pc/(0,0)*+{_{p}}*\frm{o};(0,-10)*{\circ}*\frm{}
   \ar@/^{-0.6pc}/(0,0)*+{_{p}}*\frm{o};(0,-10)*{\circ}*\frm{}
 \endxy}
 \Ea}_{k\ \mathrm{edges}}
$$
Then we compute
\Beqrn
\rho\left(\delta \Ba{c}\resizebox{4mm}{!}{ \xy
(0,5)*{};
(0,0)*+{_a}*\cir{}
**\dir{-};
(0,-5)*{};
(0,0)*+{_a}*\cir{}
**\dir{-};
\endxy}\Ea\right)&
=&\sum_{a=b+c\atop b,c\geq 1}\rho\left(
\Ba{c}\resizebox{3mm}{!}{  \xy
(0,6,3)*{};
(0,0)*+{_c}*\cir{}
**\dir{-};
(0,-5)*{};
(0,0)*+{_c}*\cir{}
**\dir{-};
(0,13)*{};
(0,8)*+{_b}*\cir{}
**\dir{-};
\endxy}
\Ea\right)
= \sum_{a=b+c\atop b,c\geq 1} \sum_{b+1=p+k\atop k\geq 1,p\geq 0} \sum_{c+1=q+l\atop l\geq 1,q\geq 0}
\left(
\Ba{c}\resizebox{7mm}{!}{ \xy
(0.0,-12.2)*{_1},
(0,-3.5)*{_k},
(0,-5)*{...},
   \ar@/^1pc/(0,0)*+{_{p}}*\frm{o};(0,-10)*{\circ}*\frm{}
   \ar@/^{-1pc}/(0,0)*+{_{p}}*\frm{o};(0,-10)*{\circ}*\frm{}
   \ar@/^0.6pc/(0,0)*+{_{p}}*\frm{o};(0,-10)*{\circ}*\frm{}
   \ar@/^{-0.6pc}/(0,0)*+{_{p}}*\frm{o};(0,-10)*{\circ}*\frm{}
 \endxy}
 \Ea\right) \circ_1
\left(
\Ba{c}\resizebox{7mm}{!}{ \xy
(0.0,-12.2)*{_1},
(0,-3.5)*{_l},
(0,-5)*{...},
   \ar@/^1pc/(0,0)*+{_{q}}*\frm{o};(0,-10)*{\circ}*\frm{}
   \ar@/^{-1pc}/(0,0)*+{_{q}}*\frm{o};(0,-10)*{\circ}*\frm{}
   \ar@/^0.6pc/(0,0)*+{_{q}}*\frm{o};(0,-10)*{\circ}*\frm{}
   \ar@/^{-0.6pc}/(0,0)*+{_{q}}*\frm{o};(0,-10)*{\circ}*\frm{}
 \endxy}
 \Ea\right) \\
 &=&
  \sum_{a=b+c\atop b,c\geq 1} \sum_{b+1=p+k\atop k\geq 1} \sum_{c+1=q+l\atop l\geq 1}
  \sum_{k=k'+k'' } \Ba{c}\resizebox{10mm}{!}{ \xy
(0,-5)*{\stackrel{l}{...}},
(0,5)*{\stackrel{k'}{...}},
(8,0)*{\stackrel{k''}{...}},
   \ar@/^{-1pc}/(0,0)*+{_{q}}*\frm{o};(0,-10)*{\circ}
   \ar@/^0.6pc/(0,0)*+{_{q}}*\frm{o};(0,-10)*{\circ}
   \ar@/^{-0.6pc}/(0,0)*+{_{q}}*\frm{o};(0,-10)*{\circ}
   \ar@/^{-1pc}/(0,10)*+{_{p}}*\frm{o};(0,0)*+{_{q}}*\frm{o}
   \ar@/^0.6pc/(0,10)*+{_{p}}*\frm{o};(0,0)*+{_{q}}*\frm{o}
   \ar@/^{-0.6pc}/(0,10)*+{_{p}}*\frm{o};(0,0)*+{_{q}}*\frm{o}
   %
   \ar@/^{2.4pc}/(0,10)*+{_{p}}*\frm{o};(0,-10)*{\circ}*\frm{}
   \ar@/^{1.3pc}/(0,10)*+{_{p}}*\frm{o};(0,-10)*{\circ}*\frm{}
 \endxy}
 \Ea
=  \sum_{a=p+q+k'+k''+l-2\atop {p,q,k',k''\geq 0, k'+k'',l\geq 1,
\atop k'+k''+p\geq 2, l+q\geq 2}}
\Ba{c}\resizebox{10mm}{!}{ \xy
(0,-5)*{\stackrel{l}{...}},
(0,5)*{\stackrel{k'}{...}},
(8,0)*{\stackrel{k''}{...}},
   \ar@/^{-1pc}/(0,0)*+{_{q}}*\frm{o};(0,-10)*{\circ}
   \ar@/^0.6pc/(0,0)*+{_{q}}*\frm{o};(0,-10)*{\circ}
   \ar@/^{-0.6pc}/(0,0)*+{_{q}}*\frm{o};(0,-10)*{\circ}
   \ar@/^{-1pc}/(0,10)*+{_{p}}*\frm{o};(0,0)*+{_{q}}*\frm{o}
   \ar@/^0.6pc/(0,10)*+{_{p}}*\frm{o};(0,0)*+{_{q}}*\frm{o}
   \ar@/^{-0.6pc}/(0,10)*+{_{p}}*\frm{o};(0,0)*+{_{q}}*\frm{o}
   %
   \ar@/^{2.4pc}/(0,10)*+{_{p}}*\frm{o};(0,-10)*{\circ}*\frm{}
   \ar@/^{1.3pc}/(0,10)*+{_{p}}*\frm{o};(0,-10)*{\circ}*\frm{}
 \endxy}\, .
 \Ea\\
\Eeqrn
Here we hide the binomial prefactors in the notation by assuming that a picture is preceded by a factor $\frac 1 {p!q! |G|}$ where $G$ is the symmetry group of the picture.
Note that terms with $k'=0$ can be dropped from the sum on the right-hand side by symmetry.
Indeed, the piece $k'=0$ of the sum is symmetric under interchange of $(p,k'')$ and $(q,l)$, and can hence be written as a linear combination of anticommutators of anticommuting operators.
Pictorially, the signs are best verified by thinking of the vertices in the pictures to be odd, so that in particular graphs which have symmetries acting by an odd permutation on the vertices vanish, and we have
\begin{align*}
 \resizebox{10mm}{!}{ \xy
(0,-5)*{\stackrel{l}{...}},
%(0,5)*{\stackrel{k'}{...}},
(8,0)*{\stackrel{k''}{...}},
   \ar@/^{-1pc}/(0,0)*+{_{q}}*\frm{o};(0,-10)*{\circ}
   \ar@/^0.6pc/(0,0)*+{_{q}}*\frm{o};(0,-10)*{\circ}
   \ar@/^{-0.6pc}/(0,0)*+{_{q}}*\frm{o};(0,-10)*{\circ}
   %\ar@/^{-1pc}/(0,10)*+{_{p}}*\frm{o};(0,0)*+{_{q}}*\frm{o}
   %\ar@/^0.6pc/(0,10)*+{_{p}}*\frm{o};(0,0)*+{_{q}}*\frm{o}
   %\ar@/^{-0.6pc}/(0,10)*+{_{p}}*\frm{o};(0,0)*+{_{q}}*\frm{o}
   %
   \ar@/^{2.4pc}/(0,10)*+{_{p}}*\frm{o};(0,-10)*{\circ}*\frm{}
   \ar@/^{1.3pc}/(0,10)*+{_{p}}*\frm{o};(0,-10)*{\circ}*\frm{}
 \endxy}
 \,
 +
 \,
  \resizebox{10mm}{!}{ \xy
(0,-5)*{\stackrel{k''}{...}},
%(0,5)*{\stackrel{k'}{...}},
(8,0)*{\stackrel{l}{...}},
   \ar@/^{-1pc}/(0,0)*+{_{p}}*\frm{o};(0,-10)*{\circ}
   \ar@/^0.6pc/(0,0)*+{_{p}}*\frm{o};(0,-10)*{\circ}
   \ar@/^{-0.6pc}/(0,0)*+{_{p}}*\frm{o};(0,-10)*{\circ}
   %\ar@/^{-1pc}/(0,10)*+{_{p}}*\frm{o};(0,0)*+{_{q}}*\frm{o}
   %\ar@/^0.6pc/(0,10)*+{_{p}}*\frm{o};(0,0)*+{_{q}}*\frm{o}
   %\ar@/^{-0.6pc}/(0,10)*+{_{p}}*\frm{o};(0,0)*+{_{q}}*\frm{o}
   %
   \ar@/^{2.4pc}/(0,10)*+{_{q}}*\frm{o};(0,-10)*{\circ}*\frm{}
   \ar@/^{1.3pc}/(0,10)*+{_{q}}*\frm{o};(0,-10)*{\circ}*\frm{}
 \endxy}
 &=0
 &&\text{and}
 &
  \resizebox{10mm}{!}{ \xy
(0,-5)*{\stackrel{k''}{...}},
%(0,5)*{\stackrel{k'}{...}},
(8,0)*{\stackrel{k''}{...}},
   \ar@/^{-1pc}/(0,0)*+{_{p}}*\frm{o};(0,-10)*{\circ}
   \ar@/^0.6pc/(0,0)*+{_{p}}*\frm{o};(0,-10)*{\circ}
   \ar@/^{-0.6pc}/(0,0)*+{_{p}}*\frm{o};(0,-10)*{\circ}
   %\ar@/^{-1pc}/(0,10)*+{_{p}}*\frm{o};(0,0)*+{_{q}}*\frm{o}
   %\ar@/^0.6pc/(0,10)*+{_{p}}*\frm{o};(0,0)*+{_{q}}*\frm{o}
   %\ar@/^{-0.6pc}/(0,10)*+{_{p}}*\frm{o};(0,0)*+{_{q}}*\frm{o}
   %
   \ar@/^{2.4pc}/(0,10)*+{_{p}}*\frm{o};(0,-10)*{\circ}*\frm{}
   \ar@/^{1.3pc}/(0,10)*+{_{p}}*\frm{o};(0,-10)*{\circ}*\frm{}
 \endxy}
 &=0\, .
\end{align*}
We hence find
\begin{equation}\label{equ:rhodeltasimpl}
 \rho\left(\delta \Ba{c}\resizebox{4mm}{!}{ \xy
(0,5)*{};
(0,0)*+{_a}*\cir{}
**\dir{-};
(0,-5)*{};
(0,0)*+{_a}*\cir{}
**\dir{-};
\endxy}\Ea\right)
=
  \sum_{a=p+q+k'+k''+l-2\atop {p,q,k''\geq 0, k',l\geq 1,
 \atop k'+k''+p\geq 2, l+q\geq 2}}
\resizebox{10mm}{!}{ \xy
(0,-5)*{\stackrel{l}{...}},
(0,5)*{\stackrel{k'}{...}},
(8,0)*{\stackrel{k''}{...}},
   \ar@/^{-1pc}/(0,0)*+{_{q}}*\frm{o};(0,-10)*{\circ}
   \ar@/^0.6pc/(0,0)*+{_{q}}*\frm{o};(0,-10)*{\circ}
   \ar@/^{-0.6pc}/(0,0)*+{_{q}}*\frm{o};(0,-10)*{\circ}
   \ar@/^{-1pc}/(0,10)*+{_{p}}*\frm{o};(0,0)*+{_{q}}*\frm{o}
   \ar@/^0.6pc/(0,10)*+{_{p}}*\frm{o};(0,0)*+{_{q}}*\frm{o}
   \ar@/^{-0.6pc}/(0,10)*+{_{p}}*\frm{o};(0,0)*+{_{q}}*\frm{o}
   %
   \ar@/^{2.4pc}/(0,10)*+{_{p}}*\frm{o};(0,-10)*{\circ}*\frm{}
   \ar@/^{1.3pc}/(0,10)*+{_{p}}*\frm{o};(0,-10)*{\circ}*\frm{}
 \endxy}
\end{equation}



On the other hand
%(WHAT IS THE DIFFERENTIAL IN THE GRADING WHEN ONE UNIT OF WEIGHT HAS GRADING $-2$?)
\Beqrn
\delta \rho\left(
\Ba{c}\resizebox{4mm}{!}{ \xy
(0,5)*{};
(0,0)*+{a}*\cir{}
**\dir{-};
(0,-5)*{};
(0,0)*+{a}*\cir{}
**\dir{-};
\endxy}\Ea\right)
&=&  \sum_{a+1=p+k\atop k\geq 1,p\geq 0} \delta
\Ba{c}\resizebox{7mm}{!}{ \xy
(0.0,-12.2)*{_1},
(0,-3.5)*{_k},
(0,-5)*{...},
   \ar@/^1pc/(0,0)*+{_{p}}*\frm{o};(0,-10)*{\circ}*\frm{}
   \ar@/^{-1pc}/(0,0)*+{_{p}}*\frm{o};(0,-10)*{\circ}*\frm{}
   \ar@/^0.6pc/(0,0)*+{_{p}}*\frm{o};(0,-10)*{\circ}*\frm{}
   \ar@/^{-0.6pc}/(0,0)*+{_{p}}*\frm{o};(0,-10)*{\circ}*\frm{}
 \endxy}
 \Ea
\\
&=&
\underbrace{
 \sum_{a+1=p+k'+k'' \atop {p+1=p'+p''+l \atop k',l\geq 1}}
 \Ba{c}\resizebox{12mm}{!}{ \xy
(0,-5)*{\stackrel{k'}{...}},
(0,5)*{\stackrel{l}{...}},
(8,0)*{\stackrel{k''}{...}},
   \ar@/^{-1pc}/(0,0)*+{_{p''}}*\frm{o};(0,-10)*{\circ}
   \ar@/^0.6pc/(0,0)*+{_{p''}}*\frm{o};(0,-10)*{\circ}
   \ar@/^{-0.6pc}/(0,0)*+{_{p''}}*\frm{o};(0,-10)*{\circ}
   \ar@/^{-1pc}/(0,10)*+{_{p'}}*\frm{o};(0,0)*+{_{p''}}*\frm{o}
   \ar@/^0.6pc/(0,10)*+{_{p'}}*\frm{o};(0,0)*+{_{p''}}*\frm{o}
   \ar@/^{-0.6pc}/(0,10)*+{_{p'}}*\frm{o};(0,0)*+{_{p''}}*\frm{o}
   %
   \ar@/^{2.4pc}/(0,10)*+{_{p'}}*\frm{o};(0,-10)*{\circ}*\frm{}
   \ar@/^{1.3pc}/(0,10)*+{_{p'}}*\frm{o};(0,-10)*{\circ}*\frm{}
 \endxy}
 \Ea
  }_{=0\text{ by $\Gamma_\hbar *_\hbar \Gamma_\hbar=0$}}
 -
  \sum_{a+1=p+k \atop {k'+k''=k %\atop k'\geq 1
  }}
 \Ba{c}\resizebox{12mm}{!}{ \xy
(0,5)*{\stackrel{k'}{...}},
(8,0)*{\stackrel{k''}{...}},
   %\ar@/^{-1pc}/(0,0)*+{_{0}}*\frm{o};(0,-10)*{\circ}
   %\ar@/^0.6pc/(0,0)*+{_{0}}*\frm{o};(0,-10)*{\circ}
   \ar@/^{-0.0pc}/(0,0)*+{_{0}}*\frm{o};(0,-10)*{\circ}
   \ar@/^{-1pc}/(0,10)*+{_{p}}*\frm{o};(0,0)*+{_{0}}*\frm{o}
   \ar@/^0.6pc/(0,10)*+{_{p}}*\frm{o};(0,0)*+{_{0}}*\frm{o}
   \ar@/^{-0.6pc}/(0,10)*+{_{p}}*\frm{o};(0,0)*+{_{0}}*\frm{o}
   %
   \ar@/^{2.4pc}/(0,10)*+{_{p}}*\frm{o};(0,-10)*{\circ}*\frm{}
   \ar@/^{1.3pc}/(0,10)*+{_{p}}*\frm{o};(0,-10)*{\circ}*\frm{}
 \endxy}
 \Ea
-
 \sum_{a+1=p+k}
 \Ba{c}\resizebox{7mm}{!}{ \xy
(0,-5)*{\stackrel{k}{...}},
%(0,5)*{\stackrel{l}{...}},
%(8,0)*{\stackrel{k''}{...}},
   \ar@/^{-1pc}/(0,0)*+{_{p}}*\frm{o};(0,-10)*{\circ}
   \ar@/^0.6pc/(0,0)*+{_{p}}*\frm{o};(0,-10)*{\circ}
   \ar@/^{-0.6pc}/(0,0)*+{_{p}}*\frm{o};(0,-10)*{\circ}
   \ar@/^{-0pc}/(0,10)*+{_{0}}*\frm{o};(0,0)*+{_{p}}*\frm{o}
   %\ar@/^0.6pc/(0,10)*+{_{p}}*\frm{o};(0,0)*+{_{q}}*\frm{o}
   %\ar@/^{-0.6pc}/(0,10)*+{_{p}}*\frm{o};(0,0)*+{_{q}}*\frm{o}
   %
   %\ar@/^{2.4pc}/(0,10)*+{_{p}}*\frm{o};(0,-10)*{\circ}*\frm{}
   %\ar@/^{1.3pc}/(0,10)*+{_{p}}*\frm{o};(0,-10)*{\circ}*\frm{}
 \endxy}
 \Ea
  -
  \sum_{a+1=p+k}
 \Ba{c}\resizebox{12mm}{!}{ \xy
(0,-5)*{\stackrel{k}{...}},
%(0,5)*{\stackrel{l}{...}},
%(8,0)*{\stackrel{k''}{...}},
   \ar@/^{-1pc}/(0,0)*+{_{p}}*\frm{o};(0,-10)*{\circ}
   \ar@/^0.6pc/(0,0)*+{_{p}}*\frm{o};(0,-10)*{\circ}
   \ar@/^{-0.6pc}/(0,0)*+{_{p}}*\frm{o};(0,-10)*{\circ}
   \ar@/^{2pc}/(0,10)*+{_{0}}*\frm{o};(0,-10)*{\circ}
   %\ar@/^0.6pc/(0,10)*+{_{p}}*\frm{o};(0,0)*+{_{q}}*\frm{o}
   %\ar@/^{-0.6pc}/(0,10)*+{_{p}}*\frm{o};(0,0)*+{_{q}}*\frm{o}
   %
   %\ar@/^{2.4pc}/(0,10)*+{_{p}}*\frm{o};(0,-10)*{\circ}*\frm{}
   %\ar@/^{1.3pc}/(0,10)*+{_{p}}*\frm{o};(0,-10)*{\circ}*\frm{}
 \endxy}
 \Ea
\Eeqrn

There are several cancellations in this expression. First, the terms with $k'=0$ from the second sum cancel the fourth sum by the same symmetry argument as above. The remaining terms of the second sum, and the third sum together kill those terms of the first sum for which either $p''=0$, $k'=1$ or $p'=0$, $l=1$. $k''=0$.
We hence find that
\[
 \delta \rho\left(
\Ba{c}\resizebox{4mm}{!}{ \xy
(0,5)*{};
(0,0)*+{a}*\cir{}
**\dir{-};
(0,-5)*{};
(0,0)*+{a}*\cir{}
**\dir{-};
\endxy}\Ea\right)
=
 \sum_{a= p'+p''+l +k'+k''-2 \atop{ k',l\geq 1 \atop{p''+k'\geq 2, p'+l+k''\geq 2} }}
 \resizebox{12mm}{!}{ \xy
(0,-5)*{\stackrel{k'}{...}},
(0,5)*{\stackrel{l}{...}},
(8,0)*{\stackrel{k''}{...}},
   \ar@/^{-1pc}/(0,0)*+{_{p''}}*\frm{o};(0,-10)*{\circ}
   \ar@/^0.6pc/(0,0)*+{_{p''}}*\frm{o};(0,-10)*{\circ}
   \ar@/^{-0.6pc}/(0,0)*+{_{p''}}*\frm{o};(0,-10)*{\circ}
   \ar@/^{-1pc}/(0,10)*+{_{p'}}*\frm{o};(0,0)*+{_{p''}}*\frm{o}
   \ar@/^0.6pc/(0,10)*+{_{p'}}*\frm{o};(0,0)*+{_{p''}}*\frm{o}
   \ar@/^{-0.6pc}/(0,10)*+{_{p'}}*\frm{o};(0,0)*+{_{p''}}*\frm{o}
   %
   \ar@/^{2.4pc}/(0,10)*+{_{p'}}*\frm{o};(0,-10)*{\circ}*\frm{}
   \ar@/^{1.3pc}/(0,10)*+{_{p'}}*\frm{o};(0,-10)*{\circ}*\frm{}
 \endxy} \, .
\]

Comparing this formula with \eqref{equ:rhodeltasimpl}, we see that both expressions agree, up to a relabelling of the summation indices.
%In particular, the three terms on the right in the second formula just kill those terms that were excluded in the first formula due to the restrictions $k'+k''+p\geq 2, l+q\geq 2$.
%follows from equation (\ref{Appendix: equation for Gamma-h}).
\end{proof}





\bip

\bip
\appendix

%\appendix
%\renewcommand{\thesubsection}{{\bf A.\arabic{subsection}}}
%\renewcommand{\thesubsubsection}{{\bf A.\arabic{subsection}.\arabic{subsubsection}}}
%\chapter{First Appendix}

\renewcommand{\thesection}{{\Alph{section}}}
\renewcommand{\thesubsection}{{\bf\Alph{section}.\arabic{subsection}}}
\renewcommand{\thesubsubsection}{\bf\Alph{section}.\arabic{subsection}.\arabic{subsubsection}}

\section{ Proof of Proposition \ref{2: toy problem}}\label{app:koszulnessproof}
In this section we show that the quadratic algebra $\cA_n$ of section {\ref{sec:extracomplexes}} is Koszul. In fact, we will show the equivalent statement that the Koszul dual algebra $B_n=\cA_n^!$ is Koszul.
Concretely, $B_n$ is the algebra generated by $V=\mathbb K x_1 \oplus\dots\oplus \mathbb K x_n$ with relations $x_ix_j=0$ if $|i-j|\neq 1$ and $x_ix_{i+1}=-x_{i+1}x_{i}$.

We denote by $C_n \otimes_{\kappa} B_n$ the Koszul complex of $B_n$ (see \cite{LV} for details), i.~e., the complex $(C_n\otimes B_n, d=d_{\kappa})$, where $C_n = B_n^{\text{!`}}$ is the coalgebra generated by the elements $x_i$ in degree $1$, with quadratic corelations $R = \text{span}( \{x_ix_j \mid |i-j|\ne 1\} \cup
\{x_ix_{i+1} + x_{i+1}x_i\})$ and the differential is induced by the degree $-1$ map $\kappa : C_n \to B_n$ that is zero everywhere except on $V$, where it identifies $V\subset C_n$ with $V \subset B_n$.
Informally, the differential acts by ``jumping" the tensor product over the $x_i$ on its left, producing a sign coming from the degree in $C_n$.


Notice that $B_n$ and $C_n$ are weight graded and the weight $k$ component of $B_n$, $B_n^{(k)}$ is zero if $k\geq 3$.


The result will follow from the acyclicity of the Koszul complex, which will in turn be shown by constructing a contracting homotopy $h$.\\

 Let $l\geq 1$ and let $w$ be a word of length $l-1$ on the variables $x_i$ and $1\leq a,b \leq n$ be indices such that $|a-b|= 1$. We define a degree $1$ map $h\colon V^{\otimes l} \otimes B_n^{(2)} \to V^{\otimes l+1} \otimes B_n^{(1)}$, for $n\geq 1$ by \\

$h(wx_k\otimes x_ax_b) = \begin{cases}
 \frac{(-1)^l}{2} (wx_k x_a\otimes x_b - w x_kx_b\otimes x_a) & \text{ if } |k-a|\ne 1 \text{ and } |k-b|\ne 1\\
 (-1)^l wx_kx_a \otimes x_b & \text{ if }|k-b| = 1\\
\end{cases}$

Notice that all the cases are covered because $|a-b|= 1 \wedge |k-b|=1 \Rightarrow |k-a|\ne 1$. Moreover, due to the anti-symmetry in $B_n^{(2)}$, if $|k-a|=1$ we have  $h(wx_k\otimes x_ax_b)= -(-1)^lwx_kx_b \otimes x_a$.

If $l=0$ we define $h\colon B_n^{(2)} \to V \otimes B_n^{(1)}$ using the first formula from above, i.~e., we consider the non-defined differences to be different from $1$ and ignore the non-existent variables.

\begin{lemma}
The map $h$ restricts to a function $C_n^{(l)} \otimes B_n^{(2)} \to C_n^{(l+1)}\otimes B_n^{(1)}$ that satisfies $dh = id_{C_n^{(l)} \otimes B_n^{(2)}}$.
\end{lemma}

\begin{proof}
Recall that $C_n^{(1)}=V$, and $C_n^{(l)}= \bigcap_{a+b=l-2} V^{\otimes a}RV^{\otimes b}$ for $l\geq 2$.

First notice that $h$ maps $V^{\otimes a}R V^{\otimes b} \otimes B_n^{(2)} $ to $V^{\otimes a}R V^{\otimes b+1} \otimes B_n^{(1)}$, since it leaves the elements in $R$ unaltered. It is also clear by the construction of $h$ that the image of an element of $V^{\otimes l}\otimes B_n^{(2)}$ lands in $V^{\otimes l-1}R\otimes B_n^{(1)}$, therefore $h$ restricts indeed to a map $C_n^{(l)} \otimes B_n^{(2)} \to C_n^{(l+1)}\otimes B_n^{(1)}$.

To check the identity $dh = id_{C_n^{(l)} \otimes B_n^{(2)}}$ suppose first that both $|k-a|$ and $|k-b|$ are different from $1$:
$$dh(wx_k\otimes x_ax_b) = \frac{1}{2}d ((-1)^lwx_k x_a\otimes x_b - (-1)^l w x_kx_b\otimes x_a) = \frac{1}{2} wx_k \otimes x_a x_b - \frac{1}{2}wx_k \otimes x_b x_a = wx_k\otimes x_ax_b.$$

If $|k-b|=1$, $dh(wx_k\otimes x_ax_b) = d((-1)^l wx_kx_a\otimes x_b) = wx_k\otimes x_ax_b$ and an analogous calculation holds if $|k-a|=1$.

\end{proof}



To define $h\colon C_n^{(l)} \otimes B_n^{(1)} \to C_n^{(l+1)} \otimes B_n^{(0)}= C^{(l+1)}$, as before we define it on $V^{\otimes l} \otimes B_n^{(1)}$ and we verify that it restricts properly.

Let $l\geq 2$ and let us denote by $w$ some word on $x_i$ of length $l-2$. We define $h\colon V^{\otimes l} \otimes B_n^{(1)} \to V^{\otimes l+1}$, by $$h(wx_kx_a\otimes x_b)= \begin{cases}
(-1)^l wx_kx_ax_b & \text{if } |a-b|\ne 1 \\
 \frac{(-1)^l}{2} wx_k(x_ax_b+x_bx_a) & \text{if } |a-b|=1 \text{, } |a-k| \ne 1 \text{ and } |b-k| \ne 1 \\
  0 & \text{if } |a-b|=1  \text{ and } |b-k| = 1\\
 (-1)^lwx_k(x_ax_b+x_bx_a) & \text{if } |a-b|=1 \text{ and } |a-k|= 1
\end{cases}$$
Interpret this definition for $l< 2$ in the following way: Whenever some difference is not defined because $a$ or $k$ are not defined, take the case in the definition where the absolute value of the difference is different from $1$ and ignore the non-existent variables.


\begin{lemma}
$h$ restricts to a function $C_n^{(l)} \otimes B_n^{(1)} \to C_n^{(l+1)}$ that satisfies $dh+hd=id_{C_n^{(l)} \otimes B_n^{(1)}}$.
\end{lemma}

\begin{proof}

Notice that by construction, the image of $h$ sits inside $V^{\otimes l-1}R$. The $R$ part in $V^{\otimes a}R V^{\otimes b}$ is left unaltered by $h$ if $b$ is at least $1$ hence $V^{\otimes a}R V^{\otimes b}\otimes B_n^{(1)}$ is sent to $V^{\otimes a}R V^{\otimes b+1}$.

Let us suppose that $l$ is at least $2$ and let us check that $V^{\otimes l-2}R \otimes B_n^{(1)}$ is sent to $V^{\otimes l-2}R V$:

Let $w$ be a word in the variables $x_i$ of length $l-2$. $V^{\otimes l-2}R \otimes B_n^{(1)}$ is spanned by elements of the form $wx_kx_a\otimes x_b$, with $|k-a| \ne 1$ and elements of the form $w(x_kx_a+x_ax_k)\otimes x_b$, with $|k-a|=1$.

Let us consider first the first type of elements.  If $|a-b|\ne 1$, $h(wx_kx_a\otimes x_b)=(-1)^lwx_kx_a x_b\in V^{\otimes l-2}RV$.

If, on the other hand, $|a-b|=1$, then either $|b-k|=1$ and the image via $h$ is zero or $|b-k|\ne 1$ and both summands of  $h(wx_kx_a\otimes x_b)= \frac{(-1)^l}{2} wx_kx_ax_b +\frac{(-1)^l}{2} wx_kx_bx_a$ belong to $V^{\otimes l-2}RV$.

Let us now consider the elements of the form $w(x_kx_a+x_ax_k)\otimes x_b$ with $|k-a| = 1$. If both $|b-k|$ and $|b-a|$ are different from $1$, then $h(  w(x_kx_a+x_ax_k)\otimes x_b) = (-1)^lw(x_kx_a+x_ax_k)x_b$ is in $V^{\otimes l-2}RV$.

Otherwise, let us assume without loss of generality that $|b-a|=1$ (and therefore $|b-k|\ne 1$). Then

$$h( w(x_kx_a+x_ax_k)\otimes x_b) = (-1)^l(wx_k(x_ax_b+x_bx_a) +wx_ax_k x_b)  = (-1)^l(w(x_kx_a+x_ax_k)x_b + wx_kx_bx_a) \in V^{\otimes l-2}RV.$$\\


Let us now show the homotopy equation. As before, we consider a generic element $wx_kx_a\otimes x_b \in V^{\otimes l}\otimes B_n^{(1)}$ and we divide the verification into various cases.\\

If $|a-b|\ne 1$,
$$dh(wx_kx_a\otimes x_b)+hd(wx_kx_a\otimes x_b) = wx_kx_a\otimes x_b+0.$$

If  $|a-b|=1$   and  $|a-k| \ne 1$ and $|b-k|\ne 1$,
$$(dh+hd)(wx_kx_a\otimes x_b) = \frac{1}{2}(wx_kx_a\otimes x_b +wx_kx_b\otimes x_a) + \frac{1}{2}(wx_kx_a\otimes x_b - wx_kx_b\otimes x_a)=wx_kx_a\otimes x_b.$$

If  $|a-b|=1$   and  $|b-k| = 1$,

$$(dh+hd)(wx_kx_a\otimes x_b) = 0 + wx_kx_a\otimes x_b.$$

If $|a-b|=1$ and $|a-k|=1$,

$$(dh+hd)(wx_kx_a\otimes x_b)= (wx_kx_a\otimes x_b + wx_kx_b\otimes x_a) - wx_kx_b\otimes x_a= wx_kx_a\otimes x_b.$$

The cases with $l< 2$ can be easily checked.
\end{proof}

With the construction of the map $h$ finished, Proposition \ref{2: toy problem} follows from the next Lemma that, together with the previous Lemmas in this section, shows that $h$ is a contracting homotopy.

\begin{lemma}
It holds $hd = id_{C_n^{l}\otimes B_n^{(0)}}$.
\end{lemma}
\begin{proof}
If $l<2$ the statement is clear.
Let $l\geq 2$ and let $w$ be a word in the variables $x_i$ of length $l-2$.

It suffices to check the equation on elements of the form $wx_ax_b$ with $|a-b|\ne 1$ and on elements of the form $w(x_ax_b+x_bx_a)$, with $|a-b|= 1$.

For the first type of elements it is clear that $hd(wx_ax_b)=wx_ax_b$.

For the second type of elements, if $l$ is at least $3$ and $w= w'x_k$ with both $|a-k|$ and $|b-k|$ different from $1$,
$$hd(w(x_ax_b+x_bx_a)) = \frac{1}{2}w(x_ax_b+x_bx_a) + \frac{1}{2}w(x_bx_a+x_ax_b)= w(x_ax_b+x_bx_a).$$
The same calculation holds if $l=2$.

For the remaining case where (without loss of generality) $|a-k|=1$, we have

$$hd(wx_ax_b+wx_bx_a) = w(x_ax_b+x_bx_a) +0.$$
\end{proof}

\subsection{Remark:} After the submission of this manuscript Jan-Erik Roos has communicated to us a nicer and shorter proof of the Koszulness of the algebras $\cA_n$. If one uses the ordering of the generators $x_2,x_1,x_4,x_3, \dots ,x_{2k},x_{2k-1}$ (for $n=2k$ even) or $x_2,x_1,x_4,x_3, \dots ,x_{2k},x_{2k-1}x_{2k+1}$ (for $n=2k+1$ odd) instead of the standard ordering, then there is a finite Gr\"obner basis. Alternatively, one can see using the above ordering of the generators that the relations form a confluent rewriting system and hence $\cA_n$ is Koszul.

\bip

\section{ Computation of the cohomology of the operad ${\cB\cV}_\infty^{com}$}\label{app:BV_com proof}

\bip
 \subsection{\bf An equivalent definition of the operad $\cB\cV$}\label{5: subsect on L-diamond and BV}
Let $\caL^\diamond$ be an operad generated by two degree $-1$ corollas,
 $
\Ba{c}\resizebox{3.6mm}{!}{  \xy
(0,4)*{};
(0,0)*+{_1}*\cir{}
**\dir{-};
(0,-4)*{};
(0,0)*+{_1}*\cir{}
**\dir{-};
\endxy}\Ea$ and
$\Ba{c}\resizebox{7mm}{!}{
\xy
(-4,-4)*{};
(0,0)*+{_0}*\cir{}
**\dir{-};
(4,-4)*{};
(0,0)*+{_0}*\cir{}
**\dir{-};
(4,-6)*{_2};
(-4,-6)*{_1};
%
(0,5)*{};
(0,0)*+{_0}*\cir{}
**\dir{-};
\endxy}\Ea=
\Ba{c}\resizebox{7mm}{!}{ \xy
(-4,-4)*{};
(0,0)*+{_0}*\cir{}
**\dir{-};
(4,-4)*{};
(0,0)*+{_0}*\cir{}
**\dir{-};
(4,-6)*{_1};
(-4,-6)*{_2};
%
(0,5)*{};
(0,0)*+{_0}*\cir{}
**\dir{-};
\endxy}\Ea,
$
subject to the following relations,
$$
\Ba{c}
\resizebox{3.6mm}{!}{ \xy
(0,0)*+{_1}*\cir{}="b",
(0,6)*+{_1}*\cir{}="c",
%
%%%%%%%%%% edges to b %%%%%%%%%%%%
(0,-4)*{}="-1",
%%%%%%%%%% edges to c %%%%%%%%%%%%
(0,10)*{}="1'",
%%%%%%%%%%% internal edges %%%%%%%%%%%
\ar @{-} "b";"c" <0pt>
%
\ar @{-} "b";"-1" <0pt>
\ar @{-} "c";"1'" <0pt>
\endxy}
\Ea\hspace{-1mm} = 0\ \  ,\ \
%--------------------------------------------------------
%
%  new graph
%
\Ba{c}
\resizebox{6mm}{!}{ \xy
(0,0)*+{_0}*\cir{}="b",
(0,7)*+{_1}*\cir{}="c",
%
%%%%%%%%%% edges to b %%%%%%%%%%%%
(-4,-5)*{}="-1",
(-2,-5)*{}="-2",
(4,-5)*{}="-3",
%%%%%%%%%% edges to c %%%%%%%%%%%%
(0,12)*{}="1'",
%
%%%%%%%%%%% internal edges %%%%%%%%%%%
\ar @{-} "b";"c" <0pt>
%
\ar @{-} "b";"-1" <0pt>
%\ar @{-} "b";"-2" <0pt>
\ar @{-} "b";"-3" <0pt>
%
\ar @{-} "c";"1'" <0pt>
\endxy}
\Ea
+
\Ba{c}
\resizebox{8mm}{!}{ \xy
(0,7)*+{_0}*\cir{}="b",
(-4,0)*+{_1}*\cir{}="c",
%
(-4,-5)*{}="-1",
(4,-5)*{}="-2",
(4,1)*{}="-3",
(0,12)*{}="1'",
\ar @{-} "b";"c" <0pt>
%
\ar @{-} "c";"-1" <0pt>
%\ar @{-} "b";"-2" <0pt>
\ar @{-} "b";"-3" <0pt>
%
\ar @{-} "b";"1'" <0pt>
\endxy}
\Ea
+
\Ba{c}
\resizebox{8mm}{!}{ \xy
(0,7)*+{_0}*\cir{}="b",
(4,0)*+{_1}*\cir{}="c",
%
(4,-5)*{}="-1",
(-4,-5)*{}="-2",
(-4,1)*{}="-3",
(0,12)*{}="1'",
%
\ar @{-} "b";"c" <0pt>
%
\ar @{-} "c";"-1" <0pt>
\ar @{-} "b";"-3" <0pt>
%
\ar @{-} "b";"1'" <0pt>
\endxy}
\Ea
=0, \ \ \
%
\Ba{c}
\resizebox{10.5mm}{!}{ \xy
(-8,-7.5)*{_{_1}};
(-0,-7.5)*{_{_2}};
(4.5,-0.6)*{_{_3}};
%
(0,7)*+{_0}*\frm{o}="b",
(-4,0)*+{_0}*\frm{o}="c",
%
%%%%%%%%%% edges to b %%%%%%%%%%%%
(-8,-6)*{}="-1",
(0,-6)*{}="-2",
(4,1)*{}="-3",
%%%%%%%%%% edges to c %%%%%%%%%%%%
(0,12)*{}="1'",
%
%%%%%%%%%%% internal edges %%%%%%%%%%%
\ar @{-} "b";"c" <0pt>
%
\ar @{-} "c";"-1" <0pt>
\ar @{-} "c";"-2" <0pt>
\ar @{-} "b";"-3" <0pt>
%
\ar @{-} "b";"1'" <0pt>
\endxy}
\Ea
+
%%%%%%%%%%%%%%%%%%%%%%%%%%%%%%%%%%%%%%%%%%%%%%%%%%%%%%
\Ba{c}
\resizebox{10.5mm}{!}{ \xy
(-8,-7.5)*{_{_2}};
(-0,-7.5)*{_{_3}};
(4.5,-0.6)*{_{_1}};
%
(0,7)*+{_0}*\frm{o}="b",
(-4,0)*+{_0}*\frm{o}="c",
%
%%%%%%%%%% edges to b %%%%%%%%%%%%
(-8,-6)*{}="-1",
(0,-6)*{}="-2",
(4,1)*{}="-3",
%%%%%%%%%% edges to c %%%%%%%%%%%%
(0,12)*{}="1'",
%
%%%%%%%%%%% internal curved edges %%%%%%%%%%%
\ar @{-} "b";"c" <0pt>
%
\ar @{-} "c";"-1" <0pt>
\ar @{-} "c";"-2" <0pt>
\ar @{-} "b";"-3" <0pt>
%
\ar @{-} "b";"1'" <0pt>
\endxy}
\Ea
+
%%%%%%%%%%%%%%%%%%%%%%%%%%%%%%%%%%%%%%%%%%%%%%%%%%%%%%
\Ba{c}\resizebox{10.5mm}{!}{ \xy
(-8,-7.5)*{_{_3}};
(-0,-7.5)*{_{_1}};
(4.5,-0.6)*{_{_2}};
%
(0,7)*+{_0}*\frm{o}="b",
(-4,0)*+{_0}*\frm{o}="c",
%
%%%%%%%%%% edges to b %%%%%%%%%%%%
(-8,-6)*{}="-1",
(0,-6)*{}="-2",
(4,1)*{}="-3",
%%%%%%%%%% edges to c %%%%%%%%%%%%
(0,12)*{}="1'",
%
%%%%%%%%%%% internal curved edges %%%%%%%%%%%
\ar @{-} "b";"c" <0pt>
%
\ar @{-} "c";"-1" <0pt>
\ar @{-} "c";"-2" <0pt>
\ar @{-} "b";"-3" <0pt>
%
\ar @{-} "b";"1'" <0pt>
\endxy}
\Ea
=0
$$

Let $\cC om$ be the operad of commutative algebras with the generator controlling the graded commutative multiplication denoted by
$\begin{xy}
 <0mm,0.66mm>*{};<0mm,3mm>*{}**@{-},
 <0.39mm,-0.39mm>*{};<2.2mm,-2.2mm>*{}**@{-},
 <-0.35mm,-0.35mm>*{};<-2.2mm,-2.2mm>*{}**@{-},
 <0mm,0mm>*{\circ};<0mm,0mm>*{}**@{},
\end{xy}$. Define an operad, $\cB\cV$, of Batalin-Vilkovisky algebras as the free operad
generated by operads $\caL^\diamond$ and $\cC om$ modulo the following relations,
\Beq\label{5: BV operad relations 2}
\Ba{c}\resizebox{7mm}{!}{ \xy
(-4,-4)*{};
(0,0)*+{_0}*\cir{}
**\dir{-};
(4,-4)*{};
(0,0)*+{_0}*\cir{}
**\dir{-};
(4,-6)*{_2};
(-4,-6)*{_1};
%
(0,5)*{};
(0,0)*+{_0}*\cir{}
**\dir{-};
\endxy}\Ea
=
%%%%%%%%%%%%% new graph %%%%%%%%%%%%%%%%%%%%%
\Ba{c}
\resizebox{6mm}{!}{ \xy
(0,0)*{\circ}="b",
(0,6)*+{_1}*\cir{}="c",
%
%%%%%%%%%% edges to b %%%%%%%%%%%%
(-4,-5)*{}="-1",
(4,-5)*{}="-2",
%%%%%%%%%% edges to c %%%%%%%%%%%%
(0,12)*{}="1'",
%
%%%%%%%%%%% internal edges %%%%%%%%%%%
\ar @{-} "b";"c" <0pt>
%
\ar @{-} "b";"-1" <0pt>
\ar @{-} "b";"-2" <0pt>
%
\ar @{-} "c";"1'" <0pt>
\endxy}
\Ea
-
\Ba{c}
\resizebox{8mm}{!}{ \xy
(0,6)*{\circ}="b",
(-4,0)*+{_1}*\cir{}="c",
%
%%%%%%%%%% edges to b %%%%%%%%%%%%
(-4,-5)*{}="-1",
(4,-5)*{}="-2",
(4,1)*{}="-3",
%%%%%%%%%% edges to c %%%%%%%%%%%%
(0,12)*{}="1'",
%
%%%%%%%%%%% internal edges %%%%%%%%%%%
\ar @{-} "b";"c" <0pt>
%
\ar @{-} "c";"-1" <0pt>
%\ar @{-} "b";"-2" <0pt>
\ar @{-} "b";"-3" <0pt>
%
\ar @{-} "b";"1'" <0pt>
\endxy}
\Ea
-
\Ba{c}
\resizebox{8mm}{!}{ \xy
(0,6)*{\circ}="b",
(4,0)*+{_1}*\cir{}="c",
%
%%%%%%%%%% edges to b %%%%%%%%%%%%
(4,-5)*{}="-1",
(-4,-5)*{}="-2",
(-4,1)*{}="-3",
%%%%%%%%%% edges to c %%%%%%%%%%%%
(0,12)*{}="1'",
%
%%%%%%%%%%% internal curved edges %%%%%%%%%%%
\ar @{-} "b";"c" <0pt>
%
\ar @{-} "c";"-1" <0pt>
\ar @{-} "b";"-3" <0pt>
%
\ar @{-} "b";"1'" <0pt>
\endxy}
\Ea
\ \ \ ,\ \ \
\Ba{c}
\resizebox{10mm}{!}{ \xy
(-8,-7.5)*{_{_1}};
(-0,-7.5)*{_{_2}};
(4.5,-0.6)*{_{_3}};
%
(0,7)*+{_0}*\frm{o}="b",
(-4,0)*{\circ}="c",
%
%%%%%%%%%% edges to b %%%%%%%%%%%%
(-8,-6)*{}="-1",
(0,-6)*{}="-2",
(4,1)*{}="-3",
%%%%%%%%%% edges to c %%%%%%%%%%%%
(0,12)*{}="1'",
%
%%%%%%%%%%% internal edges %%%%%%%%%%%
\ar @{-} "b";"c" <0pt>
%
\ar @{-} "c";"-1" <0pt>
\ar @{-} "c";"-2" <0pt>
\ar @{-} "b";"-3" <0pt>
%
\ar @{-} "b";"1'" <0pt>
\endxy}
\Ea
-
%%%%%%%%%%%%%%%%%%%%%%%%%%%%%%%%%%%%%%%%%%%%%%%%%%%%%%
\Ba{c}
\resizebox{10mm}{!}{ \xy
(-8,-7.5)*{_{_2}};
(-0,-7.5)*{_{_3}};
(4.5,-0.6)*{_{_1}};
%
(0,7)*{\circ}="b",
(-4,0)*+{_0}*\frm{o}="c",
%
%%%%%%%%%% edges to b %%%%%%%%%%%%
(-8,-6)*{}="-1",
(0,-6)*{}="-2",
(4,1)*{}="-3",
%%%%%%%%%% edges to c %%%%%%%%%%%%
(0,12)*{}="1'",
%
%%%%%%%%%%% internal curved edges %%%%%%%%%%%
\ar @{-} "b";"c" <0pt>
%
\ar @{-} "c";"-1" <0pt>
\ar @{-} "c";"-2" <0pt>
\ar @{-} "b";"-3" <0pt>
%
\ar @{-} "b";"1'" <0pt>
\endxy}
\Ea
-
%%%%%%%%%%%%%%%%%%%%%%%%%%%%%%%%%%%%%%%%%%%%%%%%%%%%%%
\Ba{c}
\resizebox{10mm}{!}{ \xy
(-8,-7.5)*{_{_1}};
(-0,-7.5)*{_{_3}};
(4.5,-0.6)*{_{_2}};
%
(0,7)*{\circ}="b",
(-4,0)*+{_0}*\frm{o}="c",
%
%%%%%%%%%% edges to b %%%%%%%%%%%%
(-8,-6)*{}="-1",
(0,-6)*{}="-2",
(4,1)*{}="-3",
%%%%%%%%%% edges to c %%%%%%%%%%%%
(0,12)*{}="1'",
%
%%%%%%%%%%% internal curved edges %%%%%%%%%%%
\ar @{-} "b";"c" <0pt>
%
\ar @{-} "c";"-1" <0pt>
\ar @{-} "c";"-2" <0pt>
\ar @{-} "b";"-3" <0pt>
%
\ar @{-} "b";"1'" <0pt>
\endxy}
\Ea
=0.
\Eeq
In fact, the second relation in (\ref{5: BV operad relations 2}) follows from the previous ones.
We keep it the list in order to define, following \cite{GTV}, an operad $q\cB\cV$ as an operad freely generated by $\caL^\diamond$ and $\cC om$ modulo a version of relations (\ref{5: BV operad relations 2}) in which the first relation is replaced by the following one,
$$
\Ba{c}
\resizebox{6mm}{!}{ \xy
(0,0)*{\circ}="b",
(0,6)*+{_1}*\cir{}="c",
%
%%%%%%%%%% edges to b %%%%%%%%%%%%
(-4,-5)*{}="-1",
(4,-5)*{}="-2",
%%%%%%%%%% edges to c %%%%%%%%%%%%
(0,12)*{}="1'",
%
%%%%%%%%%%% internal edges %%%%%%%%%%%
\ar @{-} "b";"c" <0pt>
%
\ar @{-} "b";"-1" <0pt>
\ar @{-} "b";"-2" <0pt>
%
\ar @{-} "c";"1'" <0pt>
\endxy}
\Ea
-
\Ba{c}
\resizebox{8mm}{!}{ \xy
(0,6)*{\circ}="b",
(-4,0)*+{_1}*\cir{}="c",
%
%%%%%%%%%% edges to b %%%%%%%%%%%%
(-4,-5)*{}="-1",
(4,-5)*{}="-2",
(4,1)*{}="-3",
%%%%%%%%%% edges to c %%%%%%%%%%%%
(0,12)*{}="1'",
%
%%%%%%%%%%% internal edges %%%%%%%%%%%
\ar @{-} "b";"c" <0pt>
%
\ar @{-} "c";"-1" <0pt>
%\ar @{-} "b";"-2" <0pt>
\ar @{-} "b";"-3" <0pt>
%
\ar @{-} "b";"1'" <0pt>
\endxy}
\Ea
-
\Ba{c}
\resizebox{8mm}{!}{ \xy
(0,6)*{\circ}="b",
(4,0)*+{_1}*\cir{}="c",
%
%%%%%%%%%% edges to b %%%%%%%%%%%%
(4,-5)*{}="-1",
(-4,-5)*{}="-2",
(-4,1)*{}="-3",
%%%%%%%%%% edges to c %%%%%%%%%%%%
(0,12)*{}="1'",
%
%%%%%%%%%%% internal curved edges %%%%%%%%%%%
\ar @{-} "b";"c" <0pt>
%
\ar @{-} "c";"-1" <0pt>
\ar @{-} "b";"-3" <0pt>
%
\ar @{-} "b";"1'" <0pt>
\endxy}
\Ea=0.
$$
Being a quotient of a free operad, the operad $\cB\cV$ inherits an increasing filtration by the number of vertices in the trees. It is clear that there is a morphism,
$$
g: q\cB\cV\lon gr(\cB\cV),
$$
from $q\cB\cV$ into the associated graded operad.

\subsection{\bf Proposition \cite{GTV}}\label{5: Propos on qBV and BV} {\em The morphism $g:q\cB\cV\lon gr(\cB\cV)$ is an isomorphism. }


\subsection{\bf Remark}\label{5: remark on qBV} The relations in the operad $q\cB\cV$ are homogeneous. It is easy to see that, as an $\bS$-module, $q\cB\cV$ is isomorphic to  $\cC om \circ \caL^\diamond$,
the vector space spanned by graphs from $\cC om$ whose legs are decorated with elements from $\caL^\diamond$.





\subsection{An auxiliary dg operad}

\mip

For any natural number $a\geq 1$ define by induction (over the number, $k=1,2,\ldots, a+1$, of input legs) a collection of $a+1$ elements,
\Beq\label{5: square vertices}
\Ba{c}
{\resizebox{4.0mm}{!}{ \xy
(0,5)*{};
(0,0)*+{_a}*\frm{-}
**\dir{-};
(0,-5)*{};
(0,0)*+{_a}*\frm{-}
**\dir{-};
\endxy}}\Ea :=\Ba{c}
{\resizebox{4.0mm}{!}{ \xy
(0,5)*{};
(0,0)*+{_a}*\cir{}
**\dir{-};
(0,-5)*{};
(0,0)*+{_a}*\cir{}
**\dir{-};
\endxy}}\Ea\ \ , \ \ \ldots \ \ ,\
%%%%%%%%%%%%%%%%%%%%%\
\Ba{c}\resizebox{14mm}{!}{ \xy
(-7.5,-8.6)*{_{_1}};
(-4.1,-8.6)*{_{_2}};
(4.5,-8.6)*{_{_{k\hspace{-0.3mm}-\hspace{-0.3mm}1}}};
(9.0,-8.5)*{_{_{k}}};
%
(0.0,-6)*{...};
(0,5)*{};
(0,0)*+\hbox{$_{{a}}$}*\frm{-}
**\dir{-};
(-4,-7)*{};
(0,0)*+\hbox{$_{{a}}$}*\frm{-}
**\dir{-};
(-7,-7)*{};
(0,0)*+\hbox{$_{{a}}$}*\frm{-}
**\dir{-};
%
(8,-7)*{};
(0,0)*+\hbox{$_{{a}}$}*\frm{-}
**\dir{-};
(4,-7)*{};
(0,0)*+\hbox{$_{{a}}$}*\frm{-}
**\dir{-};
\endxy}\Ea:=
%%%%%%%%%%%%%%%%%%%
\Ba{c}
\resizebox{18mm}{!}{ \xy
(-7.5,-8.6)*{_{_1}};
(-4.1,-8.6)*{_{_2}};
(4.3,-8.6)*{_{_{k\hspace{-0.3mm}-\hspace{-0.3mm}2}}};
(6,-14.6)*{_{_{k\hspace{-0.3mm}-\hspace{-0.3mm}1}}};
(13.9,-14.6)*{_{_{k}}};
%
(0.0,-6)*{...};
(0,5)*{};
(0,0)*+\hbox{$_{{a}}$}*\frm{-}
**\dir{-};
(-4,-7)*{};
(0,0)*+\hbox{$_{{a}}$}*\frm{-}
**\dir{-};
(-7,-7)*{};
(0,0)*+\hbox{$_{{a}}$}*\frm{-}
**\dir{-};
%
(9,-7)*{\circ};
(0,0)*+\hbox{$_{{a}}$}*\frm{-}
**\dir{-};
(4,-7)*{};
(0,0)*+\hbox{$_{{a}}$}*\frm{-}
**\dir{-};
(9,-7)*{\circ};
%
 <9.3mm,-7.3mm>*{};<13mm,-13mm>*{}**@{-},
 <8.7mm,-7.3mm>*{};<6mm,-13mm>*{}**@{-},
\endxy}
\Ea
-
\Ba{c}
\resizebox{18mm}{!}{ \xy
(-7.5,-8.6)*{_{_1}};
(-4.1,-8.6)*{_{_2}};
(4.3,-8.6)*{_{_{k\hspace{-0.3mm}-\hspace{-0.3mm}2}}};
(11,-8.6)*{_{_{k\hspace{-0.3mm}-\hspace{-0.3mm}1}}};
(11,-1)*{_{_{k}}};
%
(0.0,-6)*{...};
(5,7)*{\circ};
(0,0)*+\hbox{$_{{a}}$}*\frm{-}
**\dir{-};
(-4,-7)*{};
(0,0)*+\hbox{$_{{a}}$}*\frm{-}
**\dir{-};
(-7,-7)*{};
(0,0)*+\hbox{$_{{a}}$}*\frm{-}
**\dir{-};
%
(9,-7)*{};
(0,0)*+\hbox{$_{{a}}$}*\frm{-}
**\dir{-};
(4,-7)*{};
(0,0)*+\hbox{$_{{a}}$}*\frm{-}
**\dir{-};
(9,-7)*{};
%
 <5.0mm,7.5mm>*{};<5.0mm,13.9mm>*{}**@{-},
 <5.4mm,6.6mm>*{};<10mm,1mm>*{}**@{-},
\endxy}
\Ea
-
\Ba{c}
\resizebox{18mm}{!}{ \xy
(-7.5,-8.6)*{_{_1}};
(-4.1,-8.6)*{_{_2}};
(4.3,-8.6)*{_{_{k\hspace{-0.3mm}-\hspace{-0.3mm}2}}};
(11,-1)*{_{_{k\hspace{-0.3mm}-\hspace{-0.3mm}1}}};
(10,-8.6)*{_{_{k}}};
%
(0.0,-6)*{...};
(5,7)*{\circ};
(0,0)*+\hbox{$_{{a}}$}*\frm{-}
**\dir{-};
(-4,-7)*{};
(0,0)*+\hbox{$_{{a}}$}*\frm{-}
**\dir{-};
(-7,-7)*{};
(0,0)*+\hbox{$_{{a}}$}*\frm{-}
**\dir{-};
%
(9,-7)*{};
(0,0)*+\hbox{$_{{a}}$}*\frm{-}
**\dir{-};
(4,-7)*{};
(0,0)*+\hbox{$_{{a}}$}*\frm{-}
**\dir{-};
(9,-7)*{};
%
 <5.0mm,7.5mm>*{};<5.0mm,13.9mm>*{}**@{-},
 <5.4mm,6.6mm>*{};<10mm,1mm>*{}**@{-},
\endxy}
\Ea
\Eeq
of the operad $\cB\cV_\infty^{com}$. If $\rho: \cB\cV_\infty^{com}\rar \cE nd_V$ is a representation,
then, in the notation of \S{\ref{5: subsec on order of operators}},
$$
\rho\left(\Ba{c}\resizebox{14mm}{!}{ \xy
(-7.5,-8.6)*{_{_1}};
(-4.1,-8.6)*{_{_2}};
(4.5,-8.6)*{_{_{k\hspace{-0.3mm}-\hspace{-0.3mm}1}}};
(9.0,-8.5)*{_{_{k}}};
%
(0.0,-6)*{...};
(0,5)*{};
(0,0)*+\hbox{$_{{a}}$}*\frm{-}
**\dir{-};
(-4,-7)*{};
(0,0)*+\hbox{$_{{a}}$}*\frm{-}
**\dir{-};
(-7,-7)*{};
(0,0)*+\hbox{$_{{a}}$}*\frm{-}
**\dir{-};
%
(8,-7)*{};
(0,0)*+\hbox{$_{{a}}$}*\frm{-}
**\dir{-};
(4,-7)*{};
(0,0)*+\hbox{$_{{a}}$}*\frm{-}
**\dir{-};
\endxy}\Ea\right)= F_k^{\Delta_a}.
$$
Note that $F_k^{\Delta_a}$ identically vanishes for $k\geq a+2$ as the operator $\Delta_a$ is, by its definition, of order $\leq a+1$; this is the reason why we defined the above elements of $\cB\cV_\infty^{com}$ only in the range
$1\leq k\leq a+1$: for all other $k$ these elements vanish identically due to the relations between the generators of $\cB\cV_\infty^{com}$.

\mip

Consider next a free operad, $\caL ^\diamond_{\infty}$, generated, for all integers $p\geq 0$, $k\geq 1$ with
 $p+k\geq 2$, by the symmetric corollas
$$
\Ba{c}\resizebox{14mm}{!}{ \xy
(-7.5,-8.6)*{_{_1}};
(-4.1,-8.6)*{_{_2}};
%(4.5,-8.6)*{_{_{k\hspace{-0.3mm}-\hspace{-0.3mm}1}}};
(9.0,-8.5)*{_{_{k}}};
%
(0.0,-6)*{...};
(0,5)*{};
(0,0)*+\hbox{$_{{p}}$}*\frm{o}
**\dir{-};
(-4,-7)*{};
(0,0)*+\hbox{$_{{p}}$}*\frm{o}
**\dir{-};
(-7,-7)*{};
(0,0)*+\hbox{$_{{p}}$}*\frm{o}
**\dir{-};
%
(8,-7)*{};
(0,0)*+\hbox{$_{{p}}$}*\frm{o}
**\dir{-};
(4,-7)*{};
(0,0)*+\hbox{$_{{p}}$}*\frm{o}
**\dir{-};
\endxy}\Ea=
\Ba{c}\resizebox{16.2mm}{!}{ \xy
(-8.5,-8.6)*{_{_{\sigma(1)}}};
(-3.1,-8.6)*{_{_{\sigma(2)}}};
%(4.5,-8.6)*{_{_{k\hspace{-0.3mm}-\hspace{-0.3mm}1}}};
(9.0,-8.5)*{_{_{\sigma(k)}}};
%
(0.0,-6)*{...};
(0,5)*{};
(0,0)*+\hbox{$_{{p}}$}*\frm{o}
**\dir{-};
(-4,-7)*{};
(0,0)*+\hbox{$_{{p}}$}*\frm{o}
**\dir{-};
(-7,-7)*{};
(0,0)*+\hbox{$_{{p}}$}*\frm{o}
**\dir{-};
%
(8,-7)*{};
(0,0)*+\hbox{$_{{p}}$}*\frm{o}
**\dir{-};
(4,-7)*{};
(0,0)*+\hbox{$_{{p}}$}*\frm{o}
**\dir{-};
\endxy}\Ea\ \ \ \forall\ \sigma\in \bS_n,
$$
of homological degree $3-2k-2p$, and equipped with the following differential
$$
d\Ba{c}
\resizebox{14mm}{!}{ \xy
(-7.5,-8.6)*{_{_1}};
(-4.1,-8.6)*{_{_2}};
%(4.5,-8.6)*{_{_{k\hspace{-0.3mm}-\hspace{-0.3mm}1}}};
(9.0,-8.5)*{_{_{k}}};
%
(0.0,-6)*{...};
(0,5)*{};
(0,0)*+\hbox{$_{{p}}$}*\frm{o}
**\dir{-};
(-4,-7)*{};
(0,0)*+\hbox{$_{{p}}$}*\frm{o}
**\dir{-};
(-7,-7)*{};
(0,0)*+\hbox{$_{{p}}$}*\frm{o}
**\dir{-};
%
(8,-7)*{};
(0,0)*+\hbox{$_{{p}}$}*\frm{o}
**\dir{-};
(4,-7)*{};
(0,0)*+\hbox{$_{{p}}$}*\frm{o}
**\dir{-};
\endxy}\Ea
=
%%%%%%%%%%%%%%%%%%%%%%%%%%%%%
\sum_{p=q+r\atop [k]=I_1\sqcup I_2}
\Ba{c}
%
%
%%%%%%%%%%%%%%%% two vertex graph with 1 internal edge %%%%%%%%%%
\resizebox{19mm}{!}{ \xy
(0,0)*+{q}*\cir{}="b",
(10,10)*+{r}*\cir{}="c",
%
%%%%%%%%%% edges to b %%%%%%%%%%%%
(-4,-6)*{}="-1",
(-2,-6)*{}="-2",
(4,-6)*{}="-3",
(1,-5)*{...},
(0,-8)*{\underbrace{\ \ \ \ \ \ \ \ }},
(0,-11)*{_{I_1}},
%%%%%%%%%% edges to c %%%%%%%%%%%%
%(6,16)*{}="1'",
(10,16)*{}="2'",
%(14,16)*{}="3'",
%(11,15)*{...},
(11,4)*{}="-1'",
(16,4)*{}="-2'",
(18,4)*{}="-3'",
(13.5,4)*{...},
(15,2)*{\underbrace{\ \ \ \ \ \ \ }},
(15,-1)*{_{I_2}},
%
%%%%%%%%%%% internal curved edges %%%%%%%%%%%
\ar @{-} "b";"c" <0pt>
%
\ar @{-} "b";"-1" <0pt>
\ar @{-} "b";"-2" <0pt>
\ar @{-} "b";"-3" <0pt>
%
%\ar @{-} "c";"1'" <0pt>
\ar @{-} "c";"2'" <0pt>
%\ar @{-} "c";"3'" <0pt>
\ar @{-} "c";"-1'" <0pt>
\ar @{-} "c";"-2'" <0pt>
\ar @{-} "c";"-3'" <0pt>
\endxy}
%%%%%%%%%%%%%%%%%%%%%%%%%%%%%%%%%%%%%%%%%%%
\Ea
$$
Representations, $\rho: \caL^\diamond_\infty\rar \cE nd_V$, of this operad in a dg vector space $(V,d)$
are the same thing as continuous representations of  the  operad $\caL ie_\infty\{1\}[[\hbar]]$
in the topological vector space $V[[\hbar]]$ equipped with the differential
$$
-d + \sum_{p\geq 1}\hbar^p \Delta_p, \ \ \ \Delta_p:=\rho \left(\Ba{c}
\resizebox{4mm}{!}{ \xy
(0,5)*{};
(0,0)*+{_a}*\cir{}
**\dir{-};
(0,-5)*{};
(0,0)*+{_a}*\cir{}
**\dir{-};
\endxy}\Ea\right).
$$
where the formal parameter $\hbar$ is assumed to have homological degree $2$.

\begin{proposition}\label{5: Propos on cohom of L diamnd infty}  { The cohomology of the dg operad $\caL^\diamond_\infty$ is the operad $\caL^\diamond$ defined in \S {\ref{5: subsect on L-diamond and BV}}, i.e.\
$\caL^\diamond_\infty$ is a minimal resolution of $\caL^\diamond$.
}
\end{proposition}
\begin{proof}
The dg operad $\caL_\infty^\diamond\{1\}$ is a direct summand of the graded properad $gr\LoB_\infty$
associated with the genus filtration of the properad $\LoB_\infty$ .
Hence the required result follows from the proof of  Proposition~{\ref{2: proposition on nu quasi-iso}}.
\end{proof}

\mip
We are interested in the operad $\caL_\infty^\diamond$ because of the following property.

\begin{lemma}\label{5: monomorphism chi} { There is a monomorphism of dg operads,
$$
\chi: \caL_\infty^\diamond \lon \cB\cV_\infty^{com}
$$
given on generators as follows,
$$
\chi\left(\Ba{c}
\resizebox{12mm}{!}{ \xy
(-7.5,-7.6)*{_{_1}};
(-4.1,-7.6)*{_{_2}};
(4.5,-7.6)*{_{_{k\hspace{-0.3mm}-\hspace{-0.3mm}1}}};
(9.0,-7.5)*{_{_{k}}};
%
(0.0,-5)*{...};
(0,5)*{};
(0,0)*+\hbox{$_{{p}}$}*\frm{o}
**\dir{-};
(-4,-6)*{};
(0,0)*+\hbox{$_{{p}}$}*\frm{o}
**\dir{-};
(-7,-6)*{};
(0,0)*+\hbox{$_{{p}}$}*\frm{o}
**\dir{-};
%
(8,-6)*{};
(0,0)*+\hbox{$_{{p}}$}*\frm{o}
**\dir{-};
(4,-6)*{};
(0,0)*+\hbox{$_{{p}}$}*\frm{o}
**\dir{-};
\endxy}\Ea\right)
=
\Ba{c}\resizebox{12mm}{!}{
\xy
(-7.5,-7.6)*{_{_1}};
(-4.1,-7.6)*{_{_2}};
(4.5,-7.6)*{_{_{k\hspace{-0.3mm}-\hspace{-0.3mm}1}}};
(9.0,-7.5)*{_{_{k}}};
%
(0.0,-5)*{...};
(0,5)*{};
(0,0)*+\hbox{$_{{p+k-1}}$}*\frm{-}
**\dir{-};
(-4,-6)*{};
(0,0)*+\hbox{$_{{p+k-1}}$}*\frm{-}
**\dir{-};
(-7,-6)*{};
(0,0)*+\hbox{$_{{p+k-1}}$}*\frm{-}
**\dir{-};
%
(8,-6)*{};
(0,0)*+\hbox{$_{{p+k-1}}$}*\frm{-}
**\dir{-};
(4,-6)*{};
(0,0)*+\hbox{$_{{p+k-1}}$}*\frm{-}
**\dir{-};
\endxy}\Ea
$$
}
\end{lemma}
\begin{proof} For notation reason, we prove the proposition in terms of representations: for any representation
$\rho: \cB\cV_\infty^{com}\rar \cE nd_V$ we construct an associated representation $\rho':  \caL_\infty^\diamond\rar \cE nd_V$ such that $\rho'=\rho\circ \chi$.

\sip

Let $\{\Delta_a: V\rar V[1-2a] , \mu:\odot^2 V\rar V\}_{a\geq 1}$ be a $\cB\cV_\infty^{com}$-structure
in a dg vector space $(V,d)$. Then
$$
\Delta:= -d + \sum_{a\geq 1}\hbar^a \Delta_{a}
$$
is a degree 1 differential in the graded vector space $V[[\hbar]]$, $\hbar$ being a formal parameter of homological degree $2$.  As explained in \S {\ref{5: subsec on order of operators}}, this differential makes
the graded commutative algebra $(V[[\hbar]], \mu)$ into a $\caL ie_\infty\{1\}[[\hbar]]$ algebra over the ring
$\K[[\hbar]]$, with higher Lie brackets given by,
$$
F^\Delta_k:= \frac{1}{(k-1)!}\underbrace{[\ldots[[\Delta,\mu],\mu], \ldots, \mu]}_{k-1\ \mathrm{brackets}}=
\frac{\hbar^{k-1}}{(k-1)!}\sum_{p=0}^\infty\hbar^p
\underbrace{[\ldots[[\Delta_{p+k-1},\mu],\mu], \ldots, \mu]}_{k-1\ \mathrm{brackets}}
%[\ldots[[\Delta_{p+k-1},\mu],\mu], \ldots, \mu]:= \hbar^{p-1}\sum_{k=0}^\infty\hbar^k \mu_{p}^{k}
$$
It is well-known that $\caL ie_\infty$-algebra structures are homogeneous in the sense that if $\{\mu_n\}_{n\geq 1}$ is a $\caL ie_\infty$-algebra structure in some vector space, then, for any $\la\in \K$, the collection
 $\{\la^{n-1}\mu_n\}_{n\geq 1}$ is again a  $\caL ie_\infty$-algebra structure in the same space. Therefore, the rescaled collection of operations,
 $$
 \hat{F}^\Delta_k:=\frac{1}{(k-1)!}\sum_{p=0}^\infty\hbar^p
\underbrace{[\ldots[[\Delta_{p+k-1},\mu],\mu], \ldots, \mu]}_{k-1\ \mathrm{brackets}}
 $$
 also defines a continuous representation of $\caL ie_\infty\{1\}[[\hbar]]$ in the dg space $(V[[\hbar]],\Delta)$, and hence a representation of $\caL_\infty^\diamond$ in the space $(V,d)$ given on the generators as follows,
 $$
 \rho'\left(\Ba{c}
\resizebox{12mm}{!}{ \xy
(-7.5,-7.6)*{_{_1}};
(-4.1,-7.6)*{_{_2}};
(4.5,-7.6)*{_{_{k\hspace{-0.3mm}-\hspace{-0.3mm}1}}};
(9.0,-7.5)*{_{_{k}}};
%
(0.0,-5)*{...};
(0,5)*{};
(0,0)*+\hbox{$_{{p}}$}*\frm{o}
**\dir{-};
(-4,-6)*{};
(0,0)*+\hbox{$_{{p}}$}*\frm{o}
**\dir{-};
(-7,-6)*{};
(0,0)*+\hbox{$_{{p}}$}*\frm{o}
**\dir{-};
%
(8,-6)*{};
(0,0)*+\hbox{$_{{p}}$}*\frm{o}
**\dir{-};
(4,-6)*{};
(0,0)*+\hbox{$_{{p}}$}*\frm{o}
**\dir{-};
\endxy}\Ea\right)
= \frac{1}{(k-1)!}\underbrace{[\ldots[[\Delta_{p+k-1},\mu],\mu], \ldots, \mu]}_{k-1\ \mathrm{brackets}}
 $$
As
$
\rho\left(
\Ba{c}\resizebox{12mm}{!}{ \xy
(-7.5,-7.6)*{_{_1}};
(-4.1,-7.6)*{_{_2}};
(4.5,-7.6)*{_{_{k\hspace{-0.3mm}-\hspace{-0.3mm}1}}};
(9.0,-7.5)*{_{_{k}}};
%
(0.0,-5)*{...};
(0,5)*{};
(0,0)*+\hbox{$_{{p+k-1}}$}*\frm{-}
**\dir{-};
(-4,-6)*{};
(0,0)*+\hbox{$_{{p+k-1}}$}*\frm{-}
**\dir{-};
(-7,-6)*{};
(0,0)*+\hbox{$_{{p+k-1}}$}*\frm{-}
**\dir{-};
%
(8,-6)*{};
(0,0)*+\hbox{$_{{p+k-1}}$}*\frm{-}
**\dir{-};
(4,-6)*{};
(0,0)*+\hbox{$_{{p+k-1}}$}*\frm{-}
**\dir{-};
\endxy}\Ea\right)$ equals $\frac{1}{(k-1)!} \underbrace{[\ldots[[\Delta_{p+k-1},\mu],\mu], \ldots, \mu]}_{k-1\ \mathrm{brackets}}
$
as well, the proof is completed.
\end{proof}


% \begin{theorem}\label{5: Theorem on cohom BV_com} { The dg operad $\BV_\infty^{com}$ is formal with the cohomology
% operad $H^\bu(\BV_\infty^{com})$ isomorphic to the operad, $\BV$, of
% Batalin-Vilkovisky algebras, i.e.\ there is a canonical surjective quasi-isomorphism of
% operads,
% $$
% \pi: \BV_\infty^{com} \lon \BV
% $$
% which sends to zero all generators $\xy
% (0,5)*{};
% (0,0)*+{a}*\cir{}
% **\dir{-};
% (0,-5)*{};
% (0,0)*+{a}*\cir{}
% **\dir{-};
% \endxy$ with $a\geq 2$.
%  }
%  \end{theorem}

Finally we can give the proof of Theorem \ref{5: Theorem on cohom BV_com}.




\begin{proof}[Proof of Theorem \ref{5: Theorem on cohom BV_com}] The map $\pi$ obviously induces a morphism of operads,
$$
[\pi]: H^\bu(\BV_\infty^{com})\lon \cB\cV.
$$
Therefore to prove the theorem it is enough to show that $[\pi]$ induces an isomorphism of $\bS$-modules.
\sip

Denote  the following (equivalence class of a) graph in $\cB\cV_\infty^{com}$ by
$$
\underbrace{\xy
(1,-4)*{...};
(0,5)*{};
(0,0)*+\hbox{$_k$}*\frm{.}
**\dir{-};
(-2,-5)*{};
(0,0)*+\hbox{$_k$}*\frm{.}
**\dir{-};
(-4,-5)*{};
(0,0)*+\hbox{$_k$}*\frm{.}
**\dir{-};
(4,-5)*{};
(0,0)*+\hbox{$_k$}*\frm{.}
**\dir{-};
\endxy}_{k\geq 2\ \mathrm{legs}}:=
\underbrace{
\Ba{c}
\xy
%
(0,-1)*+{}="0",
(0,-6)*{\circ}="-1",
(-3,-9)*{\circ}="-2",
(3,-9)*{}="-2'",
(-6,-12)*{}="-3",
(-0,-12)*{}="-3'",
(-6.7,-12.7)*{\cdot};
(-7.7,-13.7)*{\cdot};
(-9,-15)*{\circ}="-4",
(-12,-18)*{}="-4'",
(-6,-18)*{}="-4''",
%
%\ar @{-} "1";"2" <0pt>
%\ar @{-} "1";"2'" <0pt>
%\ar @{-} "2";"3" <0pt>
%\ar @{-} "2";"3'" <0pt>
%\ar @{-} "4";"4'" <0pt>
\ar @{-} "-1";"0" <0pt>
%
\ar @{-} "-1";"-2" <0pt>
\ar @{-} "-1";"-2'" <0pt>
\ar @{-} "-2";"-3" <0pt>
\ar @{-} "-2";"-3'" <0pt>
\ar @{-} "-4";"-4'" <0pt>
\ar @{-} "-4";"-4''" <0pt>
\endxy
\Ea}_{k\ \mathrm{legs}}
$$
and call it a {\em dashed square vertex}. Consider an operad, $\f$, freely  generated by
 the operad $\cC om$ and  a countable family of unary operations,
$
\left\{ \Ba{c}\resizebox{3.1mm}{!}{  \xy
(0,5)*{};
(0,0)*+{_a}*\cir{}
**\dir{-};
(0,-5)*{};
(0,0)*+{_a}*\cir{}
**\dir{-};
\endxy}\Ea \right\}_{a\geq 1}
$
of homological degree $1-2a$ equipped with the differential \eqref{5: diff in BV comm}, the properad
$\cB\cV_\infty^{com}$ is the quotient of $\f$ by the ideal
 encoding the requirement that each unary operation $\Ba{c}\resizebox{2.9mm}{!}{\xy
(0,5)*{};
(0,0)*+{_a}*\cir{}
**\dir{-};
(0,-5)*{};
(0,0)*+{_a}*\cir{}
**\dir{-};
\endxy}\Ea$  is of order $\leq a+1$ with
respect to the multiplication operation.
Let $t^{(1)}$ be  any tree built from the following ``corollas",
$$
\Ba{c}
\resizebox{9mm}{!}{
\xy
(-4,-8)*{_{_1}};
(-2,-8)*{_{_2}};
(4.9,-8)*{_{_{k}}};
%
(0,0)*++{_a}*\frm<8pt,8pt>{ee}="b",
%
(-4,-6)*{}="-1",
(-2,-6)*{}="-2",
(4,-6)*{}="-3",
(1,-5)*{...},
(0,7)*{}="1'",
%
\ar @{-} "b";"1'" <0pt>
%
\ar @{-} "b";"-1" <0pt>
\ar @{-} "b";"-2" <0pt>
\ar @{-} "b";"-3" <0pt>
%
\endxy}
\Ea
:=
\Ba{c}
\resizebox{6mm}{!}{ \xy
(0,0)*+{k}*\frm{.}="b",
(0,7)*+{a}*\cir{}="c",
%
%%%%%%%%%% edges to b %%%%%%%%%%%%
(-4,-6)*{}="-1",
(-2,-6)*{}="-2",
(4,-6)*{}="-3",
(1,-5)*{...},
%%%%%%%%%% edges to c %%%%%%%%%%%%
(0,13)*{}="1'",
%
%%%%%%%%%%% internal curved edges %%%%%%%%%%%
\ar @{-} "b";"c" <0pt>
%
\ar @{-} "b";"-1" <0pt>
\ar @{-} "b";"-2" <0pt>
\ar @{-} "b";"-3" <0pt>
%
\ar @{-} "c";"1'" <0pt>
\endxy}\Ea\ \mbox{with}\ \ k\geq 1, a\geq 1
$$
(where we  assume implicitly that for $k=1$ the l.h.s.\ corolla equals  $\Ba{c}
{\resizebox{3.0mm}{!}{ \xy
(0,5)*{};
(0,0)*+{_a}*\cir{}
**\dir{-};
(0,-5)*{};
(0,0)*+{_a}*\cir{}
**\dir{-};
\endxy}}\Ea$) and let $t^{(2)}$ be any graph obtained by attaching to one or more (or none) input leg of a dashed
square vertex a tree of the type $t^{(1)}$, e.g.
$$
t^{(2)}=
%%%%%%%%%%%%%%%
\Ba{c}
\resizebox{16mm}{!}{ \xy
(0,0)*+{_{t_1^{(1)}}}="b1",
(7,0)*+{_{t_2^{(1)}}}="b2",
(14,2)*+{}="b3",
(22,0)*+{_{t_k^{(1)}}}="b4",
(10,10)*+{k}*\frm{.}="c",
(10,18)*+{}="u",
%
%%%%%%%%%%% internal edges %%%%%%%%%%%
\ar @{-} "u";"c" <0pt>
\ar @{-} "b1";"c" <0pt>
\ar @{-} "b2";"c" <0pt>
\ar @{-} "b3";"c" <0pt>
\ar @{-} "b4";"c" <0pt>
\endxy}
\Ea
$$
The family $\{t^{(1)}, t^{(2)}\}$ forms a basis of $\f$ as an $\bS$-module.
%is precisely the family of reduced graphs forming a basis of $\cB\cV_\infty^{com}$.

\sip

Define for any $a,k\geq 1$ a linear combination,
\Beq\label{C: from T to T}
\Ba{c}
\resizebox{13mm}{!}{ \xy
(-7.5,-8.6)*{_{_1}};
(-4.1,-8.6)*{_{_2}};
(4.5,-8.6)*{_{_{k\hspace{-0.3mm}-\hspace{-0.3mm}1}}};
(9.0,-8.5)*{_{_{k}}};
%
(0.0,-6)*{...};
(0,5)*{};
(0,0)*+\hbox{$_{{\ a\ }}$}*\frm{-}
**\dir{-};
(-4,-7)*{};
(0,0)*+\hbox{$_{{a}}$}*\frm{-}
**\dir{-};
(-7,-7)*{};
(0,0)*+\hbox{$_{{a}}$}*\frm{-}
**\dir{-};
%
(8,-7)*{};
(0,0)*+\hbox{$_{{a}}$}*\frm{-}
**\dir{-};
(4,-7)*{};
(0,0)*+\hbox{$_{{a}}$}*\frm{-}
**\dir{-};
\endxy}\Ea
=
%%%%%%%%%%%%%%%%%%%%%\
\Ba{c}
\resizebox{9mm}{!}{
\xy
(-4,-8)*{_{_1}};
(-2,-8)*{_{_2}};
(4.9,-8)*{_{_{k}}};
%
(0,0)*++{_a}*\frm<8pt,8pt>{ee}="b",
%
(-4,-6)*{}="-1",
(-2,-6)*{}="-2",
(4,-6)*{}="-3",
(1,-5)*{...},
(0,7)*{}="1'",
%
\ar @{-} "b";"1'" <0pt>
%
\ar @{-} "b";"-1" <0pt>
\ar @{-} "b";"-2" <0pt>
\ar @{-} "b";"-3" <0pt>
%
\endxy}
\Ea
%%%%%%%%%%%%%%%%%%%
-\sum_{\sigma\in \bS_k}\frac{1}{(k-1)!}\hspace{-7mm}
\Ba{c}
\resizebox{19mm}{!}{ \xy
(-8.5,-8.6)*{_{_{\sigma(1)}}};
(-3.1,-8.6)*{_{_{\sigma(2)}}};
%(4.3,-8.6)*{_{_{k\hspace{-0.3mm}-\hspace{-0.3mm}2}}};
(11,-8.6)*{_{_{\sigma(k\hspace{-0.3mm}-\hspace{-0.3mm}1)}}};
(11,-1)*{_{_{\sigma(k)}}};
%
(0.0,-6)*{...};
(5,7)*+{2}*\frm{.};
%{\circ};
% (10,10)*+{k}*\frm{.}="c"
(0,0)*++\hbox{$_{{a}}$}*\frm<8pt,8pt>{ee}
**\dir{-};
(-4,-7)*{};
(0,0)*++\hbox{$_{{}}$}*\frm<8pt,8pt>{}
**\dir{-};
(-7,-7)*{};
(0,0)*++\hbox{$_{{}}$}*\frm<8pt,8pt>{}
**\dir{-};
%
(9,-7)*{};
(0,0)*++\hbox{$_{{}}$}*\frm<8pt,8pt>{}
**\dir{-};
(4,-7)*{};
(0,0)*++\hbox{$_{{}}$}*\frm<8pt,8pt>{}
**\dir{-};
(9,-7)*{};
%
 <5.0mm,9.0mm>*{};<5.0mm,13.9mm>*{}**@{-},
 <6.4mm,4.6mm>*{};<10mm,1mm>*{}**@{-},
\endxy}
\Ea
+
\sum_{\sigma\in \bS_k}\frac{1}{2!(k-2)!}\hspace{-3mm}
\Ba{c}
\resizebox{19mm}{!}{ \xy
(-7.5,-8.6)*{_{_1}};
(-4.1,-8.6)*{_{_2}};
(7,-1)*{_{_{\sigma(k\hspace{-0.3mm}-\hspace{-0.3mm}1)}}};
(10,-8.6)*{_{_{\sigma(k\hspace{-0.3mm}-\hspace{-0.3mm}2)}}};
(14,-1)*{_{_{\sigma(k)}}};
%
(0.0,-6)*{...};
(6,7)*+{3}*\frm{.};
(0,0)*++\hbox{$_{{a}}$}*\frm<8pt,8pt>{ee}
**\dir{-};
(-4,-7)*{};
(0,0)*++\hbox{$_{{}}$}*\frm<8pt,8pt>{}
**\dir{-};
(-7,-7)*{};
(0,0)*++\hbox{$_{{}}$}*\frm<8pt,8pt>{}
**\dir{-};
%
(9,-7)*{};
(0,0)*++\hbox{$_{{}}$}*\frm<8pt,8pt>{}
**\dir{-};
(4,-7)*{};
(0,0)*++\hbox{$_{{}}$}*\frm<8pt,8pt>{}
**\dir{-};
(9,-7)*{};
%
 <6.0mm,9mm>*{};<6.0mm,13.9mm>*{}**@{-},
 <6.0mm,4.6mm>*{};<6.0mm,1mm>*{}**@{-},
 <7.4mm,4.6mm>*{};<11mm,1mm>*{}**@{-},
\endxy}
\Ea
+ \ldots,
\Eeq
and consider (i) a set  $\{T^{(1)}\}$  of all possible trees generated
by these ``square" corollas, e.g.
%$ \Ba{c}\resizebox{12mm}{!}{ \xy
%(-7.5,-7.6)*{_{_1}};
%(-4.1,-7.6)*{_{_2}};
%(4.5,-7.6)*{_{_{k\hspace{-0.3mm}-\hspace{-0.3mm}1}}};
%(9.0,-7.5)*{_{_{k}}};
%
%(0.0,-5)*{...};
%(0,5)*{};
%(0,0)*+\hbox{$_{{p+k-1}}$}*\frm{-}
%**\dir{-};
%(-4,-6)*{};
%(0,0)*+\hbox{$_{{p+k-1}}$}*\frm{-}
%**\dir{-};
%(-7,-6)*{};
%(0,0)*+\hbox{$_{{p+k-1}}$}*\frm{-}
%**\dir{-};
%
%(8,-6)*{};
%(0,0)*+\hbox{$_{{p+k-1}}$}*\frm{-}
%**\dir{-};
%(4,-6)*{};
%(0,0)*+\hbox{$_{{p+k-1}}$}*\frm{-}
%**\dir{-};
%\endxy}\Ea$ with $p\geq 0$, $ k\geq 1$, $p+k\geq 2$, e.~g.
%%%%%%%%%%%%%%%%%%
$$
T^{(1)}=\Ba{c}
\resizebox{25mm}{!}{ \xy
(0,0)*+{a_2}*\frm{-}="b",
(0,0)*+{\ a_2 \ }*\frm{-},
(10,10)*+{a_1}*\frm{-}="c",
(10,10)*+{\ a_1\ }*\frm{-},
(22,0)*+{a_3}*\frm{-}="r",
(22,0)*+{\ a_3\ }*\frm{-},
(15,-10)*+{a_4}*\frm{-}="r1",
(15,-10)*+{\ a_4\ }*\frm{-},
(32,-10)*+{}="r2",
(15,-18)*+{}="d",
%r
%%%%%%%%%% edges to b %%%%%%%%%%%%
(-4,-6)*{}="-1",
(-2,-6)*{}="-2",
(4,-6)*{}="-3",
(1,-6)*{}="-4",
%%%%%%%%%% edges to c %%%%%%%%%%%%
(10,16)*{}="2'",
%
(11,4)*{}="-1'",
(16,4)*{}="-2'",
(18,4)*{}="-3'",
%
%%%%%%%%%%% internal curved edges %%%%%%%%%%%
\ar @{-} "b";"c" <0pt>
\ar @{-} "r";"c" <0pt>
\ar @{-} "r";"r1" <0pt>
\ar @{-} "r";"r2" <0pt>
\ar @{-} "r1";"d" <0pt>
%
\ar @{-} "b";"-1" <0pt>
\ar @{-} "b";"-2" <0pt>
\ar @{-} "b";"-3" <0pt>
\ar @{-} "b";"-4" <0pt>
%
\ar @{-} "c";"2'" <0pt>
\ar @{-} "c";"-1'" <0pt>
\endxy}\Ea
$$
\
and also (ii) a set  $\{T^{(2)}\}$ of all possible trees  obtained by attaching to (some) legs of a dashed square vertex trees from the set $\{T^{(1)}\}$, e.g.
$$
T^{(2)}=
%%%%%%%%%%%%%%%
\Ba{c}
\resizebox{16mm}{!}{ \xy
(0,0)*+{_{T_1^{(1)}}}="b1",
(7,0)*+{_{T_2^{(1)}}}="b2",
(14,2)*+{}="b3",
(22,0)*+{_{T_k^{(1)}}}="b4",
(10,10)*+{k}*\frm{.}="c",
(10,18)*+{}="u",
%
%%%%%%%%%%% internal edges %%%%%%%%%%%
\ar @{-} "u";"c" <0pt>
\ar @{-} "b1";"c" <0pt>
\ar @{-} "b2";"c" <0pt>
\ar @{-} "b3";"c" <0pt>
\ar @{-} "b4";"c" <0pt>
\endxy}
\Ea
$$
Formulae (\ref{C: from T to T}) define a natural linear map of $\bS$-modules,
$$
\phi: \mathrm{span}\langle T^{(1)}, T^{(2)}\rangle \lon  \mathrm{span}\langle t^{(1)}, t^{(2)} \rangle =
\f.
$$
The expressions  (\ref{C: from T to T}) can be (inductively) inverted,
%%%%%%%%%%%%%%%%%%%%%%%%%%%%%%%%%%%%%%
%%%%%%%%%%%%%%%%%%%%%%%%%%%%%%%%%%%%%%
\Beq\label{C: from t to T}
\Ba{c}
\resizebox{9mm}{!}{
\xy
(-4,-8)*{_{_1}};
(-2,-8)*{_{_2}};
(4.9,-8)*{_{_{k}}};
%
(0,0)*++{_a}*\frm<8pt,8pt>{ee}="b",
%
(-4,-6)*{}="-1",
(-2,-6)*{}="-2",
(4,-6)*{}="-3",
(1,-5)*{...},
(0,7)*{}="1'",
%
\ar @{-} "b";"1'" <0pt>
%
\ar @{-} "b";"-1" <0pt>
\ar @{-} "b";"-2" <0pt>
\ar @{-} "b";"-3" <0pt>
%
\endxy}
\Ea
=
%%%%%%%%%%%%%%%%%%%%%\
\Ba{c}
\resizebox{13mm}{!}{ \xy
(-7.5,-8.6)*{_{_1}};
(-4.1,-8.6)*{_{_2}};
(4.5,-8.6)*{_{_{k\hspace{-0.3mm}-\hspace{-0.3mm}1}}};
(9.0,-8.5)*{_{_{k}}};
%
(0.0,-6)*{...};
(0,5)*{};
(0,0)*+\hbox{$_{{a}}$}*\frm{-}
**\dir{-};
(-4,-7)*{};
(0,0)*+\hbox{$_{{\ a\ }}$}*\frm{-}
**\dir{-};
(-7,-7)*{};
(0,0)*+\hbox{$_{{a}}$}*\frm{-}
**\dir{-};
%
(8,-7)*{};
(0,0)*+\hbox{$_{{a}}$}*\frm{-}
**\dir{-};
(4,-7)*{};
(0,0)*+\hbox{$_{{a}}$}*\frm{-}
**\dir{-};
\endxy}\Ea
%%%%%%%%%%%%%%%%%%%
+\sum_{\sigma\in \bS_k}\frac{1}{(k-1)!}\hspace{-7mm}
\Ba{c}
\resizebox{19mm}{!}{ \xy
(-8.5,-8.6)*{_{_{\sigma(1)}}};
(-3.1,-8.6)*{_{_{\sigma(2)}}};
%(4.3,-8.6)*{_{_{k\hspace{-0.3mm}-\hspace{-0.3mm}2}}};
(11,-8.6)*{_{_{\sigma(k\hspace{-0.3mm}-\hspace{-0.3mm}1)}}};
(11,-1)*{_{_{\sigma(k)}}};
%
(0.0,-6)*{...};
(5,7)*+{2}*\frm{.};
(0,0)*+\hbox{$_{{a}}$}*\frm{-}
**\dir{-};
(-4,-7)*{};
(0,0)*+\hbox{$_{{\ a\ }}$}*\frm{-}
**\dir{-};
(-7,-7)*{};
(0,0)*+\hbox{$_{{a}}$}*\frm{-}
**\dir{-};
%
(9,-7)*{};
(0,0)*+\hbox{$_{{a}}$}*\frm{-}
**\dir{-};
(4,-7)*{};
(0,0)*+\hbox{$_{{a}}$}*\frm{-}
**\dir{-};
(9,-7)*{};
%
 <5.0mm,9mm>*{};<5.0mm,13.9mm>*{}**@{-},
 <6.4mm,4.6mm>*{};<10mm,1mm>*{}**@{-},
\endxy}
\Ea
-
\sum_{\sigma\in \bS_k}\frac{1}{2!(k-2)!}\hspace{-3mm}
\Ba{c}
\resizebox{19mm}{!}{ \xy
(-7.5,-8.6)*{_{_1}};
(-4.1,-8.6)*{_{_2}};
(7,-1)*{_{_{\sigma(k\hspace{-0.3mm}-\hspace{-0.3mm}1)}}};
(10,-8.6)*{_{_{\sigma(k\hspace{-0.3mm}-\hspace{-0.3mm}2)}}};
(14,-1)*{_{_{\sigma(k)}}};
%
(0.0,-6)*{...};
(6,7)*+{3}*\frm{.};
(0,0)*+\hbox{$_{{a}}$}*\frm{-}
**\dir{-};
(-4,-7)*{};
(0,0)*+\hbox{$_{{\ a\ }}$}*\frm{-}
**\dir{-};
(-7,-7)*{};
(0,0)*+\hbox{$_{{a}}$}*\frm{-}
**\dir{-};
%
(9,-7)*{};
(0,0)*+\hbox{$_{{a}}$}*\frm{-}
**\dir{-};
(4,-7)*{};
(0,0)*+\hbox{$_{{a}}$}*\frm{-}
**\dir{-};
(9,-7)*{};
%
 <6.0mm,9mm>*{};<6.0mm,13.9mm>*{}**@{-},
 <6.0mm,4.6mm>*{};<6.0mm,1mm>*{}**@{-},
 <7.4mm,4.6mm>*{};<11mm,1mm>*{}**@{-},
\endxy}
\Ea
+ \ldots,
\Eeq
and hence give us, again by induction, a linear map
$$
\psi: \mathrm{span}\langle t^{(1)}, t^{(2)}\rangle \lon  \mathrm{span}\langle T^{(1)}, T^{(2)} \rangle
$$
as follows. On $1$-vertex trees from the family  $\{ t^{(1)}, t^{(2)}\}$ the map $\psi$ is given
by
$$
\psi\left(\Ba{c}\xy
(1,-4)*{...};
(0,5)*{};
(0,0)*+\hbox{$_k$}*\frm{.}
**\dir{-};
(-2,-5)*{};
(0,0)*+\hbox{$_k$}*\frm{.}
**\dir{-};
(-4,-5)*{};
(0,0)*+\hbox{$_k$}*\frm{.}
**\dir{-};
(4,-5)*{};
(0,0)*+\hbox{$_k$}*\frm{.}
**\dir{-};
\endxy\Ea\right)= \Ba{c}\xy
(1,-4)*{...};
(0,5)*{};
(0,0)*+\hbox{$_k$}*\frm{.}
**\dir{-};
(-2,-5)*{};
(0,0)*+\hbox{$_k$}*\frm{.}
**\dir{-};
(-4,-5)*{};
(0,0)*+\hbox{$_k$}*\frm{.}
**\dir{-};
(4,-5)*{};
(0,0)*+\hbox{$_k$}*\frm{.}
**\dir{-};
\endxy\Ea, \ \ \ \ \ \
\psi\left(%\label{C: from t to T}
\Ba{c}
\resizebox{8mm}{!}{
\xy
(-4,-8)*{_{_1}};
(-2,-8)*{_{_2}};
(4.9,-8)*{_{_{k}}};
%
(0,0)*++{_a}*\frm<8pt,8pt>{ee}="b",
%
(-4,-6)*{}="-1",
(-2,-6)*{}="-2",
(4,-6)*{}="-3",
(1,-5)*{...},
(0,7)*{}="1'",
%
\ar @{-} "b";"1'" <0pt>
%
\ar @{-} "b";"-1" <0pt>
\ar @{-} "b";"-2" <0pt>
\ar @{-} "b";"-3" <0pt>
%
\endxy}
\Ea
\right)=\mathrm{the\ r.h.s.\ of}\ (\ref{C: from t to T}).
$$
Assume that the map $\psi$ is constructed on $n$-vertex trees from the family  $\{ t^{(1)}, t^{(2)}\}$. Let $t$ be a tree with $n+1$-vertices. The complement to the root vertex of $t$
is a disjoint union of trees, $\{t'\}$, with at most $n$ vertices. To get $\psi(t)$ apply first  $\psi$ to the subtrees $t'$ to get a linear combination of trees, $\sum t''$, where each  $t''$ is  obtained by attaching to (some) input legs of a one-vertex tree $v$ from $\{ t^{(1)}, t^{(2)}\}$ an element of the set $\{ T^{(1)}, T^{(2)}\}$; finally, apply $\psi$ to the root vertex $v$ of each summand $ t''$.
By construction,  $\psi\circ \phi=\Id$ and $\phi\circ \psi=\Id$  so that  the map $\phi$ is an isomorphism of $\bS$-modules,
$$
 \f\cong \mathrm{span}\langle T^{(1)}, T^{(2)} \rangle.
$$
The ideal $I$ in $\f$ defining the operad $\cB\cV_{\infty}^{com}$ now takes a very simple form --- this is an $\bS$-submodule of $\f$ spanned by trees
from the family $\{T^{(1)}, T^{(2)}\}$ which contain at least one ``bad" square vertex $\Ba{c}
\resizebox{13mm}{!}{ \xy
(-7.5,-8.6)*{_{_1}};
(-4.1,-8.6)*{_{_2}};
(4.5,-8.6)*{_{_{k\hspace{-0.3mm}-\hspace{-0.3mm}1}}};
(9.0,-8.5)*{_{_{k}}};
%
(0.0,-6)*{...};
(0,5)*{};
(0,0)*+\hbox{$_{{a}}$}*\frm{-}
**\dir{-};
(-4,-7)*{};
(0,0)*+\hbox{$_{{\ a\ }}$}*\frm{-}
**\dir{-};
(-7,-7)*{};
(0,0)*+\hbox{$_{{a}}$}*\frm{-}
**\dir{-};
%
(8,-7)*{};
(0,0)*+\hbox{$_{{a}}$}*\frm{-}
**\dir{-};
(4,-7)*{};
(0,0)*+\hbox{$_{{a}}$}*\frm{-}
**\dir{-};
\endxy}\Ea$ with $a<k-1$. If  $ \{T_{+}^{(1)}, T_{+}^{(2)}\}\subset  \{T^{(1)}, T^{(2)}\}$ is the subset of trees containing no bad vertices, then we can write an isomorphism of $\bS$-modules,
$$
\cB\cV_\infty^{com}\cong \mbox{span} \left\langle T_+^{(1)}  \right\rangle\ \oplus \ \mbox{span} \left\langle T_+^{(2)}  \right\rangle.
$$
The sub-module $\mbox{span} \left\langle T_+^{(1)} \right\rangle$ is the image of the dg operad $\caL^\diamond_\infty$ under the monomorphism $\chi$ (see Lemma {\ref{5: monomorphism chi}}) so that the above sum is a direct sum of {\em complexes}\, and we get eventually an isomorphism of complexes,
$$
\cB\cV_\infty^{com}\cong \cC om\circ \caL^\diamond_\infty,
$$
with the above splitting corresponding to the augmentation splitting of the operad $\cC om$,
$$
\cC om= \mbox{span}\left\langle 1\right\rangle \oplus \overline{\cC om}.
$$
Therefore, by Proposition {\ref{5: Propos on cohom of L diamnd infty}} and Remark {\ref{5: remark on qBV}},
$$
H^\bu(\cB\cV_\infty^{com})\cong \cC om \circ \caL^\diamond\cong q\cB\cV.
$$
By Proposition {\ref{5: Propos on qBV and BV}}, we get isomorphisms of $\bS$-modules,
$$
H^\bu(\cB\cV_\infty^{com})\cong Gr(\cB\cV)\cong \cB\cV
$$
which completes the proof of the Theorem.
\end{proof}































\bip




\def\cprime{$'$}
\begin{thebibliography}{10}

\bibitem[B]{Ba1} S.\ Barannikov, {\em Modular operads and Batalin-Vilkovisky geometry},
Intern. Math. Res. Notices (2007), no. 19, Art. ID rnm075, 31 pp.

\bibitem[Br]{Br} F. Brown, {\em Mixed Tate motives over $\mathbb Z$}, Ann. of Math. (2) {\bf 175} (2012), no. 2, 949--976.

\bibitem[CMW]{CMW} R.\ Campos, T.\ Willwacher and S.\ Merkulov, {\em Deformation theory of the properads of Lie bialgebras and involutive Lie bialgebras}.
   In preparation (2015).

\bibitem[Ch]{Ch} M.\ Chas, {\em Combinatorial Lie bialgebras of curves on surfaces},
    Topology {\bf 43} (2004), no. 3, 543--568.

\bibitem[CFL]{CFL} K. Cieliebak, K. Fukaya and J. Latschev, {\em Homological algebra related to sur-
faces with boundary}, preprint arxiv:1508.02741, 2015.

\bibitem[CL]{CL} K.\ Cieliebak and  J.\ Latschev, {\em The role of string topology in
    symplectic field theory}, arXiv:0706.3284 (2007).

\bibitem[CS]{ChSu} M.\ Chas and D.\ Sullivan, {\em Closed string operators in topology leading to Lie bialgebras and higher string algebra}, in: { The legacy of Niels Henrik Abel}, pp.\ 771--784, Springer, Berlin, 2004.

\bibitem[D1]{D1}
V.\ Drinfeld,
{\em Hamiltonian structures on Lie groups, Lie bialgebras and the geometric
meaning of the classical Yang-Baxter equations}, Soviet Math. Dokl. {\bf 27} (1983) 68--71.

\bibitem[D2]{D2}
V. Drinfeld, {\em On quasitriangular quasi-Hopf algebras and a group closely connected
with $Gal(\bar{Q}/Q)$}, Leningrad Math. J. {\bf 2}, No.\ 4 (1991),  829--860.

\bibitem[D3]{D3}
V.\ Drinfeld,
{\em On some unsolved problems in quantum group theory},
in: Lecture Notes in Math., Springer,  {\bf 1510} (1992), 1--8.



\bibitem[DCTT]{DCTT}
G.\ C.\ Drummond-Cole, J.\ Terilla and  T.\ Tradler,
 {\em Algebras over Cobar(coFrob)}, J.\ Homotopy Relat.\ Struct.\ {\bf 5}
 (2010), no.1, 15--36.

\bibitem[DCV]{DV} G.\ C.\ Drummond-Cole and B.\ Vallette, {\em  The minimal model for the Batalin-Vilkovisky operad}, Selecta Mathematica {\bf 19}, Issue 1 (2013), 1--47.


 \bibitem[ES]{ES} P.\ Etingof and O.\ Schiffmann,
Lectures on Quantum Groups, International Press, 2002.


\bibitem[F]{Fu} H.\ Furusho, {\em Four Groups Related to Associators}, preprint arXiv:1108.3389 (2011).

\bibitem[GTV]{GTV}
I.\ G\'{a}lvez-Carrillo, A.\ Tonks, and B.\ Vallette, {\em
Homotopy Batalin�Vilkovisky algebras}, J. Noncommut. Geom. 6 (2012), 539--602.


\bibitem[Ga]{G} W.L.\ Gan, {\em Koszul duality for dioperads}, Math.\ Res.\ Lett.
{\bf 10} (2003), 109--124.

\bibitem[Ge]{Ge} E.\ Getzler, {\em Batalin-Vilkovisky algebras and two-dimensional topological field theories},
Comm.\ Math.\ Phys.\ {\bf 159} (1994) 265--285.


%\bibitem[Go]{Go} A.\ Gonzalez, {\em The Lie bialgebra structure of the vector space of cyclic words}, $preprint arXiv:1111.4709 (2011).


\bibitem[H]{Ha} A.\ Hamilton, {\em Noncommutative geometry and compactifications of the
moduli space of curves},  J.\ Noncommut.\ Geom. 4 (2010), no. 2, 157--188.

\bibitem[HM]{HM} J.\ Hirsh and J.\ Mill$\grave{e}$s, Curved Koszul duality theory,  Math. Ann. 354 (2012), no. 4, 1465--1520.

\bibitem[JF1]{JF} T.\ Johnson-Freyd, {\em Poisson AKSZ theories and their quantizations}, in "Proceedings of the conference String-Math 2013", vol. 88 of "Proceedings of Symposia in Pure Mathematics", 291--306, Amer. Math. Soc., Providence, RI, 2014.

\bibitem[JF2]{JF2} T.\ Johnson-Freyd, {\em Tree-versus graph-level quasilocal Poincar\'e duality on $S^1$}, preprint arxiv:1412.4664 (2014).

\bibitem[KM]{KM} M.\ Kapranov and Yu.I.\ Manin, {\em Modules and Morita theorem for operads}, Amer. J. Math. {\bf 123} (2001), no. 5, 811--838.

\bibitem[LV]{LV}
J.-L. Loday and B.~Vallette,
\newblock {\em {Algebraic Operads}},
\newblock Number 346 in {Grundlehren der mathematischen Wissenschaften}.
  {Springer, Berlin}, {2012}.

 \bibitem[Ko]{Ko} M.\ Kontsevich, a letter to Martin Markl, November 2002.


 \bibitem[Kr]{Kr}
 O.\ Kravchenko, {\em Deformations of Batalin-Vilkovisky algebras}, in: Poisson geometry
 (Warsaw, 1998), Banach Center Publ.,
vol. 51, Polish Acad. Sci., Warsaw, 2000, pp. 131--139.

\bibitem[Ma]{Ma} M.\ Markl,
\newblock {\em Operads and props}, in: ``Handbook of Algebra" vol. 5, 87�140, Elsevier 2008.

\bibitem[MMS]{MMS}  M.\ Markl, S.\ Merkulov and S.\ Shadrin, {\em Wheeled props and the master
equation}, preprint math.AG/0610683, J.\ Pure and Appl.\ Algebra {\bf 213} (2009), 496--535.

 \bibitem[MaVo]{MaVo} M.\ Markl and A.A.\ Voronov,
{\em PROPped up graph cohomology}, in: Algebra, arithmetic, and geometry: in honor of Yu. I. Manin. Vol. II,
Progr. Math., 270, Birkh�user Boston, Inc., Boston, MA (2009) pp. 249--281.



\bibitem[May]{May}  J.P. May, The Geometry of Iterated Loop Spaces, volume 271 of Lecture Notes in Mathematics. Springer-
Verlag, New York, 1972.

\bibitem[Mc]{Mc} S.\ MacLane, {\em Categorical algebra},
Bull.\ Amer.\ Math.\ Soc. {\bf 71} (1965), 40--106.

 %\vspace{-1mm}

\bibitem[Me1]{Me} S.A.\ Merkulov, {\em Operads, configuration spaces and quantization},
 In: ``Proceedings of Poisson 2010, Rio de Janeiro", Bull.\ Braz.\ Math.\ Soc., New Series {\bf 42}(4) (2011), 1--99.

 \bibitem[Me2]{Me2} S.A.\ Merkulov,
   {\em Graph complexes with
  loops and wheels}, in:
 ``Algebra, Arithmetic
 and Geometry - Manin Festschrift" (eds. Yu.\ Tschinkel and Yu.\ Zarhin),
 Progress in Mathematics, Birkha\"user (2010), pp. 311-354.


 \bibitem[MeVa]{MV}  S.A.\ Merkulov and  B.\ Vallette,
{\em Deformation theory of representations of prop(erad)s I \& II},
 Journal f\"ur die reine und angewandte Mathematik (Crelle)  {\bf 634}, 51--106,
 \& {\bf 636}, 123--174 (2009).

 \bibitem[S]{Sch} T.\ Schedler, {\em A Hopf algebra quantizing a necklace Lie algebra
     canonically associated
to a quiver}. Intern. Math. Res. Notices (2005), 725--760.



 \bibitem[Ta]{Ta} D.\ Tamarkin, {\em Action of the Grothendieck-Teichmueller group on the operad of Gerstenhaber algebras}, preprint math/0202039 (2002).

 \bibitem[Tu]{Tu} V.G.\ Turaev, {\em  Skein quantization of Poisson algebras of loops on
     surfaces},
Ann. Sci. Ecole Norm. Sup. (4) 24, no. 6, (1991) 635--704.

\bibitem[V1]{Va}
B.\ Vallette,  {\em A Koszul duality for
props}, Trans.\ Amer. Math. Soc., {\bf 359} (2007), 4865--4943.

\bibitem[V2]{Va2} B.\ Vallette, {\em Algebra+Homotopy=Operad},
 in "Symplectic, Poisson and Noncommutative Geometry", MSRI Publications 62 (2014), 101--162.

\bibitem[We]{We} C.A.\ Weibel, {\em An introduction to homological algebra}, Cambridge University Press 1994.

\bibitem[W1]{Wi1} T.\ Willwacher, {\em M.\ Kontsevich's graph complex and the
 Grothendieck-Teichm\"uller Lie algebra},
 Invent. Math. 200 (2015), no. 3, 671--760.

\bibitem[W2]{Wi3} T.\ Willwacher, {\em
Stable cohomology of polyvector fields}, Math. Res. Lett. 21 (2014), no. 6, 1501--1530.

\bibitem[W3]{Wi2} T.\ Willwacher, {\em The oriented graph complexes},
 Comm. Math. Phys. 334 (2015), no. 3, 1649--1666.



 \end{thebibliography}








\end{document}
%%%%%%%%%%%%%%%%%%%%%%%%%%%%%%%%%%%%%%%%%%%%%%%%%%%%%%%%%%%%%%%%%%%%%%%%%%%%%%%%%%%%%%%%%%%%%%%%%
%%%%%%%%%%%%%%%%%%%%%%%%%%%%%%%%%%%%%%%%%%%%%%%%%%%%%%%%%%%%%%%%%%%%%%%%%%%%%%%%%%%%%%%%%%%%%%%%%%




