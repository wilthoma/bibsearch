\documentclass[a4paper]{amsart}

%\usepackage[utf8]{inputenc}
%\usepackage[latin1]{inputenc}
\usepackage{txfonts, amsmath,amstext,amsthm,amscd,amsopn,verbatim,amssymb, amsfonts}
\usepackage{fullpage}

\usepackage[bbgreekl]{mathbbol}
%\usepackage[babel=true,kerning=true]{microtype}
%\usepackage{graphicx}
%\usepackage{color}
%\usepackage{pxfonts,txfonts}
%\usepackage{pstricks, pstricks-add, pst-node, pst-coil}
%\usepackage{array}
%\usepackage[all, 2cell ]{xy}  \UseAllTwocells \SilentMatrices
%\usepackage{graphicx} 
\usepackage{tikz}
\usepackage{tikz-cd}
\usetikzlibrary{matrix}
\usetikzlibrary{shapes}
\usetikzlibrary{arrows}
\usetikzlibrary{calc,3d}
\usetikzlibrary{decorations,decorations.pathmorphing}
\usetikzlibrary{through}
\tikzset{ext/.style={circle, draw,inner sep=1pt},int/.style={circle,draw,fill,inner sep=1pt},nil/.style={inner sep=1pt}}
\tikzset{exte/.style={circle, draw,inner sep=3pt},inte/.style={circle,draw,fill,inner sep=3pt}}
\tikzset{diagram/.style={matrix of math nodes, row sep=3em, column sep=2.5em, text height=1.5ex, text depth=0.25ex}}
\tikzset{diagram2/.style={matrix of math nodes, row sep=0.5em, column sep=0.5em, text height=1.5ex, text depth=0.25ex}}
\tikzset{every picture/.append style={baseline=-.65ex}}


\usepackage{hyperref}

%\usepackage{showkeys} 
\theoremstyle{plain}
  \newtheorem{thm}{Theorem}
   \newtheorem{conj}[thm]{Conjecture}
  \newtheorem{defi}[thm]{Definition}
  \newtheorem{prop}[thm]{Proposition}
  \newtheorem{defprop}[thm]{Definition/Proposition}
  \newtheorem{cor}[thm]{Corollary}
  \newtheorem{lemma}[thm]{Lemma}
   \newtheorem{sublemma}[thm]{Sublemma}
\theoremstyle{definition}
  \newtheorem{ex}[thm]{Example}
  \newtheorem{rem}[thm]{Remark}
  \newtheorem{cons}[thm]{Construction}

\newcommand{\alg}[1]{\mathfrak{{#1}}}
\newcommand{\co}[2]{\left[{#1},{#2}\right]} % commutator
\newcommand{\aco}[2]{\left\{{#1},{#2}\right\}}
\newcommand{\eref}[1]{\eqref{#1}} % equation reference
\newcommand{\pderi}[2]{ { \frac{\partial {#1} }{\partial {#2} } } }
\newcommand{\pd}[2]{ { \frac{\partial {#1} }{\partial {#2} } } }
\newcommand{\ad}{{\text{ad}}}
\newcommand{\Ad}{{\text{Ad}}}
\newcommand{\mbf}[1]{ {\pmb{#1}} }
\newcommand{\edge}{{\rightarrow }} 
\newcommand{\morphU}{{\mathcal{U} }} 
\newcommand{\widebar}[1]{{\overline{#1}}}
\newcommand{\p}{\partial}
\newcommand{\Hom}{\mathop{Hom}}
\newcommand{\folF}{\mathcal{F}}
\newcommand{\C}{{\mathbb{C}}}
\newcommand{\R}{{\mathbb{R}}}
\newcommand{\Z}{{\mathbb{Z}}}
\newcommand{\K}{{\mathbb{K}}}
\newcommand{\Q}{{\mathbb{Q}}}
\newcommand{\U}{{\mathcal{U}}}
\newcommand{\hU}{{\mathcal{U}}}
\newcommand{\CG}{{\mathsf{CG}}}
\newcommand{\fGCc}{{\mathrm{fcGC}}}
\newcommand{\HGC}{{\mathrm{HGC}}}
\newcommand{\TCG}{{\mathsf{TCG}}}
\newcommand{\Ne}{{\mathcal{N}}} % Nerve
\newcommand{\Graphs}{{\mathsf{Graphs}}}
\newcommand{\fGraphs}{{\mathsf{fGraphs}}}
\newcommand{\pdu}{{}^*} % predual
\newcommand{\fml}{{\mathit{fml}}} % predual

\newcommand{\te}{{\tilde{\mathsf{e}}}}

\newcommand{\FreeLie}{\mathrm{Free}_{\Lie}}

\newcommand{\Poiss}{{\mathsf{Poiss}}}

\newcommand{\ICG}{\mathsf{ICG}}
\newcommand{\ICGF}{\ICG^{\rm fr}}
\newcommand{\FICG}{\ICGF}
%\newcommand{\TCG}{\mathsf{TCG}}
\newcommand{\TCGF}{\TCG^{\rm fr}}


\newcommand{\Exp}{\mathrm{Exp}}
\newcommand{\SGraphs}{{\mathsf{SGraphs}}}
\newcommand{\dGraphs}{{\mathsf{Graphs}'}}
\newcommand{\cGraphs}{{\mathsf{cGraphs}}}
\newcommand{\fcGraphs}{{\mathsf{fcGraphs}}}
\newcommand{\dfcGraphs}{{\mathsf{dfcGraphs}}}
\newcommand{\fcGraphso}{\mathsf{fcGraphs}^{or}}
\newcommand{\bGraphs}{{\mathsf{bGraphs}}}
\newcommand{\hGraphs}{{\mathsf{hGraphs}}}
\newcommand{\dSGraphs}{{\mathsf{SGraphs}'}}
\newcommand{\Gr}{{\mathsf{Gra}}}
\newcommand{\Gra}{{\mathsf{Gra}}}
\newcommand{\dGr}{{\mathsf{dGra}}}
\newcommand{\dGra}{{\mathsf{dGra}}}
\newcommand{\hGra}{{\mathsf{hGra}}}
\newcommand{\Ger}{{\mathsf{Ger}}}
\newcommand{\tGer}{\widetilde{\mathsf{Ger}}}
\newcommand{\tGernf}{\widetilde{\mathsf{Ger}_{nf}}}
\newcommand{\LieBi}{{\mathsf{LieBi}}}
\newcommand{\Graphso}{\mathsf{Graphs}^{or}}

\newcommand{\SGr}{{\mathsf{SGra}}}
\newcommand{\Aut}{{\mathrm{Aut}}}
\newcommand{\hAut}{{\mathsf{hAut}}}
\newcommand{\SGra}{{\mathsf{SGra}}}
\newcommand{\fSGra}{{\mathsf{fSGra}}}
\newcommand{\PT}{ \mathsf{PT} }

\newcommand{\Tw}{\mathit{Tw}}
\newcommand{\Def}{\mathrm{Def}}

\newcommand{\Fund}{\mathrm{Fund}}

\newcommand{\bigGra}{\mathsf{bigGra}}
\newcommand{\op}{\mathcal}
\newcommand{\hKS}{\mathsf{hKS}}
\newcommand{\KS}{\mathsf{KS}}
\newcommand{\homKS}{\mathsf{homKS}}
\newcommand{\Br}{\mathsf{Br}}
\newcommand{\hBr}{\mathsf{hBr}}
\newcommand{\Lie}{\mathsf{Lie}}
\newcommand{\ELie}{\mathsf{ELie}}
%\newcommand{\Ger}{\mathsf{Ger}}
\newcommand{\hoLie}{\mathsf{hoLie}}
\newcommand{\hoELie}{\mathsf{hoELie}}
\newcommand{\fSGraphs}{\mathsf{fSGraphs}}
\newcommand{\SC}{\mathsf{SC}}
\newcommand{\SG}{\mathsf{SG}}
\newcommand{\ESG}{\mathsf{ESG}}
\newcommand{\ESC}{\mathsf{ESC}}
\newcommand{\EESC}{\mathsf{EESC}}

\newcommand{\calc}{\mathsf{calc}}
\newcommand{\Ass}{\mathsf{Assoc}}
\newcommand{\Com}{\mathsf{Com}}
\newcommand{\bigKS}{\mathsf{bigKS}}
\newcommand{\bigChains}{\mathsf{bigChains}}
\newcommand{\bigGraphs}{\mathsf{bigGraphs}}
\newcommand{\FM}{\mathsf{FM}}
\newcommand{\EFM}{\mathsf{EFM}}
\newcommand{\cFM}{\mathsf{EFM}}
\newcommand{\hc}{\mathit{hc}}

\newcommand{\La}{\Lambda}
\newcommand{\Der}{\mathrm{Der}}
\newcommand{\BiDer}{\mathrm{BiDer}}
%\newcommand{\Def}{\mathrm{Def}}

\newcommand{\vout}{\mathit{out}}
\newcommand{\vin}{\mathit{in}}
\newcommand{\conn}{\mathit{conn}}
\newcommand{\dirout}{\mathit{dir}_{out}}
\newcommand{\opm}{\mathbf{m}}
\newcommand{\bpm}{\begin{pmatrix}}
\newcommand{\epm}{\end{pmatrix}}
\newcommand{\Tpoly}{T_{\rm poly}}
\newcommand{\Dpoly}{D_{\rm poly}}

\newcommand{\GC}{\mathrm{GC}}
\newcommand{\GCo}{\mathrm{GC}^{or}}
\newcommand{\dGC}{\mathrm{dGC}}
\newcommand{\fGC}{\mathrm{fGC}}
\newcommand{\fGCo}{\mathrm{fGC}^{or}}
%\newcommand{\fGCc}{\mathrm{fGCc}}
\newcommand{\fGCco}{\mathrm{fcGC}^{or}}
\newcommand{\dfGCc}{\mathrm{dfcGC}}
\newcommand{\dfGC}{\mathrm{dfGC}}
\newcommand{\GCor}{\mathrm{GC}^{or}}
\newcommand{\hGCor}{\widehat{\mathrm{GC}}^{or}}
\newcommand{\iGCor}{\widetilde{\mathrm{GC}}^{or}}

\newcommand{\bigV}{\mathbf{V}}
\newcommand{\mU}{\mathcal{U}}
\newcommand{\mV}{\mathcal{V}}
\newcommand{\hotimes}{\mathbin{\hat\otimes}}
\DeclareMathOperator{\dv}{div}
\DeclareMathOperator{\End}{End}
\DeclareMathOperator{\sgn}{sgn}
\newcommand{\tder}{\alg{tder}}
\newcommand{\sder}{\alg{sder}}
\newcommand{\kv}{\alg{kv}}
\newcommand{\bDelta}{\blacktriangle}%\mathbb{\Delta}}
\newcommand{\tW}{ {[[u]]\otimes_{\gf[u]} W} }

\newcommand{\grt}{\alg {grt}}
\newcommand{\hoe}{\mathsf{hoe}}
\newcommand{\e}{\mathsf{e}}

\newcommand{\lo}{\longrightarrow}
\newcommand{\BGC}{\mathsf{BGC}}
\newcommand{\BGraphs}{\mathsf{BGraphs}}
\newcommand{\BstG}{{}^*\mathsf{BGraphs}}

\newcommand{\dimens}{\mathop{dim}}
\newcommand{\gra}{\mathrm{gra}}
\newcommand{\LC}{\mathrm{LC}}
\newcommand{\stG}{{}^*\Graphs}
\newcommand{\SO}{\mathit{SO}}
\newcommand{\so}{\mathfrak{so}}
\newcommand{\lD}{\mathsf{D}}
\newcommand{\flD}{\lD^{\mathrm{fr}}}

\newcommand{\Conf}{\mathrm{Conf}}
\newcommand{\dgca}{\mathsf{Dgca}}

\renewcommand{\mod}{\mathrm{mod}}

\newcommand{\mC}{{\mathcal{C}}}
\newcommand{\mD}{{\mathcal{D}}}
\newcommand{\mT}{{\mathcal{T}}}
\newcommand{\Hop}{{\mathrm{HOp}}}
\newcommand{\Op}{{\mathrm{Op}}}
\newcommand{\Tree}{{\mathsf{Tree}}}

\newcommand{\mW}{\mathcal{W}}
\newcommand{\mF}{\mathcal{F}}
\newcommand{\dgVect}{\mathrm{dgVect}}
\newcommand{\FC}{\mathcal{FC}}
\newcommand{\gr}{\mathit{gr}}
\newcommand{\id}{\mathit{id}}


\newcommand{\DK}{{\alg p}}
\newcommand{\DKF}{\DK^{\mathrm{fr}}}
\newcommand{\stGC}{{}^*\GC}
\newcommand{\tadpole}{
\begin{tikzpicture}[baseline=-.65ex]
\node[int] (v) at (0,0) {};
\draw (v) edge[loop] (v);
\end{tikzpicture}
}

\newcommand{\beq}[1]{
\begin{equation}\label{#1}
}
\newcommand{\eeq}{
\end{equation}
}

\newcommand{\hdgca}{h\dgca}

\newcommand{\Vect}{\mathsf{Vect}}
\renewcommand{\Top}{\mathsf{Top}}

%\newcommand{\Fund}{\mathrm{Fund}}

%\renewcommand{\texorpdfstring}[2]{{#1}}

\newcommand{\oltimes}{\mathop{\text{$\ltimes$\makebox[0pt][r]{$\otimes$}}}\limits}

\begin{document}
%\title{Rational homotopy theory of group actions on operads, and framed little $n$-disks operads}
\title{Real models for the framed little $n$-disks operads}
%\author{Benoit Fresse}


\author{Anton Khoroshkin}
\address{Faculty of mathematics\\ National University Higher School of Economics,\\ 
7 Vavilova street, Moscow, Russia, 115280}
\email{akhoroshkin@hse.ru}

\author{Victor Turchin}
\address{Department of Mathematics\\
  Kansas State University\\
  138 Cardwell Hall\\
  Manhatan, KS 66506, USA}
  \email{turchin@ksu.edu}
\author{Thomas Willwacher}
\address{Department of Mathematics \\ ETH Zurich \\  
R\"amistrasse 101 \\
8092 Zurich, Switzerland}
\email{thomas.willwacher@math.ethz.ch}



\thanks{A.K. has been partially supported by RFBR grants 13-02-00478, 13-01-12401, 
by "The National Research University--Higher School of Economics" Academic Fund Program in 2013-2014,
research grant 14-01-0124, by Dynasty foundation and Simons-IUM fellowship.}
\thanks{V.T. acknowledges Max-Planck-Institut f\"ur Mathematik  (Bonn) and the Institut des Hautes Etudes Scientifique for hospitality}
\thanks{T.W. acknowledges partial support by the Swiss National Science Foundation (grant 200021\_150012 and the SwissMap NCCR). This work has been partially funded by the European Research Council, ERC StG 678156--GRAPHCPX}

% \address{Department of Mathematics\\ ETH Zurich\\ R\"amistrasse 101 \\ 8092 Zurich, Switzerland}
% \email{t.willwacher@gmail.com}

%\thanks{The author was partially supported by the Swiss National Science Foundation (grant 200020-105450).}
% \subjclass[2000]{16E45; 53D55; 53C15; 18G55}
% \date{}
%\keywords{Formality, Deformation Quantization, Operads}

\begin{abstract}
We study the action of the orthogonal group on the little $n$-disks operads.
As an application we provide small models (over the reals) for the framed little $n$-disks operads.
It follows in particular that the framed little $n$-disks operads are formal (over the reals) for $n$ even and coformal for all $n$.
\end{abstract}

\maketitle

\tableofcontents

\section{Introduction}

The framed little $n$-disks operads $\flD_n$ are operads of embeddings of ``small'' $n$-dimensional disks in the $n$-dimensional unit disk. 
These operads are of fundamental importance in algebraic topology and homological algebra. In particular, in recent years they saw a surging interest due to applications in the manifold calculus of Goodwillie-Weiss \cite{G,GW}, and, relatedly, in the study of factorization algebras in homotopy theory \cite{AF}.

Surprisingly, the rational homotopy type of the operads $\flD_n$ is currently not understood very well.
This is in sharp contrast to the rational homotopy type of the non-framed sub-operads $\lD_n\subset \flD_n$, which is well understood due to work of Kontsevich \cite{K2}, Tamarkin (for $n=2$) \cite{Tam}, Lambrechts-Volic \cite{LV} and the authors \cite{FTW}.
Furthermore, it is known that the operad $\flD_2$ is rationally formal \cite{pavolfr, GS}.
The goal of this paper is to study the real homotopy type of the topological operads $\flD_n$ for $n\geq 3$.

To this end we will study the real homotopy type of the the action of the orthogonal groups on the operads $\lD_n$, from which the framed version may be deduced.
Generally, we show that the real homotopy type of the $O(n)$-action on $\lD_n$ is described by a certain Maurer-Cartan element in the Kontsevich graph complex (dg Lie algebra) with coefficients in the cohomology $H(B\SO(n))$
\[
 m\in \left( \GC_n\hat \otimes H(B\SO(n)) \right)^{\Z_2},
\]
where $\Z_2$ should be thought of as $\pi_0(O(n))$.
We derive explicit integral formulas for the element $m$, and provide a model for the $O(n)$-framed little disks operads depending (only) on $m$. 
By a version of equivariant localization we can compute the gauge equivalence type of $m$, and hence produce explicit combinatorial models for the framed little disks operads.
The following results can be read off from the models.

%TODO: describe non connected G situation in more detail.... 

\begin{thm}\label{thm:partial framed formality}
Let $n\geq 2$. The $O(n)$-framed and $\SO(n)$-framed little $n$-disks operads are formal over $\R$ if $n$ is even, in the sense that the homotopy dg Hopf cooperads of real forms on these operads can be connected to their cohomologies by zigzags of quasi-isomorphisms.
\end{thm}
The case $n=2$ is well known and has been shown in \cite{pavolfr, GS}.

For $n$ odd the situation is more complicated as the following result shows.
\begin{thm}\label{thm:odd nonformality}
 The operads of real chains of the $\SO(n)$-framed little $n$-disks operads are not formal over for $n\geq 3$ odd. 
\end{thm}
While this work was under preparation, the case $n\geq 5$ has also been shown in \cite{Mo}.

To describe our explicit model for $\flD_n$ for odd $n$ we need some more notation.
Let us sketch here the construction, leaving a more careful discussion to the forthcoming sections.
\begin{thm}
 For $n\geq 3$ odd the Maurer-Cartan element $m\in \left( \GC_n\hat \otimes H(B\SO(n))\right)^{\Z_2}$ governing the action of $O(n)$ on $\lD_n$ has, up to gauge equivalence, the following explicit form:
 \begin{equation}\label{equ:m odd}
 \sum_{j\geq 1}
 \frac {p_{2n-2}^{j}}{4^j}
\frac{1}{2(2j+1)!} 
\begin{tikzpicture}[baseline=-.65ex]
 \node[int] (v) at (0,.5) {};
 \node[int] (w) at (0,-0.5) {};
 \draw (v) edge[bend left=50] (w) edge[bend right=50] (w) edge[bend left=30] (w) edge[bend right=30] (w);
 \node at (2,0) {($2j+1$ edges)};
 \node at (0,0) {$\scriptstyle\cdots$};
\end{tikzpicture}
 \end{equation}
 with $p_{2n-2}\in H(B\SO(n))$ the top Pontryagin class.
\end{thm}

Now the graph complex $\GC_n$ is a dg Lie algebra acting on a dg Hopf cooperad model $\stG_n$ of $\lD_n$.
The Maurer-Cartan element $m$ above hence directly encodes a homotopy co-action of the Hopf algebra $H^\bullet(\SO(n))$ on the dg Hopf cooperad $\stG_n$, given by an explicit combinatorial formula.
We may replace $H^\bullet(\SO(n))$ by a slightly larger quasi-isomorphic dg Hopf algebra $A$ which lifts this homotopy action to an honest action.
Then our dg Hopf cooperad model for $\flD_n$ has the form of a framing product
\[
 \stG_n\circ A.
\]

As a corollary one can deduce the following result.

\begin{thm}\label{thm:FE3coformal}
The operads $\flD_n$ are coformal over $\R$ for all $n\geq 2$.
\end{thm}

The explicit minimal (Quillen) graded Lie algebra model of $\flD_n$ is constructed in section \ref{sec:quillen} below.
% We conjecture that the above results in fact can be generalized to higher dimensions as follows.
% \begin{conj}\label{conj:co_formality}
% The operads $\flD_n$ are formal (and coformal) for $n\geq 2$ even, and coformal for $n\geq 3$ odd.
% \end{conj}
% If the conjecture is true, very simple and natural Sullivan and Quillen models for the operads $\flD_n$ may be constructed, cf. section \ref{sec:framed} below.
% We remark that the case $n=2$ of the above conjecture has been shown in \cite{pavolfr, GS}, while the case $n=3$ is obviously Theorem \ref{thm:FE3coformal} above.

%TODO: add discussion about sphere-framed $\lD_n$?


% \subsection*{TODO and questions to think about}
% \begin{itemize}
%  \item Applications of partially framed formality to embedding calculus
%  \item Relative (partially) framed formality
%  \item For embedding calculus application, do the desired applications require formality, or can we work with model for $\flD_n$ using the (only partially known) $m$?
% \end{itemize}


\subsection*{Overview and structure of the paper}
The paper is roughly divided into two parts. In the first part (sections \ref{sec:basic notation} and \ref{sec:dgca models operads}) we discuss generalities of group actions on operads, and in particular outline a theory of homotopy operads, elements of which we use to define the notion of \emph{real model} for a topological operad.

The main technical goal of the first part is to show the following statement: Suppose we are given a topological operad $\op T$ with a group action of a compact Lie group $G$.
We may form the $G$-framed operad $\op T\circ G$, which is again a topological operad.
The goal is then (roughly) to show that a real model for $\op T\circ G$ can be computed from knowledge of the homotopy type of the $G$-equivariant differential forms $\Omega_G(\op T)$.
Ignoring certain technicalities, this goes as follows, at least for connected $G$. The $G$-equivariant differential forms $\Omega_G(\op T)$ are a sequence of dg commutative algebras and come equipped with a map from $H(BG)$. Furthermore, from the operad structure on $\op T$ they inherit a (homotopy) cooperad structure over the ground ring $H(BG)$.
Now, the equivariant forms on a $G$-space $X$ model the homotopy quotient $X//G$, and from this homotopy quotient the original $G$-space may be recovered as a homotopy pullback
\[
\begin{tikzcd}
X \ar{r}\ar{d} & EG \ar{d} \\
X//G \ar {r} & BG 
\end{tikzcd}.
\]
Dually, the real model for the $G$-space may be given as a pushout. More concretely, the Koszul complex $K=H(G)\otimes H(BG)$ (with a natural differential) is a Hopf comodule over $H(G)$ and a model for $EG$. A model for $\op T$ as an operad in $G$-spaces may then be computed as 
\[
B := K\otimes_{H(BG)} A,
\]
where $A$ is quasi-isomorphic to $\Omega_G(\op T)$. 
The model $B$ is a cooperad in dg Hopf $H(G)$-comodules.
Finally, the (or rather one) desired real (dg Hopf operad-)model for the topological operad $\op T\circ G$ may then obtained by an algebraic version of the framing construction 
\[
 B\circ H(G).
\]
Let us however warn the reader that there are various technical problems that partially require comparatively elaborate workarounds, and hence the first part of the paper is not quite as straightforward as one might expect from the above exposition.


In the second part of this paper (sections \ref{sec:graphs}-\ref{sec:auxthmproof}) we specialize to the little disks operads $\op T=\lD_n$, with an action of $G\subset O(n)$. 
The goal of the second part is to construct a model $A$ for the equivariant forms on $\lD_n$.
(Concretely, the $A$ will appear below as $A=\stG_n\otimes H(BG)$.)
This should be seen as the main novel contribution of the present paper.

Finally, in section \ref{sec:framed} we plug this model $A$ into the general machinery of the first part, to obtain our desired real models for the framed little disks operads.
Our formality and coformality claims are then easily verified, given the explicit combinatorial models.

The appendix contains a few auxiliary technical results, computations and ``side stories'' that might be of interest to the reader. 


\section{Basic notation}\label{sec:basic notation}

\subsection{Homotopy theory}
In this paper we will do homotopy theory mostly in the $(\infty,1)$-categorical setting. Concretely, we will work with homotopical categories instead of full model categories, cf. \cite{Riehl}.
\begin{defi}
 A homotopical category is a category $\mC$ together with a class of distinguished morphisms $\mW$ (the \emph{weak equivalences}) such that the 2-out-of-6 property holds:
If $f,g,h$ are three composable morphisms such that $h\circ g\in \mW$ and $g\circ f\in \mW$ then $f,g,h,h\circ g\circ f\in \mW$.

A homotopical functor between homotopical categories is a functor which preserves the class of weak equivalences.
\end{defi}
One may define the homotopy category and the simplicial localization for homotopical categories \cite{Riehl, DK1,DK2}.
For many of the categories we treat the homotopical structure will come from a closed model structure.
However, we will not work in the model categorial framework. The main reason is that for dg Hopf cooperads with arbitrary operations of arities $\leq 1$ the model structure has not yet been constructed (cf. \cite{F} for the case without such operations), and we do not attempt to fill this gap in the theory here.



\subsection{Vector spaces, complexes, dgcas}\label{sec:vector spaces complexes}

We generally work over the ground field $\K$ of characteristic zero.
Our algebraic constructions work for $\K=\Q$. To show the main results we will however use transcendental methods (integrals) and eventually restrict to $\K=\R$.

As usual, we abbreviate the phrase \emph{differential graded} by dg.
We denote the category of unbounded, cochain graded dg vector spaces by $\dgVect$.
We equip it with the standard homotopical structure, i.e., the class of weak equivalences are the quasi-isomorphisms.
We also introduce the category $\FC$ of filtered complete dg vector spaces as follows:
\begin{itemize}
 \item Objects of $\FC$ are dg vector spaces $V$ equipped with a descending complete filtration 
\[
 V=\mF^1V \supset \mF^2V\supset \mF^3V\supset \cdots .
\]
such that the associated spectral sequence abuts on the first page, i.e., $H(\gr V)\cong H(V)$. In particular, $V$ is quasi-isomorphic to its associated graded.\footnote{This assumption is inconvenient in many practical situations, and one could relax the condition such that it is only required that the spectral sequence abuts at a finite page. However, as it stands the condition makes statements and proofs easier, and will be satisfied in our examples.}
\item Morphisms in $\FC$ are morphisms of filtered dg vector spaces.
\item We equip $\FC$ with the structure of a homotopical category by declaring the weak equivalences to be the quasi-isomorphisms.
\item We define a monoidal structure on $\FC$ by the completed tensor product.
Concretely, for $V,W$ in $\FC$, the ordinary tensor product $V\otimes W$ comes with a filtration such that 
\[
\mF^k (V\otimes W) = \sum_{p+q=k} \mF^pV\otimes \mF^qW,
\]
and we define the completed tensor product to be the completion 
\[
 V\hat \otimes W = \lim_{\leftarrow} V\otimes W / \mF^k (V\otimes W).
\]
The completed tensor product preserves weak equivalences, i.e., it is a homotopical bifunctor. (This statement uses the assumption about the associated graded above.)
%In the following we shall always work with completed tensor products and drop the hat, i.e., $\otimes := \hat \otimes$.
\end{itemize}

We shall denote by $\dgca$ the category of dg commutative algebras, with quasi-isomorphisms as weak equivalences.
Similarly to the category $\FC$ of filtered complete vector spaces above, we define the category $\hdgca$ of filtered complete dg commutative algebras.
Concretely, objects of $\hdgca$ are dg commutative algebras equipped with a descending complete filtration 
\[
  A=\mF^1A \supset \mF^2A\supset \mF^3A\supset \cdots 
\]
of algebras, such that the underlying filtered vector space $A$ is an object of $\FC$.
The tensor product is defined on $\hdgca$ by inheriting the (completed) tensor product from $\FC$.
Note that we merely require that each $\mF^pA$ is a subalgebra, and \emph{not} necessarily that $(\mF^pA)(\mF^qA) \subset \mF^{p+q}A$.
In other words, objects of $\hdgca$ are filtered algebras, not algebras in filtered vector spaces.

For a dg vector space $V$ we will denote its cohomology by $H(V)$.
For $X$ a topological space we will denote the cohomology by $H(X)$ or $H^\bullet(X)$, and the homology by $H_\bullet(X)$.

%The categories $\dgVect$ and $\FC$ are weakly equivalent in the following sense.
%\begin{lemma}
% The natural inclusion 
%\begin{gather*}
%R:\dgVect \to \FC \\
%V\mapsto V, \quad \text{ with filtration } \mF^0 R(V)=V, \mF^1 R(V)=0
%\end{gather*}
%and the quotient map
%\begin{gather*}
%L:\FC\to \dgVect  \\
%V\mapsto V / \mF^1V
%\end{gather*}
%form an adjoint pair, with $L$ left adjoint. We have $L\circ R=\id$ and the unit $\id\Rightarrow R\circ L$ is a weak equivalence.
%Furthermore, $L$ and $R$ respect the monoidal structures.
%\end{lemma}
%
%For technical reasons, we will generally work with the larger category $\FC$ rather than $\dgVect$.
%We assume by default that our algebraic objects (operads, dg commutative algebras etc.) are defined in $\FC$ rather than $\dgVect$. 
%We denote in particular the category of (unital) dg commutative algebras by $\dgca$.
%
%For an introduction to operads and cooperads we refer the reader to \cite{lodayval}, whose conventions we mostly follow.
%We consider operads and cooperads and their (co-)modules in general symmetric monoidal categories.
%Most notably we call a (co)operad in $\dgVect$ a dg (co)operad, and a cooperad in $\dgca$ a Hopf cooperad.
%More generally, and informally, the word "Hopf" shall indicate that we consider some object in the category $\dgca$.
%


\subsection{Monoidal structures}\label{sec:monoidal definitions}

For us the term monoidal functor shall always mean strong monoidal functor. On the other hand, a lax monoidal functor $F:\mC\to \mD$ is a functor between monoidal categories together with a morphism $1_{\mD} \to F(1_{\mC})$ and a natural transformation 
\beq{equ:monnt}
 F(-)\otimes_{\mD}F(-) \Rightarrow F(-\otimes_{\mC} -), 
\eeq
satisfying natural coherence relations. We say that $F$ is oplax monoidal if the functor $F^{op}:\mC^{op}\to \mD^{op}$ is lax monoidal, i.e., the arrow above points in the opposite direction.
Now suppose that $\mC$ and $\mD$ are homotopical categories such that the monoidal products are homotopical functors.
Then we call call a lax monoidal homotopical functor $F$ as above a homotopically monoidal functor if \eqref{equ:monnt} is a weak equivalence, and similarly we define the notion of homotopically comonoidal functor for an oplax monoidal $F$.

If $F$ is a contravariant functor, we say that $F$ is (lax or oplax) monoidal if the functor $F:\mC^{op}\to \mD$ is.
In particular, we use the convention that for a lax monoidal functor we always have a natural transformation as in \eqref{equ:monnt}, without reversing arrows in the target category.
(This might not be the standard convention, but seems more natural to the authors.)

\newcommand{\Set}{\mathrm{Set}}
\newcommand{\sSet}{s\Set}
\newcommand{\ob}{\mathit{ob}}
\newcommand{\Res}{\mathit{Res}}

\subsection{Simplicial sets, spaces and rational (and real) models}
We denote by $\Top$ the category of topological spaces of finite real cohomological type.
In other words we require throughout that all our spaces have finite dimensional real cohomology in each degree.
We equip the category $\Top$ with a homotopical structure by declaring the weak equivalences to be the weak homotopy equivalences.
The symmetric monoidal structure on $\Top$ is given by the cartesian product as usual.
Similarly, we equip the category of simplicial sets $\sSet$ with the standard homotopical structure such that the weak homotopy equivalences are the weak equivalences, and consider it symmetric monoidal with the cartesian product.
In general we denote the category of simplicial objects in a category $\mC$ by $s\mC$, and dually cosimplical objects by $c\mC$.
Of particular importance is the cosimplicial space formed by the simplices
\[
 \Delta^\bullet \in \ob (c\Top).
\]
One has the the following functors %(in fact, they are Quillen adjunctions if we were working with model categories)
\[
 \begin{tikzcd}[every arrow/.append style={shift left},column sep=8em]
  \Top \ar{r}{\Hom_{\Top}(\Delta^\bullet, -)} & \ar{l}{|-|} \sSet\ar{r}{\Hom_{\sSet}(-,  \Omega_{poly}(\Delta^\bullet)) }  & \ar{l}{\Hom_{\dgca}(\Res(-), \Omega_{poly}(\Delta^\bullet)) } \dgca^{op}.
 \end{tikzcd}
\]
where $\Omega_{poly}(\Delta^\bullet)$ is the simplicial dgca of polynomial differential forms on simplices.
 
We define a \emph{dgca model} for a space $X$ to be an object of $\dgca$ weakly equivalent to the dgca
\[
\Omega_{PL}(X) :=  \Hom_{\sSet}( \Hom_{\Top}(\Delta^\bullet, X), \Omega_{poly}(\Delta^\bullet)).
\]
The functor $\Omega_{PL}$ is homotopically (symmetric) monoidal, i.e., we have a weak equivalence of functors
\[
 \Omega_{PL}(-)\otimes \Omega_{PL}(-) \to \Omega_{PL}(-\times -).
\]

\begin{rem}
 Below we shall work with two subcategories of $\Top$, namely manifolds and semi-algebraic manifolds. In these cases (and restricting to $\K=\R$) the functor $\Omega_{PL}$ may be replaced by the weakly equivalent functors $\Omega(-)$ (smooth forms), or respectively $\Omega_{PA}$ (PA forms, cf. \cite{HLTV}). Both of these functors share the same monoidality properties.
\end{rem}

\begin{rem}
 Note that we apply our notion of dgca model also to non-simply connected $X$. This is a ``naive'' notion of model, the standard (in some respects better) notion would be a dgca model for the universal cover $\tilde X$, together with a (homotopy) action of $\pi_1(X)$ on this model.
\end{rem}

We shall use the following result, which can be subsumed under the slogan "the model of the pullback is the pushout of the models".
\begin{thm}[{\cite[Proposition 15.8]{FHT}}, cf. also {\cite[Theorem 2.4]{HessRHT} }]\label{thm:from hess}
Consider the pullback diagram 
\[
\begin{tikzcd}
E\times_B X \ar{r}\ar{d} & E \ar{d} \\
X\ar {r} & B 
\end{tikzcd}
\]
with $E\to B$ a Serre fibration with fiber $F$, $X$ and $B$ simply connected and $E$ path connected. Suppose further that $B$ or $F$ is of finite rational type.
Then the homotopy pushout $A$
\[
\begin{tikzcd}
\Omega_{PL}(B)  \ar{r}\ar{d} & \Omega_{PL}(E) \ar{d} \\
\Omega_{PL}(X) \ar {r} & A 
\end{tikzcd}
\]
is a dgca model for $E\times_B X$.
\end{thm}

\begin{rem}
FOR US: I think that the simple connectivity assumption on $X$ can be relaxed, but I do not know a reference. Below, I'll just restrict to $X$ simply connected when we use the result.

The simple connectivity assumption on $B$ however can in general not be dropped, for example consider the following case relevant for us:
$X=\Z_2//\Z_2=*$, $B=B\Z_2$, $E=E\Z_2$. The fibration $E\to B$ is the obvious one. We have the following pullback diagram 
\[
\begin{tikzcd}
\Z_2  \ar{r}\ar{d} & E\Z_2 \ar{d} \\
* \ar {r} & B\Z_2 .
\end{tikzcd}
\]
On the other hand the homotopy pushout 
\[
\begin{tikzcd}
\K\cong \Omega_{PL}(B)  \ar{r}\ar{d} & \K\cong\Omega_{PL}(E) \ar{d} \\
\K\cong\Omega_{PL}(X) \ar {r} & A 
\end{tikzcd}
\]
is $A=\K$. (Not $\K^2$ as it "should be".)
The example also shows that for a non-connected group $G$ acting on a space $Y$ the rational model of the homotopy quotient $Y//G$ does not encode a rational model for $Y$ with $G$-action. That is why we will have to treat specially the equivariant cohomology for non-connected $G$. 
\end{rem}

\subsection{Our convention regarding ``Hopf''}
It has become more or less standard in the operadic community to call a cooperad in the category $\dgca$ a Hopf cooperad.
Dually, one also calls an operad in cocommutative coalgebras a Hopf operad.
More generally, a (something)-object in $\dgca$ is often called a ``Hopf-(something)''.
In this paper we shall follow this naming pattern.
There is a certain notational conflict present, since a ``Hopf algebra'' is, in the standard sense, not necessarily cocommutative or commutative.
Fortunately, in this paper all occurring Hopf algebras are cocommutative, and we shall adopt the notation ``Hopf coalgebra'' for such objects, which is a coalgebra object in $\dgca$.
Mind that we ignore throughout the presence of an antipode.

\subsection{Semi algebraic sets and PA forms}\label{sec:PAforms}
Following \cite{K2} and \cite{LV} we will study the real homotopy type of the (framed or unframed) little cubes operads by considering the dgca of PA forms on (a version of) this operad.
The construction of the dgca of PA forms $\Omega_{PA}(X)$ on a semi-algebraic set $X$ was sketched in the appendix of \cite{KS}, and worked out in detail in \cite{HLTV}.
For the purposes of this paper, we will use the following properties of PA forms shown in \cite{HLTV}.
\begin{itemize}
\item The functor $\Omega_{PA}$ is a contravariant, homotopically monoidal functor from the category of semi-algebraic sets to the category of dgcas.
\item It is weakly equivalent to Sullivan's functor $\Omega_{PL}$.
\item There is a dg subalgebra $\Omega_{min}(X)\subset \Omega_{PA}(X)$ containing the semi-algebraic functions, and for $f: X\to Y$ an SA bundle (see \cite{HLTV}) there there is a push-forward (``fiber integral'') operation
\[
 f_* : \Omega_{min}(X) \to \Omega_{PA}(Y)
\]
satisfying the Stokes Theorem. We shall also denote the pushforward with an integral sign $\int$ if no confusion arises.

We note in particular that the forgetful maps $f: \FM_m(r+s)\to \FM_n(r)$ of the Fulton-MacPherson compactification of the configuration spaces of points satisfy the hypothesis, and hence give rise to push-forward operations.
\end{itemize}

We shall treat the functor $\Omega_{PA}$ mostly as a ``blackbox'', using only the above formal properties, and refer the reader to loc. cit. for more information on the construction of $\Omega_{PA}$.


\section*{A foreword for sections \ref{sec:homotopy operads}-\ref{sec:dgca models operads}}
In sections \ref{sec:homotopy operads}-\ref{sec:dgca models operads} we will outline some elements of (rational or real) homotopy theory for operads in $G$-spaces.
We want to emphasize however, that our sole goal in the constructions below is to provide a rigorous version of the statement (Theorem \ref{thm:equiv to framed}) that from a model for the equivariant forms on an operad in $G$-spaces $\op T$ one can recover a dg Hopf cooperad model for the framed operad $\op T \circ G$.
Although no expert would probably doubt that statement, it is more or less impossible to extract from the existing literature, at least to our knowledge.
It should hence be kept in mind that sections \ref{sec:homotopy operads}-\ref{sec:dgca models operads} are not an adequate treatment of the homotopy theory of operads (in $G$-spaces, and/or possibly with operations in arity 1).
Important questions like the equivalence of $(\infty,1)$-categories or recoverability of (the rationalization of) an operad from its model remain unanswered and will be left for a more thorough treatment elsewhere.


\section{Homotopy (co)operads, W construction, and dgca models for operads}\label{sec:homotopy operads}

\subsection{Motivation}

One of the main problems for the algebraic models of topological  operads is that the all known functors $\Omega:\mathrm{Top} \rightarrow Dgca$ (including de Rham differential forms, Sullivan PL forms $\Omega_{PL}$ and semialgebraic forms $\Omega_{PA}$)  which construct a dgca model for a space  have the wrong monoidality properties, in that they are lax monoidal rather than oplax monoidal. In particular, the collection of dgcas $\Omega\op P(n)$ associated to a topological operad $\op P$ does not form a cooperad; instead of a cocomposition one (only) has the following zigzag:
\[
\begin{tikzcd}
\Omega(\op P(n+m-1)) \ar{r}{\Omega(\circ_i)} 
&
\Omega({\op P}(n) \times {\op P}(m))
&
\Omega({\op P}(n)) \otimes \Omega( {\op P}(m))
 \ar{l}[above]{\sim}
\end{tikzcd}
\]
In order to go around this defect we will introduce an intermediate category of homotopy (co)operads with a functor $W$ to the category of ordinary (co)operads, in completed dg vector spaces.
In the most relevant example, we hence have the following functors
\[
\begin{tikzcd}
\text{ Operads in }{Top} \ar{r}{\Omega} 
& \text{Homotopy Cooperads in $\dgca$} \ar{r}{W}
&  \text{complete Cooperads in $\dgca$}.
\end{tikzcd}
\]
The composition of these functors can be understood as a version of differential forms on the Boardman-Vogt $W$-construction (\cite{BoVo}, see also \cite{BM}) of a topological operad. 

\subsection{Homotopy operads and cooperads}
%\subsection{Homotopy operads and cooperads}

Let $\op P$ be an operad in a symmetric monoidal category $\mC$. If $F:\mC\to \mD$ is a lax symmetric monoidal functor into another symmetric monoidal category $\mD$, then $F(\op P)$ is naturally an operad in $\mD$.
Similarly, if $G:\mC^{op}\to \mD$ is an oplax symmetric monoidal functor, then $G(\op P)$ is a cooperad in $\mD$.

However, if $F:\mC\to \mD$ is oplax monoidal (or respectively $G$ lax monoidal) then $F(\op P)$ is not a priori an operad (and $G(\op P)$ not naturally a cooperad). 
One can however go around this ``defect'' by introducing a notion of homotopy operad, as proposed in \cite[section 3]{LV} as follows.
Let $\Tree$ be the symmetric monoidal category whose objects are forests of rooted trees, and whose morphisms are generated by (i) isomorphisms of forests of trees, (ii) edge contractions and (iii) cutting of an internal edge, thus splitting a tree into two. The monoidal product is the disjoint union of trees.

\begin{defi}[variant of \cite{LV}]
Let $\mC$ be a homotopical category with monoidal structure such that the product $\otimes$ is a homotopical functor.
A (non-unital) homotopy operad in the category $\mC$ is a symmetric monoidal functor $\Tree\to \mC$ such that the images of all edge cutting morphisms are weak equivalences.
A (non-unital) homotopy cooperad is a symmetric monoidal functor $\Tree\to \mC^{op}$ such that the images of all edge cutting morphisms are weak equivalences.
\end{defi}

We denote the category of homotopy operads in $\mC$ by $\Hop_\mC$, and that of homotopy cooperads by $\Hop_\mC^c$.

\begin{ex}\label{ex:op is hop}
Let $\op P$ denote an (ordinary) operad in $\mC$. Then there is natural symmetric monoidal functor 
\begin{align*}
\Tree \to \mC \\
T\to \otimes_T \op P
\end{align*}
assigning to a forest $T$ the tree- (or forest-)like tensor product of $\op P$, and assigning the edge contraction morphisms the respective composition morphisms in the operad $\op P$.
Hence any operad may be considered as a homotopy operad.
Furthermore, the functor thus defined is clearly homotopical.
\end{ex}

\begin{ex}\label{ex:cop is hop}
Suppose that $\op C$ is a cooperad in $\FC$, or similarly a completed Hopf cooperad, i.e., a cooperad in $\hdgca$.
Then the assignment
\begin{align*}
T\to \hat \otimes_T \op P
\end{align*}
defines a homotopy cooperad in $\dgVect$, or, respectively, $\dgca$. In other words, a complete (Hopf) cooperad becomes a (non-complete) homotopy cooperad, so that we have functors
\begin{align*}
\Op_{\FC}^c &\to \Hop^c_{\dgVect} \\
\Op_{\hdgca}^c &\to \Hop^c_{\dgca} 
\end{align*}
These functors are homotopical since so is the completed tensor product functor.
\end{ex}

It is furthermore clear from the definition that composition with a symmetric homotopically comonoidal functor $F:\mC\to \mD$ takes homotopy operads in $\mC$ to homotopy operads in $\mD$, and composition with a symmetric homotopically comonoidal functor $G:\mC^{op}\to \mD$ takes homotopy operads in $\mC$ to homotopy cooperads in $\mD$.

The notion of morphism between homotopy operads is defined in the obvious manner as a natural transformation of functors.
We make the categories $\Hop_\mC$ and $\Hop_{\mC}^c$ into homotopical categories by declaring the weak equivalences to be the morphisms that are objectwise weak equivalences (i.e., the weak equivalences of functors).


Example \ref{ex:op is hop} demonstrates the existence of a "forgetful" functor
\[
F : \Op_\mC \to \Hop_\mC.
\]

As a special case of the notion of homotopy operad we introduce the notion of homotopy $\mC$-algebra.
\begin{defi}
 Let $\mC$ be a homotopical monoidal category as above and assume $\mC$ has a final object $0$.
Then we say that a homotopy operad $\op P\in \Hop_\mC$ is a (non-unital) homotopy $\mC$-algebra if all forests except the linear ones (i.e., with all vertices of valence $\neq 1$) are sent to $0$.
Similarly we define the notion of homotopy $\mC$-coalgebra.
\end{defi}
\begin{rem}
 FOR US: The requirement that 0 exists is clearly not necessary, one can just define the notion as a functor from the subcategory... but this way is faster.
\end{rem}


\begin{rem}[Relation to other notions of homotopy operad]
There are various notions of \emph{homotopy operad} in the literature. The notion we use here is closely related to dendroidal objects in $\mC$ \cite{MW} and also the notion used in \cite{Ho}.
Loosely speaking, the difference is that for dendroidal objects one does not have the cutting morphisms in the category $\Tree$, but rather forgetful morphisms which remove vertices instead.
Also, one restricts to the case where the monoidal structure is the categorial product.
In this way one obtains a map from the image of a tree into the product of its corollas.
Put differently, our approach is essentially equivalent but allows for the tensor product on $\mC$ to differ from the categorial one.
\end{rem}

\subsection{Dgca model for topological operads}\label{sec:dgca mod for op}
In particular note that the functor $\Omega_{PL}$ is homotopically monoidal.
Hence, given a topological operad $\op T$ it gives rise to a dg Hopf homotopy cooperad, i.e., a homotopy cooperad in the category $\dgca$. We denote this homotopy cooperad by
\[
 \Omega_{PL}(\op T).
\]

\begin{defi}
 Let $\op T$ be a topological operad. Then we define a \emph{dg Hopf cooperad model} or short \emph{dgca model} for $\op T$ to be any homotopy dg Hopf cooperad quasi-isomorphic to $\Omega_{PL}(\op T)$.
\end{defi}
In particular, for $\op T$ an operad in smooth manifolds or semi-algebraic sets, we will use the models given by the smooth or PA forms $\Omega(\op T)$, $\Omega_{PA}(\op T)$ below.
Note that the notation is slightly abusive since $\Omega_{PA}(\op T)$ is not just the collection of dgcas $\Omega_{PA}(\op T(r))$ indexed by natural numbers, but a collection of dgcas, one for each forest, with suitable maps between them.



\subsection{Unital variant}
Presently we have been considering a notion of homotopy operad without operadic units.
There is a unital version as well. Define the category $\Tree_1$ to have the same objects as $\Tree$, but the morphisms are larger in that one adds the additional generating morphism of cretaing a univalent (i.e., one input, one output) vertex anwhere in a forest. This includes adding one new tree to a (possibly empty) forest, composed of just that one vertex.
A unital homotopy operad is then defined as a monoidal functor from $\Tree_1$, such that all cutting morphisms are sent to weak equivalences.



\subsection{Homotopy modules}
There is an extension of the notion of homotopy operad to operadic modules.
Let $\Tree_*$ be the category defined similarly to $\Tree$, but such that at most one root of one tree can be marked (or carry a different color, say).
The morphisms are defined as before, with the mark preserved.
Let $\op P$ be a homotopy operad in $\mC$. Then we define a homotopy operadic right module as an extension of the corresponding symmetric monoidal functor 
\[
 \op P: \Tree \to \mC
\]
to a symmetric monoidal functor 
\[
 \op P_*: \Tree_* \to \mC,
\]
such that all cutting morphisms are sent to weak equivalences.

We similarly define the notion of homotopy operadic right comodule.
Again, if the functors are trivial on trees with vertices of valence $\geq 2$ this notion reduces to that of a homotopy (co)algebra and a homotopy (co)module.
If $\mC=\dgca$ we will often use the alternative name homotopy Hopf (co)algebra and homotopy Hopf (co)module for these notions.

We impose the structure of a homotopical category on the homotopy (right) modules by declaring a natural transformation a weak equivalence if it is a weak equivalence objectwise.


\begin{rem}
FOR US: Here there is potential notational clash as one calls the stuff above operadic right module, while in the algebra setting we want to think of left modules rather than right modules.
The ``convention'' here is hence that for algebras we think of the (linear) trees as extending from right (root) to the left (leaves).
\end{rem}


% \subsection{The dgca (comodule) model of a $G$-space}
% A topological group (or monoid) $G$ is in particular a homotopy algebra in $\Top$ as discussed above.
% Similarly, any $G$-space $X$ gives rise to a homotopy module over this homotopy algebra.
% Applying the functor $\Omega_{PL}$ we obtain a homtopy Hopf coalgebra which we denote by $\Omega_{PL}(G)$, and a homotopy Hopf comodule over $\Omega_{PL}(G)$ which we denote by $\Omega_{PL}(X)$.
% Note that this notation is slightly abusive, since a homotopy coalgebra of homotopy comodule is not only one space, but a collection of such, one for each ``string-like'' tree.
% 
% The homotopy Hopf comodule $\Omega_{PL}(X)$ is our second version of dgca model for the $G$-space $X$.

\subsection{A model for \texorpdfstring{$G$}{G}-spaces (comodule model)}\label{sec:comodule model}
A topological group (or monoid) $G$ is in particular a homotopy algebra in $\Top$ as discussed above.
Similarly, any $G$-space $X$ gives rise to a homotopy module over this homotopy algebra.
Applying the functor $\Omega_{PL}$ we obtain a homotopy Hopf coalgebra which we denote by $\Omega_{PL}(G)$, and a homotopy Hopf comodule over $\Omega_{PL}(G)$ which we denote by $\Omega_{PL}(X)$.

\begin{defi}
We define a \emph{comodule model} of the $G$-space $X$ to be a pair consisting of a weak dg Hopf coalgebra and a dg Hopf comodule, weakly equivalent to the pair $(\Omega_{PL}(G), \Omega_{PL}(X))$.
\end{defi}
Note that again this notation is slightly abusive, since a homotopy coalgebra or homotopy comodule is not only one vector space, but a collection of such, one for each ``string-like'' tree.



% Let us remark on two variations:
% \begin{itemize}
% \item In the relevant examples for us, $X$ is a semi-algebraic space acted upon by the algebraic (compact Lie) group $G$.
% In this case, we may replace the pair $(\Omega_{PL}(G), \Omega_{PL}(X))$ by the quasi-isomorphic PA version $(\Omega_{PA}(G), \Omega_{PA}(X))$.
% \item Using the $W$-construction above, we may construct from a comodule model $(A,M)$ as above an honest dg Hopf coalgebra $W(A)$ and an honest $W(A)$-dg Hopf comodule $W(M)$ (in the category $\hdgca$, i.e., using complete complexes and complete tensor products).
% 
% Conversely, given a a pair of honest coalgebra and comodule $(A,M)$ in $\hdgca$ which is quasi-isomorphic to the pair $(W(\Omega_{PL}(G)), W(\Omega_{PL}(X)))$, then clearly $(A,M)$ considered as a homotopy Hopf coalgebra/Hopf module pair is quasi-isomorphic to $(\Omega_{PL}(G), \Omega_{PL}(X))$. Hence may equivalently define a comodule model to be a pair $(A,M)$ of honest coalgebra and comodule (in $\hdgca$), quasi-isomorphic to the pair $(W(\Omega_{PL}(G))=A_G, W(\Omega_{PL}(X)))$.
% 
% \end{itemize}

% Let us show that this second notion is equivalent to the first one of section \ref{}, in a precise sense.
% To this end, we have the following diagram of categories and functors.
% \[
%  \begin{tikzcd}
%   G\Top \ar{r}{\Omega_{PL}} \ar{d} & \ar{d} \Omega_{PL}(G)-hcomod  \\
%  \Top/BG \ar{r}{\Omega_{PL}} \ar{u}& \Omega_{PL}(BG)/\dgca  \ar{u}
%  \end{tikzcd}.
% \]
% The two vertical functors on the left have been discussed in section \ref{}. The two vertical functors on the right are as follows.
% 
% 
% 
% 
% Finally, we note for later reference that the category of homotopy Hopf comodules admits a natural monoidal structure, in such a way that the functor $\Omega_{PL}$ from $G\Top$ is lax monoidal.
% (TODO describe)



%\subsection{Simplifications for compact Lie groups $G$}
%Let let us return to the homotopy Hopf coalgebra $\Omega_{PL}(G)$
%Furthermore let us work with the (weakly equivalent) smooth version $\Omega(G)$ of the homotopy Hopf coalgebra $\Omega_{PL}(G)$ above.


\subsection{\texorpdfstring{$W$}{W} construction}\label{sec:W}

Now assume that $\mC$ is a symmetric monoidal homotopical category. Let $*\in \ob \mC$ be the monoidal unit.

\begin{defi}[{\cite[Definition 4.1]{BM}}]
A segment $I$ in $\mC$ is a factorization $*\sqcup *\xrightarrow{(0,1)} I \xrightarrow{\epsilon} *$, together with an associative product
\beq{equ:veemap}
\vee : I\otimes I\to I
\eeq
for which 0 is neutral and 1 is absorbing.
\end{defi}

\begin{rem}
The cases most relevant to this paper are the following: (i) The category $\mC$ is $dgVect$, and the segment $I$ is the (three-dimensional) complex of simplicial chains of the interval, (ii) the category $\mC$ is topological spaces and the segment is a topological interval $[0,1]$ and, dually, (iii) the category $\mC=\dgca$ is dg commutative algebras and the (co-)segment is the space of polynomial differential forms on the interval $[0,1]$.
\end{rem}

Given a segment we may define functors
\begin{align*}
W: \Hop_\mC &\to \Op_\mC \\
W_1: \Hop^1_\mC &\to \Op^1_\mC 
\end{align*}
such that the composition $\Op_\mC \to \Hop_\mC\xrightarrow{W} \Op_\mC$ agrees with the Berger-Moerdijk $W$-construction of operads in $\mC$ \cite{BM}.
Concretely, for a (nonunital) homotopy operad $\op P$ the operad $W(\op P)$ is defined as follows.
For $S$ a finite set let $\mT_S$ be the category whose objects are trees with leafs (bijectively) labelled by $S$, and with some subset of internal edges distinguished. The distinguished internal edges we will call "cut edges".
The morphisms are generated by the operation of contracting a non-cut edge, and of adding a non-cut edge to the set of cut edges, pictorially, marking a cut edge by a dashed line:
\begin{align*}
\begin{tikzpicture}
\coordinate (v) at (0,0.5);
\coordinate (w) at (0,-0.5);
\draw (v) edge (w) edge +(-.5,-.5)  edge +(.5,-.5)  edge +(0,.5)
         (w)   edge +(-.5,-.5)  edge +(.5,-.5)  edge +(0,-.5);
\end{tikzpicture}
&\mapsto
\begin{tikzpicture}
\coordinate (v) at (0,0);
\draw (v)  edge +(-.5,-.5)  edge +(.5,-.5)  edge +(0,.5) edge +(-.25,-.5)  edge +(.25,-.5)  edge +(0,-.5);
\end{tikzpicture}
& 
\begin{tikzpicture}
\coordinate (v) at (0,0.5);
\coordinate (w) at (0,-0.5);
\draw (v) edge (w) edge +(-.5,-.5)  edge +(.5,-.5)  edge +(0,.5)
         (w)   edge +(-.5,-.5)  edge +(.5,-.5)  edge +(0,-.5);
\end{tikzpicture}
&\mapsto
\begin{tikzpicture}
\coordinate (v) at (0,0.5);
\coordinate (w) at (0,-0.5);
\draw (v) edge[dashed] (w) edge +(-.5,-.5)  edge +(.5,-.5)  edge +(0,.5)
         (w)   edge +(-.5,-.5)  edge +(.5,-.5)  edge +(0,-.5);
\end{tikzpicture}\ .
\end{align*}
Clearly, cutting the tree along all cut edges produces a forest, and thus a homotopy operad $\op P$ induces a functor
\[
\op P : \mT_S\to \mC
\]
by restriction.
Furthermore, we may define a functor
\[
E_I : \mT_S \to \mC^{op}
\]
by sending a tree $T$ to 
\[
E_I(T) = \bigotimes_{e} I,
\]
where the tensor product is over non-cut edges.
The functor $E_I$ sends the contraction morphism contracting an edge $e$ to the "initial endpoint" $*\xrightarrow{0} I$, applied to the factor $I$ corresponding ot $e$, and the cutting morphism to the "terminal endpoint" $*\xrightarrow{1} I$.
Finally, we define the functor $W$ to send the homotopy operad $\op P$ to the operad $W(\op P)$ given by the collection
\[
W(\op P)(S) := \int^{T\in \mT_S} \op P(T)\otimes E_I(T).
\]
\begin{rem}
Let us describe the above coend also in more concrete terms, assuming that the underlying category $\mC$ is concrete, as is always the case for the examples of interest here.
The above coend can then be understood as a space of decorated trees.
A tree (with some cut edges) is decorated as follows:
\begin{itemize}
\item Cutting the tree $T$ along the cut edges produces a forest of sub-trees $T_1,\dots,T_n$, each "decorated" by an element of $\op P(T_1),\dots, \op P(T_n)$.
\item Additionally each non-cut edge is decorated by an element of $I$.
\end{itemize}
The coend construction enforces the following relations on these data.
\begin{itemize}
\item Suppose that the tree $T'$ is obtained from the tree $T$ by contracting the non-marked edge $e$.
Then a decoration of $T$ in which $e$ is decorated by the "left" endpoint $*\in I$ is considered equivalent to the decoration of $T'$ obtained by applying a contraction morphism to the decoration in $\op P(T_j)$ of the subtree in which $e$ lies.
\item Similarly, suppose $e$ is a non-cut edge in subtree $T_j$ of $T$, decorated by the "right" endpoint $*\in I$. Then the decorated tree is considered the same as the tree $T''$ with edge $e$ cut, with the decoration obtained by applying the "splitting" morphism to the decoration in $\op P(T_j)$. 
\end{itemize}
The following picture shall illustrate the various decorations, with elements of $I$ on non-cut edges and the subtree decorations.
\[
\begin{tikzpicture}
\coordinate (v1) at (0,2);
\coordinate (v2) at (-1,1);
\coordinate (v3) at (1,1);
\coordinate (v4) at (0,0);
\coordinate (v5) at (0,-1);
\draw (v1) edge node[auto]{$\scriptstyle I$} (v2) edge node[auto]{$\scriptstyle I$} (v3) edge +(-.5,-.5)  edge +(.5,-.5)  edge +(0,.5)
	 (v2) edge +(-.5,-.5)  edge +(.5,-.5)  edge +(0,-.5)
          (v3) edge[dashed] (v4) edge +(.5,-.5)  edge +(0,-.5)
          (v4) edge node[auto]{$\scriptstyle I$} (v5) edge +(-.5,-.5)  edge +(.5,-.5) 
          (v5) edge +(-.5,-.5)  edge +(.5,-.5)  edge +(0,-.5);
 \draw[dotted] (0,1.5) ellipse (2 and 1.5);
 \node at (-2.3,1.5) {$\scriptstyle T_1$};
  \draw[dotted] (0,-1) ellipse (1 and 1);
   \node at (-1.3,-1) {$\scriptstyle T_2$};
   \node at (2,1.5) {$\scriptstyle \op P(T_1)$};
   \node at (1,-1) {$\scriptstyle \op P(T_2)$};
\end{tikzpicture}
\]
In any case, note that $W(\op P)$ is a free operad.
\end{rem}

The operadic composition in $W(\op P)$ is just the grafting of trees, with the newly added edge being part of the set of cut edges.

In the unital case the construction is similar except for the following modification:
One enlarges the category $\mT_S$ by allowing for the insertion of a vertex with one input and out put, similarly to the extension of our category $\Tree$ to $\Tree_1$. Call the category generated $\mT_S^1$.
Assuming that $\op P$ is a unital homotopy operad, i.e., a monoidal functor 
\[
\op P\colon \Tree_1 \to \mC,
\]
it readily induces a functor 
\[
\op P :  \mT_S^1\to \mC,
\]
which we abusively denote by the same symbol.
Furthermore, we extend the functor $E_I : \mT_S \to \mC^{op}$ from above by sending the additional morphism of inserting a vertex to the product map \eqref{equ:veemap}. (It is only in the unital case that the map $\vee$ is used.)

%Dually we may define the $W$ construction of homotopy cooperads in $\mC$, taking values in cooperads in $\mC$.


\begin{ex}
If $\mC=dgVect$, then a homotopy operad $\op P$ determines a cooperad $\op C$ such that 
\[
\op C(S) = \oplus_{T\in ob\mT_S} \op P(T)\otimes (\K[1])^{\otimes |VT|},
\]
with the differential being determined by the contraction morphisms, and the cooperad structure by the edge cut morphisms. The operad $W(T)$ is then the cobar construction of the cooperad $\op C$.
\end{ex}

\begin{ex}
If $\op P$ is a homotopy operad arising from an operad $\op P_0$ through  the forgetful functor of example \ref{ex:op is hop}, then $W(\op P)$ is identical to the Berger-Moerdijk $W$-construction of $\op P_0$.
\end{ex}
In particular, this means that there is a natural transformation (and, under good conditions weak equivalence) $W\circ F \to \mathit{id}$, where we denote by $F$ the forgetful functor. 


\subsection{\texorpdfstring{$W$}{W} construction for homotopy cooperads}
We note that the $W$ construction of the previous section does not readily dualize to the case of homotopy cooperads.
The reason is that while one can impose a natural operad structure on the coend appearing there, one cannot readily impose a cooperad structure on the corresponding end, due to completion issues.
Our solution is to resort to a completed version. We do not know how to do the construction in full generality.
However, in all cases relevant to this paper the category $\mC$ in which our cooperads take values is an enriched version of the category of cochain complexes $\dgVect$.
We will then define their $W$ construction to be an operad in the corresponding category of complete objects $\hat \mC$. For example, to a homotopy cooperad in $\mC=\dgVect$ we will assign a cooperad in the complete filtered cochain complexes $\FC$.

So assume now that $\mC$ is either of the category $\dgVect$ or $\dgca$ and define the filtered complete version $\hat \mC$ as $\FC$ or $\hdgca$ accordingly.

For our (co-)segment object $I$ we take the polynomial forms on the interval $\Omega_{poly}([0,1])$ if $\mC=\dgca$ or the subspace 3-dimensional sub-complex of forms at most linear in the coordinate if $\mC=\dgVect$.

Now define for a homotopy cooperad $\op C$ the symmetric sequence
\[
W(\op C)(S) := \int_{T\in \mT_S} \op C(T)\otimes E_I(T),
\]
where the functor $E_I(-)$ is defined dually to above, and equip it with the descending complete filtration by the number  of vertices in trees.
More concretely, the space $W(\op C)(S)$ may be interpreted as a space of functions on the set of trees, assigning to every tree $T$ a decoration in $\op C(T)\otimes E_I(T)$, that satisfy certain coherence relations.
The filtration is such that $\mF^pW(\op C)(S)$ consists of all functions supported on trees with at least $p$ nodes.

Note that due to the filtration we may now define the cooperadic cocomposition dually to the operadic composition in the previous section by de-grafting trees. Due to the completion, and since there are only finitely many trees with given sets of leaves and number of vertices, the result takes values in the completed tensor product space.

Furthermore, the functor $W$ has good homotopical properties, as detailed in the following result.
\begin{thm}\label{thm:W properties}
Let $\mC$ be one of the categories above (i.e., $\dgVect$, $\dgca$), and $\hat \mC$ its completed version as above (i.e., $\FC$, $\hdgca$).
\begin{enumerate}
\item For any homotopy cooperad $\op C$ the cohomology of $\gr W(\op C)$ is concentrated in grading degree 1.
Hence $W(\op C)$ indeed takes values in $\Op_{\hat C}$ as the conditions of section \ref{sec:vector spaces complexes} are satisfied.
\item The functor 
\[
W\colon  \Hop^c_{\mC} \to \Op^c_{\hat \mC}
\]
is homotopical.
\item There is a natural weak equivalence 
\[
\iota \Rightarrow \cdot \Leftarrow W\circ F\circ \iota,
\]
where the functor $\iota:\Op_{\mC}^c\to \Op^c_{\hat \mC}$ is the inclusion from cooperads to complete cooperads, i.e., from cooperads in $\mC$ to cooperads in $\hat \mC$, and $F:\Op^c_{\hat \mC}\to \Hop^c_{\hat \mC}$ is the ``forgetful'' functor assigning to an honest Hopf cooerad the corresponding homotopy Hopf cooperad. 
%with the following property:
%Suppose that $\op C$ is a cooperad in $\mC$, which we consider as an operad in $\hat \mC$ via Remark \ref{}. 
%(Alternatively, we may take for $\op C$ a cooperad in $\hat \mC$ with $\gr^p\op C$ acyclic for $p\neq 1$.)
%Then the map 
%\[
%\op C\to W(F(\op C))
%\]
%is a weak equivalence.
\item There is a natural weak equivalence
\[
F\circ W \Rightarrow \id
\]
\end{enumerate}
\end{thm}

\begin{proof}
We will conduct the proof for the case $\mC=\dgca$, which is most relevant for this paper. The case $\mC=\dgVect$ is simpler and can be treated in the same way.

 (1) For the first item, we fix some arity $r$ consider the spectral sequence associated to the filtration by number of vertices in trees as introduced above.
 The original complex $W(\op C)(r)$ can be seen as a space of forms on metric trees: For each tree $T$ we assign a form depending on the length of edges, with values in $\op C(T)$, with conditions on the boundary values as edge length go to 0 or 1. Concretely, when the edge length of edge $e$ is zero, the decoration agrees with the one obtained from the decoration on $T/e$ via the contraction morphism, and is the edge length becomes 1, the decoration factors into decorations of the two components of $T$ obtained by cutting $e$. 
 The associated graded complex then can again be understood as forms on metric trees, with the condition that the decoration vanishes upon the edge length approaching zero, and a(n unaltered) factorization condition when the length approaches 1. 
 Consider first a tree $T$ with a single egde splitting $T$ into $T_1$ and $T_2$.
 Then the relevant complex (call it $V$) is the pullback
 \[
  \begin{tikzcd}
  V  \ar{r}\ar{d} 
  & 
  \op C(T_1) \otimes \op C(T_2) \ar{d}{\sim}
  \\
   \Omega_{poly}([0,1],0)\otimes \op C(T) \ar{r}{\mathit{ev}_1}
   & 
   \op C(T)
  \end{tikzcd},
 \]
where $\mathit{ev}_1$ is the evaluation of the form at the endpoint of the interval, and $\Omega_{poly}([0,1],0)$ are the polynomial forms on the unit interval vanishing at the starting point of the interval.
The right-hand arrow is a quasi-isomorphism by the axioms for homotopy cooperads.
The complex in the lower left is acyclic since $\Omega_{poly}([0,1],0)$ is.
Since pullbacks along fibrations preserve quasi-isomorphisms we conclude that $V$ is acyclic as well.

For a more complicated tree $T$ with $>1$ edges we proceed similar (iterating on edges) to show that the corresponding piece of the associated graded complex is acyclic. We conclude that all cohomology is concentrated in $\gr^1(W\op C)$, thus showing item (1).
 
(2) We are given a map of homotopy cooperads $f:\op C\to \op D$ and we have to show that the induced map $F: W\op C\to W\op D$ is a quasi-isomorphism.
Clearly $F$ is compatible with the filtrations (by number of vertices) on both sides, so may consider the spectral sequences associated to those filtrations on both sides.
As we have just seen, the $E_1$ is concentrated in degree 1, corresponding to trees $T$ with one vertex.
The map (induced by) $F$ there just agrees with the map $f_T: H(\op C(T))\to H(\op D(T))$ given by $f$, which is an isomorphism by assumption. Hence the statement (2) follows.

(4) Given a homotopy cooperad $\op C$, note that the homotopy cooperad $F W\op C$ is a functor which assigns to a tree $T$ a space which can be understood as forms on the metrized refinements of $T$, with suitable boundary conditions.
In particular one can evaluate such a form on the tree $T$ (i.e., the trivial refinement of $T$). This gives an element of the space
\[
 \otimes_T \op C
\]
(a tensor product of spaces $\op C(T_j)$ for $T_j$ running over corollas in $T$).
In fact this evaluation is a quasi-isomorphism, as one quickly shows by using statement (1) above.
However, the space $\otimes_T \op C$ comes equipped with a natural quasi-isomorphism into $\op C(T)$ which is part of the data of a homotopy cooperad.
This map then gives our desired quasi-isomorphism of homotopy cooperads $F W\op C \to \op C$.

(3) The argument for the third assertion is slightly more complicated as it invlolves a non-trivial zigzag of functors rather than a direct map.
Concretely, the quasi-isomorphism is realized by the zigzag
\beq{equ:FWzigzag}
F(W(\op C)) \to \hat W(\op C) \leftarrow \op C,
\eeq
where $\hat W(\op C)$ is defined as the following functor $\mT\to \dgca$.

Given a tree $T$ in $\mT$ with $k$ vertices we define a category $Fo_T$ of \emph{refinements} of the tree $T$.
A refinement is a tree $T'$ with a surjective morphism $T'\to T$ by contracting some subset $S_{T'}$ of the edges.
Additionally, we consider as the data of such tree a subset $S'_{T'}\subset S_{T'}$ of marked edges.
For each vertex $v$ of $T$ we have a functor $\pi_v:Fo_T\to \mT_{C_v}$, where $C_v$ is the set of children of $v$.
Now we define
\[
\hat W(\op C)(T) = \int^{T'\in Fo_T} \op C(T') \otimes \bigotimes_{v\in T} E_I(\pi_v(T')).
\]
There are natural maps \eqref{equ:FWzigzag} that are easily checked to be quasi-isomorphisms under the strong-ness condition.

To be more precise:
\begin{itemize}
\item $F(W(\op C))$ can be understood as a space of decorated 3-level trees. We have decorations on the innermost edges by elements of $I$, and we decorate each of the innermost trees $T''$ by an element of $\op C(T'')$.
\item Similarly $\hat W(\op C)$ may be understood as a space of decorated 3-level trees. We still decorate the innermost edges by $I$, but in contrast to $F(W(\op C))$ we decorate the whole ("flattened") tree $T'$ by one element of $\op C(T')$. 
\item There is a natural map $F(W(\op C))\to \hat W(\op C)$ by merging the decorations on the innermost trees into one decoration of $T'$ by using the "gluing" maps of $\op C$.
\item The map $\op C\to \hat W(\op C)$ is defined by using the "splitting" maps of $\op C$ to obtain from a decoration in $\op C(T)$ of the outermost tree a decoration of the flattened (refined) tree $T'$.
\end{itemize}
(TODO: add detail or clear?).

\end{proof}







% \begin{prop}
% Let $\op C$ be a strong homotopy dg Hopf cooperad. Then the weak homotopy dg Hopf cooperad $F(W(\op C))$ is quasi-isomorphic to $\op C$.
% \end{prop}
% \begin{proof}
% The quasi-isomorphism is realized by the zigzag
% \beq{equ:FWzigzag}
% F(W(\op C)) \to \hat W(\op C) \leftarrow \op C,
% \eeq
% where $\hat W(\op C)$ is defined as the following functor $\mT\to \dgca$.
% Given a tree $T$ in $\mT$ with $k$ vertices we define a category $Fo_T$ of \emph{refinements} of the tree $T$.
% A refinement is a tree $T'$ with a surjective morphism $T'\to T$ by contracting some subset $S_{T'}$ of the edges.
% Additionally, we consider as the data of such tree a subset $S'_{T'}\subset S_{T'}$ of marked edges.
% For each vertex $v$ of $T$ we have a functor $\pi_v:Fo_T\to \mT_{C_v}$, where $C_v$ is the set of children of $v$.
% Now we define
% \[
% \hat W(\op C)(T) = \int^{T'\in Fo_T} \op C(T') \otimes \bigotimes_{v\in T} E_I(\pi_v(T')).
% \]
% There are natural maps \eqref{equ:FWzigzag} that are easily checked to be quasi-isomorphisms under the strong-ness condition.
% 
% To be more precise:
% \begin{itemize}
% \item $F(W(\op C))$ can be understood as a space of decorated 3-level trees. We have decorations on the innermost edges by elements of $I$, and we decorate each of the innermost trees $T''$ by an element of $\op C(T'')$.
% \item Similarly $\hat W(\op C)$ may be understood as a space of decorated 3-level trees. We still decorate the innermost edges by $I$, but in contrast to $F(W(\op C))$ we decorate the whole ("flattened") tree $T'$ by one element of $\op C(T')$. 
% \item There is a natural map $F(W(\op C))\to \hat W(\op C)$ by merging the decorations on the innermost trees into one decoration of $T'$ by using the "gluing" maps of $\op C$.
% \item The map $\op C\to \hat W(\op C)$ is defined by using the "splitting" maps of $\op C$ to obtain from a decoration in $\op C(T)$ of the outermost tree a decoration of the flattened (refined) tree $T'$.
% \end{itemize}
% (TODO: add detail).
% \end{proof}

% We will need the following result below, which shows that the functor $W$ is exact on well behaved homotopy (co)operads.
% 
% \begin{prop}
% Suppose that $f:\op C\to \op D$ is a quasi-isomorphism of homotopy cooperads in the category $\mC=\dgca$.
% Suppose that the cohomological degrees of $\op C$ and $\op D$ are bounded below in each arity.\footnote{We say that the degrees are bounded below in each arity if for each finite set $S$ there is a number $N$ such that for each tree $T\in \mT_S$ we have that $\op C(T)$ is concentrated in degrees $\geq N$.}
% Then the induced map of dg cooperads
% \[
% W(f) : W(\op C)\to W(\op D)
% \]
% is a quasi-isomorphism.
% \end{prop}
% \begin{proof}
% Under the boundedness assumption the complexes $W(\op C)$ and $W(\op D)$ are first quadrant double complexes, the two gradings being given by those on the interval objects on the one hand and those on $\op C$ and $\op D$ on the other. Taking the obvious spectral sequences whose first differential is that on $\op C$ (or $\op D$), the map $W(f)$ induces an isomorphism on the $E^1$-pages, and hence an isomorphism on cohomology by standard arguments.
% \end{proof}
% 
% \begin{rem}
% We similarly define the $W$ constructions for homotopy cooperadic right comodules.
% \end{rem}


\section{\texorpdfstring{$G$}{G}-spaces and equivariant cohomology}

\subsection{A notational remark}
In this section we shall introduce several pieces of notation related to $G$-spaces, the classifying space, the homotopy quotient and dgca models thereof.
These objects are more or less standard, and may be constructed or defined in one of several ways.
For example, the equivariant forms on a $G$-space $X$ may be defined as the dgca $\Omega_{PL}(X//G)$. Alternatively (and equivalently) in the smooth and compact setting, we may consider instead the Cartan model, or (still equivalently) the Cartan model of a compact subgroup of $G$.
To complicate matters further (notationally at least), in the semi-algebraic (resp. smooth) category, we may make sense of $\Omega_{PA}(X//G)$ (resp. the smooth forms on $X//G$).
Overall we have for one object (e.g., forms on $X//G$) several explicit models and realizations, that we will have to keep track of and introduce notation for.
We will use the following guidelines:
\begin{itemize}
 \item We will use the notation $B_G$ to refer to some model of forms on $\Omega(BG)$. We use the superscript to distinguish several concrete models we introduce below: For example $B_G^s$ shall denote the forms on a simplicial construction of $BG$.
\item We use the notation $\Omega_G(X)$ to denote some version of equivariant forms on $X$ (i.e., forms on $X//G$). Again, via the superscript we shall distinguish several explicit models.
\item We use an additional superscript $PL$, $PA$ or $sm$ if we want to designate the PL, PA or smooth version of our model.
\end{itemize}

We realize that the notation is thus somewhat cumbersome.
However, most of the objects thus denoted will be used only for intermediate steps.


\subsection{\texorpdfstring{$G$}{G}-spaces}
Let $G$ be a topological group. We denote the category of $G$-spaces by $G\Top$.
It comes with an obvious forgetful functor $G\Top\to \Top$. We equip $G\Top$ with the homotopical structure from $\Top$, i.e., a morphism is a weak equivalence if it induces a weak homotopy equivalence on spaces. (This is sometimes called the coarse homotopical structure.)
The monoidal product is again the cartesian product, equipped with the diagonal action.

The homotopy quotient of the $G$-space $X$ is the space
\[
 X//G := X\times_G EG.
\]
It comes with a natural map $X//G\to BG$ and hence defines a functor into the over-category $\Top/BG$.
We equip the over-category with the homotopical structure induced from the forgetful functor to $\Top$, i.e., a morphism is a weak equivalences if the underlying morphism in $\Top$ is a weak homotopy equivalence.
Conversely, given an element $(\tilde X\to BG)$ of $\Top/BG$ we can assign a $G$-space $F(\tilde X)$ as the pullback
\[
 \begin{tikzcd}
  F(\tilde X) \ar{r}\ar{d} & EG \ar{d} \\
  \tilde X \ar{r} & BG 
 \end{tikzcd}.
\]
We have the following Lemma.
\begin{lemma}
 The functors $(-//G)$ and $F$ are homotopic and realize a weak equivalence of homotopical categories between $G\Top$ and $\Top/BG$.
\end{lemma}
\begin{proof}
To see that the first functor preserves weak equivalences we apply the five Lemma to the long exact sequences associated to the fiber sequences
\[
 G \to X\times EG \to X//G.
\]
To see the corresponding statement for the second functor one similarly applies the five Lemma to the long exact sequences from the fiber sequences
\[
 G\to F(\tilde X) \to \tilde X.
\]

Furthermore, we have the natural weak equivalences
\[
 F(X//G) \leftarrow X\times EG \rightarrow X
\]
and 
\[
 F(\tilde X) // G \to \tilde X.
\]
\end{proof}

Finally note that $\Top/BG$ admits a monoidal structure by the homotopy fiber product over $BG$, $(-\hat \times_{BG}-)$.
Mostly we will work within the subcategory of fibrations over $BG$, for which the homotopy fiber product may be replaced by the ordinary fiber product, which is then also symmetric.


\subsection{A model for \texorpdfstring{$G$}{G}-spaces (equivariant model)}
\label{sec:Gspacemodels}
Suppose that $G$ is connected.
We define our second notion of dgca model of the $G$-space $X$ to be the morphism 
\[
 \Omega_{PL}(BG) \to \Omega_{PL}(X//G),
\]
%which we understand as an object in the over-category $\Omega_{PL}(BG)/\dgca$,
or any weakly equivalent morphism.

More generally, suppose $G$ is possibly not connected, with $G_0\subset G$ the connected component of the identity. 
Let $X$ be a $G$-space. Note that $X//G_0$ and $BG_0= *//G_0$ carry natural actions of $\pi_0(G)=G/G_0$.
We define a dgca model for the $G$-space $X$ to be the morphism 
\[
 \Omega_{PL}(BG_0) \to \Omega_{PL}(X//G_0)
\]
of dgcas with a $\pi_0(G)$-action, or a weakly equivalent morphism.

We call a model as above for the $G$-space $X$ an \emph{equivariant model} or \emph{equivariant forms model}.

%There is an alternative, equivalent notion of dgca model (\emph{comodule model}), which is roughly $\Omega_{PL}(X)$ with a coaction by $\Omega_{PL}(G)$. 
%However, to make this notion precise we need further technical tools, due to the failure of the functor $\Omega_{PL}$ to be strongly monoidal.


\begin{rem}(FOR US)
The above definitions are likely not so good for $G$ with ``pathological'' topology, for example they don't take into account the topology on the set of connected components at all.
However, since for this paper we will eventually exclusively deal with compact Lie groups $G$, we ignore this issue (for the purposes of this paper).
The worried reader may even restrict to compact Lie groups $G$ from the start.
\end{rem}

\subsection{Concrete (simplicial) models for \texorpdfstring{$BG$}{BG} and the homotopy quotient}\label{sec:simpl models BG}
Let $G$ be a topological group. The standard way to construct (or even define) $BG$ is as the fat geometric realization of the topological nerve $G^\bullet = G^{\times \bullet}$  of $G$.
A cosimplicial dgca model for $G$ is then $\Omega_{PL}(G^{\bullet})$.
A dgca model of the classifying space $BG$ may then be constructed as the ``fat totalization'' thereof, i.e., as the end
\[
B_G := \int_{[j]\in {\Delta_+}} \Omega_{PL}(G^j) \otimes \Omega_{poly}(\Delta^j).
\]
(Here $\Delta_+$ is the semi-simplicial category. In other words the object is akin to forms on the fat geometric realization of the nerve of $G$.)
Let $X$ be a $G$-space. Then a model for the homotopy quotient ("equivariant differential forms") is 
\[
\Omega_G^{'s}(X) := \int_{[j]\in \Delta_+} \Omega_{PL}(G^j\times X) \otimes \Omega_{poly}(\Delta^j).
\]
It comes equipped with a natural map
\[
B_G \to \Omega_G^{'s}(X).
\]

The category of dgcas under $B_G$ comes equipped with a natural monoidal structure, the derived tensor product $\bar\otimes_{B_G}$ over $B_G$.
Unfortunately, this monoidal structure is not symmetric, or more precisely, symmetric only up to homotopy.
Since we do not want to deal with monoidal structures up to homotopy, we will merely work with the category of dgcas under $B_G$, free as $B_G$ modules.
Then we can equip this sub-category with the symmetric monoidal product the (non-derived) tensor product $\otimes_{B_G}$ over $B_G$.
To land in this subcategory we replace our functor $\Omega_G^{'s}$ by the quasi-free resolution
\[
 \Omega_G^{s}(X) := B_G \bar \otimes_{B_G} \Omega_G^{'s}(X).
\]
Here we take for the derived tensor product the ``bar complex''-realization, explicitly:
For $A$ a commutative algebra, $M$ and $N$ modules we set
\begin{equation}\label{equ:derived tp def}
 M\bar \otimes_A N := \bigoplus_{k\geq 0} M\otimes A[1]^{\otimes k} N
\end{equation}
with the usual differential.
The functor $\bar \otimes_A$ is symmetric monoidal through the shuffle product.
Hence our $\mod_G(X)$ above in particular retains a commutative algebra structure.
Clearly, we also have the explicit map 
\[
 B_G \to \Omega_G^{s}(X)
\]
which lands in the first summand ($k=0$) in the expression \eqref{equ:derived tp def}.

\begin{lemma}
The functor $\Omega_G^{s}$ is homotopically (symmetric) monoidal, into the category of of dgcas under $B_G$, with monoidal structure the tensor product over $B_G$.
\end{lemma}

\begin{rem}
In case $X$ is a smooth manifold acted upon by a Lie group $G$, we may replace PL forms $\Omega_{PL}(-)$ by smooth forms $\Omega(-)$ above to obtain an explicit equivariant model.
Similarly, in case $X$ is a semi-algebraic space acted upon by the (semi-)algebraic group $G$, we may replace the PL forms $\Omega_{PL}(-)$ by PA forms $\Omega_{PA}(-)$. 
If we have to distinguish these variants, we will use the notation $\Omega_G^{s,PL}(X)$, $\Omega_G^{s,PA}(X)$ and $\Omega_G^{s,sm}(X)$.
Furthermore, we will discuss below several simplifications of $\Omega_G^{s}$ under simplifying assumptions.
\end{rem}

Next, denote the connected component of the identity of $G$ by $G_0$.
We will be interested not in the $G$-equivariant forms, but in the $G_0$-equivariant forms, with an action of $G/G_0$.
Note that for $X$ a $G$-space $\Omega_{G_0}^s(X)$ carries an action of $G$, acting on all factors $G_0$ by the adjoint action and on $X$ from the left.
This action factors unfortunately does not factor readily through $G/G_0$.
On remedy is to consider the $G_0$-invariant subspace (cf. the next section).
This subspace is quasi-isomorphic in the smooth setting. However, we cannot show the corresponding quasi-isomorphism statement inb the PA setting, due to the pushforward not being defined on all PA forms, cf. section \ref{sec:PAforms}.
To work around, we will assume that we can pick a one-sided inverse $G/G_0\to G$ to the projection $G\to G/G_0$.
Then an action of $G/G_0$ on $\Omega_{G_0}^s(X)$ is defined.
More concretely, The (only) example we are interested in here is $G=O(n)$.
Then the map $\Z_2\cong G/G_0\to G$ can be easily realized by assigning the non-trivial element of $\Z_2$ a coordinate reflection.



\subsubsection{Invariant variant}\label{sec:invariant variant}
Let us also define the sub-dgca invariant under the $G$-actions "between the factors".\footnote{For example, on $*\times G\times M$ the group $G\times G$ would act as 
\[
(g_1,g_2) \cdot (h, m) = (g_1h g_2^{-1}, g_2 m).
\]
}
\[
\Omega_G^{'s,inv}(X) := \int_{[j]\in \Delta_+} \Omega_{PL}(G^j\times X)^{G^{\times j+1}} \otimes \Omega_{poly}(\Delta^j).
\]
It comes with a map from the ``invariant'' version of $B_G$:
\[
B_G^{s,inv} := \Omega_G^{'s,inv}(*).
\]
As above, to ensure freeness, we then define 
\beq{equ:invinclusion}
 \Omega_G^{s,inv}(X) := B_G^{s,inv} \bar \otimes_{B_G^{s,inv}} \Omega_G^{'s,inv}(X) \subset \Omega_G^{s}(X).
\eeq



%\beq{equ:invinclusion}
%\Omega_G^{s,inv}(X) := Tot \left( \Omega_{PL}(G^\bullet \times X )^{G^{\times \bullet+1} }  \right)\subset \Omega_G^{s}(M).
%\eeq

Let $G_0\subset G$ be the connected component of the identity.
Then $G$ acts on $\Omega_{G_0}^{s, inv}(X)$ diagonally, i.e., by the adjoint action on each $G$ and by simultaneously on $X$. We define the invariant subspace
\[
\Omega_{G_0,G}^{s, inv}(X) := \left( \Omega_{G_0}^{s, inv}(X) \right)^{G}.
\] 
In fact, the $G$-action clearly factors through $\pi_0(G)=G/G_0$ and we could replace $G$ above by $G/G_0$ if desired.
Again, we denote the smooth or PA variants of the above construction by a superscript ``$PA$'' or ``$sm$''.

% Under good conditions, the invariant version is also a model for the equivariant forms. 
% We shall show state this result only in the setting used here, namely the PA setting.
% \begin{lemma}
% If $G$ is a compact algebraic group acting on the semi-algebraic manifold $X$, then there is a natural isomorphism
% \[
% \Omega_{G_0,G}^{s, inv, PA}(X) \cong \Omega_G^{s, inv,PA}(X).
% \]
% \end{lemma}
% \begin{proof}
% Indeed, as $G$ splits into say $k$ connected components, the $M^G_j$ split into $k^j$ components. However, given the required invariance the PA form is completely determined by its value on the identity component, i.e., by an element of $\Omega_{G_0}^{sPA, inv}(M)$. Furthermore, to indeed yield an invariant form, the form in latter space has to be further invariant under the diagonal $G$ action, hence the statement is shown.
% \end{proof}

% In fact, we claim that the inclusion \eqref{equ:invinclusion} is a quasi-isomorphism, such that all three dgcas discussed are quasi-isomorphic.%, though we shall not use that fact here.
% \begin{lemma}
% %(TODO: This is not needed, remove?)
% The map 
% \[
% \Omega_{G}^{s, inv, PA}(X) \to \Omega_G^{s, PA}(X).
% \]
%  is a quasi-isomorphism of dgcas.
% \end{lemma}
% \begin{proof}[Proof sketch:]
% FOR US: This is essentially an application of the Lyndon-Hochschild-Serre spectral sequence, there should be a reference in the literature (TODO). Still, let me sketch the argument for completeness.
% 
% One takes a spectral sequence on the descending complete filtration by the cosimplicial degree on both sides.
% The $j$-th associated graded complex on the right has the form 
% \[
% \int_{[j] \in \Delta_+} \Omega_{PA}(M^G_j )  \otimes \Omega_{poly}^\partial(\Delta^j)
% \]
% where $\Omega_{poly}^\partial(\Delta^j)$ are forms vanishing on the boundary. (Slight lie here...)
% The cohomology is $(H(G)[-1])^{\otimes j}\otimes H(M)$. Altogether we find the co-Hochschild complex of the coalgebra $H(G)$ with coefficients in $H(M)$ on the $E^1$ page of our sequence on the right.
% On the left, we similarly find the $\pi_0$-invariant piece of the co-Hochschild complex of $H(G_0)$.
% The claim is that the map between these two Hochschild complexes is a quasi-isomorphism, thus proving the Lemma.
% 
% To see this take a further spectral sequence by the filtration on the "number of non-identity components". On the $E_1$ pages we find on the left the invariants of $\pi_0$-invariants of the Hochschild cohomology of $H(G_0)$ with coefficients in $H(M)$, say $Hoch(H(G_0),H(M))$. On the right we find on the $E_1$ page the complex computing the group cohomology of $\pi_0(G)$ with coefficients in $Hoch(H(G_0),H(M))$. However, since $\pi_0(G)$ is finite by assumption and we are working over a field of characteristic zero, that latter group cohomology is just the space of $\pi_0(G)$-invariants in $Hoch(H(G_0),H(M))$ as well, thus concluding the proof.
% \end{proof}


%Finally we may furthermore simplify each of these models by replacing $G$ by the normalizer $K\subset G$ of a maximal torus $T\subset G$.

\subsection{Recollections for compact Lie groups \texorpdfstring{$G$}{G}}\label{sec:compactGrecollection}
Now suppose that $G$ is a compact Lie group. 
For a compact Lie group $G$ we have that $H^\bullet(G)$ is a (strict) Hopf coalgebra, and furthermore
\beq{equ:HG}
 H^\bullet(G)\cong \K[\pi_0(G)]^* \otimes H(G_0)
\eeq
where $G_0$ is the connected component of the identity, the product is the standard commutative (``pointwise'') product of functions, and the coproduct is induced from the map $G\times G\to G$.
Concretely, $H(G_0)\cong \K[p_1,\dots, p_r]$ is a free commutative and cocommutative Hopf algebra in generators $p_1,\dots,p_r$, of odd degrees determined by the exponents of the Lie algebra of $G$, and $r$ is the rank. 
The group $\pi_0$ acts on $H(G_0)$ (by conjugation with arbitrary representatives), and the Hopf structure on \eqref{equ:HG} is such that the coproduct is ``twisted'' by this action.
Note also that $p_1,\dots, p_r$ can be identified with the (dual of the) generators of the rational homotopy groups of $G_0$.

An alternative characterization of the Hopf algebra $H(G)$ is that 
\[
 H(G) = \Omega(G)^{G_0\times G_0}.
\]

% There is an explicit right inverse to the inclusion $H(G)\subset \Omega(G)$ given by the symmetrization map
% \begin{gather*}
%  S_{G_0} \colon \Omega(G) \to H(G) \\
% \alpha \mapsto \int_{G_0\times G_0} dgdh l_g^* r_g^* \alpha,
% \end{gather*}
% where $dg$ denotes an invariant measure on $G_0$ of volume 1, and $l$ and $r$ denote the left and right multiplication maps.
% The Hopf algebra structure on $H(G)$ is then as follows: The product is the commutative product and the coproduct is given by the composition
% \[
%  H(G) \to \Omega(G) \to \Omega(G\times G) \xrightarrow{S^2_{G_0}} \Omega(G\times G)^{G_0\times G_0 \times G_0 \times G_0} \cong H(G)\otimes H(G),
% \]
% where now $S^2_{G_0}:\Omega(G\times G)\to \Omega(G\times G)^{G_0\times G_0 \times G_0 \times G_0}$ is the symmetrization operator with respect to the action of $G_0^{\times 4}$.
% (In fact, since the image is already symmetric under the action of $G_0^{\times 3}$, a symmetrization wrt. an action of one $G_0$ would suffice here.)
% 
% Note however that the symmetrization maps do not readily commute with the commutative products.

\begin{ex}
The most relevant case for our present paper is $G=\SO(n)$ or $G=O(n)$.
For $n=2k+1$ odd the rational homotopy groups of $\SO(n)$ are generated by the Pontryagin classes $p_{4j-1}$ in degree $4j-1$, for $j=1,2,\dots, k$. We have $H(BSO(n))=\K[p_{4},p_8,\dots, p_{4k}]$ where $p_{4j}$ are generators of degree $4j$, which we also refer to as Pontryagin classes.
The action of $\pi_0(O(n))=\Z_2$ on $H(B\SO(n))$ is trivial so that in particular $H(B\SO(n))=H(BO(n))$.

For $n=2k$ the rational homotopy groups of $\SO(n)$ are generated by the Pontryagin classes $p_{4j-1}$ in degree $4j-1$, for $j=1,2,\dots, k-1$, and the Euler class $e$ in degree $n-1$. We have $H(B\SO(n))=\K[p_{4},p_8,\dots, p_{4k-4}, E]$, where $E$ is of degree $n$.
The action of $\pi_0(O(n))=\Z_2$ on $H(B\SO(n))$ is trivial on the $p_{4j}$, but by sign on $E$. Hence $H(BO(n))=H(B\SO(n))^{\Z_2}=\K[p_{4},p_8,\dots, p_{4k-4}, E^2]$.

Finally, we have maps $\SO(n-1)\to \SO(n)$. The induced maps $H(B\SO(n))\to H(B\SO(n-1))$ are such that for even $n$ the Pontryagin classes are mapped to Pontryagin classes and the Euler class to 0. 
For odd $n$ the non-top Pontryagin classes are mapped to Pontryagin classes, while the the top Pontryagin class is mapped to the square of the Euler class, $p_{2n-2}\mapsto E^2$.
\end{ex}

\subsection{Recollection: Cartan model for equivariant forms}
Note that the space $\Omega_{PL}(X//G)$ is the cochain complex computing the $G$-equivariant cohomology of $X$.
Now suppose that we are working over $\K=\R$, $G$ is a compact (possibly not connected) Lie group with Lie algebra $\alg g$, and $X=M$ is a smooth manifold.
In this case it is known by a Theorem of H. Cartan that $\Omega_{PL}(X//G)$ is quasi-isomorphic to the Cartan model 
\[
\Omega_G^{Cartan}(M) :=(S (\alg g^*[-2]) \otimes \Omega(M))^G,
\]
with differential 
\beq{equ:dudef}
d_u = d + \sum_{j} u_j \iota_{e_j},
\eeq
where $d$ is the de Rham differential, $e_j$ range over a basis of $\alg g$ with dual basis $u_j\in \alg t^*$, and the last operator in the formula is the contraction with the corresponding vector field generating the action, cf. \cite[Theorem 21]{Libine}.
Furthermore, if $K\subset G$ is a compact subgroup with Lie algebra $\alg k$, then the inclusion into the $K$-invariants and restriction to $\alg k\subset \alg g$ induces a map 
\[
\Omega_G^{Cartan}(M)= (S (\alg g^*[-2 ]) \otimes \Omega(M))^G \to (S ( \alg k^*[-2]) \otimes \Omega(M))^K=\Omega_K^{Cartan}(M).
\]
This map is a quasi-isomorphism if $K$ is the normalizer of a maximal torus $T\subset G$, giving us a second model for the complex of equivariant differential forms.
We furthermore note that the above models of equivariant differential forms are functorial in $M$, and in particular from the map $M\to *$ we get maps 
\begin{align*}
H(BG) \cong \Omega_K^{Cartan}(*) &\to \Omega_K^{Cartan}(M) \\
H(BG) \cong \Omega_G^{Cartan}(*) &\to \Omega_G^{Cartan}(M)
\end{align*}
modeling the maps $M//G \to BG$.
Finally, note that using the notation here we can identify
\[
H(BG) = \K[u_1,\dots,u_r]^W
\]
where $W$ is the Weyl group and $r$ is the rank of $G$.

In each of these cases, replacing $G$ by the connected component of the identity $G_0$ the dgcas carry natural actions of $G/G_0$. Hence in the case of a compact Lie group $G$ and a manifold $M$, we can simplify the (real) dgca models for  the $G$-space $M$ as discussed in section \ref{sec:Gspacemodels}.

% \begin{rem}
% For non-connected $G$ one has to be careful in that $BG$ is not simply connected, and hence "modelling" should be understood in the coarse sense used above.
% \end{rem}

\subsection{Cartan model and PA setting: An unsatisfying ``hack''}\label{sec:PA Cartan model}
In the relevant situation for this paper $G$ is a compact Lie (algebraic) group, namely $G=O(n)$ or $G=\SO(n)$.
We would hence much prefer to work with the small Cartan models of the previous subsection, rather than the unwieldy simplicial models of subsection \ref{sec:simpl models BG}.
However, for technical reasons apparent later we are forced to work in the semi-algebraic setting, with PA forms instead of smooth \cite{HLTV}.
Unfortunately, for such forms the definition of the Cartan model does not readily carry over since the contraction operators $\iota_{e_j}$ of \eqref{equ:dudef} are a priori not defined on the PA forms.\footnote{There is, in fact, a candidate replacement for the contraction operator. 
Consider a semi algebraic action 
\[
\rho: S^1\times M\to M.
\]
The operator $\iota_{t}$ of ``contraction with the generating vector field'' may then be defined on \emph{$S^1$-invariant} PA forms $\Omega_{PA}(M)^{S^1}$ as the pullback-pushforward along $\rho$,
\[
\iota_t := \frac 1 {2\pi} \rho_* \rho^*.
\]  
Note in particular that this reduces to the standard contraction operator on the ($S^1$-invariant) smooth forms. The obvious generalization from $S^1$ to the torus could be used to define an equivariant Cartan model in the PA setting. 
However, first the pushforward is a priori not well defined on general PA forms according to \cite{HLTV}.
Secondly, verifying that this definition of the contraction operator satisfies the required properties is itself not trivial. Hence we leave the study of this approach to future work. }
We will hence resort to a workaround, that will allow us to work with small ``models'' in practice nevertheless, but is somewhat unsatisfying conceptually.

% In this section and the next we want to construct models for the $G$-equivariant PA forms on a semi-algebraic manifold $M$, for $G$ a compact algebraic group.
% We begin by describing a cosimplicial model.
% A simplicial-space model for the homotopy quotient of $M$ is the simplicial space
% \[
% M^G_\bullet := G^\times \bullet \times M.
% \]
% Here the simplicial maps are given by the group multiplication and by acting with $G$ on $M$ and (implicitly) on the point $*$.
% The spaces $M^G_k$ are all semi-algebraic manifolds, so may obtain a cosimplicial dgca model 
% \[
% \Omega_{PA}(M^G_\bullet ).
% \]
% The desired dgca model is simply its totalization
% \[
% \Omega_G^{sPA}(M) := Tot \Omega_{PA}(M^G_\bullet )=\int_{[j] \in \Delta} \Omega_{PA}(M^G_j )  \otimes \Omega_{poly}(\Delta^j).
% \]
% % Let us also define the sub-dgca invariant under the $G$-actions "between the factors".\footnote{For example, on $*\times G\times M$ the group $G\times G$ would act as 
% % \[
% % (g_1,g_2) \cdot (h, m) = (g_1h g_2^{-1}, g_2 m).
% % \]
% % }
% % \beq{equ:invinclusion}
% % \Omega_G^{sPA, inv}(M) := Tot \left( \Omega_{PA}(M^G_\bullet )^{G^{\times \bullet+1} }  \right)\subset \Omega_G^{sPA}(M).
% % \eeq
% % 
% % Let $G_0\subset G$ be the connected component of the identity.
% % Then $G$ acts on $\Omega_G^{sPA, inv}(M)$ diagonally, i.e., by the adjoint action on each $G$ and by simultaneously on $M$. We define the invariant subspace
% % \[
% % \Omega_{G_0,G}^{sPA}(M) := \left( \Omega_{G_0}^{sPA, inv}(M) \right)^{G}.
% % \] 
% % (In fact, we could here replace $G$ by $\pi_0(G)=G/G_0$ when taking the invariants.)
% % \begin{lemma}
% % There is a natural isomorphism
% % \[
% % \Omega_{G_0,G}^{sPA}(M) \cong \Omega_G^{sPA, inv}(M).
% % \]
% % \end{lemma}
% % \begin{proof}
% % Indeed, as $G$ splits into say $k$ connected components, the $M^G_j$ split into $k^j$ components. However, given the required invariance the PA form is completely determined by its value on the identity component, i.e., by an element of $\Omega_{G_0}^{sPA, inv}(M)$. Furthermore, to indeed yield an invariant form, the form in latter space has to be further invariant under the diagonal $G$ action, hence the statement is shown.
% % \end{proof}
% % 
% % In fact, we claim that the inclusion \eqref{equ:invinclusion} is a quasi-isomorphism, such that all three dgcas discussed are quasi-isomorphic, though we shall not use that fact here.
% % \begin{lemma}
% % (TODO: This is not needed, remove?)
% % The map \eqref{equ:invinclusion} is a quasi-isomorphism of dgcas.
% % \end{lemma}
% % \begin{proof}[Proof sketch:]
% % This is essentially an application of the Lyndon-Hochschild-Serre spectral sequence, there should be a reference in the literature. Still, let me sketch the argument for completeness.
% % 
% % One takes a spectral sequence on the descending complete filtration by the cosimplicial degree on both sides.
% % The $j$-th associated graded complex on the right has the form 
% % \[
% % \int_{[j] \in \Delta} \Omega_{PA}(M^G_j )  \otimes \Omega_{poly}^\partial(\Delta^j)
% % \]
% % where $\Omega_{poly}^\partial(\Delta^j)$ are forms vanishing on the boundary. (Slight lie here...)
% % The cohomology is $(H(G)[-1])^{\otimes j}\otimes H(M)$. Altogether we find the co-Hochschild complex of the coalgebra $H(G)$ with coefficients in $H(M)$ on the $E^1$ page of our sequence on the right.
% % On the left, we similarly find the $\pi_0$-invariant piece of the co-Hochschild complex of $H(G_0)$.
% % The claim is that the map between these two Hochschild complexes is a quasi-isomorphism, thus proving the Lemma.
% % 
% % To see this take a further spectral sequence by the filtration on the "number of non-identity components". On the $E_1$ pages we find on the left the invariants of $\pi_0$-invariants of the Hochschild cohomology of $H(G_0)$ with coefficients in $H(M)$, say $Hoch(H(G_0),H(M))$. On the right we find on the $E_1$ page the complex computing the group cohomology of $\pi_0(G)$ with coefficients in $Hoch(H(G_0),H(M))$. However, since $\pi_0(G)$ is finite by assumption and we are working over a field of characteristic zero, that latter group cohomology is just the space of $\pi_0(G)$-invariants in $Hoch(H(G_0),H(M))$ as well, thus concluding the proof.
% % \end{proof}
% % 
% % 
% % Finally we may furthermore simplify each of these models by replacing $G$ by the normalizer $K\subset G$ of a maximal torus $T\subset G$.
% 
% 
% \subsubsection{Construction: Link to Cartan model}
% It is not obvious how to write down a PA Cartan model since PA forms do not allow for the operation of contraction with a vector field.
% However, for our purposes, the following statements shall suffice.

To this end, suppose that $A\subset \Omega_{PA}(M)$ is a sub-dgca of smooth forms closed under the action of $K$ and under the contraction with the vector fields generating the $T$-action.
Then we define the dgca
\[
A_K := (S(\alg t^*[-2])\otimes A)^K
\]
with the differential \eqref{equ:dudef}.

The claim is that there is a map of dgcas
\begin{equation}\label{equ:Phi}
\Phi: (A_K, d_u)  \to \Omega_{K}^{s,PA}(M),
\end{equation}
using the notation of the previous subsection.
In fact, we will construct a map into the subspace
\[
 \Omega_{T,K}^{s, inv,PA}(M) \subset  \Omega_{K}^{s,PA}(M).
\]

%Let us first describe the construction of $\Phi$ schematically.
Pick a $\alg t$-valued $K$-invariant smooth connection $\eta$ on $ET$. 
Denote the components of $\eta$ by $\eta_1,\dots, \eta_r$ and consider them as degree 1 elements of 
\[
\int_{[j] \in \Delta} \Omega_{PA}(T^{ j+1}\times M)\otimes \Omega_{poly}(\Delta^j)
\]
by trivial extension to $M$. Then the 2-forms
$u_j':=d\eta_j$ are $K$-basic and represent (within the basic forms) the pullback of appropriate Euler classes on the classifying space $BS^1$ for the $j$-th circle action. 
For $\alpha\in A_K$ let $\Phi'(\alpha)$ be the form on the space above obtained by replacing each $u_j$ by $u_j'$.
Then we note that the form
\begin{equation}\label{equ:Phipdef}
\Phi(\alpha) := \left(\prod_{j=1}^r(1+ \eta_j \otimes \iota_{\xi_{j}} ) \right) \Phi'(\alpha)
\end{equation}
is $T$-basic and hence descends to (or is) a form on the quotient, and is furthermore $K$-invariant so that we obtain a form in $\Omega_{T,K}^{s,PA}(M)$ as desired.


Furthermore, note that picking $M=*$ (and $A=\R$) the above prescription realizes explicitly a quasi-isomorphism
\begin{equation}\label{equ:HBGtoOmega}
H(BG)\cong\R[u_1,\dots, u_r]^W \to \Omega(BG).
\end{equation}


% \begin{rem}
% On the Cartan model, there is a natural filtration ``by the number of $u$'s'' that is often helpful in computing the equivariant cohomology.
% In our context, we define the subspaces $\mF^p A_K:= (S^{\geq p}(\alg t^*[-2])\otimes A)^K$.
% Clearly, the differential on the associated graded is the piece $d$ of $d_u$.
% 
% Similarly, on our simplicial model $\Omega_{K}^{s,inv,PA}(M)$ there is a filtration by the total degree of forms on the various $\Omega_{poly}(\Delta^j)$ appearing in the end defining those forms. We define the subspace $\mF^p \Omega_{K}^{s,inv,PA}(M)$ by replacing $\Omega_{poly}(\Delta^j)$ by $\Omega_{poly}^{\geq p}(\Delta^j)$.
% The differential on $\Omega_{K}^{s,inv,PA}(M)$ splits as $d=d_1+d_2$, where $d_1$ is induced by the (PA) de Rham differential on the factors $\Omega_{PA}(G^{\times j}\times M)$ and $d_2$ acts similarly on $\Omega_{poly}(\Delta^j)$.
% Then the differential on the associated graded is given by $d_1$.
% The $E_1$ page in the associated spectral sequence is hence given by the 
% \end{rem}



%Note also that the map $\Phi$ is compatible with the inclusion of $H(BG)$. More concretely, we can identify $H(BG)$ with the Weyl group invariant polynomials
%\[
% H(BG)\cong \R[u_1,\dots, u_r]^W,
%\]
%and applying the 
%In particular, if we define $\Omega(BG):=\Omega_G(*)=\Omega_G(\FM_n(1))$, then we have a map 
%\begin{equation}\label{equ:HBGtoOmega}
%H(BG)\cong\R[u_1,\dots, u_n]^W \to \Omega(BG).
%\end{equation}

%
%
%%Concretely, as a model for the forms on $ET$ we may pick 
%%\[
%%\Omega_{T,K}^{sPA}()
%%\]
%
%
% We will construct the map $\Phi$ as a composition 
%\begin{multline*}
%A_K
%\stackrel{\Phi'}{\to} 
%\left(\int_{[j] \in \Delta} \Omega_{smooth}^{T-\text{basic}}(T^{ j+1}\times \FM_n)\otimes \Omega_{poly}(\Delta^j)\right)^W
%\\
%\cong 
%\left( \int_{[j] \in \Delta} \Omega_{smooth}(T^{ j}\times \FM_n)\otimes \Omega_{poly}(\Delta^j)\right)^W
%\to \Omega_G(\FM_n),
%\end{multline*}
%where $\Omega_{smooth}^{T-\text{basic}}(\cdots)$ denotes the smooth \emph{basic} forms, i.e., forms $\alpha$ that are $T$-invariant and in addition satisfy $\iota_{\xi_j}\alpha=0$ for $j=1,\dots,r$.
%Now the map $\Phi'$ is defined as follows.
%Denote the components of $\eta$ by $\eta_1,\dots, \eta_r$, consider them as degree 1 elements of 
%\[
%\left(  \int_{[j] \in \Delta} \Omega_{smooth}(T^{ j+1}\times \FM_n)\otimes \Omega_{poly}(\Delta^j)\right)^W
%\]
%by trivial extension to $\FM_n$. Then the 2-forms
%$u_j':=d\eta_j$ are basic and represent (within the basic forms) the pullback of appropriate Euler classes on the classifying space $BS^1$ for the $j$-th circle action. We define $\Phi'(u_j):=u_j'$. Furthermore, for a smooth $T$-invariant form $\alpha$ on $\FM_n$ we set 
%
%
%One easily checks that the right-hand side is indeed a basic form, and that the map $\Phi'$ commutes with the differentials. We will use the map $\Phi$ thus defined several times below.
%

%We note that the algebra $H(BG)$ may be identified with the $W$-invariant polynomials
%\[
% H(BG)\cong \R[u_1,\dots, u_n]^W
%\]
%and that we hence have a natural map of dgcas 
%\[
% H(BG) \to \Omega_G(\FM_n).
%\]
%In particular, if we define $\Omega(BG):=\Omega_G(*)=\Omega_G(\FM_n(1))$, then we have a map 
%\begin{equation}\label{equ:HBGtoOmega}
%H(BG)\cong\R[u_1,\dots, u_n]^W \to \Omega(BG).
%\end{equation}
%





% \begin{rem}
% FOR US: If we decide to prove the proposition in the next subsection, we could get rid of the present subsubsection.
% However, to keep a complete proof, I kept it for now.
% \end{rem}



%Conceptually, the precise model for the equivariant forms we use does not play a central role in the following arguments.
%However, technically, we are still somewhat restricted because the equivariant forms we use have to allow for a notion of push-forward or fiber integral along the projections $\FM_n(r+k) \to \FM_n(r)$. For this reason we will use PA forms described in \cite{HLTV}. Unfortunately, these forms do not naturally allow for the operation of contraction with a smooth vector field, so one cannot simply use a version of the Cartan model for equivariant forms built on PA forms.
%
%Thus we will use one of the following explicit models.
%\begin{itemize}
%\item We may consider the cosimpicial algebra
%\[
%\Omega_{PA}(G^{ \bullet}\times \FM_n(r))^{G}.
%\] 
%The realization is the dg commutative algebra given by the end
%\[
%\Omega_{G}^{nerve}(\FM_n(r)) 
%=
%\int_{[j] \in \Delta} \Omega_{PA}(G^{ j}\times \FM_n(r))^{G}\otimes \Omega_{poly}(\Delta^j)
%\subset \prod_{j} \Omega_{PA}(G^{ j}\times \FM_n(r))^{G}\otimes \Omega_{poly}(\Delta^j).
%\]
%\item An alternative model for the $G$-equivariant form which we will use are the $W\ltimes T$-equivariant differential forms, where $T$ is a maximal torus and $W$ is the Weyl group. Concretely, a model for these forms are the Weyl group invariant $T$-equivariant forms, i.e., 
%\[
%\Omega_{G}^{torus}(\FM_n(r))=\left(  \int_{[j] \in \Delta} \Omega_{PA}(T^{ j}\times \FM_n(r))^{T}\otimes \Omega_{poly}(\Delta^j)
%\right)^W.
%\]
%\end{itemize}
%
%The precise model of equivariant differential forms we use does in principle not matter. 
%However, for concreteness' sake, and to have an explicit as possible definition, we will use the smallest model
% \[
% \Omega_G(\FM_n) := 
% \Omega_{G}^{torus}
% \]
% in the following sections.
 
%Furthermore, let $\Omega_{smooth}(\cdots)\subset \Omega_{PA}(\cdots)$ be the smooth PA forms. 
%Consider $G$ as above with maximal torus $T$, rank $r=\mathit{dim}(T)$ and Weyl group $W$, and let $u_1,\dots, u_r$ be formal variables of degree 2.
%Let $\xi_1,\dots, \xi_r$ be the generating vector fields for the torus action, and let $\iota_{\xi_j}$ be the corresponding operators on $\Omega_{smooth}(\FM_n)$ given by contraction with these vector fields. We then define the Cartan algebra of equivariant forms 
%\[
%\Omega_G^{smooth}(\FM_n):= (\Omega_{smooth}(\FM_n)[u_1,\dots, u_r]^{W\ltimes T}, d_u),
%\]
%where the equivariant differential $d_u$ is defined as
%\[
% d_u = d + \sum_{j=1}^r u_j \iota_{\xi_j}.
%\]
%Furthermore, denote the Lie algebra of $T$ by $\alg t$, and pick a $\alg t$-valued $W$-equivariant smooth connection $\eta$ on $ET$. Then there is a map of dg commutative algebras
%\begin{equation}\label{equ:Phi}
%\Phi: (\Omega_{smooth}(\FM_n)[u_1,\dots, u_r]^{W\ltimes T}, d_u)  \to \Omega_G(\FM_n).
%\end{equation}
%We will construct this map as a composition 
%\begin{multline*}
%(\Omega_{smooth}(\FM_n)[u_1,\dots, u_r]^{W\ltimes T}, d_u)
%\stackrel{\Phi'}{\to} 
%\left(\int_{[j] \in \Delta} \Omega_{smooth}^{T-\text{basic}}(T^{ j+1}\times \FM_n)\otimes \Omega_{poly}(\Delta^j)\right)^W
%\\
%\cong 
%\left( \int_{[j] \in \Delta} \Omega_{smooth}(T^{ j}\times \FM_n)\otimes \Omega_{poly}(\Delta^j)\right)^W
%\to \Omega_G(\FM_n),
%\end{multline*}
%where $\Omega_{smooth}^{T-\text{basic}}(\cdots)$ denotes the smooth \emph{basic} forms, i.e., forms $\alpha$ that are $T$-invariant and in addition satisfy $\iota_{\xi_j}\alpha=0$ for $j=1,\dots,r$.
%Now the map $\Phi'$ is defined as follows.
%Denote the components of $\eta$ by $\eta_1,\dots, \eta_r$, consider them as degree 1 elements of 
%\[
%\left(  \int_{[j] \in \Delta} \Omega_{smooth}(T^{ j+1}\times \FM_n)\otimes \Omega_{poly}(\Delta^j)\right)^W
%\]
%by trivial extension to $\FM_n$. Then the 2-forms
%$u_j':=d\eta_j$ are basic and represent (within the basic forms) the pullback of appropriate Euler classes on the classifying space $BS^1$ for the $j$-th circle action. We define $\Phi'(u_j):=u_j'$. Furthermore, for a smooth $T$-invariant form $\alpha$ on $\FM_n$ we set 
%\begin{equation}\label{equ:Phipdef}
%\Phi'(\alpha) := \left(\prod_{j=1}^r(1+ \eta_j \otimes \iota_{\xi_{j}} ) \right) \alpha.
%\end{equation}
%
%One easily checks that the right-hand side is indeed a basic form, and that the map $\Phi'$ commutes with the differentials. We will use the map $\Phi$ thus defined several times below.
%
%We note that the algebra $H(BG)$ may be identified with the $W$-invariant polynomials
%\[
% H(BG)\cong \R[u_1,\dots, u_n]^W
%\]
%and that we hence have a natural map of dgcas 
%\[
% H(BG) \to \Omega_G(\FM_n).
%\]
%In particular, if we define $\Omega(BG):=\Omega_G(*)=\Omega_G(\FM_n(1))$, then we have a map 
%\begin{equation}\label{equ:HBGtoOmega}
%H(BG)\cong\R[u_1,\dots, u_n]^W \to \Omega(BG).
%\end{equation}
%



% \subsection{Equivariant PA forms}
% (TODO: Either we give complete proofs here or we should remove this section. Presently I assume the section will be removed... but kept it for now in case you have other preferences. )
% 
% 
% For the purposes of this paper we want to work not with smooth manifolds and smooth forms $\Omega(-)$ but rather with (smooth) semi-algebraic manifolds and the (quasi-isomorphic to $\Omega(-)$) PA forms $\Omega_{PA}$, cf. \cite{HLTV}.
% One would write down a Cartan model in this setting as well, by simply replacing the smooth forms $\Omega(M)$ above by PA forms $\Omega_{PA}(M)$. However, the difficulty here is that the contraction operator $\iota_\xi$ is generally not defined on PA forms.
% However, one can partially remedy as follows.
% Let again $G$ be a compact Lie group, $T$ a maximal torus with Lie algebra $\alg t$ and $K$ the normalizer of $T$. (Then $W:=K/T$ is the Weyl group of $G$.) 
% Let $t\in \alg t$ be the generator of an $S^1\subset T$. Consider the action 
% \[
% \rho: S^1\times M\to M.
% \]
% The contraction operator $\iota_{t}$ may then be defined on \emph{$T$-invariant} PA forms $\Omega_{PA}(M)^T$ as the pullback-push-forward along $\rho$,
% \[
% \iota_t := \frac {2\pi} \rho_* \rho^*.
% \]  
% Note in particular that this reduces to the standard contraction operator on the ($T$-invariant) smooth forms.
% Now since $T\subset K$ acts trivially on $\alg t$ we may define the equivariant Cartan model in the PA setting as the dgca
% \[
% \Omega_K^{PA, Cartan}(M) := (S(\alg t[-2])\otimes \Omega_{PA}(M))^K = (S(\alg t^*[-2])\otimes \Omega_{PA}(M)^T)^K,
% \]
% with differential 
% \[
% d + \sum_j t_j^* \iota_{t_j},
% \]
% where the $t_j$ range over a basis of $\alg t$.
% TODO: Show that this $\iota_{t_j}$ is indeed a derivation.
% 
% Again by functoriality we have a natural map 
% \[
% H(BG) = \Omega_K^{PA, Cartan}(*) \to \Omega_K^{PA, Cartan}(M). 
% \]
% Furthermore, there is the following Proposition:
% \begin{prop}
%  The dgca $\Omega_K^{PA, Cartan}(M)$ is a dgca model for $X//G$.
% \end{prop}
% \begin{proof}
% One constructs a map $\Omega_K^{PA, Cartan}(M)$ into $\Omega_{T,K}^{sPA}(M)$ similarly to the map from $A_K$ in the preceding subsection, and shows it is a quasi-isomorphism by invoking the spectral sequence on the filtration by simplicial degree. 
% 
% FOR US: Unfortunately, I do not know a more direct proof, by a comparison map between PA and smooth forms, although it should definitely exist: Roughly, one should copy the comparison of PA and PL forms from \cite{HLTV}, but replace the semi-algebraic and singular chains with a form of equivariant chains, as in \cite{GKM}.
% \end{proof}
% 
% 
% \begin{rem}
% Note that the alternative construction for the full Cartan model (i.e., using $G$ instead of $K$ and $\alg g$ instead of $\alg t$) fails since the action of the connected component $G_0\subset G$ on $\alg g$ is non-trivial.
% \end{rem}
% 



%\subsection{Complete cochain complexes}


%\section{Rational homotopy theory of G-spaces - to be merged /removed}\label{sec:actions}
%
%\subsection{Overview - What this section should roughly contain}
%We consider $G$-spaces, topological spaces $X$ with an action of the Lie group $G$ (say reductive).
%We assume first that $G$ is connected. We denote the classifying space of $G$ by $BG$ and the universal bundle by $EG$ as usual.
%The homotopy quotient is denoted by $X/G:=EG \times_G X$.
%There is a canonical map $X/G\to BG$ classifying the principal $G$ bundle $X\simeq EG\times X \to X/G$.
%We have a homotpy pullback square 
%\[
% \begin{tikzcd}
%  X \ar{d} \ar{r} & EG \ar{d}\\
%  X/G  \ar{r} & BG 
% \end{tikzcd}.
%\]
%In particular, this shows how the homotopy type of the $G$-space $X$ may be recovered from that of $X/G$ and the map $X/G\to BG$.
%
%Now let us consider the algebraic version of the above setup. We generally work over the ground field $\R$.
%Deviating from the standard notation we write $\alg g:=\pi(G)\otimes_{\mathbb{Z}}\R$. (Note the $\alg g$ is \emph{not} the Lie algebra of $G$.) 
%We will consider $\alg g$ as an abelian Lie algebra.
%
%A real Sullivan model for the topological group $G$ is the Hopf algebra 
%\[
% \La_{\alg g} := S(\alg g^*).
%\]
%A real model for $BG$ is given by 
%\[
% S_{\alg g} = S(\alg g^*[-1]) = H^\bullet(BG).
%\]
%A model for the $G$-space $EG$ is the Koszul complex 
%\[
% K_{\alg g} := (\La_{\alg g} \otimes S_{\alg g}, d)
%\]
%where the differential $d$ is obtained by taking a coproduct in $\La_{\alg g}$, projecting one tensor factor to generators, and multiplying into $S_{\alg g}$.
%
%A dgca model for $X/G$ (in case $X$ is a manifold or space with a good notion of differential forms) are the equivariant differential forms $\Omega_G(X)$.
%They come with a natural map $S_{\alg g}\to \Omega_G(X)$ modelling the canonical map $X/G\to BG$.
%In general, we will denote by $M_{X/G}$ some model for $X/G$ to have a little flexibility below.
%
%A model $M_X$ for $X$ as $G$-space can be obtained from the homotopy pushout square dual to the pullback square above.
%\[
%  \begin{tikzcd}
%  S_{\alg g} \ar{d} \ar{r} & M_{X/G} \ar{d}\\
%  K_{\alg g}  \ar{r} & M_X
% \end{tikzcd}.
%\]
%Concretely, $M_X= K_{\alg g} \otimes_{S_{\alg g}} M_{X/G}$ is obtained by taking a relative tensor product with the Koszul complex, and the $\La_{\alg g}$-coaction is obtained from that on $K_{\alg g}$.
%
%Summarizing, we can model the $G$-space $X$ algebraically in one of the following two ways:
%\begin{itemize}
% \item As a dgca $M_{X}$ with a compatible coaction of the Hopf algebra $\La_{\alg g}$.
% \item As a dgca $M_{X/G}$ with a dgca map $S_{\alg g}\to M_{X/G}$.
%\end{itemize}
%We can pass from the second to the first representation by the functor $K_{\alg g} \otimes_{S_{\alg g}} -$, and vice versa by taking a Chevalley complex with values in the comodule $M_{X}$.
%
%We can hence represent $G$-spaces algebraically as (i) differential graded commutative algebras (dgcas) with a $\La_{\alg g}$ coaction, or (ii) dgca's under $S_{\alg g}$. Note that the first category carries a natural symmetric monoidal structure given by the tensor product. In case (ii) the category carries a product by the homotopy tensor product relative to $S_{\alg g}$. In practice, we will work with the sub-category of dgca's under $S_{\alg g}$ which are free as $S_{\alg g}$ modules. Within that category the monoidal structure may be defined as the ordinary $S_{\alg g}$-relative tensor product.
%
%In both cases the monoidal structure allows us to talk about (co)operads or (co)algebras in these categories. The functors above are symmetric monoidal and hence we can use them to pass between the respective categories of (co)operads.
%
%Note also that $\La_{\alg g}$ is the co-enveloping algebra of the abelian Lie coalgebra $\alg g^*$. A $\La_{\alg g}$ coaction is hence equivalent to a $\alg g$-action. Homotopy equivalently, we may hence consider dgca's with an $L_\infty$-$\alg g$-action. Still equivalently, we may consider dgca's $A$ with an $L_\infty$ map
%\[
% \alg g\to \Der(A).
%\]
%A cooperad in such objects is a Hopf cooperad with a compatible $\alg g$ action.
%Let us state this result for later reference:
%\begin{lemma}
% An algebraic model for operads in $G$-spaces are Hopf cooperads $X$ together with an $L_\infty$ map
% \[
%  \alg g\to \BiDer(X).
% \] 
%
%\end{lemma}
%Note furthermore that such an $L_\infty$ map may alternatively be described as a Maurer-Cartan element in 
%\[
%  \BiDer(X)\otimes S_{\alg g} =\BiDer(X)\otimes H^{\bullet}(BG).
%\]
%
%% \begin{rem}
%% Let us briefly explain how to obtain from an $L_\infty$ map such as the above the corresponding cooperads in any of the aforementioned two categories (dgcas with a $\La_{\alg g}$-coaction or dgcas over $S_{\alg g}$). 
%% \end{rem}
%
%
%Now to an operad in $G$-spaces $\op T$ we may associate the $G$-framed operad
%\[
% \op T \circ G.
%\]
%The algebraic analog of this construction is well defined and (I think) yields a model for $\op T \circ G$ (we will have to show this carefully)
%\[
% M_{\op T \circ G} = M_{\op T} \circ \La_{\alg g}.
%\]
%
%\begin{rem}
%Suppose we are given an $L_\infty$ map $\alg g\to \BiDer(X)$, for $X$ a Hopf cooperad.
%This does then not immediately describe a $\hat \La_{\alg g}$-coaction on $X$. Rather, we have to consider the canonical free resolution of $\alg g$ given by the bar-cobar construction
%\[
% \hat{\alg g}=\Omega(B(\alg g)).
%\]
%Concretely $\hat{\alg g}=(\FreeLie( (S^+ \alg g[1])[-1]) , D)$ is the free Lie algebra generated by the cofree non-counital cocommutative coalgebra cogenerated by $\alg g$, with a suitable differential.
%The data of an $L_\infty$-map $\alg g\to \BiDer(X)$ is equivalent to the data of a dg Lie algebra map $\hat{\alg g}\to \BiDer(X)$, i.e., an action of $\hat{\alg g}$ on $X$. Dually, we obtain a coaction on $X$ of the co-enveloping algebra 
%\begin{equation}\label{equ:hatLaDef}
%\hat \La_{\alg g} := \mU \hat{\alg g}^*.
%\end{equation}
%
%In this situation we shall build a model for the framed operad associated to $X$ as 
%\[
% X\circ \hat \La_{\alg g}.
%\]
%\end{rem}
%
%% TODO: What if G is not simply connected? Version of rational homotopy theory?
%
%
%\subsection{What we need from this section below}
%For the intended applications below we need the following result.
%I state it as ``desired'' because there are a few technical points to be checked, which I suppressed in the text above.
%\begin{thm}[(Desired)]\label{thm:equiv model to framed}
% Suppose that $M$ is a dg Hopf cooperad under $H(BG)$ quasi-isomorphic to $\Omega_G(\FM_n)$ (equivariant PA forms), then the ``framed model'' associated to $M$ under the above algebraic equivalences is indeed a model for the framed operad $\FM_n\circ G$.
%\end{thm}



%\subsection{Lie group actions in topology and algebra}
%\subsection{Algebras and operads in G-spaces}
%\subsection{Equivariant homology and Koszul duality}
%\subsection{Algebraic generalizations for modules over Hopf algebras}
%\subsection{The framed operad}

\subsection{Strictifying: A model for the Hopf algebra associated to a topological group}
Let $G$ be a topological group (or monoid). Morally, the dg commutative algebra $\Omega_{PL}(G)$ is (or ``wants to be'') a Hopf algebra, the coproduct being given by the pullback of the composition $G\times G\to G$. However, because the functor $\Omega_{PL}$ is not monoidal, $\Omega_{PL}(G)$ is a dg Hopf algebra only up to homotopy.

To make this precise we apply the construction of section \ref{sec:homotopy operads} to $G$, considered as a topological operad with only unary operations.
Hence $G$ gives rise to a homotopy cooperad $\Omega_{PL}(G)$ in the category of dgcas, with only unary cooperations.
We will consider the complete bialgebra
\[
A_G := W(\Omega_{PL}(G))
\]
or any dg commutative bialgebra quasi-isomorphic to $A_G$,
as an algebraic model of the topological group\footnote{Or rather, for the topological monoid, since we disregard here the inverse of the group.} $G$.

% Note that there is a natural map of dgca's
% \beq{equ:AGtoG}
% A_G \to \Omega_{PL}(G).
% \eeq
% The map is not a map of Hopf algebras as the right hand side is not a Hopf algebra.
% However, it is compatible with the Hopf structure up to homotopy, though we do not want to make this statement precise, and just take $A_G$ as the ad hoc definition of the Hopf algebra model of $G$. Yet, we can show the following result.
% 
% \begin{prop}\label{prop:qisoresG}
% The map \eqref{equ:AGtoG} is a quasi-isomorphism.
% \end{prop}
% \begin{proof}
% We have to compute the cohomology of $A_G= W_{\dgca}(\Omega_{PL}(G))$.
% There is the forgetful functor $\dgca\to dg\Vect$. In $dg\Vect$ we may consider the (3 dimensional) interval object 
% \[
% V = \K t \oplus \K(1-t) \oplus \K dt \subset \Omega_{poly}(I),
% \]
% and the inclusion $V\subset \Omega_{poly}(I)$ induces a map
% \[
% W_{dg\Vect}(\Omega_{PL}(G)) \to W_{\dgca}(\Omega_{PL}(G)) =A_G.
% \]
% This map is easily seen to be a quasi-isomorphism, by noting that 
% \[
%  \Omega_{poly}(I) = V \oplus \tilde V,
%  \]
%  where $\tilde V\subset \Omega_{poly}(I)$ is an acyclic subcomplex consisting of forms vanishing at both endpoints (and having zero volume).
% Hence it suffices to show that the composite map of complexes
% \beq{equ:WdgWdgcaG}
% W_{dg\Vect}(\Omega_{PL}(G)) \to W_{\dgca}(\Omega_{PL}(G)) \to \Omega_{PL}(G)
% \eeq
% is a quasi-isomorphism. 
%  
% Next note that the complex $W_{dg\Vect}(\Omega_{PL}(G))$ (as well as $W_{\dgca}(\Omega_{PL}(G))$ is a first quadrant double complex. The differential splits as 
% \[
% d = d_\Delta + d_\Omega,
% \]
% where $d_\Omega$ is induced by the differential on $\Omega_{PL}(\cdots)$ and  $d_\Delta$ originates from the differential on the factors $\Omega_{poly}(I)$. We consider the spectral sequence such that the first differential is $d_\Omega$. Provided the finite type assumption $H(G)$ is a coalgebra and the $E_1$ page is identified with its bar-cobar resolution. (This also uses that the maps $\Omega(X)\otimes\Omega(Y)\to \Omega(X\times Y)$ are injections.)
% Hence on the $E_2$-page the map \eqref{equ:WdgWdgcaG} induces an isomorphism and is hence a quasi-isomorphism to begin with by standard spectral sequence results.
% \end{proof}

\subsection{Hopf Formality for compact Lie groups}

Let us also note the following fact.

\begin{prop}\label{prop:Hopf formality}
Let $G$ be a compact Lie group.
Then the Hopf algebra $A_G$ is formal, i.e., it is weakly equivalent to the Hopf algebra $H(G)$.
\end{prop}
\begin{proof}[Proof sketch]
We will show the result by an obstruction theoretic argument. 

As recalled in section \ref{sec:compactGrecollection} the Hopf algebra $H(G)$ in this case is a semidirect product of the dual of the group ring of $\pi_0(G)$ with the commutative and cocommutative polynomial algebra $H(G_0)=\K[p_1,\dots,p_r]$.
Put differently, suppose that the cardinality of $\pi_0(G)$ is $n$, and that the elements have been numbered.
Then $H(G)$ can be identified as a graded commutative algebra 
\[
\K[p_{i,j} \text{ for } i=1,\dots,r \text{ and } j=1,\dots,n]/\langle p_{i,j}p_{i',j'}=0 \text{ if $j\neq j'$} \rangle .
\]
Here $p_{i,j}$ is represented by forms supported on the $j$-th connected component of $G$.
Let us construct a quasi-isomorphism $H(G) \to A_G$.
Note first that $A_G$ also naturally splits into a direct sum of $n$ isomorphic dg vector spaces, according to the $n$ connected components of $W$-construction of $G$.
We construct the desired map $f:H(G) \to A_G$ by specifying the images of each $p_{i,j}$, such that it lands in the $j$-th such component. For any such choice of $f(p_{i,j})$ the resulting map $f$ will be a map of dgcas.
Now we construct $f(p_{i,j})$ inductively, using the filtration on $A_G$.
I.e., in the first step of the induction we determine $f(p_{i,j})$ up to $\mF^2 A_G$ by picking arbitrary closed representatives in $\mF^1 A_G/\mF^2 A_G\cong \Omega(G)$.
At the $p$-th step of the induction we extend our choice of $f(p_{i,j})$ up to $\mF^p A_G$ to a choice up to $\mF^{p+1} A_G$, such that (i) the elements are closed and (ii) the cocompositions agree modulo elements in $\mF^{p+1} A_G$.

Concretely, the problem to be solved at the $p$-th induction step translates into the following:
We have to prescribe the value of $f(p_{i,j})$ on a $(p-1)$-cube, with values in $\Omega(G^p)$, i.e., we have to provide an element of 
\[
\Omega_{poly}([0,1]^{p-1})\otimes \Omega(G_0^p).
\]
By previous induction steps the value on the various boundary faces is given, by forms closed on those boundary faces, such that the top components in $\Omega(G^p)$ represent the same cohomology class, namely the $p-1$-fold coproduct of $p_{i,j}$. (TODO:make precise) Our task is to extend the form to the interior.
The obstruction for this to be possible lives in
\[
H^{|p_{i,j}|+1}(\Delta^{p-1}\times G_0^p, \partial\Delta^{p-1}\times G_0^p)\cong H^{|p_{i,j}|+p+1}(G_0^p).
\]
At each stage of the induction we have choices parametrized by the same space, in one less degree (up to exact forms).
Looking at how the choice affect potential obstructions at a one later stage, we find that the obstructions can removed, except for those taking values in 
\[
H^{|p_{i,j}|+2}(BH(G))
\]
where here $B$ is the bar construction of coassociative coalgebras, But now $H(BH(G))=H(BG)$ is concentrated in even degrees, while $|p_{i,j}|$ is odd, so no obstruction remains.

TODO: This argument should be cleaned up.

\end{proof}


\begin{rem}
FOR US: In the slang we use here "Hopf something" always means "something object in dgca". This is slightly abusive as Hopf algebra here means bialgebra actually, and not a true Hopf algebra in the conventional sense. (I.e., we never discuss antipodes.)
\end{rem}




%Then we can construct from $X$ a Hopf $A_G$-comodule $\mod_{A_G}(X)=W(X)$ by the $W$-construction comodules.
%Concretely, the result will be an object in in $\hdgca$, defined such that 
%\[
%\mod_{A_G}(X) = \int_{[j]\in \Delta_+} A_G \otimes \Omega_{PL}(G^j\times X) \otimes \Omega_{poly}(\Delta^j).
%\]
%(TODO: describe in more detail)

% \begin{rem}
% FOR US: The above model is not the smallest possible, but choosing the above definition simplifies some statements in the next subsections.
% \end{rem}


\subsection{The comodule model revisited: Monoidality}
Recall the notion of comodule model for a $G$-space from section \ref{sec:comodule model}.
It is a homotopy comodule over the homotopy Hopf coalgebra $\Omega_{PL}(G)$.
Unfortunately, we presently do not know a good symmetric monoidal structure on the category of such comodules.\footnote{There is a natural candidate: For two homotopy comodules $A$ and $B$ we would like to define $A\otimes B$ such that on a string $T$ of length $k+1$ we have 
\[
 (A\otimes B)(T)= A(T) \bar \otimes_{\Omega(G^k)} B(T).
\]
However, this is construction is a priori not symmetric, at least as long as we are unwilling to relax the notion of a symmetric monoidal structure to an ``up-to-homotopy'' version.}

Hence, in the context operads, we will mostly work with the strictified version of the comodule model.
We apply the W construction to obtain from the homotopy $\Omega_{PL}(G)$-comodule $\Omega_{PL}(X)$ an (honest) $A_G$ comodule 
\[
 \mod_{A_G}(X) = W(\Omega_{PL}(X)).
\]
We remind the reader again that this comodule lives in a category of complete vector spaces with completed tensor product as the monoidal structure.
Fortunately, $A_G$ is commutative and hence the $A_G$ comodules naturally form a symmetric monoidal category.

\begin{lemma}
 The functor $\mod_{A_G}$ is homotopically symmetric monoidal.
\end{lemma}
\begin{proof}
 Given spaces $X$ and $Y$ we have to produce a quasi-isomorphism 
\beq{equ:desired mon t}
 \mod_{A_G}(X)\otimes \mod_{A_G}( Y) \to \mod_{A_G}(X\times Y).
\eeq
The elements of each dg vector space $\mod_{A_G}(Z)$ are collections of forms in 
\[
 \Omega(G^j\times Z) \otimes \Omega(I^j)
\]
for $j=0,1,2,\dots$, satisfying suitable boundary conditions on the boundaries of the cube $I^j$.
The map \eqref{equ:desired mon t} is then induced by the obvious (multiplication) map
\[
 \left( \Omega(G^j\times X) \otimes \Omega(I^j) \right)\otimes\left( \Omega(G^j\times Y) \otimes \Omega(I^j) \right)
\to 
\left( \Omega(G^j\times X\times Y) \otimes \Omega(I^j) \right).
\]
One checks easily that the required boundary conditions for the image hold if they hold for the factors on the left.
Since multiplication is commutative the symmetry in $X$ and $Y$ is preserved.
Finally, the induced map in cohomology is the multiplication
\[
 H(X)\otimes H(Y) \to H(X\times Y)
\]
and is an isomorphism. 
\end{proof}





\subsection{Recovering the space from the quotient and passing between the equivariant and comodule models}\label{sec:space from quotient}
The homotopy type of the $G$-space $X$ may be recovered from the homotopy quotient $X//G$ and the map $X//G\to BG$ via the homotopy pullback square.
\[
 \begin{tikzcd}
  X \ar{d} \ar{r} & EG \ar{d}\\
  X//G  \ar{r} & BG 
 \end{tikzcd}.
\]
Using Theorem \ref{thm:from hess} one sees that one may dually recover the model for $X$ from the model for $X//G$ by a homotopy pushout, if $BG$ is simply connected, i.e., if $G$ is connected.
Assume first that $G$ is indeed connected.
We introduce the "Koszul complex dgca"
\[
K_G:=  \int_{[j]\in \Delta_+} A_G \otimes \Omega_{PL}(G^j) \otimes \Omega_{poly}(\Delta^j).
\]
It is an $A_G$ Hopf comodule which models $EG$, and in particular comes with a natural map $B_G\to K_G$. It can be used to recover the model of $X$ from the equivariant model as follows.
\begin{prop}\label{prop:space from quotient}
Let $X$ be a $G$-space, with $G$ connected. Then the $A_G$-Hopf comodules $\mod_{A_G}(X)$ and 
\[
K_G \otimes_{B_G} \Omega_G^s(X)
\]  
are quasi-isomorphic.
%, where $\bar \otimes_{B_G}$ denotes the derived tensor product, defined such that for two dgcas $M,N$ with maps $B_G\to M$, $B_G\to N$ we have 
%\[
%M\bar \otimes_{B_G} N = \bigoplus_{j\geq 0} M\otimes B_G^j \otimes N,
%\]
%with the commutative product defined using the shuffle product.
\end{prop}
We note that here we may use the ordinary tensor product instead of the derived one because we defined the equivariant forms already including a resolution, see section \ref{sec:simpl models BG}.
\begin{proof}
We will construct a zigzag between the two functors $\mod_{A_G}(-)$ and $K_G \otimes_{B_G} \Omega_G^s(-)$.
To do that we consider, for $X$ a $G$-space, a resolution 
\[
 \hat X := \int^{[j]} G^{\times j+1} \times X \times \Delta^j
\]
of the $G$-space $X$. Then via the map $\hat X\to X$ we have a quasi-isomorphism
\[
 \mod_{A_G}(X) \to \mod_{A_G}(\hat X).
\]
Furthermore, note that $K_G$ is (essentially) the space of forms on $\mod_{A_G}(\hat *)$. Hence, via the map $X\to *$ we obtain a map
\[
 K_G\to \mod_{A_G}(\hat X).
\]
Next, using the map $G^{\times j+1}\to *$ one can construct the morphism
\[
 \Omega_G^{'s}(X) \to \mod_{A_G}(\hat X).
\]
The latter two morphisms are compatible with the maps from $B_G^s$, and hence we obtain the desired zigzag
\[
 K_G\otimes_{B_G^s}\Omega_G^{s}(X)\to  K_G\otimes_{B_G^s} \Omega_G^{'s}(X) \to \mod_{A_G}(\hat X) \leftarrow \mod_{A_G}(X).
\]

All dgcas here have cohomology $H(X)$, and one checks that the morphisms induce the identity map on cohomology.
Furthermore, the construction is evidently functorial in $X$.
% The map
% \[
% %K\bar \otimes_{B_G} \mod_G(X) \to 
% K_G \otimes_{B_G} \Omega_G^s(X)
% \to \mod_{A_G}(X)
% \]
% is a map of $A_G$ Hopf comodules (and in particular dgca's). Computing the cohomology of the left-hand side we see that the composition induces an isomorphism on cohomology.
\end{proof}

Furthermore, we need the following result later:
\begin{lemma}\label{lem:module from equiv monoidal}
The functor 
\[
(K_G \otimes_{B_G} -) \colon B_G/\dgca \to A_G-mod^c
\]
is homotopically symmetric monoidal, and the quasi-isomorphism of functors constructed in (the proof of) Proposition \ref{prop:space from quotient} respects the (lax) symmetric monoidality.
\end{lemma}
\begin{proof}
 The monoidal structure is given by the natural morphism 
\[
(K_G \otimes_{B_G} M) \otimes (K \otimes_{B_G} N) \to K\bar \otimes_{B_G} (M\otimes_{B_G} N),
\]
for $M,N$ objects in $B_G/\dgca$, using the commutative product on $K$.

Next consider the second statement, and recall the zigzag in the proof of Proposition \ref{prop:space from quotient}.
Note that the functor $\hat{(-)}$ is oplax symmetric monoidal, via the map $\widehat{X\times Y}\to \hat X\times \hat Y$ (using the diagonal on $G$).
It follows that the functor $\mod_{A_G}(\hat {(-)})$ is lax symmetric monoidal. Furthermore, the natural transformation 
$\mod_{A_G}(\hat {(-)}) \leftarrow \mod_{A_G}(\hat {(-)})$
respects the symmetric monoidal structures.

Finally, we claim that the natural transformation $(K_G \otimes_{B_G} -)\to \mod_{A_G}(\hat {(-)})$ respects the symmetric monoidal structures.
Indeed, unpacking the definitions one verifies that the diagram
\[
 \begin{tikzcd}
  (K_G \otimes_{B_G} \Omega_G^s(X))\otimes (K_G \otimes_{B_G} \Omega_G^s(X)) \ar{d} \ar{r}
  & \mod_{A_G}(\hat X)\otimes \mod_{A_G}(\hat Y) \ar{d} 
  \\
  K_G \otimes_{B_G} \Omega_G^s(X) \otimes_{B_G} \Omega_G^s(Y) \ar{d}
  &
  \mod_{A_G}(\hat X\times \hat Y) \ar{d}
  \\
  K_G \otimes_{B_G} \Omega_G^s(X\times Y)\ar{r}
  &
  \mod_{A_G}(\widehat{X\times Y})
 \end{tikzcd}
\]
commutes.

\end{proof}


\subsubsection{Non-connected variant}
Suppose now that $G$ is non-connected with connected component of the identity $G_0\subset G$.
Let us in fact assume that $G$ is a compact Lie group. (This is a stronger assumption than necessary, but includes the sole case we care about here, $G=O(n)$.)
Then, as noted in section \ref{sec:simpl models BG}, the ``correct'' equivariant model for a $G$-space $X$ is not the $G$-equivariant forms on $X$, but the $G_0$-equivariant forms on $X$ wrt. $G_0$, together with an action of the finite group $G/G_0$.
Here we use the ``invariant'' models from section \ref{sec:invariant variant} to have a manifest action of $G/G_0$ (rather than just $G$).
As in section \ref{sec:invariant variant} we have the map of dgcas and $G/G_0$ modules
\[
 B_{G_0}^{s,inv} \to \Omega_{G_0}^{s,inv}(X).
\]
We define the version of the ``Koszul complex''
\[
K_{G,G_0}:=  \int_{[j]\in \Delta_+} \left(A_G \otimes \Omega_{PL}(G_0^j)\right)^{G_0^{j+1}} \otimes \Omega_{poly}(\Delta^j).
\]
It comes with an $A_G$-coaction, a map from $B_{G_0}^{inv}$ and an action of $G/G_0$.
Topologically this models $G//G_0$ which comes with an action of $G$ a map (fibration) to $BG_0$, and an action of $G/G_0$.\footnote{To exhibit the $G/G_0$-action, realize $G//G_0$ as  
\[
 G \times_{G_0} EG.
\]
On this space $h\in G$ acts as
\[
 h\cdot (g,x) = (gh^{-1},hx).
\]
The action evidently factors through $G/G_0$. The $G/G_0$ action is free and the quotient is $G$.
}
We may now recover the comodule model from the equivariant model in this setting as follows.
\begin{prop}\label{prop:space from quotient inv}
 Let $X$ be a $G$-space with $G$ a compact Lie group with connected component of the identity $G_0\subset G$.
 Then the $A_G$ Hopf comodules $\mod_{A_G}(X)$ and 
 \[
  \left( K_{G,G_0} \otimes_{B_{G_0}^{s,inv}}  \Omega_{G_0}^{s,inv}(X) \right)^{G/G_0}
 \]
are quasi-isomorphic.
\end{prop}
\begin{proof}
 One replaces the functors in the proof of Proposition \ref{prop:space from quotient} by their ``invariant'' counterparts to 
 construct a zigzag.
\end{proof}

Finally let us note:
\begin{lemma}\label{lem:module from equiv monoidal inv}
The functor 
\[
(K_{G,G_0} \otimes_{B_{G_0}^{s,inv}} -)^{G/G_0} %\colon B_G/\dgca \to A_G-mod^c
\]
is homotopically monoidal. Furthermore, the quasi-isomorphism of functors leading to Proposition \ref{prop:space from quotient inv} respects the symmetric monoidal structures.
\end{lemma}
\begin{proof}
 Again replace all objects in the proof of Lemma \ref{lem:module from equiv monoidal} by their invariant versions.
\end{proof}



\newcommand{\HK}{\tilde K}
\subsubsection{Simplification for compact Lie groups }\label{sec:Koszul cx for compact Lie}
Let now $G$ be a compact Lie group and $G_0\subset G$ be the connected component of the identity.
We define the $H(G)$ Hopf comodule 
$$
\HK = H(G) \otimes H(BG_0)
$$
and endow it with the Koszul differential. Concretely, using Sweedler notation for the coproduct on $H(G)$ and denoting by $\pi:H(G)\to \pi_{\R}(G)^*$ the projection to cogenerators, and 
by $\iota:\pi_{\R}(G)^*[-1]\to H(BG)$ the inclusion of (commutative algebra) generators differential is
$$
d(\alpha\otimes \beta) = \sum \alpha' \otimes \pi(\alpha'')\beta.
$$
By standard Koszul duality theory we have
\[
 H(\HK,d) = \K[G/G_0]^*.
\]
Furthermore, $\HK$ comes equipped with an action of $G/G_0$ induced by the action of $G$ on $G$ by right multiplication, and that on $G_0$ through the adjoint action.
\begin{prop}
The quasi-isomorphism of (homotopy) Hopf algebras $H(G)\to A_G$ from Proposition \ref{prop:Hopf formality} may be extended to a quasi-isomorphism
\[
 f : \HK \to K
\]
compatible with all algebraic structures, i.e.:
\begin{itemize}
 \item $f$ is a map of dg commutative algebras.
 \item There is a quasi-isomorphism of dgcas $H(G)\to B_G$ such that the diagram
\beq{equ:KtK1}
 \begin{tikzcd}
  H(G) \ar{r} \ar{d} & \HK \ar{d}{f} \\
B_G \ar{r} & K
 \end{tikzcd}
\eeq
commutes.
Furthermore, the maps can be chosen compatibly with the $G$ actions (factoring through $G/G_0$ in the top row) on all objects.
 \item $f$ intertwines the $H(G)$ coaction on the left and the $A_G$ coaction on the right (using the map $H(G)\to A_G$ from Proposition \ref{prop:Hopf formality}).
\end{itemize}
\end{prop}
\begin{proof}
This follows from an obstruction theoretic argument akin to the proof of Proposition \ref{prop:Hopf formality}.

TODO: Expand this.
\end{proof}




\section{Dgca models for operads in \texorpdfstring{$G$}{G}-spaces, and the framed operad}\label{sec:dgca models operads}

\subsection{Dgca models of operads in \texorpdfstring{$G$}{G}-spaces}\label{sec:op mod in G space}
Let $G$ be again a topological group, with $G_0\subset G$ the connected component of the identity. (In fact, for this paper, we only care about $G=O(n)$, $G/G_0=\mathbb{Z}_2$.)
Let $\op T$ be an operad in $G$-spaces.
Using the two types of models for $G$-spaces (the equivariant and comodule model) we may define two types of dgca model for $\op T$.
First, the equivariant forms functor $\Omega_G^s(-)$ is homotopically comonoidal.
Hence the collection $\mod_G(\op T)$ is a homotopy cooperad in the category of dgcas under $B_{G_0}$, with a compatible action of $G/G_0$.

\begin{defi}\label{def:equivariant model}
An \emph{equivariant model} for the operad in $G$-spaces $\op T$ is a pair consisting of (i) a dgca $B$ with an action of $G/G_0$ and (ii) a homotopy cooperad in the category of dgcas under $B$, with a compatible action of $G/G_0$, quasi-isomorphic to the pair $(B_G^s,\Omega_G^s(\op T))$.
\end{defi}

Again, we will also call an honest cooperad in the aforementioned category a model, if it satisfies the above condition, considered as a homotopy cooperad. 

Secondly, we consider comodule models for operads.
Here a further technical complication arises: We would like to say that a comodule model for $\op T$ is a pair consisting of a homotopy Hopf coalgebra $A$ quasi-isomorphic to $\Omega_{PL}(G)$, and a homotopy cooperad $\op C$ in homotopy $A$-comodules quasi-isomorphic to $\Omega_{PL}(\op T)$.
Unfortunately, this definition is invalid since the category of homotopy $A$-comodules is not (strictly) symmetric monoidal.
There are ways to repair this defect. However, for the sake of simplicity we adopt here a somewhat crude solution and strictify the Hopf algebra and module.
As above, the strictification comes at the cost of having to work with filtered complete vector spaces and completed tensor products as before.

\begin{defi}
 A \emph{comodule model} for the operad in $G$-spaces $\op T$ is a pair consisting of (i) a dg Hopf coalgebra $A$ (i.e., a coalgebra object in $\hdgca$) and (ii) a homotopy operad in dg Hopf $A$-comodules, which is quasi-isomorphic to the pair $(A_G, \mod_{A_G}(\op T))$.
\end{defi}

Here we use that $A_G$ is commutative and hence the $A_G$ comodules form a symmetric monoidal category, and furthermore that $\mod_{A_G}$ is a lax symmetric monoidal functor.

% 
% \begin{rem}
%  We note that we abuse the deceptively too simple notation $\Omega_{PL}(\op T)$ for the homotopy cooperad in homotopy comodules.
% Unpacked, this object is in fact a collection, for every tree $T$ and integer $n$ of dgcas
% \[
%  \Omega_{Pl}( \times_T \op T \times G^{\times n})
% \]
% with suitable maps between them encoding the $G$-action and operadic (co)composition.
% \end{rem}


% for $G$ and $\op T$ as above the functor 
% fix a Hopf algebra model $A_G$ for $G$, and a dgca model $B_G$ for $BG$. For $\op T$ an operad in $G$-spaces (i.e., a topological operad with a compatible $G$-action) we define a dgca model over $\K$ of $\op T$ to be a cooperad in $A_G$-Hopf comodules, weakly equivalent as a homotopy cooperad to the homotopy cooperad $\Omega_{PL}(\op T)$.
% 
% We define an equivariant model of $\op T$ to be a cooperad in dgca's under $B_G$ weakly equivalent (as a homotopy cooperad) to $\Omega_{PL}(T//G)$.\footnote{The homotopy quotient here is defined arity-wise, not as a homotopy quotient of a topological operad.}
% \end{defi}
% 
% \begin{rem}
% We remark that the category of homotopy operads is not a suitable category for doing homotopy theory of operads.
% Hence we are careful to not define the model above to be a homotopy cooperad quasi-isomorphic to $\Omega_{PL}(\op T,\K)$, which would yield the wrong notion. (TODO: find a better name for "homotopy cooperad", make this remark more precise.)
% \end{rem}

%\begin{rem}[

\subsection{Strictifying by the W construction}
We have defined homotopy cooperad models for an operad in $G$-spaces $\op T$.
We may strictify these models using the W construction of section \ref{sec:homotopy operads}.
As before this comes at the cost of having to work with filtered complete vector spaces and completed tensor products.

In principle, we have to treat two versions of the $W$ construction: one for the equivariant model and one for the comodule model.
However, below we will need only the $W$ construction for the comodule model, which we shall hence consider exclusively.
% 
% Consider first the equivariant model.
% Assume for simplicity the $G$ is connected.
% Let $\op C$ be a cooperad in dgcas under $B$, flat as $B$ modules, for $B$ some dgca, quasi-isomorphic to $B_G$.
% The W construction then produces an cooperad in the under-category $B/\hdgca$. 
% If $G$ is not connected, we require (i) $B$ to be quasi-isomorphic to $B_{G_0}$ instead, and (ii) compatible actions of $G/G_0$ on all objects as before.

As the interval object occurring in the $W$ construction we will again use $\Omega_{poly}([0,1])$, considered as a Hopf comodule with the trivial coaction, with the trivial filtration.
The $W$ construction produces a cooperad in filtered complete $A$-comodules (with $A$ a dg Hopf coalgebra quasi-isomorphic to $\Omega(G)$).
In particular, the cooperations in the resulting cooperad carry two compatible complete filtrations: One arises because the $A$-comodules had a filtration to start with.
The other arises because of the $W$ construction, and is induced by the number of vertices in trees.
We shall need below that the $W$ construction is an exact functor, more concretely:
\begin{thm}\label{thm:W for comodule op}
 Let $A$ be a dg Hopf coalgebra, i.e., a coalgebra in $\hdgca$. Let $\mC$ be the category of $A$-comodules. Let $\hat \mC$ be the category of $A$-comodules with an additional filtration satisfying the conditions of section \ref{sec:vector spaces complexes}.
Then the statement of Theorem \ref{thm:W properties} is valid for these categories $\mC$, $\hat \mC$.
\end{thm}
\begin{proof}
 The proof of Theorem \ref{thm:W properties} carries over one-to-one to this setting.
\end{proof}




% 
% 
% Given models for an operad $\op T$ in $G$-spaces as above, we may of course strictify these models by applying the $W$ construction.
% This produces an honest cooperad, albeit at the cost of using completed tensor products for the underlying vector spaces.
% For the equivariant model in particular we can obtain in this way a dg Hopf cooperad under $B_G^s$
% \[
%  \mod_{G}(\op T) := W(\Omega_G^s(\op T)).
% \]
% For the comodule model one has to strictify first the weak coaction of $\Omega_{PL}(G)$ on $\Omega_{PL}(\op T)$. Applying the $W$ construction this result in a homotopy cooperad in
% dg Hopf $A_G$ comodules
% \[
%   W_G(\Omega_{PL}(\op T)).
% \]
% Here we used the notation $W_G$ to emphasize that the W construction is only applied to the homotopy modules, not touching the cooperad structure yet.
% Then, in a second step, one can also strictify the cooperad structure and obtain a cooperad in
% dg Hopf $A_G$ comodules
% \[
%  \mod_{A_G}(\op T) := W_{op}(W_G(\Omega_{PL}(\op T))).
% \]
% Here we use the notation $W_{op}$ to denote the cooperadic W construction, to distinguish it from $W_G$.
% 

% We define the model functors 
% \[
% \mod : \Op_{\Top} \to \Op^c_{dgca}
% \]
% and 
% \[
% \mod_G : \Op_{G\Top} \to \Op^c_{A_G-mod^c}
% \]
% by the assignments 
% \[
% \mod(\op T) = W(\Omega_{PL}(\op T))
% \]
% and 
% \[
% \mod_G(\op T) = W_G(\mod_{A_G}^{space}(\op T)),
% \]
% where in the latter definition 
% \[
% \mod_{A_G}^{space} :  \Op_{G\Top} \to \Hop^c_{A_G-mod^c}
% \]
% is the functor to homotopy cooperads in Hopf $A_G$ comodules induced by our ($G$-space-)model functor $\mod_{A_G}$ in spaces, while 
% \[
% W_G : \Hop^c_{A_G-mod^c} \to \Op^c_{A_G-mod^c}
% \]
% is the functor $W$ from before, specialized to the category $A_G-mod^c$. (N.b.: In $A_G-mod^c$ the interval object is the same as in $\dgca$, i.e., the polynomial forms on $[0,1]$, equipped with the trivial $A_G$ coaction.)
% 
% There are natural maps of dg Hopf collections $\mod(\op T)\to \Omega_{PL}(\op T)$ and $\mod_G(\op T)\to \Omega_{PL}(\op T)$ by restriction. They are in some sense compatible up to homotopy with the cooperadic structure and the $\Omega_{PL}(G)$ coaction (which are defined only up to homotopy), however we will not make this precise but again just define ad hoc the model of $\op T$ to be $\mod(\op T)$ or $\mod_G(\op T)$ respectively. Still, the following result shall serve as motivation for doing so.
% \begin{prop}
% The maps $\mod(\op T)\to \Omega_{PL}(\op T)$ and $\mod_G(\op T)\to \Omega_{PL}(\op T)$ are quasi-isomorphisms.
% \end{prop}
% \begin{proof}
% The proof is an exact copy of the proof of Proposition \ref{prop:qisoresG}.
% \end{proof}

% \subsection{Relating the operadic equivariant and comodule models}
% We saw in section \ref{} how to recover the comodule model for a topological $G$-space from the equivariant, by taking a tensor product with a model $K$ for $EG$.
% It was noted there that the functor $K_G\otimes_{B_G}-$ is homotopically monoidal.
% Hence, given an operad in $G$-spaces $\op T$, we use that functor to recover from the cooperadic equivariant model the cooperadic comodule model.
% Let us state this finding in the following result.
% 
% \begin{prop}
% The cooperads in $A_G$ Hopf comodules given by $W(K_G \otimes_{B_G}\Omega_G(\op T))$ and $W(\mod_{A_G}(\op T))$ are quasi-isomorphic. 
% \end{prop}
% \begin{proof}
% It follows by combining Proposition \ref{prop:framedmodel} TODO:wrong ref!!! and Proposition \ref{prop:space from quotient}.
% 
% \end{proof}


% 
% REMOVE STUFF BELOW
% 
% 
% We have functors 
% \[
% \begin{tikzcd}
% G\Top \ar{r}{\mod_{A_G}} \ar{dr}[below]{\mod_{G} } & (A_G-mod^c)^{op} \ar[bend left]{d}{-\otimes^{A_G} K} \\
% & (B_G /\dgca)^{op} \ar{u}{-\otimes_{B_G} K}
% \end{tikzcd}
% \]
% These functors have the following monoidality properties.
% \begin{itemize}
% \item The functors $\mod_{A_G}$ and $\mod_G$ are symmetric comonoidal.
% \item The functor $-\otimes^{A_G} K$ is symmetric monoidal (mind the ${}^{op}$).
% \item The functor $-\otimes_{B_G}K$ is symmetric comonoidal.
% \end{itemize}
% In particular, for a topological operad $\op T$ with a $G$ action we may build the homotopy cooperads (in $A_G-mod^c$) $\Omega_G(\op T)\otimes_{B_G} K$ and $\Omega(\op T)$. 
% 
% \begin{prop}
% The cooperads in $A_G-mod^c$ given by $W_G(\Omega_G(\op T)\otimes_{B_G} K)$ and $W_G(\Omega(\op T))=\mod_G(\op T)$ are quasi-isomorphic. 
% \end{prop}
% \begin{proof}
% It follows by combining Proposition \ref{prop:framedmodel} TODO:wrong ref!!! and Proposition \ref{prop:space from quotient}.
% 
% \end{proof}


\subsection{Framed operads and the framing product}\label{sec:framed operads}
Let $\op T$ be a topological operad with an action of a topological group (or monoid) $G$. Then one may build the corresponding framed operad $\op T\circ G$ such that 
\[
(\op T\circ G)(r) = \op T(r) \times G^{\times r},
\]
with the natural composition structure defined using the $G$-action. Concretely, the composition is such that for $t\in \op T(r)$, $t'\in \op T(s)$, $g_1,\dots,g_r,g_1',\dots,g_s'\in G$, $j=1,\dots,r$ we have
\[
 (t,g_1,\dots, g_r) \circ_j (t',g_1',\dots, g_s')
:=
(t\circ_j (g\cdot t'), g_1,\dots,g_{j-1}, g_jg_1',\dots, g_j g_s', g_{j+1},\dots, g_r).
\]

We call the operation ``$\circ$'' which associates to an operad in $G$-spaces the corresponding framed operad the \emph{framing product}.
On the underlying symmetric sequences of spaces it is the same as the plethysm, hence we use the same symbol. Mind however that we understand $\op T\circ G$ to come quipped with the strutcure of a topological operad.

Similarly, let $\op C$ be a Hopf cooperad with a Hopf coaction of the Hopf algebra $A$. 
Then we may build the corresponding framed cooperad $\op C\circ A$ such that 
\[
(\op C \circ A)(r) = \op C(r) \otimes A^{\otimes r},
\]
with the cocomposition naturally defined using the Hopf coaction of $A$ on $\op C$.
Mind that for our application (e.g., $A=A_G$) the tensor product here is a completed tensor product, in order for the cocomposition to make sense.

If $\op C$ is only a homotopy cooperad in $A$ comodules we do not know how to define the framing product ``directly'' in the category of homotopy cooperads.
Rather, we will first strictify $\op C$ using the $W$ construction as in the previous subsection to a cooperad $W\op C$ in Hopf $A$-comodules. Then we may apply the framing construction $(W\op C)\circ A$. The result is a cooperad in $\hdgca$, which we may of course interpret as a homotopy cooperad in $\dgca$.

The main result is that the topological and algebraic framing constructions above are related to each other via our model functor.

\begin{prop}\label{prop:framedmodel}
Let $\op T$ be a topological operad acted upon by the topological group $G$. Then the homotopy Hopf cooperads $\Omega_{PL}(\op T\circ G)$ and $W(\mod_{A_G}(\op T)) \circ A_G$ are quasi-isomorphic.
\end{prop}
\begin{proof}
We will connect the homotopy Hopf cooperads $\Omega_{PL}(\op T \circ G)$ and $\mod_{A_G}(\op T) \circ A_G$ by a zigzag of quasi-isomorphisms.
To describe the intermediate objects, we need some items of notation.
First, for a topological operad $\op T$ we denote by $W_{op}\op T$ its $W$ construction in the sense of section \ref{sec:W}, which agrees with the usual Boardman-Vogt $W$-construction.
As a group is in particular an operad, we may also consider the resolution $WG$ of $G$.
Since $G$ acts on $\op T$ and on $W_{op}\op T$, the same is true for $WG$.

However, we may as well consider $\op T$ as an operad in $G$-spaces.
Hence we may consider the homotopy operad $\op T$, and then apply the $W$ construction for $G$-modules. The result is a homotopy operad in $WG$-spaces which we denote by $W_G(\op T)$.
Concretely, for a tree $T$ the space $(W_G\op T)(T)$ looks like the bar resolution of the $G$-space $\times_T \op T$.
Next, we may apply the $W$ construction again to such a homotopy operad in $WG$-spaces, yielding an operad in $WG$-spaces, which we call $W_{G-op}(W_G(\op T))$.
Finally we may take the framing product with $WG$.
We note that we have a direct map of topological operads 
\beq{equ:contractW}
W_{G-op}(W_G(\op T))\circ WG\to (W_{op}\op T) \circ G
\eeq
by contracting the various $W$ resolutions.

The chain of quasi-isomorphisms of homotopy Hopf cooperads is then
\[
\Omega_{PL}(\op T \circ G) \to \Omega_{PL}(W_{op}(\op T)\circ G) \to \Omega_{PL}(W_{G-op}(W_G(\op T))\circ W(G)) \leftarrow \mod_{A_G}(\op T) \circ A_G.
\]
Here the first map is induced by $W_{op}(\op T)\to \op T$. The second comes from \eqref{equ:contractW}.
The last is the natural inclusion, noting that the construction $W_{G-op}(W_G(\op T))\circ W(G)$ is the topological version of $\mod_{A_G}(\op T) \circ A_G$.

\end{proof}



\subsection{Remark: Semi-algebraic variant}
In the above generality we were using the Sullivan functor $\Omega_{PL}$ to define our models for topological objects.
However, if, for some sub-category of topological spaces there is a quasi-isomorphic functor $\Omega \simeq \Omega_{PL}$, we may equivalently use $\Omega$ instead of $\Omega_{PL}$ in all constructions above, obtaining quasi-isomorphic dgca models.
For the concrete problem we are considering in this paper, i.e., to study the $\SO(n)$ action on $\FM_n$ all objects are semi-algebraic manifolds, and hence we may consider PA instead of PL forms on them, cf. \cite{HLTV}.
Below this point we will tacitly (and abusively) replace the functor $\Omega_{PL}$ in the above constructions by $\Omega_{PA}$. (TODO: shall we invent better notation?)


\subsection{Models for framed operads from equivariant models}
The only result of this section we will need below is summarized in the following Theorem.
In fact, we will strictly speaking only need the second statement: We will be able to produce a small equivariant model for the little $n$-disks operad with the $O(n)$-action, and we want to use the following Theorem to produce from that equivariant model a dgca model for the framed little $n$-disks operad.

\begin{thm}\label{thm:equiv to framed}
 Suppose that $\op T$ is a semi-algebraic operad acted upon by the compact algebraic group $G$, with connected component $G_0$.
Let $H(BG_0)\to \Omega_{PA}^G(\op T)$ be the homotopy cooperad (in the category $\mC$ of dgcas under $H(BG_0)$ with an action of $G/G_0$) formed by the PA equivariant forms, cf. section \ref{sec:PA Cartan model}.
Let $X$ be a cooperad in the category $\mC$, which comes with a quasi-isomorphism (of homotopy operads in $\mC$)
\[
 X\to \Omega_{PA}^G(\op T).
\]
($X$ is hence an equivariant model for $\op T$ in the sense of section \ref{sec:op mod in G space}.)
Then:
\begin{enumerate}
 \item The cooperad in $H(G)$-comodules
$$
X' := \left(\HK \otimes_{H(BG)} X\right)^{G/G_0}
$$ 
is a comodule model for $\op T$ in the sense of section \ref{sec:op mod in G space}, where $\HK$ is the Koszul complex as in section \ref{sec:Koszul cx for compact Lie}.
\item Furthermore, the framed cooperad
$$
X'' := X' \circ H(G)
$$
is quasi-isomorphic (in the category of homotopy cooperads in dgcas) to the homotopy cooperad $\Omega_{PA}(\op T\circ X)$.
In other words, $X''$ is a dg Hopf cooperad model for $\op T\circ G$ in the sense of section \ref{sec:dgca mod for op}.
\end{enumerate}
\end{thm}
\begin{proof}
Considering the result of section \ref{sec:Koszul cx for compact Lie}, $X'$ fits into a zigzag of quasi-isomorphisms of Hopf coalgebras and homotopy cooperads in Hopf comodules
\beq{equ:modulezigzag}
\begin{tikzcd}
 H(G) \ar[dashed]{r} \ar{d}{=} & X'=\left(\HK \otimes_{H(BG)} X\right)^{G/G_0} \\
 H(G) \ar[dashed]{r} \ar{d} & \cdot \ar{u} \ar{d} \\
A_G \ar[dashed]{r} & \mod_{A_G}(\op T) %K \hotimes_{B_G} mod_G(\op T) =: Y'
\end{tikzcd}
\eeq
where dashed arrows denote a coaction. This shows the first statement.


Next, to show the second statement, we proceed in the following steps.
First note that due to Theorem \ref{thm:W for comodule op} $W(X')\xrightarrow{\sim} X'$. Due to the exactness of the framing product, we then have that
\[
 X'' = X'\circ H(G) \xleftarrow{\sim} W(X') \circ H(G).
\]
Again using both exactness statements the zigzag \ref{equ:modulezigzag} then provides us with a zigzag of quasi-isomorphisms
\[
 X'' \leftarrow \cdot \rightarrow W(\mod_{A_G}(\op T))\circ A_G.
\]
Finally, using (the PA variant of) Proposition \ref{prop:framedmodel}, we conclude that the latter object provides indeed a dgca model for the framed operad $\op T\circ G$.
\end{proof}







\section{Graph operads and graph complexes}\label{sec:graphs}
The goal of this section is to construct an equivariant model for the little disks operads, using diagrams.
The construction is essentially merely the equivariant version of a construction employed by Kontsevich \cite{K2} in proving the real formality of these operads.

\subsection{Definitions}\label{sec:GraDefinitions}
We recall here the definition of the Kontsevich's graph complexes and graph operads. The original definitions may be found in \cite{K2, Knoncomm}, whereas we follow the approach of \cite{Will}.

We denote by $\gra_{N,k}$ the set of directed graphs with vertex set $[N]=\{1,\dots,N\}$ and edge set $k$.
It carries an action of the group $S_N\times S_k \ltimes S_2^k$ by permuting the vertex and edge labels and changing the edge directions. The graphs operads $\Gra_n$ are defined such that
\[
\Gra_n(N) = \oplus_k (\K \langle \gra_{N,k}\rangle [(n-1)k])_{S_k \ltimes S_2^k}
\]
where the action of $S_k$ is with sign if $n$ is even and the action of $S_2^k$ is with sign if $n$ is odd.
For all $n$ one has a map of operads
\begin{align*}
\Poiss_n &\to \Gra_n \\
\wedge &\mapsto 
\begin{tikzpicture}
\node[ext] at (0,0) {};
\node[ext] at (0.5,0) {};
\end{tikzpicture} 
\\
[,] &\mapsto 
\begin{tikzpicture}
\node[ext] (v) at (0,0) {};
\node[ext] (w) at (0.5,0) {};
\draw (v) edge (w);
\end{tikzpicture} .
\end{align*}
In particular, one obtains maps 
\[
\hoLie_n\to \Lie_n \to \Gra_n.
\]
We define the full graph complex as the deformation dg Lie algebra as the operadic deformation complex
\[
\fGC_n := \Def_{Op}(\hoLie_n\to \Gra_n).
\]
One may consider two interesting dg Lie subalgebras:
\begin{itemize}
\item The connected graphs with at least bivalent vertices form the dg Lie subalgebra $\GC^2_n$.
\item The connected graphs with at least trivalent vertices form the dg Lie subalgebra $\GC_n$.
\end{itemize}
One can check that (see \cite{Knoncomm} or \cite[Proposition XX]{Will})
\[
H(\GC_n^2)= H(\GC_n)\oplus \bigoplus_{1\leq r \equiv 2n-1 \text{ mod 4}} \K L_r
\]
where $L_r$ denotes the ``loop'' class of degree $r-n$, represented by a "loop" graph consisting of $r$ bivalent vertices.
\[
L_r
=
\begin{tikzpicture}[baseline=-.65ex]
\node[int] (v1) at (0:1) {};
\node[int] (v2) at (72:1) {};
\node[int] (v3) at (144:1) {};
\node[int] (v4) at (216:1) {};
\node (v5) at (-72:1) {$\cdots$};
\draw (v1) edge (v2) edge (v5) (v3) edge (v2) edge (v4) (v4) edge (v5);
\end{tikzpicture}
\quad \quad \quad \quad \quad \quad \text{($r$ vertices and $r$ edges)}
\]

We may use the formalism of operadic twisting \cite{DolWill} to twist the operad $\Gra_n$ to an operad $\fGraphs_n$. Elements of $\fGraphs_n(N)$ are series of graphs with two sorts of vertices, external vertices labelled $1,\dots,N$ and internal unlabeled vertices.
We again identify two useful suboperads
\begin{itemize}
\item The graphs with at least bivalent internal vertices and no connected components entirely internal vertices form the sub-operad $\Graphs^2_n$.
\item The graphs with at least bivalent internal vertices and no connected components entirely internal vertices form the sub-operad $\Graphs_n$.
\end{itemize}

The formalism of operadic twisting furthermore ensures that there is an action of the dg Lie algebra $\fGC_n$ on $\fGraphs_n$. One easily checks that the action restricts to an action of $\GC_n^2$ on $\Graphs_n^2$ and of $\GC_n$ on $\Graphs_n$.
Furthermore, the multiplicative group $\K^\times\ni \lambda$ acts on $\Graphs_n^2$ and $\Graphs_n$ by multiplying a graph $\Gamma$ by the number
\[
 \lambda^{\#(\text{internal vertices})-\#(\text{edges})}.
\]


There is a natural map $\Poiss_n\to \Graphs_n$ given by the same formulas as the map $\Poiss_n\to \Gra_n$ above. We will use the following well known result:
\begin{prop}[\cite{K2},\cite{LV},\cite{Will}]
The maps 
\[
\Poiss_n\to \Graphs_n^2 \to\Graphs_n 
\]
are quasi-ismorphisms.
\end{prop}

Finally, there is a natural topology and a continuous Hopf operad structure on $\Graphs_n$ and $\Graphs_n^2$ and the above maps and actions are compatible with that structure.

By (pre-)duality one can define dg Hopf $\La$ cooperads $\stG_n$ and $\stG_n^2$ such that 
\begin{align*}
 \Graphs_n &= (\stG_n)^* & \Graphs_n^2 &= (\stG_n^2)^*.
\end{align*}
Concretely, elements of $\stG_n(r)$ are linear combinations of graphs with $r$ numbered ``external'' vertices and an arbitrary number of internal vertices, of the same form as those generating $\Graphs_n(r)$. 
(The difference is that elements of $\Graphs_n(r)$ are formal series of graphs instead of (finite) linear combinations.)
The dg Hopf $\La$ cooperad structure is determined by duality. We have quasi-isomorphisms of dg Hopf $\La$ cooperads
\[
 \stG_n^2 \to \stG_n\to \Poiss_n
\]
and the graph complex $\GC_n$ (resp. $\GC_n^2$) acts on $\stG_n$ (resp. $\stG_n^2$), respecting all structures.


\subsection{Kontsevich's proof of real formality of \texorpdfstring{$\lD_n$}{Dn}}
M. Kontsevich showed in \cite{K2} that the operads of real chains on the little disks operads are formal.
Some of the steps and undrlying technicalities where however only sketched in his paper and later developed more carefully by Hardt, Lambrechts, Voli\'c, and Turchin \cite{LV,HLTV}. 
The main step of the proof is to construct a quasi-isomorphism 
\begin{equation}\label{equ:Kmap}
 \stG_n \to \Omega_{PA}(\FM_n)
\end{equation}
between the graphical cooperad $\stG_n$ introduced above and the PA forms \cite{HLTV} on the Fulton-MacPherson-Axelrod-Singer compactification of the moduli space of points on $\R^n$ introduced in \cite{GJ}.
Before recalling the definition of the map above, let us recall some details on the topological operad $\FM_n$.
Let $\Conf_{N}(R^n)$ be the space of configurations of $N$ distinguishable points on $\R^n$. It is acted upon freely by the group $\R_{>0}\ltimes \R^n$ by scaling and translation.
The spaces $\FM_n(N)$ are compactifications (iterated real bordifications) of the quotient space under this action.
\[
 \FM_n = \overline{ (\Conf_N(\R^n)/\R_{>0}\ltimes \R^n)}
\]
Concretely, the compactification is defined such that the $\FM_n$ as an operad in sets rather than spaces is the free operad generated by $\Conf_N(\R^n)/\R_{>0}\ltimes \R^n)$. From this description the definition of the operadic composition in $\FM_n$ is also obvious. The topological operad $\FM_n$ is homotopic to the little $n$-disks operad $\lD_n$.
For more details on the definition we refer the reader to the original reference \cite{GJ} or \cite{Si}.

Now let us turn to the definition of Kontsevich's map \eqref{equ:Kmap}. 
For a graph $\Gamma\in \stG_n(N)$ with $k$ internal vertices the map is defined by the formula 
\begin{equation}\label{equ:Kintegral}
 \Gamma \mapsto \omega_\Gamma:= \int_f \bigwedge_{(i,j)\text{ edge}} \pi_{ij}^* \Omega_{S^{n-1}} \in \Omega_{PA}(\FM_n(N))
\end{equation}
where 
\[
\pi_{ij}: \FM_n(N+k) \to \FM_{n}(2)=S^{n-1}
\]
is the forgetful map forgetting all vertices in a configuration except for the $i$-th and $j$-th, and the integral is over the fiber of the forgetful map 
\[
 \FM_n(N+k) \to \FM_n(N).
\]
The fiber integral does in general not produce a smooth differential form, and that is the reason why one has to work with PA forms instead of smooth forms.
It can be checked by using Stokes' Theorem that the map \eqref{equ:Kmap} respects the differentials and is compatible with the cooperad structure on $\stG_n$ and the operadic composition on $\FM_n$ in a natural way. It is furthermore a quasi-isomorphism.
By dualizing the map \eqref{equ:Kmap} one obtains a quasi-isomorphism of operads
\begin{gather*}
 C(\FM_n) \to \Graphs_n \\
 c\mapsto \sum_{\Gamma} \Gamma \int_c \omega_{\Gamma}
\end{gather*}
where $C(\FM_n)$ is the operad of semi-algebraic chains (see again \cite{HLTV}) on $\FM_n$, and the sum is over a set of graphs forming a basis of $\stG_n$. We use the notation $\int_c \alpha$ to denote the pairing of a semi-algebraic chain $c$ and a PA form $\alpha$.

The desired real formality morphism linking $C(\FM_n)$ to its homology operad $\Poiss_n$ is hence realized by the zigzag of quasi-isomorphism of operads
\[
 C(\FM_n) \to \Graphs_n \leftarrow \Poiss_n.
\]

The purpose of the rest of this section is to construct an equivariant version of the Kontsevich map \eqref{equ:Kmap}. Naively speaking this may be done by simply replacing PA forms by equivariant PA forms, while essentially retaining the formula \eqref{equ:Kintegral}, which, in its equivariant form, will re-appear as \eqref{equ:Kintegralequiv} below. However, in practice various steps of the proof that the map \eqref{equ:Kmap} is compatible the differential and cooperad structure will (at least naively) fail in the equivariant setting, the ``defects'' accounting exactly for the rational nontriviality of the action of $\SO(n)$ on $\FM_n$.

\begin{rem}
 Note that a priori the formula \eqref{equ:Kintegral} is defined without restrictions on the arity of vertices in the graph $\Gamma$. In particular, it in fact defines a map (and a quasi-isomorphism)
 \[
  \stG_n^2\to \Omega_{PA}(\FM_n).
 \]
As part of Kontsevich's construction of \eqref{equ:Kmap} one then has to check that this map indeed factors through the quotient cooperad $\stG_n\leftarrow \stG_n^2$. In other words, one has to check that the intergrals corresponding to graphs with bivalent internal vertices vanish.
In fact, it turns out to be sufficient to check that for the graph
\[
 \Gamma = 
 \begin{tikzpicture}[baseline=-.65ex]
  \node[ext] (v) at (0,0) {};
  \node[int] (x) at (0.5,0) {};
  \node[ext] (w) at (1,0) {};
  \draw (x) edge (v) edge (w);
 \end{tikzpicture}
\]
we obtain $\omega_\Gamma=0$, which was shown by Kontsevich, cf. Lemma \ref{lem:bivalentvanish} in the Appendix.
%In the equivariant setting discussed below we will not be able to provide such a vanishing result and hence have to work with the larger cooperad $\stG_n^2$ throughout, instead of passing to $\stG_n$. 
\end{rem}


% \subsection{Rational deformation theory of $\lD_n$ and (philosophical) outlook}\label{sec:philosophy}
% In this section, let us give a conceptual overview and interpretation of the results to be derived below.
% (TODO: maybe this should be part of an overview at the end of the introduction)
% 
% Quite generally (and vaguely) the action of a group $G$ on an object $X$ may be understood as a map 
% \[
%  G\to \Aut(X).
% \]
% Supposing that both groups above are topological, we may ask for the homotopy type of the map above, which we may define as the homotopy class of the map of topological spaces
% \[
%  BG\to B\hAut(X).
% \]
% (The map of groups, up to homotopy, may be recovered by taking loop spaces on both sides.)
% 
% In our setting we want to take $X=\lD_n^\Q$, the rationalization of the little disks operad. The homotopy theory of these operads is developed in \cite{FTW}. It is shown there that $\hAut(\lD_n^\Q)$ may be modeled (in a sense to be recalled here...) by the Lie algebra
% \[
%  \Q \ltimes \GC_n^{2,trunc}
% \]
% where the factor $\Q$ controls the rescalings of the Lie bracket in homology, and $\GC_n^{2,trunc}$ is a truncated version of the graph complex, which acts on the model $\stG_n$ for $\lD_n$ according to the formulas \eqref{}.
% To be able to work with pro-unipotent groups, let us pass to the subgroup of homotopy automorphisms $\hAut'(\lD_n^\Q)\subset \hAut(\lD_n^\Q)$ inducing the identity map on homology. This is a minor modification since automorphism of $H(\e_n)$ are in bijection to the invertible numbers $\Q^\times$.
% 
% Then a map $BG\to B\hAut'(\lD_n^\Q)$ may be modelled by a flat $\GC_n^{2,trunc}$-connection on $BG$, which in turn can be seen as a Maurer-Cartan element in the graph complex with coefficients in $H(BG)$:
% \[
%  m\in \GC_n^{2}\hat \otimes H(BG). % TODO: define truncation more carefully and say why MC elements are the same
% \]
% TODO: truncated GC here?
% The gauge equivalence class of $m$ may be understood as an encoding of the ``rational homotopy type'' of the action of $G$ on $\lD_n$.
% The nice fact is that a simple and explicit integral formula for $m$ may be provided, see \eqref{equ:MCelement} below.
% 
% 
\subsection{Equivariant forms on \texorpdfstring{$\FM_n$}{FMn}}\label{sec:equivariant forms FMn}
Before we discuss the equivariant version of Kontsevich's construction, let us set up the model of equivariant forms on $\FM_n$.
We consider the action on $\FM_n$ of the group $G= O(n)$.
%In fact, we may mostly restrict to the ``most difficult'' case $G=O(n)$, for all smaller $G$ the desired results will follow by restriction of our results for $O(n)$.
We denote by $G_0=\SO(n)$ the connected component of the identity.
We pick a maximal torus $T\subset G_0$ with Lie algebra $\alg t$. Denote by $K\subset G_0$ the normalizer of $T$, and by $W=K/T$ the Weyl group.
We would like to use the toric Cartan model for the equivariant forms. 
However, due to technical difficulties with the PA version of that model discussed in section \ref{sec:PA Cartan model}, we have to resort to the workaround described in that section.
To this end let us define the subalgebras $A_r\subset \Omega_{PA}(\FM_n(r))$ consisting of smooth algebraic forms.
Clearly, the contraction with a generating vector field is then again algebraic and smooth.
Then we define the ``pseudo''-Cartan differential forms $(A_{K,r},d_u)$ to be 
\[
A_{r,K} := (S(\alg t^*[-2])\otimes A_r)^K
\]
with differential \eqref{equ:dudef}. As in \eqref{equ:Phi} these smooth algebraic equivariant forms come equipped with a map $\Phi$ into the PA equivariant PA differential forms on $\FM_n$.
(Note that we explicitly not claim that the map $\Phi$ is a quasi-isomorphism.)


% By the result of the previous sections we may understand the real homotopy type of the $G$-framed $E_n$ operad through understanding the $G$-equivariant forms on $E_n$.
% It is hence the purpose of the present section to construct a "manageable" combinatorial model for the latter.
% The construction here is essentially just an equivariant upgrade of a construction by Kontsevich \cite{K2, LV}.
% 


\subsection{A propagator}\label{sec:propagator}
We choose a smooth algebraic $K$-equivariant differential form $\Omega_{sm}\in A_{2,K}$ on the $(n-1)$-sphere such that
\begin{enumerate}
\item $\Omega_{sm}$ is of degree $n-1$.
\item The image of $\Omega_{sm}$ under the map $A_{2,K}\to K$ is a volume form of area 1.
\item $d_u\Omega_{sm}\in H(BG)$. In practice, this means that for $n$ even and $G=\SO(n)$, $d_u\Omega_{sm}=E$ is the Euler class in $H(B\SO(n))$, and $d_u\Omega=0$ in all other cases.
\item If $f:S^{n-1}\to S^{n-1}$ is the inversion (i.e., $f(x)=-x$) then $f^*\Omega_{sm}=(-1)^{n}\Omega_{sm}$.
\item Note that by being in $A_{2,K}$ the form $\Omega_{sm}$ is required to be invariant under the action of $K$. Furthermore, let us require that it is invariant under the action of $\pi_0(G)$.
\end{enumerate}
We will call this form the (equivariant) propagator.
An explicit formula for $\Omega_{sm}$ is given in Appendix \ref{sec:explicitpropagator}.

We will also define the element 
\[
 \Omega = \Phi(\Omega_{sm}) \in \Omega_K^{s,PA}(\FM_n(2))
\]
where $\Phi$ is the map \eqref{equ:Phi}.


\newcommand{\Grav}{\mathsf{Grav}}
\subsection{Equivariant cohomology of \texorpdfstring{$\FM_n$}{FMn}}
Let us pause here and evaluate the $G_0=\SO(n)$-equivariant cohomology of $\FM_n(r)$.
For the moment, we disregard the operad structure, we care only about the cohomology of the dg vector space of equivariant forms.
This cohomology is easily computed using the smooth Cartan model.
There is an evident spectral sequence whose $E_1$ page reads
\beq{equ:E1equiv}
 E_1 = H(B\SO(n))\otimes H(\FM_n(r)).
\eeq
Recall that by results of F. Cohen the cohomology of $\FM_n(r)$ is described as a commutative algebra by generators and relations as follows:
The generators are (classes represented by) forms 
\[
 \alpha_{ij} = \pi_{ij}^* \Omega_{S^{n-1}},
\]
where $1\leq i\neq j\leq r$ and $\Omega_{S^{n-1}}$ is a form on $S^{n-1}$ generating $H(S^{n.1})$.
The relations are the following
\begin{align*}
 \alpha_{ij}&=(-1)^n\alpha_{ji}
\\
\alpha_{ij}^2&=0
\\
\alpha_{ij}\alpha{jk}+\alpha_{jk}\alpha{ki}+\alpha_{ki}\alpha{ij}&=0.
\end{align*}

Now, if $n$ is odd, all the $\alpha_{ij}$ may in fact be extended to equivariantly closed forms, for example we can take for the extension
\[
 \pi_{ij}^*\Omega_{sm},
\]
where $\Omega_{sm}$ is our propagator from the preceding subsection.
Hence we conclude that for odd $n$ the spectral sequence abuts at this stage.

For even $n$ we may proceed similarly using our propagator to extend the forms, but since $d_u\Omega_{sm}=E$ the spectral sequence does not abut here.
Rather, defining the operator $T: H(\FM_n(r))\to H(\FM_n(r))$ as
\[
 T=\sum_{i\neq j} \frac{\partial}{\partial \alpha_{ij}}.
\]
Hence the next (distinct) page in the spectral sequence reads
\[
 \R[p_4,\cdots,p_{2n-4}]\otimes H( H(\FM_n(r))[E], ET).
\]
It is known that 
\[
\Grav(r) :=\ker(T) \to (H(\FM_n(r))[E], ET)  
\]
is a quasi-isomorphism. (In fact, $\Grav$ is the gravity operad.)
Since we now have closed representatives for all remaining classes on the present page our spectral sequence, the spectral sequence abuts here.

Let us summarize our finding.
\begin{prop}\label{prop:FMequiv}
 The $\SO(n)$-equivariant cohomology of $\FM_n(r)$ is 
\[
 \begin{cases}
  H(B\SO(n))\otimes H(\FM_n(r)) & \text{if $n$ is odd} \\
  H(B\SO(n-1))\otimes \Grav(r) &\text{if $n$ is even}
 \end{cases}
\]
\end{prop}

Note also that the explicit representatives we constructed are algebraic, by our choice of $\Omega_{sm}$ as an algebraic form.
The corresponding representatives in $\Omega_K^{s,PA}(\FM_n(r))$ may be obtained by just replacing $\Omega_{sm}$ by $\Omega=\Phi(\Omega_{sm})$.




\subsection{A Maurer-Cartan element}
Fix some choice of propagator $\Omega_{sm}$ as in the last section. We denote $E:=d_u\Omega_{sm}\in H(BG_0)$. Concretely, $E$ is either the Euler class or 0, depending on $G$.
We define a Maurer-Cartan element $m\in \GC_n\hat \otimes H(BG_0)$ by the sum-of-graphs-formula
\begin{equation}\label{equ:MCelement}
m = E \tadpole + \sum_{\gamma} \gamma^* \int_{\FM_n(|V\gamma|)} \bigwedge_{(i,j)\in E\Gamma} \pi_{ij}^* \Omega_{sm},
\end{equation}
where the sum is over graphs $\gamma$ forming a basis of $\stGC_n$, while $\gamma^*$ are the dual basis elements in $\GC_n$. The projection $\pi_{ij}: \FM_n(|V\gamma|)\to \FM_n(2)=S^{n-1}$ is the forgetful map, forgetting the locations of all points in a configuration except for the $i$-th and $j$-th.


\begin{prop}\label{prop:mMC}
The element $m$ is indeed a Maurer-Cartan element.
\end{prop}

\begin{proof}
It follows from applying Stokes' Theorem.
\end{proof}

We claim that the gauge equivalence class of $m$ completely characterizes the (real) homotopy type of the action of $G$ on $\FM_n$. To see this, we will use $m$ to build a model for the $G$-equivariant differential forms in the next section.

Before we do this, let us however define the similar (in fact, identical) Maurer-Cartan element $\tilde m\in \GC_n\hat \otimes \Omega(BG)$
\begin{equation}\label{equ:MCelement2}
 \tilde m = E \tadpole + \sum_{\gamma} \gamma^* \int_{\FM_n(|V\gamma|)} \bigwedge_{(i,j)\in E\Gamma} \pi_{ij}^* \Omega.
\end{equation}
Of course, this element is defined in the same way as $m$ before, except that one uses the propagator $\Omega$ instead of $\Omega_{sm}$.
However, one checks that the two elements $m$, $\tilde m$ are in fact identical.
\begin{lemma}\label{lem:m to OBG}
 The element $\tilde m$ is a Maurer-Cartan element.
 It is the image of $m$ under the map of dg Lie algebras $\GC_n\hat \otimes H(BG)\to \GC_n\hat \otimes \Omega(BG)$ induced by the map \eqref{equ:HBGtoOmega}.
\end{lemma}
\begin{proof}
 The first statement clearly follows from the second and Proposition \ref{prop:mMC}.
 To see the second statement, denote by $I_{sm}$ and $I$ the two integrands appearing in \eqref{equ:MCelement} and \eqref{equ:MCelement2}. Then by \eqref{equ:Phipdef} the integrands differ only by contractions with vector fields on $\FM_n$, i.e., $I=\Phi(I_{sm})$ is the same as $\left(\prod_{j=1}^r(1+ \eta_j \otimes \iota_{\xi_{j}} ) \right) I_{sm}$ up to an identification of basic forms with forms on the quotient. In particular, the contractions necessarily produce forms that are not of top degree along $\FM_n$, and hence do not contribute to the integral. Hence the only surviving terms in the integrals in \eqref{equ:MCelement2} are those already present in \eqref{equ:MCelement}.
\end{proof}



\begin{rem}
 The good way to interpret the Maurer-Cartan element $m\in\GC_n\hat \otimes H(BG_0)$ above is as follows.
The action of $G_0=\SO(n)$ on $\FM_n$ may be modelled by a a homotopy action of the Hopf algebra $H_\bullet(G_0)$ on a real model $E_n^\R$ for $\FM_n$.
Since $H_\bullet(G_0)$ may be understood as the universal enevolping algebra of $\alg g:= \pi^\R(G_0)$, with the trivial Lie bracket, such an action can be modelled by an $L_\infty$ map from $\alg g$ into the homotopy derivations of $E_n^\R$. However, the graph complex $\GC_n$ (essentially) models those homotopy derivations \cite{FTW}.
And indeed, one way to interpret a Maurer-Cartan element $m\in\GC_n\hat \otimes H(BG_0)$ is as an $L_\infty$ map $\alg g\to \GC_n$. (Mind that, at this point we have not seen yet that our $m$ is really ``the correct one'' for the purpose of describing the $G_0$-action.)
Furthermore, one can see that the correct way to extend this interpretation to the non-simply connected $O(n)$ is to require that the $L_\infty$ morphism is $\pi_0(O(n))=\Z_2$-equivariant. Concretely, this means that $m$ is $\Z_2$-invariant, with $\Z_2$ acting on $H(B\SO(n))$ by flipping the sign of the Euler class, and on graphs in $\GC_n$ by multiplying with $(-1)^{\text{loop order}}$. Indeed, a quick calculation shows that, choosing an (anti-)symmetric propagator, $m$ is indeed $\Z_2$ invariant.
\end{rem}




% 




\subsection{A model for the equivariant forms on \texorpdfstring{$\FM_n$}{FMn}}
Recall from section \ref{sec:graphs} that the dg Lie algebra $\GC_n$ acts on the Hopf cooperad $\stG_n$. It follows that the dg Lie algebra 
\[
\BGC_n := \GC_n\hat \otimes H(BG_0)
\]
acts on the Hopf cooperad
\begin{align*}
\BstG_n &:= \stG_n \otimes H(BG_0) %& \text{and} & &  \stG_n \otimes H(BG)
\end{align*}
in such a way that the images of the natural commutative algebra maps $H(BG)\to \BstG_n(r)$ are preserved.
Given a Maurer-Cartan element $m\in \BGC_n$ as above, and using the action we may hence twist $\BstG_n$ to a Hopf cooperad $\BstG_n^m$ under $H(BG)$.

We claim that this is a an equivariant model for $\FM_n$ in the sense of Definition \ref{def:equivariant model}.
Indeed, there is a map
\[
 F: \BstG_n^m \to \Omega_{K}^{s,PA}(\FM_n)
\]
given by Feynman rules. Concretely, to a graph $\Gamma\in \stG_n(N)$ with $k$ internal vertices we associate the (semi-algebraic) differential form
\begin{equation}\label{equ:Kintegralequiv}
 \omega_\Gamma = \int_f \bigwedge_{(i,j)} \pi_{ij}^* \Omega
\end{equation}
where $\Omega$ is the propagator from above, the product is over edges and the integral is the fiber integral along
\[
 E\SO(n)\times_{\SO(n)} \FM_n(N+k) \to E\SO(n)\times_{\SO(n)} \FM_n(N).
\]

We note that this map is well defined, in the sense that it vanishes on graphs with bivalent internal vertices by Lemma \ref{lem:bivalentvanish}.
Furthermore, we let $\Z_2=\pi_0(G)$ act on $\BstG_n$ by multiplying a graph $\Gamma$ with $k$ internal vertices and $e$ edges by $(-1)^{l-e}$.
This action readily extends to $\BstG_n$, and it is elementary to check that the map $F$ is $\Z_2$-equivariant, given that our propagator is reflection anti-invariant.

\begin{thm}\label{thm:equivariant model}
The map $F$ above realizes $\BstG_n^m$ as an equivariant model for $\FM_n$ as an operad in $G=O(n)$-spaces.
 Concretely, $F$ respects the differentials, the (co)operad structure, the map from $H(BG_0)$, the $\Z_2$ action and is a quasi-isomorphism. 
 \end{thm}
\begin{proof}
The proof the same as Kontsevich's proof of the corresponding non-equivariant statement, except for three points.

First, In checking that the map $F$ commutes with the differentials one proceeds as follows. As in Kontsevich's proof, one applies Stokes' Theorem for PA forms.
\beq{equ:tempdF}
 dF(\Gamma) = \int_{\partial f} \bigwedge_{(i,j)} \pi_{ij}^* \Omega
+
\int_{f} d\bigwedge_{(i,j)} \pi_{ij}^* \Omega,
\eeq
where the first integral is over the fiberwise boundary. Again as in Kontsevich's proof the fiberwise boundary consists of several strata corresponding to bunches of points colliding.
Now, however, the integrals associated to these strata do not vanish. Rather, they produce precisely the terms of the Maurer-Cartan element $\tilde m$, except for the term $m_0:=E\tadpole$. Using 
Lemma \ref{lem:m to OBG} these terms are accounted for by taking the twist with $m-m_0$ in 
\beq{equ:tempBstGm}
\BstG_n^m=(\BstG_n^{m_0})^m.
\eeq
Next, the second term of \eqref{equ:tempdF} can be simplified as follows:
\begin{align*}
 d\bigwedge_{(i,j)} \pi_{ij}^* \Omega
&= 
 d\bigwedge_{(i,j)} \pi_{ij}^* \Phi(\Omega_{sm})
\\&= 
\sum_{e=(p,q)} (-1)^e \pi_{pq}^*\Phi(d_u \Omega_{sm}) \wedge \bigwedge_{(i,j)\neq e} \pi_{ij}^* \Phi(\Omega_{sm}),
\\&= 
\sum_{e=(p,q)} (-1)^e  E \wedge \bigwedge_{(i,j)\neq e} \pi_{ij}^* \Phi(\Omega_{sm}).
\end{align*}
In the last lines we sum over edges $e=(p,q)$ in our graph $\Gamma$, and we set $(-1)^e$ to be 1 for the first edge in the ordering $-1$ for the second etc.
For the last simplification we furthermore used that $d_u\Omega_{sm}=E$ by construction of the propagator.
Inserting back into \eqref{equ:tempdF}, the second term of that equation may be identified with 
\[
\sum_{e=(p,q)} (-1)^e E F(\Gamma-e) = F(m_0\cdot \Gamma).
\]
Hence this term reproduces precisely the twist by $m_0$ in \eqref{equ:tempBstGm}.

Finally we claim that the map $F$ is a quasi-isomorphism. Indeed, recall the computation of the equivariant cohomology of $\FM_n$ from Proposition \ref{prop:FMequiv}.
On the other hand, we may compute the cohomology of $\BstG_n^m$ by using the spectral sequence on the ``number of $u$'s''.
The first convergent is 
\[
 H(B\SO(n))\otimes H(\Graphs_n).
\]
Using that $H(\Graphs_n)\cong H(FM_n)$ this agrees with \eqref{equ:E1equiv}.
Furthermore, one immediately checks that the further pages of the spectral sequence agree, so that indeed $H(\BstG_n^m)\cong H_G(\FM_n)$.
Finally, it is clear from looking at the representatives of the cohomology of both sides that $F$ induces an isomorphism on cohomology.
\end{proof}

\begin{rem}
Note that by its construction the dg Hopf cooperad $\BstG_n^m$ (in $H(BG)/\dgca$) is acted upon by the dg Lie algebra
\[
(\GC_n\otimes H(BG_0))^m.
\]
Of course, if one wants to preserve also the $\Z_2$-module structure on $\BstG_n^m$, one has to restrict to the $\Z_2$-invariant dg Lie subalgebra 
\[
\left( (\GC_n\otimes H(BG_0))^m \right)^{\Z_2}.
\]
\end{rem}

\begin{rem}
 We will see later that for $n$ even the Mauer-Cartan element $m$ is gauge equivalent to $m_0$. It follows that $\BstG_n^m$ is quasi-isomorphic to $\BstG_n^{m_0}$.
Furthermore, there is a direct map (and quasi-isomorphism) $\BstG_n^{m_0}\to H(B\SO(n-1))\otimes \Grav$, where $\Grav$ is the gravity cooperad as in Proposition \ref{prop:FMequiv}.
This then shows that $\FM_n$ is equivariantly formal for $n$ odd.
\end{rem}


% \subsection{A model for the $G$-framed little cubes operads}
% According to the general theory (Theorem \ref{thm:equiv model to framed} above) we may use Theorem \ref{thm:equivariant model} to build a real model for the $G$-framed versions of $\FM_n$. 
% Concretely, we have the following immediate corollary.
% \begin{cor}\label{cor:framed models}
% $\stG_n\circ_m \hat \La_{\alg g} $ is a real model for the $G$-framed operads $\FM_n^{G-fr}$, where $\hat \La_{\alg g}$ is as in \eqref{equ:hatLaDef}.
% \end{cor}
% %TODO: mention the explicit map between these two models
% \begin{proof}
% TODO.
% \end{proof}


\section{The Maurer-Cartan element \texorpdfstring{$m$}{m}}\label{sec:MC}
Above we have seen that the study of the (real) homotopy type of the $O(n)$-action boils down to understanding the Maurer-Cartan element $m\in \BGC_n$. This section is hence devoted to studying (the gauge equivalence class of) $m$. 
In fact, we will see that $m$ is (gauge equivalent to) a quite trivial graphical Maurer-Cartan element.

\begin{thm}\label{conjthm:main}
\begin{itemize}
 \item For $n$ even and $G=O(n)$, the Maurer-Cartan element $m$ is gauge equivalent to $E\tadpole$, where $E\in H(B\SO(n))$ is the Euler class.
 \item For $n$ odd and $G=O(n)$, the Maurer-Cartan element $m$ is gauge equivalent to 
 \begin{equation}\label{equ:conjectured m odd}
 \sum_{j\geq 1}
 \frac{p_{2n-2}^{j}}{4^j}
\frac{1}{2(2j+1)!} 
\begin{tikzpicture}[baseline=-.65ex]
 \node[int] (v) at (0,.5) {};
 \node[int] (w) at (0,-0.5) {};
 \draw (v) edge[bend left=50] (w) edge[bend right=50] (w) edge[bend left=30] (w) edge[bend right=30] (w);
 \node at (2,0) {($2j+1$ edges)};
 \node at (0,0) {$\scriptstyle\cdots$};
\end{tikzpicture}
 \end{equation}
 with $p_{2n-2}\in H(B\SO(n))$ the top Pontryagin class.
 \end{itemize}
\end{thm}
 Since our construction is somewhat\footnote{The construction is as functorial as the version of equivariant forms we use.} functorial in the group $G\subset O(n)$, the above result also covers several other such groups as immediate corollaries.
 In particular, let us assemble some special cases in the following result.
 \begin{cor}\label{conjthm:main2}
\begin{itemize}
 \item For $n$ even and $G\subset \SO(n-1)$, $m$ is gauge trivial.
 \item For $n$ odd and $G\subset \SO(n-2)$, $m$ is gauge trivial.
 \item For $n$ odd and $G= \SO(n-1)$, $m$ is gauge equivalent to 
  \[
 \sum_{j\geq 1}
 \frac{E^{2j}}{4^j}
\frac{1}{2(2j+1)!} 
\begin{tikzpicture}[baseline=-.65ex]
 \node[int] (v) at (0,.5) {};
 \node[int] (w) at (0,-0.5) {};
 \draw (v) edge[bend left=50] (w) edge[bend right=50] (w) edge[bend left=30] (w) edge[bend right=30] (w);
 \node at (2,0) {($2j+1$ edges)};
 \node at (0,0) {$\scriptstyle\cdots$};
\end{tikzpicture}
 \]
 where $E\in H(BG)$ is the Euler class.
\end{itemize}
\end{cor}
\begin{proof}
 The result may be obtained from Theorem \ref{conjthm:main} by just restricting the coefficient ring from $H(B\SO(n))$ to its quotient $H(B\SO(m))$, $m<n$.
\end{proof}


We will prove Theorem \ref{conjthm:main} in several steps.


\subsection{Explicit computations of leading order terms}
We can use the explicit integral formulas \eqref{equ:MCelement} to understand the leading order terms of the MC element $m$.
Concretely, for $n=2k+1$ odd one may compute the coefficients of the graphs of the form 
\[
\begin{tikzpicture}[baseline=-.65ex]
 \node[int] (v) at (0,.5) {};
 \node[int] (w) at (0,-0.5) {};
 \draw (v) edge[bend left=50] (w) edge[bend right=50] (w) edge[bend left=30] (w) edge[bend right=30] (w);
 \node at (2,0) {($2r+1$ edges)};
 \node at (0,0) {$\scriptstyle\cdots$};
\end{tikzpicture}.
\]
\begin{lemma}
 The integral weight of the above graph is
\[
 \left(\frac{u_1\cdots u_k}{2(2\pi)^k}\right)^{2r} =: \frac{p_{2n-2}^r}{4^r}.
\]
Hence the coefficient of the same graph in $m$ (as in \eqref{equ:MCelement}) is
\[
 \frac 1 {2(2r+1)!} \frac{p_{2n-2}^r}{4^r}
\]

\end{lemma}
\begin{proof}
 The integral weight is the integral appearing in \eqref{equ:MCelement}.
In our case this integral takes the form
\[
 \int_{S^{2k}} \Omega_{sm}^{2r+1}.
\]
It is an integral of an equivariantly closed form over a manifold without boundary.
Hence we may use the Berline-Vergne equivariant localization formula (see \cite[Theorem 46]{Libine}) to evaluate the integral.
The fixed point set of the torus action consists of two points, the north and south pole of the sphere. 
By symmetry, both points contribute the same value in the localization formula. Denoting the north pole by $N$ temporarily, the integral hence evaluates to
\[
 2 \frac{(2\pi)^k}{u_1\cdots u_k} (\Omega_{sm}(N))^{2r+1},
\]
where the $2$ accounts for the contribution of the south pole and the remaining prefactor comes from the localization formula.
Using now Lemma \ref{lem:prop at north pole} in the Appendix, we evaluate the expression to
\[
 2 \frac{(2\pi)^k}{u_1\cdots u_k} \left(\frac {u_1\cdots u_k} {2(2\pi)^k}\right)^{2r+1}
=
\left(\frac {u_1\cdots u_k} {2(2\pi)^k}\right)^{2r}
=
\frac{p_{2n-2}^r}{4^r},
\]
where we defined the top Pontryagin class as 
\[
 p_{2n-2} := \frac {u_1\cdots u_k} {(2\pi)^k}.
\]
This immediately yields the coeffient of that graph in the formula for $m$, which differs only by a conventional combinatorial prefactor, which is the size of the symmetry group of the graph.
\end{proof}

\begin{rem}
 Let us quickly comment on the somewhat ``strange'' combinatorial prefactor occurring in the Lemma.
Note that in sum-of-graphs formulas such as \eqref{equ:MCelement} there appears over basis elements $\gamma$ of a space of graphs, and the corresponding dual elements $\gamma^*$ in the dual graph space.
Now, spaces of linear combinations of graphs come with a natural basis, given by (individual) graphs, and hence so do their dual spaces.
However, conventionally, in the identification of a graph as an element of the primal space, or as an element of the dual space, one often introduces a conventional combinatorial prefactor of size the order of the symmetry group of the graph.
This makes formulas for the differential and bracket in the dual complex more pretty.
We note however that this prefactor is purely conventional and could be absorbed in different conventions.
\end{rem}


% 
% Using the explicit form of the propagator of Appendix \ref{sec:explicitpropagator} we may compute the coefficients of these graphs by evaluating integrals of the form 
% \[
% \int_{S^{2k}} \Omega_{sm}^{2r+1} = \int_{\Delta_k\times (S^1)^k}
% \left(
% \sum_{I\subset \{1,\dots,k\}} \frac{C_n}{\Gamma(|I|+\frac 1 2)}  (\sqrt{\sigma_0})^{2|I|-1} \left( \prod_{i\in I} u_i\right) \left( \prod_{j\notin I}d\sigma_j d\phi_j \right)
% \right)^{2r+1}
% =
% c_r
% (u_1\cdots u_k)^{2r}
% \]
% where $c_r$ is a combinatorial prefactor. Let us note immediately that because all the constants appearing in the integrand are positive, so will be be the $c_r$.
% Note also that $(u_1\cdots u_k)^2=p_{2n-2}$ is the top Pontryagin class.
% 
% Concretely, the value of $c_r$ may be computed by the Berline-Vergne equivariant localization formula \cite[Theorem 46]{Libine}.
% Applying the formula in our case yields
% \begin{align*}
% \int_{S^{2k}} \Omega_{sm}^{2r+1}
% &= 
% 2 (-2\pi)^{k} \frac{(\Omega_{sm}^{2r+1})(p) }{u_1\cdots u_k}
% \\&=
% 2 (-2\pi)^{k} \frac{\left( \frac{C_n u_1\cdots u_k}{\Gamma(k+\frac 1 2)}  \right)^{2r+1} }{u_1\cdots u_k}
% \\&=
% 2 (-2\pi)^{k} \left( \frac{1}{\sqrt{2} \pi^{\frac {n-3} 2}} \right)^{2r+1} (u_1\cdots u_k)^{2r}.
% \end{align*}
% (TODO: Check again.)

% Moreover, one can show:
% 
% \begin{prop}
% Possibly after changing $m$ to a gauge equivalent MC element the following holds:
% For $n\geq 2$ even, the coefficients in $m$ of all primitive classes, i.e., of $E, p_4,p_8,\dots, p_{2n-4}$ are trivial.
% For $n\geq 5$ odd, the coefficients in $m$ of the classes in $p_4,p_8,\dots, p_{2n-6}$ are trivial, while the coefficient of the top Pontryagin class is a non-zero multiple of the theta graph 
% \[
%  \begin{tikzpicture}[baseline=-.65ex]
%  \node[int] (v) at (0,.5) {};
%  \node[int] (w) at (0,-0.5) {};
%  \draw (v) edge[bend left=50] (w) edge[bend right=50] (w) edge (w);
% \end{tikzpicture} .
% \]
% For $n\geq 3$, the top Pontryagin class is sent to a non-zero multiple of the theta graph, plus possibly higher order (i.e., with more vertices) corrections.
% \end{prop}
% 
% \begin{cor}
%  Possibly after changing $m$ to a gauge equivalent MC element the following holds:
%  \[
%   m = 
%  \sum_{j\geq 1}
%  p_{2n-2}^{j}
% \frac{1}{(2j+1)!} 
% \begin{tikzpicture}[baseline=-.65ex]
%  \node[int] (v) at (0,.5) {};
%  \node[int] (w) at (0,-0.5) {};
%  \draw (v) edge[bend left=50] (w) edge[bend right=50] (w) edge[bend left=30] (w) edge[bend right=30] (w);
%  \node at (2,0) {($2j+1$ edges)};
%  \node at (0,0) {$\scriptstyle\cdots$};
% \end{tikzpicture}
%  +(\text{higher order terms}).
%  \]
% \end{cor}
% 

\subsection{Proof of Theorem \ref{conjthm:main} for \texorpdfstring{$n=2$}{n=2} and \texorpdfstring{$n=3$}{n=3}}

\begin{prop}\label{prop:conjmain23}
Theorem \ref{conjthm:main} holds for $n=2$  and $n=3$.
\end{prop}
\begin{proof}
It is well known for $n=2$ \cite{pavolfr, GS}. In this case all integrals vanish by the Kontsevich vanishing Lemma \cite[Lemma 6.4]{K1}.

For $n=3$ we will see below that the Maurer-Cartan element $m$ is a deformation of the conjectured one (say $m^c$) above.
The result follows by showing that the $\Z_2$-invariant subspace of $\BGC_3=\GC_3[[u]]$ with $u=p_4$ the Pontryagin class of degree 4 and with the differential induced by $m^c$ has (essentially) no cohomology in degree 1, and hence the Maurer-Cartan element $m^c$ is not deformable.

To see the vanishing of the cohomology, note that by results of \cite{KWZ} the complex $\GC_3((u))$ is (essentially) acyclic.
In particular, the subcomplex of even loop order is acyclic.
Hence our complex $\GC_3[[u]]$ is quasi-isomorphic to $u^{-1}\GC_3[u^{-1}]=\GC_3((u))/\GC_3[[u]]$ up to a degree shift by one. 
But in $u^{-1}H(\GC_3)[u^{-1}]$ all classes of even loop order are represented by graphs with $\geq 3$-valent vertices, and hence are in too negative degrees.
\end{proof}



\subsection{An auxiliary Theorem}
Let us use the following notation:

\begin{itemize}
\item $Z_G^n\in \GC_n\otimes H(BG)$ is the Maurer-Cartan element describing the $G$ action on $E_n$, for $G$ a compact Lie group. Say $G$ is connected here for simplicity, if it is not, there is a slight adaptation. The tensor product here and below is a completed tensor product.
\item We abbreviate $Z_m^n:=Z_{\SO(m)}^n$ and $Z_{k,l}^n:=Z_{\SO(k)\times \SO(l)}^n$ for $k+l\leq n$.
\end{itemize}

Theorem \ref{conjthm:main} hence states that:
\begin{itemize}
\item For $n$ even, $Z_n^n$ is gauge equivalent to 
\[
Z^n_{conj} = E_n\tadpole.
\]
\item For $n$ odd, $Z_n^n$ is gauge equivalent to 
 \[
 Z^n_{conj} = (const)
 \sum_{j\geq 1}
 p_{2n-2}^{j}
\frac{1}{(2j+1)!} 
\begin{tikzpicture}[baseline=-.65ex]
 \node[int] (v) at (0,.5) {};
 \node[int] (w) at (0,-0.5) {};
 \draw (v) edge[bend left=50] (w) edge[bend right=50] (w) edge[bend left=30] (w) edge[bend right=30] (w);
 \node at (2,0) {($2j+1$ edges)};
 \node at (0,0) {$\scriptstyle\cdots$};
\end{tikzpicture}
\]
\end{itemize}

In particular, let us rephrase Corollary \ref{conjthm:main2} in this language.
\begin{cor}[Triviality of actions, special case of Corollary \ref{conjthm:main2}]
\label{cor:trivial}
We have $Z_m^n\sim 0$, where $\sim$ denotes gauge equivalence, in either of the two cases (i) $m\leq n-1$ and $n$ even or (ii) $m\leq n-2$ and $n$ odd. 
\end{cor}


Note that on $\GC_n$ we have a grading by loop order. Call the generator $L$ (it maps a graphs to the number of loops times that graph).
Then we have a map of dg Lie algebras
\begin{align*}
\Phi_{k,l}^n:  \GC_{n-k}\otimes H(B \SO(l)) &\to  \GC_n\otimes H(B(\SO(k)\times \SO(l)))
\\
\Gamma &\mapsto E_k^L \Gamma
\end{align*}
where $E_k$ is the Euler class in $H(B\SO(k))$.

\begin{rem}
Denote the (at this point of our proof conjectural) conjectural form of $Z_n^n$ of Theorem \ref{conjthm:main} by $Z^n_{conj}$. Then in particular
\beq{equ:PhiZ}
\Phi_{2,n-2}^n(Z_{conj}^{n-2}) = Z_{conj}^n.
\eeq
\end{rem}

The main claim is that using a version of equivariant localization one can show the following Theorem.
\begin{thm}\label{thm:locmain}\label{thm:mainloc}
We have that for $k\geq 0$ even and $l\geq 0$ such that $k<n$ and $k+l\leq n$
\[
Z_{k,l}^n \sim_{E_k} \Phi_{k,l}^n (Z_{l}^{n-k}).
\]
Here $\sim_{E_k}$ means "gauge equivalent after formally inverting $E_k$". In other words this is gauge equivalence in 
$\GC_n\otimes H(B(\SO(k)\times \SO(l)))_{E_k}$.
\end{thm}

Theorem \ref{thm:mainloc} will be proven in section \ref{sec:auxthmproof} below. For now, let us believe the statement and use it to derive our main Theorem \ref{conjthm:main}.

\subsection{Derivation of Theorem \ref{conjthm:main} from the auxiliary Theorem \ref{thm:locmain} }

\subsubsection{First "exercise": derivation of Corollary \ref{cor:trivial} from Theorem \ref{thm:locmain}}

Although not strictly speaking necessary, let us give an independent proof of Corollary \ref{cor:trivial} from Theorem \ref{thm:locmain}. Let us also proceed in unnecessary detail to prepare for the similar but more complicated proof of the general case.
To this end, use the case of the Theorem for $l=0$ and $k=n-2$ ($n$ even) or $k=n-3$ ($n$ odd).
 We find that in each of these cases
 \[
 Z_{k}^n \sim_{E_k} 0,
 \]
 using that by the standard Lemmas $Z_0^n=0$ for $n \geq 2$. (Note: $Z_0^1\neq 0$!)
 
Our remaining task is hence to get rid of the localization in $E_k$. In other words we want to show that the localized gauge equivalence implies the non-localized.

To this end we use the filtration on $\GC_n$ by the number of vertices and proceed by induction.
Evidently, or by explicit computation of the integrals, $Z_{k}^n$ does not contain terms with 1 or 2 vertices.
Assume inductively that, possibly after some gauge transformation we can bring $Z_{k}^n$ into a form without graphs with $< r$ vertices. To simplify the notation, we will then assume that $Z_{k}^n$ is of that form to start with.

The potential leading order term is of the form 
\[
\gamma = \sum_j e_j \gamma_j
\]
where $\gamma_j\in \GC_n$ are linear combinations of graphs with exactly $r$ vertices and $e_j$ range over a fixed (say monomial) basis of $H(B\SO(k))$.
The Maurer-Cartan equation then implies that $\gamma$ is closed, i.e., $\delta \gamma=0$. Our goal is to show that $\gamma$ is exact. If we can show that, say $\gamma=\delta \nu$, then we perform a gauge transformation by $\exp(\nu)$ and have shown our induction hypothesis for one larger $l$. (Hence we would be done by induction.)
Of course, since $\delta$ does not involve any non-trivial polynomial in the Euler and Pontryagin classes, closedness of $\gamma$ actually means that $\delta \gamma_j=0$ for each $j$, and we need to show that each $\gamma_j$ is separately exact.

Now we use that $Z_k^n\sim_{E_k} 0$. Concretely, the leading oder (with $l$ vertices) terms in $Z_k^n$ considered as element in the localized dg Lie algebra 
\[
\GC_n \otimes H(B\SO(k))_{E_k}
\]
are evidently also $\gamma$. Now by the gauge triviality $Z_k^n\sim_{E_k} 0$ we conclude that $\gamma$ is exact as an element of of the previous (localized) dg Lie algebra, i.e., there is a 
\[
\kappa \in   \GC_n \otimes H(B\SO(k))_{E_k}
\]
such that $\delta\kappa = \gamma$.
Concretely, we may extend the bases $e_j$ of $H(B\SO(k))$ above to a basis $H(B\SO(k))_{E_k}$ by adding some monomials $f_j$ which contain negative powers of $E_k$. (Here we use that the map $H(B\SO(k))\to H(B\SO(k))_{E_k}$ is injective.)
Then $\kappa$ will have the form 
\[
\kappa = \sum_j e_j\kappa_j + \sum_j f_j \kappa_j'.
\]
The equation $\delta\kappa = \gamma$ then says that 
\[
\delta \kappa_j =\gamma_j
\]
for each $j$ while $\delta \kappa_j'=0$.
But then we are done, we just pick 
\[
\nu := \sum_j e_j \kappa_j,
\] 
and this will satisfy $\delta\nu = \gamma$ as desired. So we can continue the induction and hence show the Corollary, except for one small issue:
When $n$ is even, we have shown that $Z_{n-2}^n\sim 0$, while we want $Z_{n-1}^n\sim 0$.
However, the only difference is that $H(B\SO(n-1))= H(B\SO(n-2))^{\Z_2}$, and picking the $\nu$ above $\Z_2$ invariantly (say by averaging) we can run the same proof working with $\Z_2$-invariant elements only. 

For later use, let us also remark that the main ingredient in the above proof was showing the following Lemma.
\begin{lemma}
The map
\[
(\GC_n \otimes H(B\SO(k)), \delta)
\to
(\GC_n \otimes H(B\SO(k))_{E_k}, \delta)
\]
induces an injective map in cohomology.
\end{lemma}

\subsubsection{Derivation of Conjecture \ref{conjthm:main} from Theorem \ref{thm:locmain}}
We proceed by induction on $n$. For $n=1$, $n=2$ and $n=3$ Theorem \ref{conjthm:main} is known.
Now we invoke Theorem \ref{thm:locmain} for $k=2$, $l=n-2$, assuming $n\geq 4$.\footnote{In fact, the case $n=3$ may also be tackled in this way, giving a second proof of the conjecture for $n=3$. However, in the interest of uniformity of notation, let us assume $n\geq 4$.}
The Theorem then states that
\[
 Z_{2,n-2}^n \sim_{u} \Phi_{2,n-2}^n (Z_{n-2}^{n-2}),
\] 
where we abbreviate the orthogonal Euler class by $u$ (it has degree $+2$).
Now by our induction hypothesis $Z_{n-2}^{n-2}\sim Z_{conj}^{n-2}$, and hence, using \eqref{equ:PhiZ} we find that
\beq{equ:tmp11}
Z_{2,n-2}^n \sim_{u}  R_{2,n-2}(Z_{conj}^n).
\eeq
%Next, note that clearly $Z_{2,n-2}^n$ is the image of $Z_n^n$ under the map 
Where $R_{2,n-2}$ is the map
\[
R_{2,n-2}\colon\GC_n\otimes H(B\SO(n)) \to \GC_n\otimes H(B(\SO(2)\times \SO(n-2))).
\]
Note also that clearly $Z_{2,n-2}^n=R_{2,n-2}(Z_n^n)$.
Let us be explicit how the underlying map of the coefficient rings looks like.
For $n$ even we have
\begin{gather*}
H(B\SO(n)) = \R[P_4,\dots,P_{2n-4}, E_n] \to H(B(\SO(2)\times \SO(n-2))) \cong \R[u, P_4,\dots,P_{2n-8},E_{n-2}]
\\
E_n \mapsto u E_{n-2} \\
P_{2n-4}\mapsto u^2 P_{2n-8}+ E_{n-2}^2 \\
P_j\mapsto u^2 P_{j-4} + P_j \quad \text{(for $j\neq 2n-4$)}\, .
\end{gather*}
In the above and in the formulas for $n$ odd below, we assume $P_0=1$.
For $n$ odd we have
\begin{gather*}
H(B\SO(n)) = \R[P_4,\dots,P_{2n-2}] \to H(B(\SO(2)\times \SO(n-2))) \cong \R[u, P_4,\dots,P_{2n-6}]
\\
P_{2n-2}\mapsto u^2 P_{2n-6} \\
P_j\mapsto u^2 P_{j-4} + P_j \quad \text{(for $j\neq 2n-2$)} \, .
\end{gather*}

Now localize the rings on the right hand side over $u$. We can then exchange the generator $E_{n-2}$ by $E_n:=uE_{n-2}$, respectively $P_{2n-6}$ by $P_{2n-2} := u^2P_{2n-6}$. This will make uniform the formula
for the differential for both left- and right-hand sides.
The maps above then change in that for $n$ even
\begin{align*}
E_n&\mapsto E_n \\ 
P_{2n-4}&\mapsto u^2 P_{2n-8}+ u^{-2} E_{n}^2 \\
P_j&\mapsto u^2 P_{j-4} + P_j 
\end{align*}
and for $n$ odd
\begin{align*}
P_{2n-2} &\mapsto P_{2n-2} \\  
P_{2n-6}&\mapsto u^2 P_{2n-10}+u^{-2} P_{2n-2} \\
P_j&\mapsto u^2 P_{j-4} + P_j . 
\end{align*}


Now we want to use \eqref{equ:tmp11}, or equivalently $R_{2,n-2}(Z_{n}^n) \sim_{u}  R_{2,n-2}(Z_{conj}^n)$, to show that $Z_{n}^n \sim  Z_{conj}^n$ by a similar but slightly more complicated argument than in the preceding subsection.
We have to make a case distinction according to whether $n$ is even or odd.
Suppose first that $n$ is even.
Then we perform an induction on the number of vertices in graphs, plus the power of $E_n$ in the coefficient.
Let us call the corresponding grading Euler-vertex degree.
%Call the resulting number the weight temporarily.
 Assume that $Z_{n}^n \sim Z_{conj}^n + (\dots)$, where $(\dots)$ are terms of Euler-vertex (EV-)degree $\geq r$.
To simplify the notation we will in fact assume that $Z_{n}^n = Z_{conj}^n + (\dots)$ (i.e., change the gauge so that the equation holds before proceeding).
We denote the terms of EV-degree exactly $r$ in $Z_{n}^n$ by $\gamma$.

We write 
\[
\gamma =\sum_j e_j \gamma_j
\]
where now the $e_j$ range over a basis of $H(B\SO(n-1))$ while 
\[
\gamma_j\in \GC_n \otimes \R[E_{n}].
\]
Note that this sum is finite essentially because for even $n$ graphs with multiple edges do not appear. 

The Maurer-Cartan equation implies that $D\gamma=0$, where $D=\delta+[Z_{conj}^n,-]=\delta+E_n\nabla$.
Our task is to show that $\gamma$ is $D$-exact. Since $D$ does not involve any polynomial in the Pontryagin classes we in fact have $D\gamma_j=0$ for each $j$ separately, and our goal is equivalent to showing that each $\gamma_j$ is separately $D$-exact.

Now by \eqref{equ:tmp11} we know that the image of $\gamma$ in  $(\GC_n \otimes \R[u,u^{-1}, P_4,\dots,P_{2n-8},E_n], D)$ is exact. We have hence reached our goal of showing exactness of $\gamma$ if we can show the following:
\begin{lemma}
The map of complexes
\beq{eq:map_even}
(\GC_n \otimes \R[P_4,\dots,P_{2n-4},E_n], D)
\to 
(\GC_n \otimes \R[u,u^{-1}, P_4,\dots,P_{2n-8},E_n], D)
\eeq
by mapping the coefficient ring according to the above prescription induces an injective map in cohomology.
\end{lemma}
\begin{proof}
To see this one proceeds as follows:
\begin{itemize}
\item As complexes, both sides have the form $(\dots)\otimes (\GC_n\otimes \R[E_n],D)$. Here 
the tensor product is the completed one. One should be a little bit careful: we have that the left-hand side
is a direct product of complexes isomorphic to $ (\GC_n\otimes \R[E_n],D)$ labeled by the basis of monomials of 
$\R[P_4,\dots,P_{2n-4}]$ and shifted respectively in their degree; while the right-hand side is something in between a direct sum and a direct product (labeled by monomials of $\R[u,u^{-1}, P_4,\dots,P_{2n-8}]$) as
a coefficient in front of any graph in $\GC_n \otimes \R[u,u^{-1}, P_4,\dots,P_{2n-8},E_n]$ is a finite sum
of monomials. But at the end it won\rq{}t matter. In our argument showing that a non-zero homology class
is sent to non-zero, we will be projecting to one of such factors/summands in the target. And it will be clear that the image of the projection is non-zero.
\item Pick the obvious monomial bases of $\R[P_4,\dots,P_{2n-4}]$ and $ \R[u,u^{-1}, P_4,\dots,P_{2n-8}]$. Impose the lexicographic ordering on these bases, with the ordering of the generating symbols such that  $P_i>P_j$ if $i>j$ and $P_i>u>u^{-1}$ for all $i$.
\item Consider any cocycle in the source that is not exact. Let $r$ be the smallest Euler-vertex degree in which this cocycle is non-trivial.
\item  Note that the map~\eqref{eq:map_even} respects the Euler-vertex degree only as a filtration.
Thus we will be looking below only at the factors/summands in the target that contribute non-trivially to
the Euler-vertex degree $r$.
%\item It is sufficient to show that the map induced by the leading order piece of the map $\R[P_4,\dots,P_{2n-4},E_n] \to \R[u,u^{-1}, P_4,\dots,P_{2n-8},E_n]$ is injective on cohomology.
\item Checking the formulas, the lexicographic leading order piece (landing in the Euler-vertex degree $r$) is given by the assignment 
\begin{align*}
P_{2n-4}\mapsto u^2 P_{2n-8} \\
P_j\mapsto P_j .
\end{align*}
It is clear that this induces an injective map on basis elements. Hence it is enough to project to the factor/summand corresponding to the leading order piece to see that the image homology class is non-zero.
\end{itemize}
\end{proof}
Thus we have shown Theorem \ref{conjthm:main} in the case of even $n$.

Next consider the case of odd $n$. Here we proceed similarly, but we pick the initial filtration on graphs on the number of vertices. The number of vertices will be referred as {\it  vertex grading}.
Assume that $Z_{n}^n \sim Z_{conj}^n + (\dots)$, where $(\dots)$ are terms with graphs with $\geq r$ vertices.
To simplify the notation we will in fact assume again that $Z_{n}^n = Z_{conj}^n + (\dots)$.

We call the term with exactly $r$ vertices $\gamma$ again.
We write 
\[
\gamma =\sum_j e_j \gamma_j
\]
where now the $e_j$ range over a basis of $H(B\SO(n-2))$ wile 
\[
\gamma_j\in \GC_n \otimes \R[P_{2n-2}].
\]
However, the difference with the previous case is that this sum might be infinite (because graphs with multiple edges are allowed compensating to the degree arising from products of Pontryagin classes). 

Now the Maurer-Cartan equation implies that $D\gamma=0$, where $D=\delta+[Z_{conj}^n,-]$.
Our task is to show that $\gamma$ is $D$-exact. Since $D$ does not involve any polynomial in the lower Pontryagin classes we in fact have $D\gamma_j=0$ for each $j$ and our goal is equivalent to showing that each $\gamma_j$ is separately $D$-exact.

Using \eqref{equ:tmp11} and proceeding as for even $n$ before, we end up with having to show the following result:
\begin{lemma}\label{l:map_odd}
The map 
\beq{eq:map_odd}
(\GC_n \otimes \R[P_4,\dots,P_{2n-2}], D)
\to 
(\GC_n \otimes \R[u,u^{-1}, P_4,\dots,P_{2n-10},P_{2n-2}], D)
\eeq
induces an injective map on cohomology. 
\end{lemma}

\begin{proof} 
As for the case of even $n$, we get that both the source and the target have the form $(\dots)\otimes (\GC_n\otimes \R[P_{2n-2}],D)$. Notice also that the vertex degree is preserved by the map~\eqref{eq:map_odd}.  Thus it is enough to prove injectivity for any cocycle concentrated
in a given vertex degree  of the source. However, we can not proceed similarly to the case of even $n$ as
one can now have coefficient monomials of arbitrary length, hence a top down induction is not permitted. 
%One is tempted to proceed as for even $n$, but this is not correct since now there can be infinitely many graphs with given number of vertices. (One has coefficient monomials of arbitrary length, hence a top down induction is not permitted.)
 To repair, we use the descending filtration in $(\GC_n\otimes \R[P_{2n-2}],D)$ by loop order. We will need the following.
 \begin{sublemma}\label{l:loop}
 For any cocycle $\alpha$ in $(\GC_n\otimes \R[P_{2n-2}],D)$, which is not exact, there exist an integer number $\ell<\infty$, such that $\alpha$
 is homologous to a cocycle $\alpha\rq{}$ represented by a sum of graphs of loop order $\geq \ell$, and is not homologous to any cocycle
 given by a sum of graphs of loop order $\geq \ell+1$.
 \end{sublemma}
 The number $\ell$ corresponding to a cocylce $\alpha$ given by the lemma above, will be called the loop order of a cocycle.
 
We finish the proof of Lemma~\ref{l:map_odd} by contradiction: Pick a cocycle $x$ in $(\GC_n \otimes \R[P_4,\dots,P_{2n-2}], D)$ of vertex degree $r$ which is not exact, but is sent to an exact element under the above map. The complex $(\GC_n \otimes \R[P_4,\dots,P_{2n-2}], D)$ is a product of complexes
isomorphic to $(\GC_n\otimes \R[P_{2n-2}],D)$ and labeled by monomials ${\overline P}^{\overline j}$ of $\R[P_4,\dots,P_{2n-6}]$. 
Thus $x$ is an infinite sum  of cocycles $x_{{\overline P}^{\overline j}}$ corresponding to each such factor. Now we look only at those factors for which $x_{{\overline P}^{\overline j}}$  has the minimal loop order. 
By dimensional reasons there will be only finitely many such factors. Among them we choose the one corresponding to the lexicographically maximal monomial $M$. Finally, we project the image of $x$ in the target
complex to the factor/summand labeled by the monomial obtained from $M$ by replacing
\begin{align*}  
P_{2n-6}&\mapsto u^2 P_{2n-10} \\
P_j&\mapsto  P_j . 
\end{align*}
The result is a cocycle $y$ that might be non-homologous to $x_M$, but still such that $y-x_M$ has a higher loop order than $x_M$. Thus $y$ is not exact, which brings in a contradiction.
% 
%We may assume that the representative has been chosen of maximal loop order within all representatives.
%(Furthermore, we may also assume the class has been chosen so as to minimize that loop order.)
%Then, proceeding as for even $n$ one shows that the leading (by loop order) piece can be killed, i.e., one can find a representative of higher loop order, and hence find a contradiction.
%(If we can find representatives of arbitrary high loop order the class is zero.)
%(The last statement uses that the spectral sequence abuts on a finite page, if we restrict to fixed number of vertices.)
\end{proof}

This then also shows Theorem \ref{conjthm:main} for odd $n$.
\hfill\qed

\begin{proof}[Proof of Sublemma~\ref{l:loop}]
This follows from the standard fact that if 
\[
F_0C\supset F_1C\supset F_2C \supset\ldots
\]
is a complete and Hausdorff  filtration in a (co)chain complex, such that all the terms $F_nC/F_{n+1}C$ are of finite type, then the induced filtration $F_\bullet H(C)$ in the (co)homology 
is also complete and Hausdorff. In fact we need  only Hausdorffness, meaning $\bigcap_n F_n H(C) =0$. Given a (co)cycle $x\in F_0C$, assume that it\rq{}s possible to subtract (co)boundaries $\partial y_0$, $\partial y_1$, etc,  so that
$(x-\partial y_0)\in F_1C$, $(x-\partial y_0-\partial y_1)\in F_2C$, etc. We want to show that $x$ is exact. Let us each time instead of  $y_i$ choose $z_i$ so that $\partial z_i=\partial y_i$ and $z_i$ lies in the maximal possible filtration term. Notice that the fact that it\rq{}s possible to choose such $y_i$  means that the differential in the  spectral sequence associated to this filtration is non-trivial. And moreover the filtration order of $z_i$ is responsible for the place from where this differential is sent.  Since each term 
$F_nC/F_{n+1}C$ is of finite type, there will be finitely many non-trivial arrows in this spectral sequence from any given cell. Thus for any $n$ there will be only finitely many $z_i$\rq{}s not lying in $F_nC$. As a result, the sum $z_0+z_1+z_2+\ldots$ is well defined (also by completeness). 
\end{proof}

% \begin{rem}
% A point that I am uncertain about: One would think that the statement of Theorem \ref{thm:locmain} is stronger the larger $k$ you take.
% However, if for $n$ odd we take the maximum $k=n-1$, then actually I cannot derive the main Conjecture.
% It is a bit puzzling, so one must check carefully there is no mistake.
% \end{rem}
% 
% FOR US TODO: The above arguments are somewhat messy and I would like to have a cleaner (or at least as clean as possible) derivation.

\newcommand{\hFM}{\widehat{\FM}}


\section{Equivariant localization and proof of Theorem \ref{thm:locmain}} \label{sec:auxthmproof}
\subsection{A relative version of configuration space}
Consider $\R^m$ as a fixed subset of $\R^n$ by embedding it as a coordinate hyperplane for the first $m$ coordinates.
We define the space
\[
\FM_{m,n}(r,s) \subset \FM_n(r+s)
\]
as the subspace for which the last $s$ points lie on a plane $\R^m\subset \R^n$.
More precisely, the space $\FM_{m,n}(r,s)$ fits into a pullback diagram
\[
 \begin{tikzcd}
\FM_{m,n}(r,s) \ar{r}\ar{d} & \FM_n(r+s) \ar{d} \\
\FM_m(s) \ar{r} & \FM_n(r)
 \end{tikzcd}\, .
\]

Following Kontsevich's notation we call the $r$ first points \emph{type I points} and the others $s$ points (which lie in a plane) \emph{type II points}.
The totality of spaces $\FM_{m,n}(-,-)$ together with $\FM_n$ forms a colored operad $\hFM_{m,n}$ such that
\begin{itemize}
\item The operations with output in color 1 are
\[
\hFM_{m,n}^1(r,s)=
\begin{cases}
\FM_n(r) & \text{for $s=0$} \\
\emptyset & \text{otherwise}
\end{cases}.
\]
\item The operations with output in color 2 are
\[
\hFM_{m,n}^2(r,s)=
\FM_{m,n}(r,s) \quad \text{for $r\geq 0$ and $s\geq 1$}.
\]
\end{itemize}
The operadic compositions are inherited from those on $\FM_n$, i.e., defined by gluing one configuration into another.

\begin{rem}
There is also a variant of the above colored operad in which one allows for operations with output in color 2 but no input in color 2.
The definition of the appropriate compactification in that case is slightly more intricate, however. In this paper we only need to work with the version above.
\end{rem}

Obviously, the colored operad $\hFM_{m,n}$ is equipped with a natural action of $O(m)\times O(n-m)$, by restriction of the $O(n)$ action on $\FM_n$.


% 
% FOR US: The colored operad $\FM_{m,n}$ is a variant of the Swiss-Cheese operad we already discussed for a different project. (Namely that the relative formality is equivalent to $\FM_{m,n}$ being formal.) I forgot who introduced these operads,... was it Budney? Do you have the reference?


% \begin{rem}
%  More generally, one can define for any operad map $f: \op Q\to \op P$, with $\op P$ equipped with a $\Lambda$-structure \cite{Fresse}, a two-colored operad $\op X_f$ extending $\op P$ in color one, with the operations in color 2 fitting into a pullback diagram
% \[
%   \begin{tikzcd}
% \op X^2_f(r,s) \ar{r}\ar{d} & \op P(r+s) \ar{d}{\Lambda} \\
% \op Q(s) \ar{r}{f} & \op P(s)
%  \end{tikzcd}\, .
% \]
% In our case (i.e., $\op X=\hFM_{m,n}$) the underlying operad map is $\FM_m\to \FM_n$.
% \end{rem}

\begin{cons}\label{cons:twocolopfromop}
Let $\op P$ be an operad. Let us define a two-colored operad $\op P^{2-col}$, such that 
\begin{align*}
 \op P^{2-col,1}(r,s) &=
\begin{cases}
 \op P(r) & \text{if $s=0$} \\
0& \text{otherwise}
\end{cases}
&
 \op P^{2-col,2}(r,s) &=
\begin{cases}
 \op P(r+s) & \text{if $s\geq 1$} \\
0& \text{otherwise},
\end{cases}
\end{align*}
with the operadic compositions inherited from $\op P$. Dually, given a cooperad $\op C$, we define a two colored cooperad $\op C^{2-col}$ by the analogous construction.
\end{cons}




\newcommand{\stGra}{{}^*\Gra}
%\newcommand{\xGra}{\mathsf{xGra}}
%\newcommand{\hGra}{\widehat{\Gra}}
%\newcommand{\hxGra}{\widehat{\xGra}}

%\newcommand{\stxGra}{{}^*\xGra}
%\newcommand{\fxGC}{\mathsf{xfGC}}
%\newcommand{\xGC}{\mathsf{xGC}}

\newcommand{\sthGra}{\widehat{\stGra}}
%\newcommand{\sthxGra}{\widehat{\stxGra}}

\newcommand{\hZ}{\hat Z}
%\newcommand{\hxZ}{\hat{xZ}}


\subsection{A complex of graphs}
Recall the cooperad $\stGra_n$ from section \ref{sec:GraDefinitions}. Set $G=\SO(m)\times \SO(n-m)$, fix a maximal torus $T$ and compact subgroup $K\cong W\ltimes T$, with $W$ the Weyl group as before.
First let us define a two colored cooperad $\stGra_{m,n}=\stGra_n^{2-col}$ from $\stGra_n$ using the construction of Construction \ref{cons:twocolopfromop}.
More concretely, we define a family of graded vector spaces $\stGra_{m,n}(r,s)=\stGra_n(r+s)$ consisting of graphs in $r$ "type I" and $s$ "type II" vertices, with the same sign and degree conventions as for $\stGra_n$. 
In pictures, we shall distinguish the type II vertices by drawing them on a "baseline", which shall be thought of representing $\R^m$, as follows
\[
\begin{tikzpicture}
\draw (-1,0) -- (1,0);
\node[ext] (u) at (-.5,0) {$1$};
\node[ext] (v) at (.5,0) {$2$};
\node[ext] (w) at (-.5,.5) {$1$};
\node[ext] (x) at (.5,.5) {$2$};
\draw (u) edge[bend left] (v) edge (w) (w) edge (v) edge(x) (x) edge (v) edge (u);
\end{tikzpicture}
\]

The pair $\stGra_n(-)$ and $\stGra_{m,n}(-,-)$ is naturally a two colored Hopf cooperad, which we call $\sthGra_{m,n}$.
There is a map of Hopf cooperads
\beq{equ:sthGramap}
\sthGra_{m,n}\otimes H(BG) \to \Omega_K^{s,PA} (\hFM_{m,n})
\eeq
sending an edge between vertices $i$ and $j$ to the form 
\[
\pi_{ij}^* \Omega, 
\]
where $\pi_{ij}$ is the forgetful map forgetting all but points $i$ and $j$ from a configuration, and $\Omega$ is the ``propagator'', the $G$-equivariant form on $S^{n-1}$ as in section \ref{sec:propagator}.

%Next restrict to even $n-m$ for simplicity. (The case $n-m$ odd can be handled, but we will not use it anyway.)
%Define 
%\[
%\stxGra_{m,n}(r,s)=\stGra_{m,n}(r,s) \otimes (\wedge \R[n-m-1])^{\otimes r}.
%\]
%Combinatorially, this shall be interpreted as graphs having an optional marking of degree $n-m-1$ on the type II vertices.
%In pictures, we shall indicate the marking by drawing a dashed line to the baseline as follows.
%\[
%\begin{tikzpicture}
%\draw (-1,0) -- (1,0);
%\node[ext] (u) at (-.5,0) {$1$};
%\node[ext] (v) at (.5,0) {$2$};
%\node[ext] (w) at (0,.7) {$1$};
%\draw (w) edge (v) edge (u)  edge[dashed] (0,0);
%\end{tikzpicture}
%\]
%The vector spaces $\stGra_n(-)$ and $\stxGra_{m,n}(-,-)$ again form a two-colored coperad $\sthxGra_{m,n}$: one just extends the cocomposition naturally such possible markings "remain at the vertex as far as possible", with the convention that double markings make the graph zero.
%
%We can extend the map \eqref{equ:sthGramap} to a map of graded dgcas, compatible with the cooperadic cocomposition
%\beq{equ:sthxGramap}
%\sthxGra_{m,n}\otimes H(BG) \to \Omega_{PA}^G (\hFM_{m,n})
%\eeq
%by assigning to each marking at vertex $j$ the pullback of the $\SO(n-m)$-equivariant volume form $\Omega_{sm}^{S^{n-m-1}}$ under the projection
%\[
%\pi_j : \FM_{m,n}(r,s) \to S^{n-m-1}
%\]
%by projecting the coordinates of the $j$-th point on the orthogonal plane $\R^{n-m}$.
%Concretely, if the $j$-th point has coordinates $(x_1,\dots,x_n)$, then the corresponding point on $S^{n-m-1}$ is
%\[
%\frac {(x_{m+1},\dots,x_{n})}{\sqrt{ x_{m+1}^2 + \cdots + x_{n}^2 } }.
%\]
%A graph with markings at type II vertices is assigned the zero form, unless the type II vertex is the only vertex, then one assigns $1$.

We also define the dual two colored operad $\hGra_{m,n}$.
We consider the graded Lie algebras of invariants of those colored operads.
To describe them correctly including signs and degrees, consider the two colored operad $\Lie_{m,n}$ governing a $\Lie_m$ algebra acted upon by a $\Lie_n$ algebra. 
Concretely, we have
\begin{align*}
\Lie_{m,n}^1(r,s) &=
\begin{cases}
\Lie_n(r) &\text{for $s=0$} \\
0 &\text{otherwise}
\end{cases}
\\
\Lie_{m,n}^2(r,s) &=
\begin{cases}
\Lie_m(s) &\text{for $r=0$ and $s\geq 1$} \\
\Lie_n(r) \otimes \Lie_m(s)[n-m-1] &\text{for $r\geq 1$ and $s\geq 1$} \\
0 &\text{otherwise}
\end{cases}
\end{align*}
Note in particular that there is no operation with output in color 2, but no input in color 2.
We denote the minimal resolution by $\hoLie_{m,n}$.
Concretely, $\hoLie_{m,n}$ is generated by the following operations: 
\begin{itemize}
 \item Operations $\mu_k$ with $k\geq 2$ inputs in color 1 and the output in color one, spanning a one-dimensional representation of $S_k$ in degree $1-(k-1)n$. The operations generate $\hoLie_n$.
\item Operations $\mu_{k,l}$ with $k$ inputs in color one, $l$ inputs in color 2 and output in color 2, where $k\geq 0$, $l\geq 1$, $k+l\geq 2$. The operation $\mu_{k,l}$ has degree $1-kn-(l-1)m$, and spans a one-dimensional subspace under the action of the group $S_k\times S_l$.
The $\mu_{0,l}$ generate a copy of $\hoLie_m$ inside $\hoLie_{m,n}$.
\end{itemize}


Then the invariant Lie algebra can be defined as the deformation complex
\begin{align*}
\fGC_{m,n} := \Def(\hoLie_{m,n}\stackrel{0}\to \hGra_{m,n})
%\\
%\fxGC_{m,n} := \Def(\hoLie_{m,n}\stackrel{0}\to \hxGra_{m,n})
\end{align*}
of the trivial maps sending all generators to zero. (This is just the invariants of the total space, up to some degree shifts.)
Via the map \eqref{equ:sthGramap} we obtain a Maurer-Cartan element
\begin{align*}
\hZ_{m,n} &= \sum_\Gamma \Gamma \int \omega_{\Gamma^*} \in \fGC_{m,n}\otimes H(BG).
\end{align*}

From this point on, let us only focus on the case of even $n-m$, which is what we need below. (FOR US: The case of odd codimension $n-m$ can also be looked at, but it lacks a couple of nice features, and I do not know how to reproduce the results below in that case. We can discuss in person if you want.) 
%$\fxGC_{m,n}$ and $\hxZ_{m,n}$, which is all we need below.
To be explicit, the leading order terms of $\hZ_{m,n}$ are
\beq{equ:hZmn}
\hZ_{m,n} = 
\underbrace{
\begin{tikzpicture}
\node[int] (v) at (0,0) {};
\node[int] (w) at (0.6,0) {};
\draw (v) edge (w);
\end{tikzpicture}
+
\begin{tikzpicture}
\draw (-.5,0) -- (.5,0);
\node[int] (v) at (0,0) {};
\node[int] (w) at (0,.5) {};
\draw (v) edge (w);
\end{tikzpicture}
+
%\begin{tikzpicture}
%\draw (-.5,0) -- (.5,0);
%\node[int] (w) at (0,.5) {};
%\draw (w) edge[dashed] (0,0);
%\end{tikzpicture}
+
E_{n-m}
\begin{tikzpicture}
\draw (-.5,0) -- (.5,0);
\node[int] (v) at (-.3,0) {};
\node[int] (w) at (0.3,0) {};
\draw (v) edge[bend left] (w);
\end{tikzpicture}
%+
%E_{n-m}
%\begin{tikzpicture}
%\draw (-.5,0) -- (.5,0);
%\node[int] (v) at (0,0) {};
%\draw (v) edge [loop, dashed] (v);
%\end{tikzpicture}
}_{=:\hZ_{m,n}^0}
+ (\cdots)
\eeq
All terms are given by connected graphs, as one easily verifies. (Including the case $m=1$.)
Note that the second term reflects the fact that to the marking one assigns a form that is not equivariantly closed.
We denote the leading terms by $\hZ_{m,n}^0$ as indicated in the formula, and regard the remainder $\hZ_{m,n}-\hZ_{m,n}^0$ as a perturbation of $\hZ_{m,n}^0$. One easily verifies that $\hZ_{m,n}^0$ is itself a Maurer-Cartan element.

\begin{rem}
 The leading term $\hZ_{m,n}^0$ is the Maurer-Cartan element corresponding to the colored operad map
\[
 \hoLie_{m,n} \to \Lie_{m,n} \xrightarrow{f} \hGra_{m,n}\otimes H(BG)
\]
where $f$ maps the generators as follows:
\begin{align*}
 f(\mu_2) &=
\begin{tikzpicture}
\node[ext] (v) at (0,0) {};
\node[ext] (w) at (0.6,0) {};
\draw (v) edge (w);
\end{tikzpicture}
&
  f(\mu_{1,1}) &=
\begin{tikzpicture}
\draw (-.5,0) -- (.5,0);
\node[ext] (v) at (0,0) {};
\node[ext] (w) at (0,.5) {};
\draw (v) edge (w);
\end{tikzpicture}
&
f(\mu_{0,2}) &=
E_{n-m}
\begin{tikzpicture}
\draw (-.5,0) -- (.5,0);
\node[ext] (v) at (-.3,0) {};
\node[ext] (w) at (0.3,0) {};
\draw (v) edge[bend left] (w);
\end{tikzpicture}\, .
\end{align*}
\end{rem}


\begin{rem}
 Let us collect several maps between the operads constructed so far.
First, there are obvious inclusions 
\begin{align}\label{equ:Liemninclusions}
 \Lie_m&\to \Lie_{m,n} & \Lie_n&\to \Lie_{m,n}\, ,
\end{align}
interpreting the left-hand side in each case as a colored operad concntrated in color 2 (respectively, color 1).

Next, suppose that $R$ is any ring containing an element $\lambda$ of degree $n-m$. Recall that we require $n-m$ to be even. Then there is a colored operad map (cf. also Construction \ref{cons:twocolopfromop})
\begin{equation}
 \label{equ:Liemn2Lie}
\Lie_{m,n} \to \Lie_n^{2-col} \otimes R \,.
\end{equation}
This map is defined on generators as follows:
\begin{align*}
 \mu_2 &\mapsto \mu_2 
&
\mu_{1,1} &\mapsto \mu_2
&
\mu_{0,2} &\mapsto \lambda \mu_2.
\end{align*}

\end{rem}


\subsection{Combinatorial description of \texorpdfstring{$\fGC_{m,n}$}{fGCmn}}
The graded Lie algebra $\fGC_{m,n}$ has a semi-direct product structure owed to its definition as a deformation complex of a colored operad.
Concretely, as graded Lie algebra
\beq{equ:fsplitting}
\fGC_{m,n} = \fGC_n \ltimes \fGC_{m,n}'
\eeq
where $\fGC_n$ are graphs "without baseline", while $\fGC_{m,n}'$ is spanned by graphs with baseline, with at least one vertex on the baseline.
The Lie bracket on $\fGC_{m,n}'$ is by inserting into type II (i.e., baseline-)vertices. The Lie action of $\fGC_n$ is by insertion into type I vertices.

Changing the ground ring to $H(BG)$ and twisting by the Maurer-Cartan element $\hZ_{m,n}$ produces several terms in the differential.
Among them are the differential on $\fGC_n$, and terms sending $\fGC_n\otimes H(BG)\to \fGC_{m,n}'\otimes H(BG)$.

\subsection{Connectedness and the dg Lie subalgebra \texorpdfstring{$\GC_{m,n}$}{GCmn} }
We define the connected dg Lie subalgebra $\GC_{m,n}\subset \fGC_{m,n}$ to be composed of connected graphs.
Here a graph counts as connected if any two vertices can be connected by a path of edges, irrespective of the vertex types (I or II).


Similar to \eqref{equ:fsplitting} we have a splitting of $\GC_{m,n}$ into a semi-direct product (as graded Lie algebra) 
\beq{equ:splitting}
\GC_{m,n} = \fGCc_n \ltimes \GC_{m,n}'
\eeq
where $\fGCc_n$ is the standard (connected) graph complex, but without any valence restriction on vertices.

\begin{lemma}
The Maurer-Cartan element $\hZ_{m,n}$ lives inside the connected part $\GC_{m,n}\subset \fGC_{m,n}$.
\end{lemma}
\begin{proof}
Suppose $\Gamma=\Gamma_1\sqcup \Gamma_2$ is a non-connected graph, with the pieces $\Gamma_1,\Gamma_2$ non-empty and not connected to each other. Then the corresponding weight form $\omega_\Gamma=\omega_{\Gamma_1}\wedge \omega_{\Gamma_2}$ is 
basic under rescaling and translation of the points contributing to $\Gamma_1$ and $\Gamma_2$ \emph{separately}. 
Hence the form can not have a top form component on configuration space, which is obtained by quotienting out (only) the diagonal scaling and translation action. 
\end{proof}

\begin{rem}
For cosmetic reasons one could introduce the following further valence conditions:
(i) Every type I vertex has valence $\geq 2$ (respectively $\geq 3$) and (ii) every type II vertex that is not connected to a type I vertex has valence $\geq 2$ (resp. $\geq 3$).
One can easily check that these conditions describe a Lie subalgebra $\GC_{m,n}^{\geq 2}\subset \GC_{m,n}$ (resp. $\GC_{m,n}^{\geq 3}\subset \GC_{m,n}$).
Furthermore, it is shown in the Appendix that the non-leading piece of the MC element $\hZ_{m,n}-\hZ_{m,n}^0$ lives in the Lie subalgebra $\GC_{m,n}^{\geq 2}$.
(And in fact also in $\GC_{m,n}^{\geq 3}$ if one absorbs one further term into the leading piece.)
However, to use a somewhat unified notation, we will stick to the version of $\GC_{m,n}$ without valence condition for now.
\end{rem}


\subsection{A path object}
Let us now work over the localized coefficient ring $H(BG)_{E_{n-m}}$, formally inverting the orthogonal Euler class.
\begin{prop}\label{prop:pathobject}
The dg Lie algebra $\alg g:=(\GC_{m,n} \otimes H(BG)_{E_{n-m}})^{\hZ_{m,n}^0}$ is a path object for $\alg h=\fGCc_n \otimes H(BG)_{E_{n-m}}$.
This means that there are morphisms of dg Lie algebras factoring the diagonal
\[
\begin{tikzcd}
\alg h \ar{r}{\iota}[swap]{\sim} & \alg g 
%\ar[shift left=2]{r}{p_0} \ar[shift right=2]{r}[shift right, swap]{p_1} 
\ar[two heads]{r}{(p_0,p_1)}& \alg h \times \alg h
\end{tikzcd}
\]
with the right-hand map surjective and the left-hand map an (in our case injective) quasi-isomorphism.
\end{prop}
The left-hand map $\iota$ sends a graph to the sum of all graphs obtained by declaring an arbitrary subset of vertices to be of type II, multiplying by $E_{n-m}^k$, where $k$ is the number of type II vertices. For example:
\[
\begin{tikzpicture}
\node[int] (v1) at (45:.5){};
\node[int] (v2) at (135:.5){};
\node[int] (v3) at (225:.5){};
\node[int] (v4) at (-45:.5){};
\draw (v1) edge (v2) edge (v3) edge (v4) (v2) edge (v3) edge (v4) (v3) edge (v4);
\end{tikzpicture}
\stackrel{\iota}{\mapsto}
\begin{tikzpicture}
\node[int] (v1) at (45:.5){};
\node[int] (v2) at (135:.5){};
\node[int] (v3) at (225:.5){};
\node[int] (v4) at (-45:.5){};
\draw (v1) edge (v2) edge (v3) edge (v4) (v2) edge (v3) edge (v4) (v3) edge (v4);
\end{tikzpicture}
+
E_{n-m}
\begin{tikzpicture}
\draw (-1,-.7) -- (1,-.7);
\node[int] (v1) at (45:.5){};
\node[int] (v2) at (135:.5){};
\node[int] (v3) at (225:.5){};
\node[int] (v4) at (.5,-.7){};
\draw (v1) edge (v2) edge (v3) edge (v4) (v2) edge (v3) edge (v4) (v3) edge (v4);
\end{tikzpicture}
+
E_{n-m}^2
\begin{tikzpicture}
\draw (-1,-.7) -- (1,-.7);
\node[int] (v1) at (45:.5){};
\node[int] (v2) at (135:.5){};
\node[int] (v3) at (-.5,-.7){};
\node[int] (v4) at (.5,-.7){};
\draw (v1) edge (v2) edge (v3) edge (v4) (v2) edge (v3) edge (v4) (v3) edge[bend left] (v4);
\end{tikzpicture}
+
E_{n-m}^3
\begin{tikzpicture}
\draw (-1,-.7) -- (1,-.7);
\node[int] (v1) at (45:.5){};
\node[int] (v2) at (0,-.7){};
\node[int] (v3) at (-.5,-.7){};
\node[int] (v4) at (.5,-.7){};
\draw (v1) edge (v2) edge (v3) edge (v4) (v2) edge[bend right] (v3) edge[bend left] (v4) (v3) edge[bend left] (v4);
\end{tikzpicture}
+
E_{n-m}^4
\begin{tikzpicture}
\draw (-1,-.7) -- (1.5,-.7);
\node[int] (v1) at (1,-.7){};
\node[int] (v2) at (0,-.7){};
\node[int] (v3) at (-.5,-.7){};
\node[int] (v4) at (.5,-.7){};
\draw (v1) edge[bend right] (v2) edge[bend right] (v3) edge[bend right] (v4) (v2) edge[bend right] (v3) edge[bend left] (v4) (v3) edge[bend left] (v4);
\end{tikzpicture}
\]
The first right-hand map $p_0$ is the projection to $\alg h$ that projects to the first factor of \eqref{equ:splitting}.
The map $p_1$ is the projection to the piece where all vertices are type II, multiplying by $E_{n-m}^{-k}$, where $k$ is the number of type II vertices, sending all graphs with type I vertices to zero. For example:
\begin{align*}
\begin{tikzpicture}
\draw (-1,-.7) -- (1,-.7);
\node[int] (v1) at (45:.5){};
\node[int] (v2) at (135:.5){};
\node[int] (v3) at (-.5,-.7){};
\node[int] (v4) at (.5,-.7){};
\draw (v1) edge (v2) edge (v3) edge (v4) (v2) edge (v3) edge (v4) (v3) edge[bend left] (v4);
\end{tikzpicture}
&\stackrel{p_1}{\mapsto}
0
&
\begin{tikzpicture}[yshift=.7cm]
\draw (-1,-.7) -- (1.5,-.7);
\node[int] (v1) at (1,-.7){};
\node[int] (v2) at (0,-.7){};
\node[int] (v3) at (-.5,-.7){};
\node[int] (v4) at (.5,-.7){};
\draw (v1) edge[bend right] (v2) edge[bend right] (v3) edge[bend right] (v4) (v2) edge[bend right] (v3) edge[bend left] (v4) (v3) edge[bend left] (v4);
\end{tikzpicture}
\stackrel{p_1}{\mapsto}
E_{n-m}^{-4}
\begin{tikzpicture}
\node[int] (v1) at (45:.5){};
\node[int] (v2) at (135:.5){};
\node[int] (v3) at (225:.5){};
\node[int] (v4) at (-45:.5){};
\draw (v1) edge (v2) edge (v3) edge (v4) (v2) edge (v3) edge (v4) (v3) edge (v4);
\end{tikzpicture}
\end{align*}

\begin{proof}
It is an exercise to check that the maps respect the dg Lie structure:
The easiest way is to conduct a small graphical computation. Alternatively, to see that $p_0$ and $p_1$ are dg Lie algebra maps, one uses the representation of the graph complex as deformation complex. Then $p_0$ and $p_1$ are essentially pull-backs under the inclusions \eqref{equ:Liemninclusions}.
For the map $\iota$ a similar but slighter longer argument is possible, using the map \eqref{equ:Liemn2Lie}, inducing the second arrow in the composition
\begin{multline*}
 \fGC_{n}\otimes H(BG)_{E_{n-m}}= \Def(\hoLie_n\to \Gra_n\otimes H(BG)_{E_{n-m}})
\to \Def(\hoLie^{2-col}_n\to \Gra_n^{2-col}\otimes H(BG)_{E_{n-m}})
\\
\to 
\Def(\hoLie_{m,n}\to \Gra_n^{2-col}\otimes H(BG)_{E_{n-m}}) \cong \fGC_{m,n}\otimes H(BG)_{E_{n-m}}.
\end{multline*}


 
Next let us show that the maps $\iota, p_0, p_1$ are quasi-isomorphisms, which is a less straightforward statement.
First note that it suffices to show that $p_1$ is a quasi-isomorphism, because then (by 2-out-of-3) so is $\iota$, and then (by 2-out-of-3 again) so is $p_0$.

So let us check that $p_1$ is a quasi-isomorphism.
Consider a univalent type II vertex attached to a type I vertex as a "marking" of that type I vertex.
Now take a spectral sequence on the total number of edges plus vertices, disregarding the markings. (I.e., the univalent type II vertices and their attaching edges to a type I vertex don't contribute to the count).
Then the differential $\delta_0$ on the associated graded of $\alg g$ becomes
\[
\delta_0 : \Gamma \mapsto E_{n-m} \sum_v \Gamma \sqcup(\text{add marking at vertex $v$})
\]
where the sum is over all type I vertices. In words, we add a marking to one type I vertex, summing over all choices of such vertex.
Note also that in particular, if the graph $\Gamma$ above is in the first summand on the right-hand side of \eqref{equ:splitting}, this operation sends it to a linear combination of graphs in the second summand of the right-hand side of \eqref{equ:splitting}.
Now consider the operation $h_0'$ by summing over all vertices and removing one marking (if one is present).
By a simple computation:
\[
(\delta_0 h_0' + h_0' \delta_0)(\Gamma) = (\text{\# of type I vertices}) E_{n-m} \Gamma.
\]
Hence the operation 
\[
h_0: \Gamma \mapsto
\begin{cases}
0 & \text{if $\Gamma$ has no type I vertices}\\
\frac 1 {(\text{\# of type I vertices})} E_{n-m}^{-1} h_0'(\Gamma) &\text{otherwise}
\end{cases}
\]
is a homotopy for $\delta_0$, in the sense that $\delta_0 h_0 + h_0\delta_0=\mathit{id}-\pi$, where $\pi$ is the projection onto the subspace spanned by graphs without type II vertices. That means that on the level of associated graded spaces the map $p_1$ induces an isomorphism on cohomology
%\footnote{Concretely, this isomorphism is multiplication by $E_{n-m}^{(\text{\# of type II vertices})}$.}
and hence the spectral sequence collapses here. 
\end{proof}

The following result is evident from the definitions, but let us still call it a lemma.
\begin{lemma}\label{lem:MCelementspath}
The images of the Maurer-Cartan element $\hZ_{m,n}-\hZ_{m,n}^0$ under the maps $p_0$, $p_1$ of the preceding Proposition \ref{prop:pathobject} are as follows.
\begin{align*}
p_0(\hZ_{m,n}-\hZ_{m,n}^0) &= Z_{m,n}^n \\
p_1(\hZ_{m,n}-\hZ_{m,n}^0) &= L^{E_{n-m}} Z_{m}^m 
\end{align*}
\end{lemma}

Now the proof of Theorem \ref{thm:mainloc} is simple.

\begin{proof}[Proof of Theorem \ref{thm:mainloc}]
It suffices to show the statement for $l=n-k$.
By definition, two Maurer-Cartan elements $x,y$ in a dg Lie algebra $\alg h$ are gauge equivalent if there is a path object $\alg g$ together with a MC element $z\in \alg g$ such that $p_0(z)=x$ and $p_1(z)=y$, where $p_0,p_1:\alg g\to \alg h$ are the two maps in the definition of path object.
Typically one takes $\alg g=\alg h[t,dt]$, but any other path object is fine, see Appendix \ref{app:pathobjects}.

In our case we take the path object of Proposition \ref{prop:pathobject} (with $m=k$).
Then Lemma \ref{lem:MCelementspath} says that the MC elements are gauge equivalent in $\fGCc_n\otimes H(BG)$.
Finally, since $\fGCc_n$ and $\GC_n^{2}$ are quasi-isomorphic, one concludes the result.
(TODO: Here one should be a little more careful... the cleanest is to restrict the valence of vertices in $\GC_{m,n}$ from the start.) 
\end{proof}




%\section{A first application: Computation of equivariant cohomology}
%Let us, as a first application of the above methods, compute the $\SO(n)$-equivariant cohomology of the little $n$-cubes operads.
%We will use the model 
%\[
% \left(H(B\SO(n))\otimes \stG_n\right)^{m}.
%\]
%for the cochains on $E\SO(n)\times_{\SO(n)} \FM_n$.
%Note that there is a map 
%\[
% f: \left(H(B\SO(n))\otimes \stG_n\right)^{m} \to (H(B\SO(n))\otimes \e_n^*, E \otimes T\cdot)
%\]
%induced by the projection $\stG_n\to \e_n$.
%\begin{lemma}
%The map $f$ is a quasi-isomorphism.
%\end{lemma}
%\begin{proof}
%We may take a spectral sequence on the degree of the coefficient in $H(B\SO(n))$. On the associated graded the differential will be that of $\stG_n$. The cohomology of the associated graded is hence $H(B\SO(n))\otimes \e_n^*$. 
%\end{proof}
%
%\begin{prop}
%The $\SO(n)$-equivariant cohomology of $E_n$ is 
%\[
%H_{\SO(n)}^\bullet(E_n) = 
%\begin{cases}
%H(B\SO(n))\otimes \e_n^* & \text{$n$ odd} \\
%H(B\SO(n-1)\otimes \ker (T\cdot) \subset H(B\SO(n-1))\otimes \e_n^* &\text{for $n$ even}.
%\end{cases}
%\]
%\end{prop}
%\begin{proof}
%The only thing yet to be shown is the case of $n$ even.
%
%Claim: $( \e_n^*(N), T\cdot)$ is acyclic in each arity $N\geq 2$.
%
%Using the claim, the result follows easily. 
%
%To show the claim, note that for each arity $N$ the complex is a direct summand of the Chevalley complex of a free Lie coalgebra in $N$ generators.
%\end{proof}
%
%

\section{Models and (co)formality for the framed \texorpdfstring{$n$}{n}-disks operads}\label{sec:framed}

\subsection{Models for the little \texorpdfstring{$n$}{n}-disks operad as \texorpdfstring{$G$}{G}-space}
In the previous section we studied (versions of) the ring of $G$-equivariant forms (for $G\subset O(n)$) for the little $n$-disks operad.
Again denote by $G_0\subset G$ the connected component of the identity.
If $G=G_0$ is connected the result was that there is a Maurer-Cartan element 
\[
 m\in H(BG_0)\otimes \GC_n
\]
governing the $G$-action in the sense that 
\[
 (\stG_n\otimes H(BG_0))^m
\]
is a cooperad in the category of dgcas under $H(BG_0)$.
If $G$ is not connected, all data may be chosen equivariantly under the $G/G_0$-action.

Now, using the result of section \ref{sec:space from quotient} we see that from the above model we may obtain a model for the $E_n$ operad as a $G$-space by tensoring over $H(BG)$ with the Koszul complex $K$. 
Suppose first that $G=G_0$ is connected. Recall that $K=H(G)\otimes H(BG_0)$, with a differential making its cohomology one dimensional, and with the obvious $H(G)$-coaction.
We obtain a cooperad in the category of dg Hopf comodules over $H(G)$
\[
 K\otimes_{H(BG_0)} (\stG_n\otimes H(BG_0))^m \cong \stG_n\otimes H(BG_0) \otimes H(G),
\]
where the Maurer-Cartan element $m$ is contained in the differential. In the case of non-connected $G$ one has the same formula, but takes also invariants under the $G/G_0$ action.

There is an alternative viewpoint which allows us to simplify the above model, in particular given that the Maurer-Cartan element $m$ in practice has a very simple form, as we saw in the previous sections.
Note that $H(BG_0)=C(\alg g)$ can be viewed as the Chevalley complex of the abelian graded Lie coalgebra $\alg g$ such that
\[
 \alg g_j = \pi_j(G_0)\otimes_{\Z} \R \quad\text{for $j\geq 3$}.
\]
Under this viewpouint the Maurer-Cartan element $m$ may be interpreted as an $L_\infty$-map 
\[
 \alg g^* \to \GC_n.
\]
Since $\GC_n$ acts on $\stG_n$ by cooperadic coderivations, this $L_\infty$ map describes an $L_\infty$ action of $\alg g^*$ on $\stG_n$, or dually an $L_\infty$-coaction of $\alg g$.

Let $\hat{\alg g}$ be a resolution that lifts the $L_\infty$ coaction to an honest coaction.
For example, for $n$ even and $G=\SO(n)$ we have seen that $m$ can be taken to be linear, thus the $L_\infty$ coaction is already an honest Lie coaction from the start, and we can take $\hat{\alg g}=\alg g$.
For $n$ odd we saw that the MC element $m$ can be assumed to have the simple form \eqref{equ:conjectured m odd}. It is not linear in the Pontryagin classes, but contains only the top Pontryagin class non-trivially.
Hence we may take in this case
\begin{equation}\label{equ:hat g formula}
 \hat{\alg g} = \R p_3 \oplus \R p_6 \oplus \cdots \oplus \R p_{2n-7} \oplus \alg h,
\end{equation}
where $\alg h=\FreeLie(S(\K p_{2n-3})$ is the minimal resolution of the one-dimensional abelian graded Lie algebra spanned by $p_{2n-3}$.

Again restricting to the connected case first, note that $H(G)=\mU\alg g$ can be understood as the universal enveloping coalgebra.
The universal enveloping coalgebra $\mU\hat{\alg g}$ is a resolution of $H(G)$ (as a dg Hopf algebra) and acts on $\stG_n$.
Instead of finding a model of $E_n$ in cooperads in dg Hopf $H(G)$-comodules we may equivalently look for a model in the category of cooperads in Hopf $\mU\hat{\alg g}$-comodules.
As such, a particularly simple model is the cooperad $\stG_n$ itself, equipped with the $\mU\hat{\alg g}$ coaction described by $m$.

\begin{prop}
 Let $\alg g$ be the abelian graded Lie coalgebra above, and $\hat{\alg g}\leftarrow \alg g$ a resolution that lifts the $L_\infty$ coaction encoded in $m$ to an honest dg Lie action.
Let $\hat K=\mU \hat{\alg g}\otimes H(BG_0)$ be the corresponding ``Koszul'' complex.
Then there is the following commutative diagram of dg Hopf algebras and their dg Hopf comodules:
\[
 \begin{tikzcd}
  \mU \alg g=H(G_0) \ar[dashed]{r} \ar{d}{\sim} & K\otimes_{H(BG_0)} (\stG_n\otimes H(BG_0))^m  \ar{d}{\sim}\\
  \mU \hat{\alg g} \ar[dashed]{r}  & \hat K\otimes_{H(BG_0)} (\stG_n\otimes H(BG_0))^m \\
  \mU \hat{\alg g} \ar[dashed]{r} \ar{u}{=} & \stG_n \ar{u}{\sim}.
 \end{tikzcd}
\]
The lower vertical arrow on the right-hand side is such that the composition with the projection to comodule cogenerators is the obvious inclusion
\[
 \stG_n \to H(BG_0) \otimes \stG_n.
\]
The dashed horizontal arrows stand for ``coaction on''.
\end{prop}
\begin{proof}
Clearly $\hat K$ and $K$ are quasi-isomorphic, so that the upper right vertical arrow is a quasi-isomorphism.
The space in the middle right is isomorphic (as a graded vector space) to $\stG_n\otimes \mU \hat{\alg g} \otimes H(G_0)$. Taking a spectral sequence on the degree in $\stG_n$, the differential reduces to the Koszul differential on the ``Koszul complex'' piece $\mU \hat{\alg g} \otimes H(G_0)$. That latter complex has one dimensional cohomology, and the spectral sequence must abut here by degree reasons. Hence the vertical arrow on the right is indeed a quasi-isomorphism.
\end{proof}

Let us remark that there is again a slight extension to the case of non-connected $G$.
One merely takes as model for $H(G)$ a semidirect product of $\R[G/G_0]^*$ with $\mU \hat{\alg g}$ above, which co-acts on $\stG_n$.

Finally, for $G=O(n)$ and $n$ even it is not necessary to take resolutions as mentioned, and $H(G)$ co-acts on $\stG_n$ immediately.
The action induces an action on cohomology, and the natural projection
\[
 \stG_n \to \e_n^* 
\]
preserves the coaction of $H(G)$, thus showing the (real) formality of the little $n$-disks operad as an operad in $O(n)$-spaces.


\subsection{Models for the framed little \texorpdfstring{$n$}{n}-disks operads, and formality}\label{sec:models formality}
By the previous subsection, we have models for the little $n$-disks operad as a $G$-space, in the form of cooperads in dg Hopf $H(G)$-(or a resolution thereof-)comodules.
We may hence use the result of section \ref{sec:dgca models operads} above to generate a model for the $G$-framed little $n$-disks operad $\lD_n\circ G$ by applying the algebraic framing operation to our models.

The (arguably) smallest such is obtained by using the dg Hopf comodule $\stG_n$ over $\mU\hat{\alg g}$. A dgca model (in the sense of section \ref{sec:dgca models operads}) for the $G$-framed little $n$-disks operad is hence
\[
 \stG_n \circ \mU\hat{\alg g}
\]
using the $\circ$-product, cf. section \ref{sec:framed operads}.

In good cases we can use this model to show that the framed operad is formal.
\begin{thm}
\begin{itemize}
\item For $n\geq 2$ even the $O(n)$-framed little $n$-disks operad is formal over $\R$.
\item For $n\geq 5$ odd, $m\leq n-2$, the $O(m)$-framed little $n$-disks operad is formal over $\R$.
\end{itemize}
\end{thm}
\begin{proof}
 For the first statement we use that we can pick $\alg g_\infty=\alg g$, so that our model becomes
\[
 \stG_n \circ H(G).
\]
The explicit formality morphism is then given by the projection $\stG_n\to H(\stG_n)$, which preserves the $H(G)$ coaction, as 
\[
 \stG_n \circ H(G) \to H(\stG_n \circ H(G)) = \stG_n \circ H(G).
\]
For the second case the coaction is governed by a trivial Maurer-Cartan element by Corollary \ref{conjthm:main2}.
Hence the claimed formality is obvious.
\end{proof}


% 
% \subsection{Small models}
% \newcommand{\BV}{\mathsf{BV}}
% \newcommand{\BVGraphs}{\mathsf{BVGraphs}}
% \newcommand{\tp}{\circlearrowleft}
% 
% \subsubsection{Even $n$}
% For even $n$ a very small operad may be constructed as follows.
% We may take $\BVGraphs_n:=\Graphs_n^\tp/I$, where the operadic ideal $I$ is spanned by graphs with tadpoles at internal vertices.
% \begin{lemma}[Proposition X of \cite{}]
% The natural map $\BV_n\to \BVGraphs_n$ is a quasi-isomorphism.
% \end{lemma}
% 
% $\BVGraphs_n$ is clearly a Hopf operad. By the Lemma, $\BVGraphs_2$ is a model for $E_2^{fr}$.
% The following conjecture is a consequence of Conjecture \ref{conj:main}.
% \begin{conj}
% The Hopf operad $\BVGraphs_{2k}\circ \K[...,p_j,...]$ is a model for $E_{2k}^{fr}$.
% \end{conj}
% 
% %\subsection{Odd $n$}
% %Define $\FRGraphs_n$, similarly to $\Graphs_n$, but with an additional type of internal edge ("dashed edge"). whose differential is a double ordinary edge.
% %
% %where $J$ is he operadic ideal spanned by graphs at least one of whose  
% 
% \subsection{Formality}
% Let us finish the proof of our results on formality of the partially framed little cubes operads.
% 
% \begin{proof}[Proof of Theorem \ref{thm:partial framed formality}]
% By Theorem \ref{thm:triviality} we know that for $G=\SO(m)$, $m$ as in the Theorem, the Maurer-Cartan element $m$ is gauge trivial. Hence, by Corollary \ref{cor:framed models} one model for the framed $E_3$ operad is 
% \[
% \La_{\alg g} \otimes \stG_n.
% \]
% The cohomology Hopf cooperad is $\La_{\alg g} \otimes \e_n^*$, and via the map $\e_n^*\to \stG_n$ we obtain our desired formality morphism.
% \end{proof}

\subsection{Recollection: The Drinfeld-Kohno (Quillen) models for \texorpdfstring{$E_n$}{En}}\label{sec:drinfeld-kohno}
The higher Drinfeld-Kohno Lie algebras are Lie algebras $\DK_n(r)$ generated by symbols $t_{ij}=(-1)^n t_{ji}$ of degree $n-2$, with $1\leq i\neq j \leq n$, with relations $[t_{ij}, t_{kl}]=0$ for $\#\{i,j,k,l\}=4$ and $[t_{ij},t_{ik}+t_{il}]=0$ for $\#\{i,j,k\}=3$.
These Lie algebras assemble (for varying $r$) into an operad in Lie algebras which we call the $n$-Drinfeld Kohno operad.
Concretely, the right symmetric group action is by the obvious permutation of indices. The operadic compositions
\[
\circ_r: \DK_n(r) \oplus \DK_n(r') \to \DK_n(r+r'-1)
\]
are defined by the following rules.
\begin{align*}
 \DK_n(r) \ni t_{ij} &\mapsto t_{ij}  \in\DK_n(r+r'-1)& \text{for $i,j<r$} \\
 \DK_n(r) \ni t_{ir} &\mapsto t_{ir}+t_{i(r+1)}+\dots + t_{i(r+r'-1)}  \in\DK_n(r+r'-1) & \text{for $i<r$} \\
 \DK_n(r') \ni t_{ij} &\mapsto t_{(i+r)(j+r)}  \in\DK_n(r+r'-1) & \\ 
\end{align*}
All other operadic compositions are determined by the operad axioms and the symmetric group action.
The following result is well known.
\begin{thm}\label{thm:drinfeld-kohno}
The Drinfeld-Kohno operad $\DK_n$ forms a rational Quillen model for the operad $\lD_n$ for each $n\geq 2$.
\end{thm}
Let us give a sketch of a proof that these objects are indeed real Quillen models. The proof necessarily uses some form of the (real) formality of $\lD_n$
First, we have seen that the Hopf operad $\Graphs_n$ is a dgca model for $\lD_n$. Furthermore, $\Graphs_n=C(\ICG_n)$ is the Chevalley complex of the operad in $L_\infty$ algebras given by the internally connected graphs. We are done if we can show that $\ICG_n$ and $\DK_n$ are quasi-isomorphic. To this end, one can define an auxiliary grading on $\ICG_n$ for which a graph $\Gamma$ has auxiliary degree 
\[
2 \# (\text{vertices of }\Gamma) -  \# (\text{edges of }\Gamma) +1.
\]
The definition is chosen such that:
\begin{itemize}
\item The differential has auxiliary degree $+1$.
\item The cohomology is concentrated in auxiliary degree $0$.
\item The grading is compatible with the operad and Lie algebra structure.
\end{itemize}

Given these conditions we may define the truncated sub-operad in $L_\infty$ algebras
\[
\TCG_n \subset \ICG_n
\]
formed by all elements of auxiliary degree $<0$, the closed elements in degree $0$, and no elements in degree $>0$.
We then have the zig-zag of quasi-isomorphisms
\[
\ICG_n\leftarrow \TCG_n \to H(\ICG_n) {=}\DK_n
\]
showing that $\ICG_n$ is formal, and hence that $\lD_n$ is coformal and that $\DK_n$ is a Quillen model for $\lD_n$.

\subsection{Framed Drinfeld-Kohno models}\label{sec:quillen}
In this section we want to discuss framed analogs $\DKF_n$ of the higher Drinfeld-Kohno operads of Lie algebras. 
Concretely, let as before $\alg g=\pi(\SO(n))\otimes \R [-1]$ as abelian Lie algebra. Concretely, $\alg g$ is spanned as a vector space by the Pontryagin classes $p_{4s}$ and, if $n$ is even, the Euler class $E$. We now define, as a Lie algebra
\[
\DKF_n(r) := \DK_n(r) \oplus \underbrace{\alg g\oplus \cdots \oplus \alg g}_{r\times}.
\]
In words, $\DKF_n(r)$ is generated by the $t_{ij}$ with relations as in the previous section, and by additional commuting generators $p_{4s}^{j}$ and $E^j$ for $1\leq j\leq r$. There is an obvious action of the symmetric group by permuting indices.
We define the operad structure by extending that on $\DK_n$, such that the composition
\[
\circ_r: \DKF_n(r) \oplus \DKF_n(r') \to \DKF_n(r+r'-1)
\]
acts on the the additional generators as follows.
\begin{align*}
 \DKF_n(r) \ni E^{j} &\mapsto E^j  \in\DKF_n(r+r'-1) & \text{for $n$ even, $j<r$} \\
 \DKF_n(r) \ni E^{r} &\mapsto \sum_{j=r}^{r'} E^{j} + \sum_{r\leq i<j \leq r'} t_{ij}  \in\DKF_n(r+r'-1) & \text{for $n$ even} \\
 \DKF_n(r) \ni p_{2n-2}^{r}&\mapsto \sum_{j=r}^{r'} p_{2n-2}^{j} + \sum_{r\leq i<j \leq r'}[t_{ij},t_{ij}] +  2\sum_{r\leq i<j<k \leq r'}[t_{ij},t_{jk}] \in\DKF_n(r+r'-1)& \text{for $n$ odd} \\
 \DKF_n(r) \ni p_{4s}^{j}&\mapsto p_{4s}^{j}  \in\DKF_n(r+r'-1)& \text{in all other cases} \\
\end{align*}
Again, the other operadic compositions are determined by the symmetric group action. 
In words, the operad structure is trivial regarding all basis elements of $\alg g$ except for the Euler class (for $n$ even), and the top Pontryagin class (for $n$ odd), for which we modified the composition rules.
We will show below that $\DKF_n$ is a Quillen model for $\flD_n$. For now, we show the following:

\begin{thm}\label{thm:homFICG}
The framed Drinfeld-Kohno operad $\DKF_n$ is the homology of a real Quillen model for $\flD_n$ for each $n\geq 2$.
\end{thm}
Note that the Theorem shows that $\DKF_n$ is a Quillen model for $\flD_n$ provided we can show coformality.
\begin{proof}
By section \ref{sec:models formality} we already have a dgca model 
\[
\stG_n \circ \mU\hat{\alg g}
\]
for $\flD_n$, depending on the graphical Maurer-Cartan element $m$ defined in section \ref{sec:MC}, which we assume to be taken in the form of Theorem \ref{conjthm:main}. 
Nevertheless, the above model is quasi-free as a (collection of) dgca. The dual space of the generators is 
\[
\FICG_n:= \ICG_n \circ \hat{\alg g}
\]
where $\hat{\alg g}$ is (up to a degree shift) isomorphic to a (quasi-)free Lie algebra with generators indexed by elements of $H(B\SO(n))^*$.
The differential on $\FICG_n$ does not depend on $m$, and hence it is immediate that, as a collection of graded vector spaces
\begin{equation}\label{equ:ICGFDKF}
H(\ICGF_n) \cong \DKF_n.
\end{equation}
However, note that the operadic compositions ($L_\infty$ morphisms) in $\FICG_n$ are relatively complicated and do depend on $m$, and it is a priori not obvious that the isomorphism \eqref{equ:ICGFDKF} is compatible with the operadic compositions, irrespective of $m$.
To show this statement let us note two facts: (i) The only contributions to the homology of $\ICG_n$ can come from trivalent trees without internal loops. (ii) Looking at the way the action of $\GC_n$ is defined the only graphs in $\GC_n$ that can possibly create such objects are graphs which after deletion of one vertex become trivalent forests without loops.
(iii) For $n$ even there is only one such graph: the tadpole. For $n$ odd, many such graphs do exists, but by Proposition \ref{prop:veryloopy} we may assume that only one such, namely the theta graph, is present in $m$. 

Now, we know the coefficients of the tadpole and theta graphs in $m$, and hence we can understand the operad structure on $H(\ICGF_n)$. The operad structure on $\DKF_n$ is defined precisely such that \eqref{equ:ICGFDKF} is compatible with the compositions.

TODO: discuss bivalent/trivalent graph issue somewhere
\end{proof}


\subsection{ Coformality of \texorpdfstring{$\lD_n$}{Dn}}
Now we are ready to prove the coformality of $\lD_n$, and hence check that $\DKF_n$ is indeed a Quillen model.

\begin{proof}[Proof of Theorem \ref{thm:FE3coformal}]
After the discussion in the previous subsection, it suffices to construct a quasi-isomorphism between the operad in $L_\infty$ algebras $\ICGF_n$ and its homology $H(\ICGF_n)=\DKF_n$. We can further assume (-by Theorem \ref{conjthm:main}-) that the Maurer-Cartan element $m\in \BGC_n$ used to define $\ICGF_n$ is as in \eqref{equ:conjectured m odd}.

Now we desire to copy the truncation trick in the proof of the non-framed analogous result Theorem \ref{thm:drinfeld-kohno} above. Looking at that proof, it is clear that we are done if we can define an auxiliary grading on $\ICGF_n$ with the same formal properties, i.e.: 

\begin{itemize}
\item The differential has auxiliary degree $+1$.
\item The cohomology is concentrated in auxiliary degree $0$.
\item The grading is compatible with the operad and Lie algebra structure.
\end{itemize}

Note that, as a vector space $\ICGF_n=\hat{\alg g}\oplus \ICG_n$. We define our auxiliary grading on $\ICG_n$ as in the proof of Theorem \ref{thm:drinfeld-kohno} above as 
\[
2 \# (\text{vertices of }\Gamma) -  \# (\text{edges of }\Gamma) +1.
\]
Furthermore, the auxiliary grading on the piece $\hat{\alg g}$ is defined as follows. 
In the case of even $n$ we declare all of $\hat{\alg g}=\alg g$ to be in auxiliary degree $0$. 
Recall the formula \eqref{equ:hat g formula} for $\hat{\alg g}$ for odd $n$. In this case $\hat{\alg g}$ is (up to degree shift) composed of an abelian piece, generated by the lower Pontryagin classes and a free Lie algebra generated by powers of the top Pontryagin class $P$. We now define the auxiliary grading by declaring the abelian piece (i.e., all the lower Pontryagin classes) to live in degree zero, and by declaring the power  $P^r$ of the top Pontryagin class to live in degree $r-1$.
One can check that in each case this grading satisfies the requirements, using the explicit form of $m$. Hence we can construct the desired quasi-isomorphism as
\[
\ICGF_n\leftarrow \TCGF_n \to H(\ICGF_n) \stackrel{\text{Thm. \ref{thm:homFICG}}}{=}\DKF_n.
\]
\end{proof}



\subsection{Non-formality for odd \texorpdfstring{$n$}{n}: proof of Theorem \texorpdfstring{\ref{thm:odd nonformality}}{} }

Let $n\geq 3$ be an odd integer, and consider the $\SO(n)$-framed little $n$-disks operad.
We obtain a model by dualizing our model for differential forms of section \ref{sec:models formality}.
Concretely, the operad of chains is quasi-isomorphic (as a homotopy operad) to 
\[
 \op P := \Graphs_n \circ A,
\]
where 
\[
 A=H_\bullet(\SO(n-2)) \otimes F,
\]
and $F=\R\langle P_1,P_2,\dots \rangle$ is a free algebra in symbols $P_k$ of degree $-k(2n-2)+1$, with differential $dP_k=\sum_{i+j=k}P_iP_j$.
In particular, $P_1$ represents the top Pontryagin class.
The action of the Hopf algebra $A$ on the operad $\Graphs_n$ is such that $H_\bullet(\SO(n-2))$ acts trivially, and the action of $P_j$ is (up to an unimportant prefactor) represented by the graph with two vertices and $2j+1$ edges, cf. Theorem \ref{conjthm:main}.

On the other hand, the homology operad is \cite{SW}
\[
 \op H := \Poiss_n \circ H_\bullet(\SO(n)),
\]
with $H_\bullet(\SO(n))$ acting trivially on $\Poiss_n$.

Our goal is to show that $\op P$ is not quasi-isomorphic to $\op H$ as a dg operad.
If it was we could find a quasi-isomorphism
\beq{equ:nfo1}
 \op H_\infty \to \op P
\eeq
for $\op H_\infty$ a cofibrant replacement of $\op H$.
We want to show by obstruction theory that such a morphism cannot exist.
First, the operad $\op H$ is Koszul, and we may pick 
\[
 \op H_\infty := \Omega(\op H^{\vee})
\]
to be the cobar construction of the Koszul dual cooperad.
More explicitly, the Koszul dual operad is identified with 
\[
 \op H^! = H(B\SO(n)) \otimes \Poiss_n\{n\}.
\]

We will try to construct \eqref{equ:nfo1} inductively on the arity $r$ and hit an obstruction in arity 3.
We will impose a filtration on $\op P$ as follows.
We say that the weight of a graph in $\Graphs_n$ is the number of edges.
We say that the lower Pontryagin classes (cogenerators of $H_\bullet(\SO(n))$) have some weight $>3$, say $4$.
We say that $P_j$ has weight $2j+1$.
This imposes a filtration by weight on $\op P$. 

To simplify the obstruction argument, we (try to) will construct \eqref{equ:nfo1} only up to weight 3, i.e., we ignore terms of weight $\geq 4$ in $\op P$.
Mind that only a small (finite dimensional in each arity) subspace of $\op P$ lives in weight $\leq 3$:
\begin{itemize}
 \item We can have an empty graph decorated by one copy of $P_1$.
\item We can have a graph with at most three edges decorated by the trivial element of $A$.
\end{itemize}

In particular, in arity one the subspace of elements of weight $\leq 3$ is 3-dimensional.
To simplify further, we note that $H(B\SO(n))\cong \R[p_4,p_8,\dots,p_{2n-2}]$. Hence we may equip $\op H^!$ with a (``co-weight'') grading such that each $p_j$ ($j\leq 2n-6$) has co-weight 2 and $p_{2n-2}$ has co-weight 1.
We only consider terms of co-weight $\leq 1$.

Now, in arity $r=1$, we have to provide a map from the generators $\op H^{\vee}(1)$ to $\op P(1)$, or dually an element of $\op H^!(1)\otimes \op P(1)$.
In co-weight $\leq 1$ we have only the $p_{2n-2}$ (of degree $2n-2$) as generator.
It has to be sent to a closed element of $\op P(1)$. In weight $\leq 3$, the only closed element is $P_1$, in degree $3-2n$.
Hence, using that the map must be a quasi-isomorphism, modulo terms of co-weight $\geq 2$ or weight $\geq 4$ the arity 1 map is described by 
\[
 p_{2n-2} \otimes P_1 + (\cdots) \in \op H^!(1)\otimes \op P(1).
\]
Next, consider the arity $r=2$ part.
In coweight $0$ we have the bracket ($=:b$) and product ($=:p$) generators in $\op H^!(2)$.
In coweight 1 we have the $p_{2n-2} b$ and $p_{2n-2}p$.
In $\op P(2)$ one can list all elements of weight $\leq 3$ (we draw only one graph for each $S_2$ orbit):
\begin{align*}
 &\begin{tikzpicture}
  \node[ext] (v1) at (0,0) {};
  \node[ext] (v2) at (.5,0) {};
 \end{tikzpicture}
& &
\begin{tikzpicture}
  \node[ext] (v1) at (0,0) {};
  \node[ext] (v2) at (.5,0) {};
\draw (v1) edge (v2);
 \end{tikzpicture}
& &
\begin{tikzpicture}
  \node[ext] (v1) at (0,0) {};
  \node[ext] (v2) at (.5,0) {};
\draw (v1) edge[bend left] (v2) (v1) edge[bend right] (v2);
 \end{tikzpicture}
& &
\begin{tikzpicture}
  \node[ext] (v1) at (0,0) {};
  \node[ext] (v2) at (.5,0) {};
\draw (v1) edge (v2) edge[bend left] (v2) edge[bend right] (v2);
 \end{tikzpicture}
\\
&
\begin{tikzpicture}
  \node[ext, label={$\scriptstyle P_1$}] (v1) at (0,0) {};
  \node[ext] (v2) at (.5,0) {};
 \end{tikzpicture}
& &
\begin{tikzpicture}
  \node[ext] (v1) at (0,0) {};
  \node[ext] (v2) at (.5,0) {};
  \node[int] (v3) at (0,0.5) {};
\draw (v1) edge (v3) edge[bend left] (v3) edge[bend right] (v3);
 \end{tikzpicture}
& &
\begin{tikzpicture}
  \node[ext] (v1) at (0,0) {};
  \node[ext] (v2) at (.5,0) {};
  \node[int] (v3) at (0.25,0.5) {};
\draw (v1) edge[bend left] (v3) edge[bend right] (v3) (v2) edge (v3);
 \end{tikzpicture}\, .
\end{align*}
Writing down the requirement that \eqref{equ:nfo1} should commute with the differentials and induce an isomorphism (say the identity) on cohomology, one quickly checks that in arity $2$ the map \eqref{equ:nfo1} must be described by
\[
 b\otimes p + p \otimes b + 
p_{2n-2}p \otimes 
\begin{tikzpicture}
  \node[ext] (v1) at (0,0) {};
  \node[ext] (v2) at (.5,0) {};
\draw (v1) edge (v2) edge[bend left] (v2) edge[bend right] (v2);
 \end{tikzpicture}
+
p_{2n-2}b \otimes
\begin{tikzpicture}
  \node[ext] (v1) at (0,0) {};
  \node[ext] (v2) at (.5,0) {};
\draw (v1) edge[bend left] (v2) edge[bend right] (v2);
 \end{tikzpicture}
+ (\cdots).
\]

Next consider arity $r=3$.
The co-weight $\leq 1$ elements are built using 0,1 or 2 brackets, possibly times $p_{2n-2}$.
Again one computes that the double bracket $m_{1,23}m_{23}$ must be paired with an element $x\in \op P(3)$ whose differential is the graph 
\[
\begin{tikzpicture}
 \node[ext] (v1) at (0,0) {$\scriptstyle 1$};
\node[ext] (v2) at (0.5,0) {$\scriptstyle 2$};
\node[ext] (v3) at (1,0) {$\scriptstyle 3$};
\draw (v1) edge [bend right] (v2) edge[bend left] (v3);
\end{tikzpicture}
\]
Since no such element exists we have found our obstruction.

\hfill\qed






\begin{rem}
Here we show that the operad of chains is not formal as dg operad, using our model.
Likely, the proof can be simplified significantly in the Hopf (co)operadic setting, and made independent of the developments of the present paper, by merely noting that the action of $\SO(n)$ on $S^{n-1}$ is not formal over $\R$ for $n\geq 3$ odd.
\end{rem}



%\section{Deformation theory of the framed $n$-disks operads}
%Once we have combinatorial models for the framed $n$-disks operads, the Koszul duals and the $\BGC_n$  action we can use the standard machinery to show the following result.
%
%\begin{thm}
%\[
%H(\Der_{Op}(E_n^{fr})= \cong S^+(H(\BGC_n^{ext}))
%\]
%(TODO: degree shifts) where $\BGC_n^{ext}$ is the same as $\BGC_n$, but with extra elements added representing the derivations of the algebra $H(B\SO(n))$.
%\end{thm}
%
%Given that we have the Hopf operad model (...the details for which I have yet to write...) we can also more or less certainly show that
%\[
%H(\Der_{Hopf}(E_n^{fr})= \cong H(\BGC_n^{ext}).
%\]

%\section{The spherically framed $E_n$ operads}
%TODO




\appendix

\section{An explicit formula for the propagator}
\label{sec:explicitpropagator}
One may provide an explicit formula for the propagator $\Omega_{sm}$ of section \ref{sec:propagator}.
To this end, we parameterize the sphere $S^{n-1}$ by a torus and a simplex as follows.
For $n$ odd we parameterize each hemisphere separately, and get
\begin{gather*}
 \{\pm 1\}\times (S^1)^k \times \Delta_{k'} \to S^{n-1}\subset \R^n \\
 (\epsilon, \phi_1,\dots,\phi_k,\sigma_0,\dots,\sigma_{k}) 
\mapsto (\epsilon \sqrt{\sigma_0}, \sqrt{\sigma_1}\cos \phi_1,\sqrt{\sigma_1}\sin\phi_1,\dots , \sqrt{\sigma_k}\cos \phi_k,\sqrt{\sigma_k}\sin\phi_k) 
\end{gather*}
where $k=(n-1)/2$, and where we use the standard coordinates on the simplex $\sigma_0,\dots,\sigma_{k}\geq 0$ such that $\sum_{j=0}^{k}\sigma_j=1$. In the following, we will forget $\epsilon$ and restrict to the upper hemisphere (i.e., $\epsilon=+1$), the formula for the lower hemisphere can then be recovered by reflection anti-invariance.

For $n$ even we use the similar parameterization
\begin{gather*}
(S^1)^k \times \Delta_{k'} \to S^{n-1}\subset \R^n \\
 (\phi_0,\dots,\phi_{k'},\sigma_0,\dots,\sigma_{k'}) 
\mapsto(\sqrt{\sigma_0}\cos \phi_0,\sqrt{\sigma_0}\sin\phi_0,\dots , \sqrt{\sigma_{k'}}\cos \phi_{k'},\sqrt{\sigma_{k'}}\sin\phi_{k'}) 
\end{gather*}
where now $k=n/2$ and $k':=k-1$.
In the above parameterization the $T=(S^1)^k$-action is obvious.

Again in these coordinates the round volume form on the sphere $S^{n-1}$ has the form\footnote{It is an elementary but nice exercise to compute the surface area of the $(n-1)$-sphere in these coordinates.}
\begin{align}
&\left(\frac 1 2\right)^{k-1} \iota_E \left( d\sqrt{\sigma_0}\prod_{j=1}^{k} (d\phi_j d\sigma_j) \right) = \left(\frac 1 2\right)^{k}\frac 1 {\sqrt{\sigma_0}} \prod_{j=1}^k (d\phi_j d\sigma_j) && \text{$n$ odd} \\
\label{equ:roundvol2}
&\left(\frac 1 2\right)^{k-1} \iota_E \prod_{j=0}^{k'} (d\phi_j d\sigma_j)  = \pm \left(\frac 1 2\right)^{k-1} d\phi_0\cdots d\phi_{k'} d\sigma_1\cdots d\sigma_{k'}   && \text{$n$ even},
\end{align}
where we choose the orientation on the parameter space such that the above forms are positive, and where $\iota_E$ is the operator of contraction with the Euler vector field
\begin{align*}
E&= \sum_{j=0}^{k} \sigma_j \frac \partial {\partial \sigma_j} & \text{$n$ odd} \\
E&= \sum_{j=0}^{k'} \sigma_j \frac \partial {\partial \sigma_j} & \text{$n$ even}.
\end{align*}
Note that the Euler vector field is defined not on the simplex but on the larger space $\hat \Delta_{k'} = \{\sigma_0,\dots,\sigma_{k'}\geq 0\}\supset \Delta_{k'}$, and the notation in \eqref{equ:roundvol2} above shall silently mean the restriction of the stated form on $\hat \Delta_{k'}$ to $\Delta_{k'}$, {\em after} the contraction of the vector field.

To state the formula for the propagator, let us introduce the following notation. For $K$ a subset of indices we shall set
\begin{align*}
u^K &:= \prod_{j\in K} u_j & 
(d\phi d\sigma)^K &:= \prod_{j\in K}(d\phi_j d\sigma_k).
\end{align*}
Furthermore, we denote the complement of the subset $K$ by $\bar K$, and the number of elements by $|K|$.

The explicit formula for the propagator is then 
\begin{align*}
 \Omega_{sm}
&= 
C_n \iota_E
\left( 
d\sqrt{\sigma_0}
\sum_{K\subset \{1,\dots,k\}}
 {(|\bar K|-\frac 1 2)!} u^K (d\phi d\sigma)^{\bar K}
\right)
 && \text{$n$ odd} \\
 \Omega_{sm}
&= 
C_n \iota_E
\left(
\sum_{K\subsetneq \{0,\dots,k'\}}
 {(|\bar K|-1)!} u^K (d\phi d\sigma)^{\bar K}
\right)
&& \text{$n$ even},
\end{align*}
where $x!:=\Gamma(x+1)$ with $\Gamma$ the Euler $\Gamma$-function, and $C_n$ is an unimportant normalization constant chosen such that the integral over the sphere of the above form is $1$. Concretely,
\begin{align*}
C_n = \frac{1}{2^{k-1} \Gamma(n/2) \mathit{vol}(S^{n-1})} = 
\begin{cases}
\frac{1}{\sqrt{\pi}(2\pi)^k } & \text{$n$ odd} \\ 
\frac{1}{(2\pi)^k } & \text{$n$ even} 
\end{cases}.
%C_n &= \frac {2 \Gamma(\frac 1 2)} { \mathit{vol}(S^{n-1})} = \frac {(n-2)!!}{(2\pi)^{\frac n 2 -1}} && \text{$n$ odd} \\
%C_n &= \frac 1 {(k')! \mathit{vol}(S^{n-1})} = \frac{(n-2)!!}{(k')!(2\pi)^{\frac n 2}} && \text{$n$ even}.
\end{align*}

\begin{lemma}
 The above propagator is well defined and non-singular on the sphere, and satisfies the conditions of section \ref{sec:propagator}, in particular
\begin{align}
\label{equ:equivdprop1}
  (d+\sum_{i=1}^k u_i\iota_i) \Omega_{sm} &= 0 & \text{$n$ odd} \\
  \label{equ:equivdprop2}
(d+\sum_{i=0}^{k'} u_i\iota_i) \Omega_{sm} &= -C_n u_0\cdots u_{k'} =: E 
& \text{$n$ even}
\end{align}
\end{lemma}
\begin{proof}
The above form is obviously smooth away from the singular loci of our parameterization, which are the union of the sets $\{\sigma_j=0\}$. The functions $\sigma_j$ are smooth functions on the sphere, and hence are the forms $d\sigma_j$.
The forms $d\phi_j$ has a singularity at $\{\sigma_j=0\}$, however one easily checks that the combinations $\sigma_jd\phi_j$ and $d\sigma_jd\phi_j$ occurring in our formula are smooth forms on the sphere.
Hence the only possible source of a singularity stems from the power of $\sigma_0$ in the formula for $n$ odd.
However, $\sqrt{\sigma_0}$ is one of the Euclidean coordinate functions on the sphere, and hence smooth, and hence so are all of its non-negative powers and the differential $d\sqrt{\sigma_0}$. 

Next, let us consider the stated formulas for the equivariant differentials.
% Consider first the case $n$ odd and calculate.
% \begin{align*}
% d\Omega_{sm} &= -C_n\sum_{K\subset \{1,\dots,k \}} 
% \frac 1 {\Gamma(|K|+\frac 1 2)}
% (|K|-\frac 1 2)(\sigma_0)^{|K|-1-\frac 1 2} u^K (d\phi d\sigma)^{\bar K} \sum_{i=1}^kd\sigma
% \\&= -C_n
% \sum_{K\subset \{1,\dots,k \}} 
% \frac 1 {\Gamma(|K|-\frac 1 2)}
% (\sigma_0)^{|K|-1-\frac 1 2} u^K (d\phi d\sigma)^{\bar K} \sum_{i=1}^kd\sigma
% \\
% \sum_{j=1}^k u_j \iota_j \Omega_{sm}
% &=
% C_n\sum_{K\subset \{1,\dots,k \}} 
% \frac 1 {\Gamma(|K|+\frac 1 2)}
% (\sigma_0)^{|K|-\frac 1 2} u^K \sum_{i\in \bar K} u_id\sigma_i (d\phi d\sigma)^{\bar K\setminus \{i\}}
% \end{align*}
% Collecting powers of $u$ in the last expression (or, put differently, changing the summation variable $K\mapsto K\cup \{i\}$) we obtain
% \[
% C_n\sum_{K\subset \{1,\dots,k \}} 
% \frac 1 {\Gamma(|K|-\frac 1 2)}
% (\sigma_0)^{|K|-1-\frac 1 2} u^K (d\phi d\sigma)^{\bar K} \sum_{i\in K} d\sigma_i
% = 
% C_n\sum_{K\subset \{1,\dots,k \}} 
% \frac 1 {\Gamma(|K|-\frac 1 2)}
% (\sigma_0)^{|K|-1-\frac 1 2} u^K (d\phi d\sigma)^{\bar K} \sum_{i=1}^k d\sigma_i
% .
% \]
% This agrees with the negative of the expression for the de Rham differential, and hence \eqref{equ:equivdprop1} is shown. Note also that in the last expression we would strictly speaking have to reduce to $K\neq \emptyset$ in the sum, but we may drop this condition since the corresponding summand is zero by degree reasons anyway.

Consider first the case of even $n$, for which we compute:
\begin{align*}
d\Omega_{sm} &=C_n  L_E
\sum_{K\subsetneq \{0,\dots,k'\}}
 {(|\bar K|-1)!} u^K (d\phi d\sigma)^{\bar K}
= C_n \sum_{K\subsetneq \{0,\dots,k'\}}
{|\bar K|!} u^K (d\phi d\sigma)^{\bar K}
\\
\sum_{j=0}^{k'} u_j \iota_j \Omega_{sm}
&=
-
C_n
\iota_E
\sum_{K\subsetneq \{0,\dots,k'\}}
 {(|\bar K|-1)!} u^K \sum_{i\in \bar K} u_i d\sigma_i (d\phi d\sigma)^{\bar K\setminus \{i\}}
\end{align*}
In the first line we denoted the Lie derivative with respect to the Euler vector field by $L_E$.
Now collect powers of $u$ in the final expression in the second line (i.e., change summation variables $K\mapsto K\cup\{i\}$), yielding
\[
-
C_n
\iota_E
\sum_{\emptyset \neq K\subset \{0,\dots,k'\}}
 {|\bar K|!} u^K (d\phi d\sigma)^{\bar K} \sum_{i\in K} d\sigma_i 
=
-
C_n
\iota_E
\sum_{\emptyset \neq K\subset \{0,\dots,k'\}}
|\bar K|! u^K (d\phi d\sigma)^{\bar K} \sum_{i=0}^{k'} d\sigma_i  \, .
\] 
To simplify further we need to carry out the contraction and obtain
\[
\iota_E \left((d\phi d\sigma)^{\bar K} \sum_{i=0}^{k'} d\sigma_i\right)
=
(\iota_E (d\phi d\sigma)^{\bar K})  \underbrace{\sum_{i=0}^{k'} d\sigma_i}_{=0\text{ on }\Delta_{k'}}
+
(d\phi d\sigma)^{\bar K} \underbrace{\iota_E \sum_{i=0}^{k'} d\sigma_i}_{=1\text{ on }\Delta_{k'}}
=
(d\phi d\sigma)^{\bar K}.
\]
Collecting the previous computations we find that for even $n$
\begin{align*}
(d+\sum_{i=0}^{k'} u_i\iota_i) \Omega_{sm} &=
C_n \sum_{K\subsetneq \{0,\dots,k'\}}
{|\bar K|!} u^K (d\phi d\sigma)^{\bar K}
-
C_n \sum_{\emptyset \neq K\subset \{0,\dots,k'\}}
{|\bar K|!} u^K (d\phi d\sigma)^{\bar K}
\\&=
-
C_n \, u_0\cdots u_{k'},
\end{align*}
and thus \eqref{equ:equivdprop2} is shown. (Here we note that the term $K=\emptyset$ in the first sum does not contribute, since the restriction of that summand to the simplex vanishes.)

Next, let us turn to $n$ odd, and compute similarly:

\begin{align*}
d\Omega_{sm} &=C_n  L_E
\left( 
d\sqrt{\sigma_0}
\sum_{K\subset \{1,\dots,k\}}
 {(|\bar K|-\frac 1 2)!} u^K (d\phi d\sigma)^{\bar K}
\right)
= C_n \left( 
d\sqrt{\sigma_0}
\sum_{K\subset \{1,\dots,k\}}
 {(|\bar K|+\frac 1 2)!} u^K (d\phi d\sigma)^{\bar K}
\right)
\\
\sum_{j=1}^{k} u_j \iota_j \Omega_{sm}
&=
C_n
\iota_E
\left( 
d\sqrt{\sigma_0}
\sum_{K\subset \{1,\dots,k\}}
 {(|\bar K|-\frac 1 2)!} u^K \sum_{i\in \bar K} u_i d\sigma_i (d\phi d\sigma)^{\bar K\setminus \{i\}} 
\right)
\end{align*}

Collect again powers of $u$ in the last expression (i.e., change summation variables $K\mapsto K\cup\{i\}$), yielding
\[
C_n
\iota_E
\left( 
d\sqrt{\sigma_0}
\sum_{\emptyset \neq K\subset \{1,\dots,k\}}
 {(|\bar K|+\frac 1 2)!} u^K (d\phi d\sigma)^{\bar K} \sum_{i\in K}d \sigma_i 
\right)
=
C_n
\iota_E
\left( 
d\sqrt{\sigma_0}
\sum_{\emptyset \neq K\subset \{1,\dots,k\}}
 {(|\bar K|+\frac 1 2)!} u^K (d\phi d\sigma)^{\bar K} \sum_{i=0}^k d \sigma_i 
\right) \, .
\] 
Now carry out the contraction as for even $n$ and obtain
\[
-
C_n
\left( 
d\sqrt{\sigma_0}
\sum_{K\subset \{1,\dots,k\}}
 {(|\bar K|+\frac 1 2)!} u^K (d\phi d\sigma)^{\bar K}
\right) \, .
\]
Comparing terms, \eqref{equ:equivdprop1} follows.

\end{proof}

% \begin{rem} TODO:
% The above formula in the odd case is suboptimal....
% 
% It is better to use a different version, that also unifies even and odd case
% 
% \[
%  \Omega_{sm}
% = 
% C_n \iota_E
% \left( 
% d\sqrt{\sigma_0}
% \sum_{K\subsetneq \{0,\dots,k'\}}
%  {(|\bar K|-\frac 1 2)!} u^K (d\phi d\sigma)^{\bar K}
% \right)
% \]
% where $x!:=\Gamma(x+1)$.
% \end{rem}

\begin{rem}
We note that the forms $\Omega_{sm}$ are stable under restriction to the $n-2$-dimensional subspace defined by $\sigma_{k'}=0$, in the sense that 
\[
\Omega_{sm}^{n\text{-dim}} = (const) u_{k'} \Omega_{sm}^{n-2\text{-dim}}.
\]
(For $n$ odd set $k'=k$.)
\end{rem}


In order to facilitate explicit computations, let us also note the following.
\begin{lemma}\label{lem:prop at north pole}
 If $n$ is odd the value of $\Omega_{sm}$ at the north pole of the sphere (i.e., at $\sigma_0=1$, $\sigma_1=\cdots=\sigma_k=0$) is 
\[
C_n \frac 1 2 \Gamma(\frac 1 2) u_1\cdots u_k= \frac 1 {2(2\pi)^k} u_1\cdots u_k.
\]
\end{lemma}



% \begin{rem}(TODO: We do not use this remark and can safely remove it)
% One can also write down a version of the propagator using the Cartan model instead of the toric Cartan model of equivariant forms
% \[
% (S(\alg g^*[-2])\otimes \Omega(S^{n-1}))^G.
% \]
% A basis for $\alg g^*[-2]$ (with $G=\SO(n)$) is denoted by symbols $u_{ij}=-u_{ji}$, with $1\leq i\neq j \leq n$.
% Define the operator 
% \[
% I := \sum_{i<j}u_{ij} \iota_i \iota_j.
% \]
% Then define the equivariant volume form to be 
% \[
% \Omega_{sm}^{C}:= C_n \iota_E \sum_{0\leq k < n/2} (n-2k-2)!! I^k (dx_1\cdots dx_n),
% \]
% where $E=\sum_{j=1}^n x_j \frac{\partial}{\partial x_j}$ is the Euler vector field, and it is understood that (only) after contraction one restricts the form to the sphere. We set $(-1)!!:=1$.
% 
% Now verify that the given form satisfies the defining equations.
% First
% \[
% d \Omega_{sm}^{C} = C_n L_E \sum_{0\leq k < n/2} (n-2k-2)!! I^k (dx_1\cdots dx_n) 
% = C_n \sum_{0\leq k < n/2} (n-2k)!! I^k (dx_1\cdots dx_n)
% = C_n \sum_{1\leq k < n/2} (n-2k)!! I^k (dx_1\cdots dx_n)
% \]
% Next
% \begin{align*}
% \sum_{i,j} u_{ij}x_i \iota_j  \Omega_{sm}^{C}
% &=
% -C_n\iota_E \sum_{0\leq k < n/2} (n-2k-2)!! I^k \sum_{i,j} u_{ij}x_i \iota_j  (dx_1\cdots dx_n)
% \\&=
% -C_n\iota_E \sum_{0\leq k < n/2} (n-2k-2)!!I^k \sum_{i,j} u_{ij}x_idx_i \iota_i \iota_j  (dx_1\cdots dx_n)
% \\
% &= 
% -C_n\frac 1 2 \iota_E \sum_{0\leq k < n/2} (n-2k-2)!!I^k \sum_{l=1}^n x_ldx_l \sum_{i,j} u_{ij} \iota_i \iota_j  (dx_1\cdots dx_n)
% \\&= 
% -C_n\iota_E \sum_{0\leq k < n/2} (n-2k-2)!! I^{k+1} \sum_{l=1}^n x_ldx_l  (dx_1\cdots dx_n)
% \\ 
% &=
% -C_n\sum_{0\leq k < n/2} (n-2k-2)!! I^{k+1}  (dx_1\cdots dx_n)
% \\&=
% -C_n\sum_{1\leq k < n/2+1} (n-2k)!! I^{k}  (dx_1\cdots dx_n).
% \end{align*}
% Hence we find that 
% \[
% (d+
% \sum_{i,j} u_{ij}x_i \iota_j)\Omega_{sm}^{C}
% =
% \begin{cases}
% 0 & \text{for $n$ odd} \\
% -C_n I^{n/2}(dx_1\cdots dx_n) \propto E & \text{for $n$ even}
% \end{cases}.
% \]
% 
% \end{rem}

\section{Auxiliary result on graph complex}
The goal of this section is to show the following small auxiliary result used in the proof of Theorem \ref{thm:homFICG} above. We call a graph \emph{very loopy} if the complement of every vertex has at least one loop.
In other words, the only graphs which are not very loopy are trees all of whose leafs are fused to one vertex. 

\begin{prop}\label{prop:veryloopy}
Suppose that $n\geq 3$ is odd and let $m\in \BGC_n$ be a Maurer-Cartan whose 2-vertex part agrees with the Maurer-Cartan element \eqref{equ:conjectured m odd} above. Then we may change $m$ to a gauge equivalent Maurer-Cartan element that contains only very loopy graphs, apart from the 2-vertex piece.
\end{prop}

Let us first note the following Lemma, from which the Proposition can be derived.
\begin{lemma}
Let $n$ be odd and $\GC_n^{vl}\subset \GC_n$ be the subcomplex of very loopy graphs. Then the quotient $\GC_n':=\GC_n/\GC_n^{vl}$ has one-dimensional cohomology in odd degrees, spanned by the theta-graph. (TODO: at least trivalent graph complex here, make notation consistent.)
\end{lemma}
\begin{proof}
First, one checks that if $\Gamma$ is a non-very loopy graph with at least 3 vertices, there is a unique vertex whose complement is a tree. Hence one can check that $\GC_n'$ is quasi-isomorphic to a complex of trees, with indistinguishable leafs, except that trees with one vertex and an even number of leaves are forbidden. The complex of all trees is well known to be acyclic, up to the one class given by the "tripod" tree. 
Removing the "even-pods" creates one additional class for each even-pod. However, these classes all live in even degree. 
\end{proof}

\begin{proof}[Proof of Proposition \ref{prop:veryloopy}]
We will split $m=m_0+m_1$ with $m_0$ being the element \eqref{equ:conjectured m odd} above.
Our goal is to perform a gauge change transforming $m$ into $m'=m_0+m_1'$, with $m_1$ consisting of very loopy graphs only.

The complex $\BGC_n$ carries a grading derived from that in $H(B\SO(n))$, call it the aux-grading temporarily.
We will perform an induction on the aux-degree in $H(B\SO(n))$. We assume inductively that the first non-very loopy graphs in $m_0$ appear in aux-degree $r$. (Necessarily, the cohomological degree must be odd.)  
By the previous Lemma, these terms are $\delta$-exact. Hence we may gauge change to kill these terms. The gauge transformation will produce some additional "garbage", which however lives in higher aux-degree. Hence we are done.
\end{proof}


\section{Some results about lifts etc.}
\subsection{Path objects and gauge equivalence}\label{app:pathobjects}
The results here are more or less for confirmation / explicit formulas on how to construct a gauge transformation in the naive or classical sense from a path object. They are not needed strictly speaking.
\begin{lemma}
Suppose $\alg h$ is a dg Lie (or $L_\infty$-)algebra and $\alg g$ is a path object with maps 
\[
\begin{tikzcd}
\alg h \ar{r}{\iota} & \alg g \ar{r}{p_0} \ar{r}{p_1} & \alg h
\end{tikzcd}
\]
Then the following holds:
\begin{enumerate}
\item There is a "naive" homotopy between the maps $\mathit{id}_{\alg g}$ and $\iota\circ p_0$, i.e., an $L_\infty$-morphism
\[
F: \alg g\to \alg g[t,dt]
\]
which agrees with $\mathit{id}_{\alg g}$ at $t=1$ and $\iota\circ p_0$ at $t=0$.
\item There is an $L_\infty$-quasi-isomorphism
\[
G: \alg g\to \alg h[t,dt]
\]
such that $p_0=ev_{t=0}\circ G$ and $p_1=ev_{t=1}\circ G$.
\end{enumerate}
\end{lemma}
\begin{proof}
To show the second statement given the first we simply set 
\[
G = p_1\circ F,
\]
where we quietly extend $p_1$ $t$- and $dt$-linearly. 
Then 
\[
ev_{t=1}\circ G=ev_{t=1}\circ p_1 \circ F = p_1 
\]
and 
\[
ev_{t=0}\circ G=p_1 \circ \iota \circ p_0 = p_0. 
\]
For the first statement one has to pick some homotopy $h$ on $\alg g$ such that
\[
dh+hd = \mathit{id}_{\alg g} - \iota\circ p_0.
\]
Then the $L_\infty$-morphism $F$ can be recursively constructed.
TODO: It would be nice if here there is an explicit formula, akin to homotopy lifting or homotopy transfer.
TODO: Can one ensure that $F\circ \iota$ is constant in $t$?
\end{proof}

Now let $x,y\in \alg h$ be two MC elements. Suppose $z\in \alg g$ is an MC element such that $p_0(z)=x$ and $p_1(z)=y$.  
Given the Lemma one can then construct a naive gauge equivalence between $x$ and $y$.
Concretely, $G(z)\in \alg h[t,dt]$ is a family of Maurer-Cartan elements interpolating between $x$ and $y$ at $t=0$ and $t=1$, together with a family of homotopies. Integrating the flow generated by those homotopies (TODO: add detail here?) we find the explicit gauge transformation between $x$ and $y$.


\section{Vanishing Lemmas for graphs with (certain) bivalent and univalent vertices}
The goal here is to show that the integral weights of graphs involving bivalent vertices of several types vanish. This can be done by using the standard reflection argument due to Kontsevich.
%To this end, it is easiest to exploit the full $O(n)$-, respectively $O(m)\times O(n-m)$-symmetry, and pick our equivariant forms in the model 
%\[
%(S(\alg g^*[-2]) \otimes \Omega(M))^{G_0}.
%\]
%We can return to the smaller toric model simply by restriction to $\alg t\subset \alg g$.
%The main tool from which the vanishing results can be shown without actually computing any integrals is the following.
%\begin{lemma}
%There is no $O(n)$-anti-invariant $O(n)$-equivariant form on $S^{n-1}$ of degree $\leq n-2$.
%\end{lemma}
%\begin{proof}
%This is a consequence of classical invariant theory. Any $O(n)$-anti-invariant (equivariant) form on the sphere is uniquely determined by its value at one point of the sphere, which is an element of 
%\[
%\bigoplus_{i,j} (\wedge^2 \R^n)^i \otimes \wedge^j \R^{n-1}.
%\]
%Now this restriction has to still be $O(n-1)$-anti-invariant. But classical invariant theory (and the fact that $\wedge^2 \R^n\cong \R^{n-1}\otimes \wedge^2 \R^{n-1}$) dictates that any such tensor can be written using an odd number of copies of completely antiymmetric $n-1$-tensors, and any number of copies of the Euclidean product (i.e., symmetric 2-tensors).
%In particular, the lowest-degree $O(n-1)$-anti-invariant element occurs in degree $n-1$, when one uses precisely one antysymmetric $n-1$ tensor.
%\end{proof}

%The above result can be used to deduce various vanishing Lemmas, if we choose our propagators to always be maximally invariant.
%As an exercise, let us start with the basic vanishing Lemma.



%(It also follows with weaker pre-condition from an argument of Kontsevich, cf. \cite{}, however, let us rederive it as an exercise.)
\begin{lemma}\label{lem:bivalentvanish}
The following form vanishes:
\[
\begin{tikzpicture}
\node[int](v) at (0,0.2){};
\node[ext](v1) at (-0.5,0){1};
\node[ext](v2) at (0.5,0.5){2};
\draw (v) edge (v1) edge (v2);
\end{tikzpicture}
=0
\]
\end{lemma}
\begin{proof}
We can use the argument of \cite[Lemma 2.2]{KFeynman}, which we reproduce here for completeness, and to verify that it also works in the equivariant setting.
Number the black point as 3, and denote by $\alpha_{ij}$ the propagator between points $i$ and $j$, where $i,j\in \{1,2,3\}$.
The form above is then obtained as a fiber integral, integrating out point 3,
\[
 \int_3 \alpha_{13}\alpha_{23}.
\]
Note that we have chosen our propagator (anti-)invariantly under the inversion, hence $\alpha_{ij}=(-1)^n\alpha_{ji}$.
Now apply an inversion through the midpoint between points 1 and 2 to the integration variable, i.e., reflect the position of point 3 at that midpoint.
As is quickly verified, this change of variables sends out integral to
\[
 (-1)^n\int_3 \alpha_{32}\alpha_{31}=(-1)^n\int_3 \alpha_{23}\alpha_{13}=-\int_3 \alpha_{13}\alpha_{23}.
\]
In the last equality we have used that the two forms are of degree $n-1$. It follows that the integral equals minus itself and is hence zero.
\end{proof}

Similarly, one shows the following:

\begin{lemma}
The following form vanishes:
\[
\begin{tikzpicture}
\draw (-1,0)--(1,0);
\node[int](v) at (0,0){};
\node[ext](v1) at (-0.5,0){1};
\node[ext](v2) at (0.5,0){2};
\draw (v) edge[bend right] (v1) edge[bend left] (v2);
\end{tikzpicture}
=0
\]
\end{lemma}
\begin{proof}
The edge between the two type II vertices is assigned the form 
\[
E_{n-m} \alpha,
\]
where $\alpha$ is proportional to the $m$-dimensional propagator. Hence applying the previous Lemma (with $n$ replaced by $n-m$) gives the result.
\end{proof}


%\begin{lemma}
%The following form vanishes:
%\[
%\begin{tikzpicture}
%\draw (-.5,0) -- (.5,0);
%\coordinate(w) at (0,0);
%\node[int](v) at (0,0.5){};
%\node[ext](v2) at (.5,1){1};
%\draw (v) edge[dashed] (w) edge (v2);
%\end{tikzpicture}
%=0
%\]
%\end{lemma}
%\begin{proof}
%Choosing invariant propagators, the form is an $O(n-m)$-anti-invariant equivariant $n-m-2$-form on $S^{n-m-1}$. 
%\end{proof}
% 
% \begin{lemma}
% The following forms vanish:
% \begin{align*}
% &
% \begin{tikzpicture}
% \node[int](v) at (0,0.5){};
% \node[ext](v2) at (.5,1){1};
% \draw (v) edge (v2);
% \end{tikzpicture}
% &&
% \begin{tikzpicture}
% \draw (-.5,0) -- (.5,0);
% \node[int](v) at (0,0.5){};
% \node[ext](v2) at (.5,1){1};
% \draw (v) edge (v2);
% \end{tikzpicture}
% \end{align*}
% \end{lemma}
% \begin{proof}
% This is purely by degree reasons, the form would have degree $-1$.
% \end{proof}


\begin{lemma}
The weights of all graphs containing univalent vertices vanish, except for the graphs occurring in $\hZ^0_{m,n}$ in \eqref{equ:hZmn} above, and except for one-valent type II vertices that may be attached to type I vertices:
\[
\begin{tikzpicture}
\draw (-.5,0) -- (.5,0);
\node[int](w) at (0,0){};
\node[ext](v2) at (0,.5){1};
\draw (v2) edge (w);
\end{tikzpicture}
\]
\end{lemma}
\begin{proof}
Consider a graph with a univalent vertex $v$. We distinguish several cases.
First, suppose $v$ is of type I, and the graph has at least 2 other vertices.
The the vanishing of the configuration space integral is purely due to degree reasons: Consider the points in the configuration other than that (say $x$) corresponding to $v$ fixed.
Then $x$ traces out an $n$-dimensional space, but there are at most $n-1$ form degrees along $x$, hence the integral is zero.
The same argument works for the case that there is one other type I vertex (and the graph is of the type with a baseline).
This settles the case of $v$ of type I.

Next suppose $v$ is of type II, with the single edge connecting it to another type II vertex.
If there is at least one more vertex in the graph, the integral vanishes by analogous reasoning as before, just in lower dimension.
If not, we have a graph occurring in \eqref{equ:hZmn}.
\end{proof}

% 
% \section{Alternative arguments}
% This section contains simpler proofs of sub-statements of the main Theorem.
% (TODO: This is the old stuff, probably we should just remove it.)
% 
% 
% \subsection{Proof for \texorpdfstring{$G=\SO(n-3)$}{G=SO(n-3)}}
% TODO: correct mis-counting, $G=\SO(n-6)$
% 
% We will use the methods above to show the following Theorem:
% \begin{thm}\label{thm:triviality}
% The $\SO(m)$ action on the operad $E_n$ is rationally trivial if $n$ is odd and $m\leq n-2$, and if $n$ is even and $m\leq n-3$.
% \end{thm}
% We conjecture that for $n$ even the action is, in fact, trivial for $m\leq n-1$. For $n$ odd the above result is optimal.
% 
% To show the theorem it suffices to show that the Maurer-Cartan element in the graph complex $\GC_n\otimes H(B\SO(m))$ constructed along the above lines is zero.
% This will be done using two vanishing Lemmas:
% 
% \begin{lemma}\label{lem:vanishing1}
% The weights of graphs containing a bivalent vertex vanishes.
% \end{lemma}
% \begin{proof}
% We consider the equivariant $(n-2)$-form $\alpha$ on the sphere $S^{n-1}$ associated to the following graph.
% \begin{tikzpicture}
% \node[ext] (v) at (0,0) {};
% \node[ext] (w) at (1,0) {};
% \node[int] (vv) at (0.5,0) {};
% \draw (vv) edge (w) edge (v);
% \end{tikzpicture}
% 
% It is easily checked that it is equivariantly closed.
% We claim that it is zero.
% We consider the sphere as equipped with an action of $\SO(2m)$ and show the claim by an induction on $m$.
% For $m=0$ the claim is well-known.\footnote{For example, note that the form must be a $\SO(n)$ invariant, $O(n)$ anti-invariant $n-2$-form on $S^{n-1}$, and there is no such thing (except 0). }
% Suppose we know that the claim holds for $\SO(2m-2)$ and we want to show that the $\SO(2m)$-equivariant form, say $\alpha_m=0$.
% Note that by the induction hypothesis the equivariant form $\alpha_m$ has to be divisible by $u_1\cdots u_m$.
% Hence it has to have degree at least $2m$, and thus for $n=2m$ and $n=2m+1$ it follows immediately that $\alpha_m=0$. 
% 
% For $n=2m+2$ the form must be $u_1\cdots u_m$ times a function.
% This function must be $\SO(2)$ invariant and $O(2)$ anti-invariant by symmetry reasons, with $O(2)$ acting orthogonally to $\SO(2m)$. (I.e., we split $\R^{n}=\R^{n-2m}\times \R^{2m}$.) 
% But there are no such functions.
% 
% For $2m\leq n-3$ the form must be of degree $\geq n-2m-2 \geq 1$ in the $\R^{n-2m}$ direction. (Before integration the integrand has form degree $\geq 2(n-2m-1)$ in these directions, then one integrates out $n-2m$ dimensions.)
% But the form must also be $\SO(n-2m)$-invariant. But the only $\SO(n-2m)$ invariant forms on $S^{n-2m-1}$ of positive degree are multiples of the volume form. Hence $\alpha_m$ must be divisible by the volume form on $S^{n-2m-1}$, and by $u_1\cdots u_m$, and hence must have degree $\geq 2m + n-2m-1=n-1$, a contradiction.
% \end{proof}
%  
%  \begin{proof}[Proof of Theorem \ref{thm:triviality}]
%  Suppose a graph $\Gamma$ is given. By Lemma \ref{lem:vanishing1} we can assume that all vertices of $\Gamma$ are at least trivalent.
%  It suffices to show the vanishing of the integrals in codimension $n-m\geq 3$. (For $n$ odd and codimension $2$ the integrals are the same as for codimension 3.)
%  
% We split $\R^n=\R^{n-m}\times \R^m$. Note that each propagator $\Omega$ contributes a form of degree $\geq n-m-1$ along the  $\R^{n-m}$-directions. Hence, using the trivalence, the total degree of the integrand along those directions is
% \[
% \geq \frac{3}{2} |V(\Gamma)| (n-m-1).
% \]
% On the other hand the configuration space along this directions has dimension
% \[
% (|V(\Gamma)|-1) (n-m).
% \]
% Hence for $n-m\geq 3$ the integrand is of too high degree and hence has to vanish.
% \end{proof}


\bibliographystyle{plain}
\begin{thebibliography}{1}

\bibitem{AF}
David Ayala, John Francis.
\newblock Factorization homology of topological manifolds.
\newblock {\em J. Topol.} 8 (2015), no. 4, 1045--1084.

\bibitem{BM}
Clemens Berger and Ieke Moerdijk.
\newblock The Boardman-Vogt resolution of operads in monoidal model categories.
\newblock {\em Topology}, 45:807--849, 2006.

\bibitem{BoVo}
Michael Boardman and Rainer Vogt.
\newblock Homotopy invariant algebraic structures on topological spaces.
\newblock {\em Lect. Notes Math.} 347 (1973).

\bibitem{DolWill}
Vasily Dolgushev and Thomas Willwacher.
\newblock {Operadic twisting - with an application to Deligne's conjecture}.
\newblock {\em Journal of Pure and Applied Algebra}, 219(5):1349--1428, 2015.

\bibitem{DK1}
William Dwyer and Daniel Kan.
\newblock Simplicial localizations of categories.
\newblock {\em J. Pure Appl. Algebra} 17 (1980), 267--284.

\bibitem{DK2}
William Dwyer and Daniel Kan.
\newblock Calculating simplicial localizations.
\newblock {\em J. Pure Appl. Algebra} 18 (1980), 17--35.

\bibitem{FHT}
Yves F\'elix, Stephen Halperin and Jean-Claude Thomas.
\newblock Rational homotopy theory. 
\newblock Graduate Texts in Mathematics, 205. Springer-Verlag, New York, 2001. xxxiv+535 pp. ISBN: 0-387-95068-0

\bibitem{F}
Benoit Fresse.
\newblock Homotopy of Operads \& Grothendieck-Teichm\"uller Groups I, II, III.
\newblock to appear in Mathematical Surveys and Monographs of the AMS, 2016.

\bibitem{FTW}
Benoit~Fresse, Victor~Turchin and Thomas Willwacher.
\newblock {Deformation theory of the $E_n$-operads}.
\newblock in preparation.

\bibitem{GJ}
Ezra Getzler and J.~D.~S. Jones.
\newblock Operads, homotopy algebra and iterated integrals for double loop
  spaces, 1994.
\newblock arXiv:hep-th/9403055.

\bibitem{GS}
Jeffrey Giansiracusa and Paolo Salvatore.
\newblock Formality of the framed little 2-discs operad and semidirect
  products.
\newblock In {\em Homotopy theory of function spaces and related topics},
  volume 519 of {\em Contemp. Math.}, pages 115--121. Amer. Math. Soc.,
  Providence, RI, 2010.

\bibitem{G}
Thomas G. Goodwillie.
\newblock Calculus II. Analytic functors.
\newblock {\em K-Theory} 5 (1991/92) no. 4, 295--332.

\bibitem{GW}
Thomas G. Goodwillie and Michael Weiss.
\newblock Embeddings from the point of view of immersion theory. {II}.
\newblock {\em Geom. Topol.} 3 (1999), 103--118.

\bibitem{GKM}
M. Goresky, R. Kottwitz and R. MacPherson.
\newblock Equivariant Cohomology, Koszul duality
and the localization theorem.
\newblock {\em Invent. Math.}
131
(1998), no. 1, 25--83.

\bibitem{HLTV}
Robert Hardt, Pascal Lambrechts, Victor Turchin, and Ismar Voli{\'c}.
\newblock Real homotopy theory of semi-algebraic sets.
\newblock {\em Algebr. Geom. Topol.}, 11(5):2477--2545, 2011.

\bibitem{HessRHT}
Kathryn Hess.
\newblock Rational homotopy theory: a brief introduction.
\newblock Interactions between homotopy theory and algebra, 175--202, {\em Contemp. Math.}  436, Amer. Math. Soc., Providence, RI, 2007.

\bibitem{Ho}
Geoffroy Horel.
\newblock Profinite completion of operads and the Grothendieck-Teichm\"uller group.
\newblock Preprint arXiv:1504.01605.

\bibitem{KWZ}
Anton Khoroshkin, Thomas Willwacher and Marko \v Zivkovi\'c
\newblock Differentials on graph complexes.
\newblock preprint arXiv:1411.2369, to appear in Adv. Math.

\bibitem{K1}
Maxim Kontsevich.
\newblock Deformation quantization of Poisson manifolds.
\newblock {\em Lett. Math. Phys.}, 66(3):157–216, 2003.

\bibitem{K2}
Maxim Kontsevich.
\newblock Operads and {M}otives in {D}eformation {Q}uantization.
\newblock {\em Lett. Math. Phys.}, 48:35--72, 1999.

\bibitem{Knoncomm}
Maxim Kontsevich.
\newblock Formal (non)commutative symplectic geometry.
\newblock Proceedings of the I. M. Gelfand seminar 1990-1992, 173--188, Birkhauser, 1993.

\bibitem{KFeynman}
Maxim Kontsevich.
\newblock Feynman diagrams and low-dimensional topology.
\newblock {\em Progr. Math.}, 120:97--121, 1994.
\newblock First European Congress of Mathematics, Vol. II, (Paris, 1992).

\bibitem{KS}
Maxim Kontsevich and Yan Soibelman.
\newblock Deformations of algebras over operads and the Deligne conjecture.
\newblock Conf\'erence Mosh\'e Flato 1999, Vol. I (Dijon), 255--307,
Math. Phys. Stud., 21, Kluwer Acad. Publ., Dordrecht, 2000. 

\bibitem{LV}
Pascal Lambrechts and Ismar Volic.
\newblock Formality of the little {N}-disks operad, 2008.
\newblock arXiv:0808.0457.

\bibitem{Libine}
Matvei Libine.
\newblock Lecture Notes on Equivariant Cohomology.
\newblock arXiv:0709.3615.

\bibitem{MW}
Ieke Moerdijk and Ittay Weiss.
\newblock Dendroidal sets.
\newblock {\em Algebraic \& Geometric Topology} 7 (2007) 1441--1470.

\bibitem{Mo}
Syunji Moriya.
\newblock Non-formality of the odd dimensional framed little balls operads
\newblock Preprint, arXiv:1609.06800.


\bibitem{pavolfr}
Pavol {\v{S}}evera.
\newblock Formality of the chain operad of framed little disks.
\newblock {\em Lett. Math. Phys.}, 93(1):29--35, 2010.

\bibitem{Riehl}
Emily Riehl.
\newblock Categorical homotopy theory.
\newblock New Mathematical Monographs, 24. Cambridge University Press, Cambridge, 2014. xviii+352 pp. ISBN: 978-1-107-04845-4

\bibitem{SW}
Paolo Salvatore and Nathalie Wahl.
\newblock Framed discs operads and Batalin-Vilkovisky algebras.
\newblock Q. J. Math. 54 (2003), no. 2, 213--231. 

\bibitem{Si}
D. Sinha.
\newblock Manifold-theoretic compactifications of configuration spaces.
\newblock {\em Selecta Math. (N.S.)} 10, no. 3, 391--428, 2004.

\bibitem{Tam}
Dmitry E. Tamarkin.
\newblock Formality of chain operad of little discs.
\newblock {\em Lett. Math. Phys.} 66, no. 1-2, 65--72, 2003.

\bibitem{Will}
Thomas Willwacher.
\newblock M. {K}ontsevich's graph complex and the
  {G}rothendieck--{T}eichm\"uller {L}ie algebra.
\newblock {\em Invent. Math.}, 200(3):671--760, 2015.

\end{thebibliography}



\end{document}
